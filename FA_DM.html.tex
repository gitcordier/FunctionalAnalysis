<!DOCTYPE html>
<html >
<head>
<meta http-equiv="Content-Type" content="text/html; charset=UTF-8">
<meta name="generator" content="hevea 2.32">
<style type="text/css">
.li-itemize{margin:1ex 0ex;}
.li-enumerate{margin:1ex 0ex;}
.dd-description{margin:0ex 0ex 1ex 4ex;}
.dt-description{margin:0ex;}
.toc{list-style:none;}
.footnotetext{margin:0ex; padding:0ex;}
div.footnotetext P{margin:0px; text-indent:1em;}
.thefootnotes{text-align:left;margin:0ex;}
.dt-thefootnotes{margin:0em;}
.dd-thefootnotes{margin:0em 0em 0em 2em;}
.footnoterule{margin:1em auto 1em 0px;width:50%;}
.caption{padding-left:2ex; padding-right:2ex; margin-left:auto; margin-right:auto}
.title{margin:2ex auto;text-align:center}
.titlemain{margin:1ex 2ex 2ex 1ex;}
.titlerest{margin:0ex 2ex;}
.center{text-align:center;margin-left:auto;margin-right:auto;}
.flushleft{text-align:left;margin-left:0ex;margin-right:auto;}
.flushright{text-align:right;margin-left:auto;margin-right:0ex;}
div table{margin-left:inherit;margin-right:inherit;margin-bottom:2px;margin-top:2px}
td table{margin:auto;}
table{border-collapse:collapse;}
td{padding:0;}
.cellpadding0 tr td{padding:0;}
.cellpadding1 tr td{padding:1px;}
pre{text-align:left;margin-left:0ex;margin-right:auto;}
blockquote{margin-left:4ex;margin-right:4ex;text-align:left;}
td p{margin:0px;}
.boxed{border:1px solid black}
.textboxed{border:1px solid black}
.vbar{border:none;width:2px;background-color:black;}
.hbar{border:none;height:2px;width:100%;background-color:black;}
.hfill{border:none;height:1px;width:200%;background-color:black;}
.vdisplay{border-collapse:separate;border-spacing:2px;width:auto; empty-cells:show; border:2px solid red;}
.vdcell{white-space:nowrap;padding:0px; border:2px solid green;}
.display{border-collapse:separate;border-spacing:2px;width:auto; border:none;}
.dcell{white-space:nowrap;padding:0px; border:none;}
.dcenter{margin:0ex auto;}
.vdcenter{border:solid #FF8000 2px; margin:0ex auto;}
.minipage{text-align:left; margin-left:0em; margin-right:auto;}
.marginpar{border:solid thin black; width:20%; text-align:left;}
.marginparleft{float:left; margin-left:0ex; margin-right:1ex;}
.marginparright{float:right; margin-left:1ex; margin-right:0ex;}
.theorem{text-align:left;margin:1ex auto 1ex 0ex;}
.part{margin:2ex auto;text-align:center}
</style>
<title>
Solutions to some exercises from Walter Rudin's Functional Analysis

</title>
</head>
<body >
<!--HEVEA command line is: hevea FA_DM -->
<!--CUT STYLE book--><!--CUT DEF chapter 1 --><table class="title"><tr><td style="padding:1ex"><h1 class="titlemain">
Solutions to some exercises from Walter Rudin’s <span style="font-style:italic">Functional Analysis</span>
</h1><h3 class="titlerest">gitcordier</h3></td></tr>
</table><!--TOC chapter id="sec1" Contents-->
<h1 id="sec1" class="chapter">Contents</h1><!--SEC END --><ul class="toc"><li class="li-toc">
<a href="#sec2">Chapter 1  Topological Vector Spaces</a>
<ul class="toc"><li class="li-toc">
<a href="#sec3">1.1  Exercise 7. Metrizability &amp; number theory</a>
</li><li class="li-toc"><a href="#sec4">1.2  Exercise 9. Quotient map</a>
</li><li class="li-toc"><a href="#sec5">1.3  Exercise 10. An open mapping theorem</a>
</li><li class="li-toc"><a href="#sec6">1.4  Exercise 14. <span style="font-style:italic">D</span><sub><span style="font-style:italic">K</span></sub> equipped with other seminorms</a>
</li><li class="li-toc"><a href="#sec7">1.5  Exercise 16. Uniqueness of topology for test functions</a>
</li><li class="li-toc"><a href="#sec14">1.6  Exercise 17. Derivation in some non normed space</a>
</li></ul>
</li><li class="li-toc">bibliography
</li></ul>
<!--TOC chapter id="sec2" Topological Vector Spaces-->
<h1 id="sec2" class="chapter">Chapter 1  Topological Vector Spaces</h1><!--SEC END -->
<!--TOC section id="sec3" Exercise 7. Metrizability &amp; number theory-->
<h2 id="sec3" class="section">1.1  Exercise 7. Metrizability &amp; number theory</h2><!--SEC END --><p>
<span style="font-style:italic">
Let be X the vector space of all complex functions on the unit interval 
</span>[0, 1]<span style="font-style:italic">, topologized by the family of seminorms 
</span></p><table class="display dcenter"><tr style="vertical-align:middle"><td class="dcell">
     

</td><td class="dcell"><table style="border-spacing:6px;border-collapse:separate;" class="cellpading0"><tr><td style="text-align:right;white-space:nowrap" >    <span style="font-style:italic">p</span><sub><span style="font-style:italic">x</span></sub>(<span style="font-style:italic">f</span>)=|<span style="font-style:italic">f</span>(<span style="font-style:italic">x</span>)|    (0≤ <span style="font-style:italic">x</span>≤ 1).

</td><td style="text-align:left;white-space:nowrap" >&nbsp;</td><td style="text-align:right;white-space:nowrap" >&nbsp;</td><td style="text-align:left;white-space:nowrap" >&nbsp;</td><td style="text-align:right;white-space:nowrap" >&nbsp;</td><td style="text-align:left;white-space:nowrap" >&nbsp;</td><td style="text-align:right;white-space:nowrap" >&nbsp;</td><td style="text-align:left;white-space:nowrap" >&nbsp;</td><td style="text-align:right;white-space:nowrap" >&nbsp;</td><td style="text-align:left;white-space:nowrap" >&nbsp;</td><td style="text-align:right;white-space:nowrap" >&nbsp;</td></tr>
</table></td></tr>
</table><p><span style="font-style:italic">
This topology is called the topology of pointwise convergence. 
Justify this terminology.
Show that there is a sequence </span>{<span style="font-style:italic">f</span><sub><span style="font-style:italic">n</span></sub>}<span style="font-style:italic"> in X such that (a) </span>{<span style="font-style:italic">f</span><sub><span style="font-style:italic">n</span></sub>}<span style="font-style:italic"> converges 
to </span>0<span style="font-style:italic"> as </span><span style="font-style:italic">n</span> →∞<span style="font-style:italic">, but (b) if </span>{γ<sub><span style="font-style:italic">n</span></sub>}<span style="font-style:italic"> is any sequence of scalars such 
that </span>γ<sub><span style="font-style:italic">n</span></sub>→∞<span style="font-style:italic"> then </span>{γ<sub><span style="font-style:italic">n</span></sub><span style="font-style:italic">f</span><sub><span style="font-style:italic">n</span></sub>}<span style="font-style:italic"> does not converge to </span>0<span style="font-style:italic">. 
(Use the fact that the collection of all complex sequences converging to </span>0<span style="font-style:italic"> 
has the same cardinality as </span>[0, 1]<span style="font-style:italic">.)
This shows that metrizability cannot be omited in (b) of Theorem 1.28.
</span>
</p><p><span style="font-weight:bold">Proof.</span>
Our justification consists in proving that τ-convergence and pointwise 
convergence are the same one. 
To do so, remark first that the family of the seminorms <span style="font-style:italic">p</span><sub><span style="font-style:italic">x</span></sub> is separating.
By [1.37], the collection <span style="font-style:italic">B</span> of all finite intersections 
of the sets 
</p><table class="display dcenter"><tr style="vertical-align:middle"><td class="dcell">
     

</td><td class="dcell"><table style="border-spacing:6px;border-collapse:separate;" class="cellpading0"><tr><td style="text-align:right;white-space:nowrap" >    <span style="font-style:italic">V</span><sup>((<span style="font-style:italic">x</span>, <span style="font-style:italic">k</span>)</sup> 
≜ 
{<span style="font-style:italic">p</span><sub><span style="font-style:italic">x</span></sub> &lt; 2<sup><span style="font-size:small">-</span> <span style="font-style:italic">k</span></sup>} 
   
(<span style="font-style:italic">x</span> ∈ [0, 1], <span style="font-style:italic">k</span> ∈ <span style="font-weight:bold"><span style="font-style:italic">N</span></span>)
</td><td style="text-align:left;white-space:nowrap" >&nbsp;</td><td style="text-align:right;white-space:nowrap" >&nbsp;</td><td style="text-align:left;white-space:nowrap" >&nbsp;</td><td style="text-align:right;white-space:nowrap" >&nbsp;</td><td style="text-align:left;white-space:nowrap" >&nbsp;</td><td style="text-align:right;white-space:nowrap" >&nbsp;</td><td style="text-align:left;white-space:nowrap" >&nbsp;</td><td style="text-align:right;white-space:nowrap" >&nbsp;</td><td style="text-align:left;white-space:nowrap" >&nbsp;</td><td style="text-align:right;white-space:nowrap" >    (1)</td></tr>
</table></td></tr>
</table><p>
is then a local base for a topology τ on <span style="font-style:italic">X</span>. Given 
{<span style="font-style:italic">f</span><sub><span style="font-style:italic">n</span></sub>: <span style="font-style:italic">n</span>=1, 2, 3, …}, 
we set

</p><table class="display dcenter"><tr style="vertical-align:middle"><td class="dcell">
     

</td><td class="dcell"><table style="border-spacing:6px;border-collapse:separate;" class="cellpading0"><tr><td style="text-align:right;white-space:nowrap" ><table class="display"><tr style="vertical-align:middle"><td class="dcell">  <span style="font-family:monospace"><span style="font-style:italic">off</span></span>(<span style="font-style:italic">U</span>) ≜ </td><td class="dcell"><table class="display"><tr><td class="dcell" style="text-align:center">∞</td></tr>
<tr><td class="dcell" style="text-align:center"><span style="font-size:xx-large">∑</span></td></tr>
<tr><td class="dcell" style="text-align:center"><span style="font-style:italic">n</span>=1</td></tr>
</table></td><td class="dcell">[<span style="font-style:italic">f</span><sub><span style="font-style:italic">n</span></sub> ∉ <span style="font-style:italic">U</span>]    (<span style="font-style:italic">U</span>∈τ),
</td></tr>
</table></td><td style="text-align:left;white-space:nowrap" >&nbsp;</td><td style="text-align:right;white-space:nowrap" >&nbsp;</td><td style="text-align:left;white-space:nowrap" >&nbsp;</td><td style="text-align:right;white-space:nowrap" >&nbsp;</td><td style="text-align:left;white-space:nowrap" >&nbsp;</td><td style="text-align:right;white-space:nowrap" >&nbsp;</td><td style="text-align:left;white-space:nowrap" >&nbsp;</td><td style="text-align:right;white-space:nowrap" >&nbsp;</td><td style="text-align:left;white-space:nowrap" >&nbsp;</td><td style="text-align:right;white-space:nowrap" >    (2)</td></tr>
</table></td></tr>
</table><p>
with the convention <span style="font-family:monospace"><span style="font-style:italic">off</span></span>(<span style="font-style:italic">U</span>)=∞ whether the sum has no finite support. 
So, 
</p><table class="display dcenter"><tr style="vertical-align:middle"><td class="dcell">
     

</td><td class="dcell"><table style="border-spacing:6px;border-collapse:separate;" class="cellpading0"><tr><td style="text-align:right;white-space:nowrap" ><table class="display"><tr style="vertical-align:middle"><td class="dcell">    <a id="Inequality boolean series"> </a>
</td><td class="dcell"><table class="display"><tr><td class="dcell" style="text-align:center"><span style="font-style:italic">m</span></td></tr>
<tr><td class="dcell" style="text-align:center"><span style="font-size:xx-large">∑</span></td></tr>
<tr><td class="dcell" style="text-align:center"><span style="font-style:italic">i</span>=1</td></tr>
</table></td><td class="dcell"> <span style="font-family:monospace"><span style="font-style:italic">off</span></span>(<span style="font-style:italic">U</span><sup>(<span style="font-style:italic">i</span>)</sup>) 
= 
</td><td class="dcell"><table class="display"><tr><td class="dcell" style="text-align:center">∞</td></tr>
<tr><td class="dcell" style="text-align:center"><span style="font-size:xx-large">∑</span></td></tr>
<tr><td class="dcell" style="text-align:center"><span style="font-style:italic">n</span>=1</td></tr>
</table></td><td class="dcell"><table class="display"><tr><td class="dcell" style="text-align:center"><span style="font-style:italic">m</span></td></tr>
<tr><td class="dcell" style="text-align:center"><span style="font-size:xx-large">∑</span></td></tr>
<tr><td class="dcell" style="text-align:center"><span style="font-style:italic">i</span>=1</td></tr>
</table></td><td class="dcell"> [<span style="font-style:italic">f</span><sub><span style="font-style:italic">n</span></sub> ∉ <span style="font-style:italic">U</span><sup>(<span style="font-style:italic">i</span>)</sup>]
≥ 
<span style="font-family:monospace"><span style="font-style:italic">off</span></span>(<span style="font-style:italic">U</span><sup>(1)</sup> ⋂ ⋯ ⋂ <span style="font-style:italic">U</span><sup>(<span style="font-style:italic">m</span>)</sup>)
</td></tr>
</table></td><td style="text-align:left;white-space:nowrap" >&nbsp;</td><td style="text-align:right;white-space:nowrap" >&nbsp;</td><td style="text-align:left;white-space:nowrap" >&nbsp;</td><td style="text-align:right;white-space:nowrap" >&nbsp;</td><td style="text-align:left;white-space:nowrap" >&nbsp;</td><td style="text-align:right;white-space:nowrap" >&nbsp;</td><td style="text-align:left;white-space:nowrap" >&nbsp;</td><td style="text-align:right;white-space:nowrap" >&nbsp;</td><td style="text-align:left;white-space:nowrap" >&nbsp;</td><td style="text-align:right;white-space:nowrap" >    (3)</td></tr>
</table></td></tr>
</table><p>
We first assume that {<span style="font-style:italic">f</span><sub><span style="font-style:italic">n</span></sub>} τ-converges to some <span style="font-style:italic">f</span> in <span style="font-style:italic">X</span>, <span style="font-style:italic">i.e.</span> </p><table class="display dcenter"><tr style="vertical-align:middle"><td class="dcell">
     

</td><td class="dcell"><table style="border-spacing:6px;border-collapse:separate;" class="cellpading0"><tr><td style="text-align:right;white-space:nowrap" >    <span style="font-family:monospace"><span style="font-style:italic">off</span></span>(<span style="font-style:italic">f</span>+<span style="font-style:italic">V</span>) &lt; ∞   (<span style="font-style:italic">V</span> ∈ <span style="font-style:italic">B</span>).
</td><td style="text-align:left;white-space:nowrap" >&nbsp;</td><td style="text-align:right;white-space:nowrap" >&nbsp;</td><td style="text-align:left;white-space:nowrap" >&nbsp;</td><td style="text-align:right;white-space:nowrap" >&nbsp;</td><td style="text-align:left;white-space:nowrap" >&nbsp;</td><td style="text-align:right;white-space:nowrap" >&nbsp;</td><td style="text-align:left;white-space:nowrap" >&nbsp;</td><td style="text-align:right;white-space:nowrap" >&nbsp;</td><td style="text-align:left;white-space:nowrap" >&nbsp;</td><td style="text-align:right;white-space:nowrap" >    (4)</td></tr>
</table></td></tr>
</table><p>
The special cases <span style="font-style:italic">V</span>=<span style="font-style:italic">V</span><sup>(<span style="font-style:italic">x</span>, <span style="font-style:italic">k</span>)</sup> mean the pointwise convergence of 
{<span style="font-style:italic">f</span><sub><span style="font-style:italic">n</span></sub>}. 
Conversely, assume that {<span style="font-style:italic">f</span><sub><span style="font-style:italic">n</span></sub>} does not τ-converges to any <span style="font-style:italic">g</span> 
in <span style="font-style:italic">X</span>, <span style="font-style:italic">i.e.</span> </p><table class="display dcenter"><tr style="vertical-align:middle"><td class="dcell">
     

</td><td class="dcell"><table style="border-spacing:6px;border-collapse:separate;" class="cellpading0"><tr><td style="text-align:right;white-space:nowrap" >    <a id="Divergence"> </a>
∀ <span style="font-style:italic">g</span> ∈ <span style="font-style:italic">X</span>, ∃ <span style="font-style:italic">V</span><sup>(<span style="font-style:italic">g</span>)</sup> ∈  <span style="font-style:italic">B</span>: 
<span style="font-family:monospace"><span style="font-style:italic">off</span></span>(<span style="font-style:italic">g</span>+<span style="font-style:italic">V</span><sup>(<span style="font-style:italic">g</span>)</sup>) = ∞. 
</td><td style="text-align:left;white-space:nowrap" >&nbsp;</td><td style="text-align:right;white-space:nowrap" >&nbsp;</td><td style="text-align:left;white-space:nowrap" >&nbsp;</td><td style="text-align:right;white-space:nowrap" >&nbsp;</td><td style="text-align:left;white-space:nowrap" >&nbsp;</td><td style="text-align:right;white-space:nowrap" >&nbsp;</td><td style="text-align:left;white-space:nowrap" >&nbsp;</td><td style="text-align:right;white-space:nowrap" >&nbsp;</td><td style="text-align:left;white-space:nowrap" >&nbsp;</td><td style="text-align:right;white-space:nowrap" >    (5)</td></tr>
</table></td></tr>
</table><p>
Given <span style="font-style:italic">g</span>, <span style="font-style:italic">V</span><sup>(<span style="font-style:italic">g</span>)</sup> is then an intersection

<span style="font-style:italic">V</span><sup>(<span style="font-style:italic">x</span><sup>(1)</sup>, <span style="font-style:italic">k</span><sup>(1)</sup>)</sup> ∩ ⋯ ∩ 
<span style="font-style:italic">V</span><sup>(<span style="font-style:italic">x</span><sup>(<span style="font-style:italic">m</span>)</sup>, <span style="font-style:italic">k</span><sup>(<span style="font-style:italic">m</span>)</sup>)</sup> 
.
Thus
</p><table class="display dcenter"><tr style="vertical-align:middle"><td class="dcell">
     

</td><td class="dcell"><table style="border-spacing:6px;border-collapse:separate;" class="cellpading0"><tr><td style="text-align:right;white-space:nowrap" ><table class="display"><tr style="vertical-align:middle"><td class="dcell">    </td><td class="dcell"><table class="display"><tr><td class="dcell" style="text-align:center"><span style="font-style:italic">m</span></td></tr>
<tr><td class="dcell" style="text-align:center"><span style="font-size:xx-large">∑</span></td></tr>
<tr><td class="dcell" style="text-align:center"><span style="font-style:italic">i</span>=1</td></tr>
</table></td><td class="dcell"> <span style="font-family:monospace"><span style="font-style:italic">off</span></span>(<span style="font-style:italic">g</span> + <span style="font-style:italic">V</span><sup>(<span style="font-style:italic">x</span><sup>(<span style="font-style:italic">i</span>)</sup>, <span style="font-style:italic">k</span><sup>(<span style="font-style:italic">i</span>)</sup>)</sup>) 
(<a href="#Inequality%20boolean%20series">3</a>)≥ 
<span style="font-family:monospace"><span style="font-style:italic">off</span></span>(<span style="font-style:italic">g</span> + <span style="font-style:italic">V</span><sup>(<span style="font-style:italic">g</span>)</sup>) 
(<a href="#Divergence">5</a>)= 
∞ .
</td></tr>
</table></td><td style="text-align:left;white-space:nowrap" >&nbsp;</td><td style="text-align:right;white-space:nowrap" >&nbsp;</td><td style="text-align:left;white-space:nowrap" >&nbsp;</td><td style="text-align:right;white-space:nowrap" >&nbsp;</td><td style="text-align:left;white-space:nowrap" >&nbsp;</td><td style="text-align:right;white-space:nowrap" >&nbsp;</td><td style="text-align:left;white-space:nowrap" >&nbsp;</td><td style="text-align:right;white-space:nowrap" >&nbsp;</td><td style="text-align:left;white-space:nowrap" >&nbsp;</td><td style="text-align:right;white-space:nowrap" >    (6)</td></tr>
</table></td></tr>
</table><p>
One of the sum <span style="font-family:monospace"><span style="font-style:italic">off</span></span>(<span style="font-style:italic">g</span> + <span style="font-style:italic">V</span><sup>(<span style="font-style:italic">x</span><sup>(<span style="font-style:italic">i</span>)</sup>, <span style="font-style:italic">k</span><sup>(<span style="font-style:italic">i</span>)</sup>)</sup>) must then be ∞. 
This implies that convergence of <span style="font-style:italic">f</span><sub><span style="font-style:italic">n</span></sub> to <span style="font-style:italic">g</span> fails at point <span style="font-style:italic">x</span><sub><span style="font-style:italic">i</span></sub>.
<span style="font-style:italic">g</span> being arbitrary, we so conclude that <span style="font-style:italic">f</span><sub><span style="font-style:italic">n</span></sub> does not converge pointwise.
We have just proved that 
τ-convergence is a rewording of pointwise convergence.
We now aim to prove the second part.
From now on, 
<span style="font-style:italic">k</span>, <span style="font-style:italic">n</span> and <span style="font-style:italic">p</span> 
run on <span style="font-weight:bold"><span style="font-style:italic">N</span></span><sub>+</sub>. Let <span style="font-family:monospace"><span style="font-style:italic">dyadic</span></span>(<span style="font-style:italic">x</span>) be the usual dyadic expansion of a real number <span style="font-style:italic">x</span>, 
so that <span style="font-family:monospace"><span style="font-style:italic">dyadic</span></span>(<span style="font-style:italic">x</span>) is an aperiodic binary sequence <span style="font-weight:bold">iff</span> <span style="font-style:italic">x</span> is irrational. 
Define
</p><table class="display dcenter"><tr style="vertical-align:middle"><td class="dcell">
     

</td><td class="dcell"><table style="border-spacing:6px;border-collapse:separate;" class="cellpading0"><tr><td style="text-align:right;white-space:nowrap" ><table class="display"><tr style="vertical-align:middle"><td class="dcell">  <a id="f_n(x) definition"> </a>
<span style="font-style:italic">f</span><sub><span style="font-style:italic">n</span></sub>(<span style="font-style:italic">x</span>) 
≜ 
</td><td class="dcell"><table class="display"><tr style="vertical-align:middle"><td class="dcell">⎧<br>
⎪<br>
⎨<br>
⎪<br>
⎩</td><td class="dcell"><table style="border-spacing:6px;border-collapse:separate;" class="cellpading0"><tr><td style="text-align:left;white-space:nowrap" ><table class="display"><tr style="vertical-align:middle"><td class="dcell">      2</td><td class="dcell"><table class="display"><tr><td class="dcell"><table class="display"><tr style="vertical-align:middle"><td class="dcell">−</td><td class="dcell"><table class="display"><tr><td class="dcell" style="text-align:center"><span style="font-style:italic">n</span></td></tr>
<tr><td class="dcell" style="text-align:center"><span style="font-size:xx-large">∑</span></td></tr>
<tr><td class="dcell" style="text-align:center"><span style="font-style:italic">k</span>= 1</td></tr>
</table></td><td class="dcell"> <span style="font-family:monospace"><span style="font-style:italic">dyadic</span></span>(<span style="font-style:italic">x</span>)<sub>−<span style="font-style:italic">k</span></sub></td></tr>
</table></td></tr>
<tr><td class="dcell">&nbsp;</td></tr>
</table></td></tr>
</table></td><td style="text-align:left;white-space:nowrap" >(<span style="font-style:italic">x</span> ∈ [0, 1]∖ <span style="font-weight:bold"><span style="font-style:italic">Q</span></span> )</td></tr>
<tr><td style="text-align:left;white-space:nowrap" >      0</td><td style="text-align:left;white-space:nowrap" >(<span style="font-style:italic">x</span> ∈ [0,1]⋂ <span style="font-weight:bold"><span style="font-style:italic">Q</span></span>)
</td></tr>
</table></td></tr>
</table></td></tr>
</table></td><td style="text-align:left;white-space:nowrap" >&nbsp;</td><td style="text-align:right;white-space:nowrap" >&nbsp;</td><td style="text-align:left;white-space:nowrap" >&nbsp;</td><td style="text-align:right;white-space:nowrap" >&nbsp;</td><td style="text-align:left;white-space:nowrap" >&nbsp;</td><td style="text-align:right;white-space:nowrap" >&nbsp;</td><td style="text-align:left;white-space:nowrap" >&nbsp;</td><td style="text-align:right;white-space:nowrap" >&nbsp;</td><td style="text-align:left;white-space:nowrap" >&nbsp;</td><td style="text-align:right;white-space:nowrap" >    (7)</td></tr>
</table></td></tr>
</table><p>
so that <span style="font-style:italic">f</span><sub><span style="font-style:italic">n</span></sub>(<span style="font-style:italic">x</span>) <span style="font-style:italic">n</span>→∞—→ 0
and take scalars 
γ<sub><span style="font-style:italic">n</span></sub> 
such that <span style="font-style:italic">n</span>→∞—→ ∞, <span style="font-style:italic">i.e.</span> at fixed <span style="font-style:italic">p</span>, γ<sub><span style="font-style:italic">n</span></sub> is greater than 2<sup><span style="font-style:italic">p</span></sup> for almost all <span style="font-style:italic">n</span>.
Next, choose <span style="font-style:italic">n</span><sup>(<span style="font-style:italic">p</span>)</sup> among those <span style="font-style:italic">almost all</span> <span style="font-style:italic">n</span> that are 
large enough to satisfy 
</p><table class="display dcenter"><tr style="vertical-align:middle"><td class="dcell">
     

</td><td class="dcell"><table style="border-spacing:6px;border-collapse:separate;" class="cellpading0"><tr><td style="text-align:right;white-space:nowrap" >    <span style="font-style:italic">n</span><sup>(<span style="font-style:italic">p</span>−1)</sup> − <span style="font-style:italic">n</span><sup>(<span style="font-style:italic">p</span>−2)</sup> &lt; <span style="font-style:italic">n</span><sup>(<span style="font-style:italic">p</span>)</sup>− <span style="font-style:italic">n</span><sup>(<span style="font-style:italic">p</span>−1)</sup> 
</td><td style="text-align:left;white-space:nowrap" >&nbsp;</td><td style="text-align:right;white-space:nowrap" >&nbsp;</td><td style="text-align:left;white-space:nowrap" >&nbsp;</td><td style="text-align:right;white-space:nowrap" >&nbsp;</td><td style="text-align:left;white-space:nowrap" >&nbsp;</td><td style="text-align:right;white-space:nowrap" >&nbsp;</td><td style="text-align:left;white-space:nowrap" >&nbsp;</td><td style="text-align:right;white-space:nowrap" >&nbsp;</td><td style="text-align:left;white-space:nowrap" >&nbsp;</td><td style="text-align:right;white-space:nowrap" >    (8)</td></tr>
</table></td></tr>
</table><p>
(start with <span style="font-style:italic">n</span><sup>(<span style="font-size:small">-</span> 1)</sup> = <span style="font-style:italic">n</span><sup>(0)</sup> = 0) and so obtain 
</p><table class="display dcenter"><tr style="vertical-align:middle"><td class="dcell">
     

</td><td class="dcell"><table style="border-spacing:6px;border-collapse:separate;" class="cellpading0"><tr><td style="text-align:right;white-space:nowrap" >    2<sup><span style="font-style:italic">p</span></sup> &lt; γ<sub><span style="font-style:italic">n</span><sup>(<span style="font-style:italic">p</span>)</sup></sub>:  
0&lt; <span style="font-style:italic">n</span><sup>(<span style="font-style:italic">p</span>)</sup> − <span style="font-style:italic">n</span><sup>(<span style="font-style:italic">p</span>−1)</sup><span style="font-style:italic">p</span>→∞—→ ∞.
</td><td style="text-align:left;white-space:nowrap" >&nbsp;</td><td style="text-align:right;white-space:nowrap" >&nbsp;</td><td style="text-align:left;white-space:nowrap" >&nbsp;</td><td style="text-align:right;white-space:nowrap" >&nbsp;</td><td style="text-align:left;white-space:nowrap" >&nbsp;</td><td style="text-align:right;white-space:nowrap" >&nbsp;</td><td style="text-align:left;white-space:nowrap" >&nbsp;</td><td style="text-align:right;white-space:nowrap" >&nbsp;</td><td style="text-align:left;white-space:nowrap" >&nbsp;</td><td style="text-align:right;white-space:nowrap" >    (9)</td></tr>
</table></td></tr>
</table><p>
The indicator χ of 
{<span style="font-style:italic">n</span><sup>(1)</sup>, <span style="font-style:italic">n</span><sup>(2)</sup>, …}
is then aperiodic, <span style="font-style:italic">i.e.</span> 
</p><table class="display dcenter"><tr style="vertical-align:middle"><td class="dcell">
     

</td><td class="dcell"><table style="border-spacing:6px;border-collapse:separate;" class="cellpading0"><tr><td style="text-align:right;white-space:nowrap" ><table class="display"><tr style="vertical-align:middle"><td class="dcell">    <span style="font-style:italic">x</span><sup>(γ)</sup> 
≜
</td><td class="dcell"><table class="display"><tr><td class="dcell" style="text-align:center">∞</td></tr>
<tr><td class="dcell" style="text-align:center"><span style="font-size:xx-large">∑</span></td></tr>
<tr><td class="dcell" style="text-align:center"><span style="font-style:italic">k</span>=1</td></tr>
</table></td><td class="dcell">χ<sub><span style="font-style:italic">k</span></sub> 2<sup><span style="font-size:small">-</span> <span style="font-style:italic">k</span></sup> 
</td></tr>
</table></td><td style="text-align:left;white-space:nowrap" >&nbsp;</td><td style="text-align:right;white-space:nowrap" >&nbsp;</td><td style="text-align:left;white-space:nowrap" >&nbsp;</td><td style="text-align:right;white-space:nowrap" >&nbsp;</td><td style="text-align:left;white-space:nowrap" >&nbsp;</td><td style="text-align:right;white-space:nowrap" >&nbsp;</td><td style="text-align:left;white-space:nowrap" >&nbsp;</td><td style="text-align:right;white-space:nowrap" >&nbsp;</td><td style="text-align:left;white-space:nowrap" >&nbsp;</td><td style="text-align:right;white-space:nowrap" >    (10)</td></tr>
</table></td></tr>
</table><p>
is irrational. Consequently,
</p><table class="display dcenter"><tr style="vertical-align:middle"><td class="dcell">
     

</td><td class="dcell"><table style="border-spacing:6px;border-collapse:separate;" class="cellpading0"><tr><td style="text-align:right;white-space:nowrap" >    <span style="font-family:monospace"><span style="font-style:italic">dyadic</span></span>(<span style="font-style:italic">x</span><sup>(γ)</sup>)<sub><span style="font-size:small">-</span> <span style="font-style:italic">k</span></sub></td><td style="text-align:left;white-space:nowrap" >= χ<sub><span style="font-style:italic">k</span></sub>.
</td><td style="text-align:right;white-space:nowrap" >&nbsp;</td><td style="text-align:left;white-space:nowrap" >&nbsp;</td><td style="text-align:right;white-space:nowrap" >&nbsp;</td><td style="text-align:left;white-space:nowrap" >&nbsp;</td><td style="text-align:right;white-space:nowrap" >&nbsp;</td><td style="text-align:left;white-space:nowrap" >&nbsp;</td><td style="text-align:right;white-space:nowrap" >&nbsp;</td><td style="text-align:left;white-space:nowrap" >&nbsp;</td><td style="text-align:right;white-space:nowrap" >    (11)</td></tr>
</table></td></tr>
</table><p>
We now easily see that
</p><table class="display dcenter"><tr style="vertical-align:middle"><td class="dcell">
     

</td><td class="dcell"><table style="border-spacing:6px;border-collapse:separate;" class="cellpading0"><tr><td style="text-align:right;white-space:nowrap" >    χ<sub>1</sub> + ⋯ + χ<sub><span style="font-style:italic">n</span><sup>(<span style="font-style:italic">p</span>)</sup></sub> = <span style="font-style:italic">p</span>, 
</td><td style="text-align:left;white-space:nowrap" >&nbsp;</td><td style="text-align:right;white-space:nowrap" >&nbsp;</td><td style="text-align:left;white-space:nowrap" >&nbsp;</td><td style="text-align:right;white-space:nowrap" >&nbsp;</td><td style="text-align:left;white-space:nowrap" >&nbsp;</td><td style="text-align:right;white-space:nowrap" >&nbsp;</td><td style="text-align:left;white-space:nowrap" >&nbsp;</td><td style="text-align:right;white-space:nowrap" >&nbsp;</td><td style="text-align:left;white-space:nowrap" >&nbsp;</td><td style="text-align:right;white-space:nowrap" >    (12)</td></tr>
</table></td></tr>
</table><p>
which, combined with (<a href="#f_n%28x%29%20definition">7</a>), yields
</p><table class="display dcenter"><tr style="vertical-align:middle"><td class="dcell">
     

</td><td class="dcell"><table style="border-spacing:6px;border-collapse:separate;" class="cellpading0"><tr><td style="text-align:right;white-space:nowrap" >    <span style="font-style:italic">f</span><sub><span style="font-style:italic">n</span><sup>(<span style="font-style:italic">p</span>)</sup></sub>(<span style="font-style:italic">x</span><sup>(γ)</sup>) = 2<sup><span style="font-size:small">-</span> <span style="font-style:italic">p</span></sup>.
</td><td style="text-align:left;white-space:nowrap" >&nbsp;</td><td style="text-align:right;white-space:nowrap" >&nbsp;</td><td style="text-align:left;white-space:nowrap" >&nbsp;</td><td style="text-align:right;white-space:nowrap" >&nbsp;</td><td style="text-align:left;white-space:nowrap" >&nbsp;</td><td style="text-align:right;white-space:nowrap" >&nbsp;</td><td style="text-align:left;white-space:nowrap" >&nbsp;</td><td style="text-align:right;white-space:nowrap" >&nbsp;</td><td style="text-align:left;white-space:nowrap" >&nbsp;</td><td style="text-align:right;white-space:nowrap" >    (13)</td></tr>
</table></td></tr>
</table><p>
Finally,
</p><table class="display dcenter"><tr style="vertical-align:middle"><td class="dcell">
     

</td><td class="dcell"><table style="border-spacing:6px;border-collapse:separate;" class="cellpading0"><tr><td style="text-align:right;white-space:nowrap" >    γ<sub><span style="font-style:italic">n</span><sup>(<span style="font-style:italic">p</span>)</sup></sub> <span style="font-style:italic">f</span><sub><span style="font-style:italic">n</span><sup>(<span style="font-style:italic">p</span>)</sup></sub>(<span style="font-style:italic">x</span><sup>(γ)</sup>) &gt; 1.
</td><td style="text-align:left;white-space:nowrap" >&nbsp;</td><td style="text-align:right;white-space:nowrap" >&nbsp;</td><td style="text-align:left;white-space:nowrap" >&nbsp;</td><td style="text-align:right;white-space:nowrap" >&nbsp;</td><td style="text-align:left;white-space:nowrap" >&nbsp;</td><td style="text-align:right;white-space:nowrap" >&nbsp;</td><td style="text-align:left;white-space:nowrap" >&nbsp;</td><td style="text-align:right;white-space:nowrap" >&nbsp;</td><td style="text-align:left;white-space:nowrap" >&nbsp;</td><td style="text-align:right;white-space:nowrap" >    (14)</td></tr>
</table></td></tr>
</table><p>
We have so established that the subsequence 
{γ<sub><span style="font-style:italic">n</span><sup>(<span style="font-style:italic">p</span>)</sup></sub><span style="font-style:italic">f</span><sub><span style="font-style:italic">n</span><sup>(<span style="font-style:italic">p</span>)</sup></sub>} 
does not tend pointwise to 0, hence neither does the whole sequence 
{γ<sub><span style="font-style:italic">n</span></sub><span style="font-style:italic">f</span><sub><span style="font-style:italic">n</span></sub>}.
In other words, (b) holds, which is in violent contrast with [1.28]: 
<span style="font-style:italic">X</span> is then not metrizable. So ends the proof.
</p><p><br>


</p>
<!--TOC section id="sec4" Exercise 9. Quotient map-->
<h2 id="sec4" class="section">1.2  Exercise 9. Quotient map</h2><!--SEC END --><p>
<span style="font-style:italic">Suppose
</span></p><ol class="enumerate" type=1><li class="li-enumerate"><span style="font-style:italic">
</span><span style="font-style:italic">X</span><span style="font-style:italic"> and </span><span style="font-style:italic">Y</span><span style="font-style:italic"> are topological vector spaces,
</span></li><li class="li-enumerate">Λ: <span style="font-style:italic">X</span>→ <span style="font-style:italic">Y</span><span style="font-style:italic"> is linear.
</span></li><li class="li-enumerate"><span style="font-style:italic">N</span><span style="font-style:italic"> is a closed subspace of </span><span style="font-style:italic">X</span><span style="font-style:italic">,
</span></li><li class="li-enumerate">π: <span style="font-style:italic">X</span>→ <span style="font-style:italic">X</span>/<span style="font-style:italic">N</span><span style="font-style:italic"> is the quotient map, and
</span></li><li class="li-enumerate">Λ <span style="font-style:italic">x</span>=0<span style="font-style:italic"> for every </span><span style="font-style:italic">x</span>∈ <span style="font-style:italic">N</span><span style="font-style:italic">.
</span></li></ol><p><span style="font-style:italic">
Prove that there is a unique </span><span style="font-style:italic">f</span>:<span style="font-style:italic">X</span>/<span style="font-style:italic">N</span>→ <span style="font-style:italic">Y</span><span style="font-style:italic"> which satisfies 
</span>Λ=<span style="font-style:italic">f</span>∘ π<span style="font-style:italic">, 
that is, 
</span>Λ <span style="font-style:italic">x</span>=<span style="font-style:italic">f</span>(π (<span style="font-style:italic">x</span>))<span style="font-style:italic"> for all </span><span style="font-style:italic">x</span>∈ <span style="font-style:italic">X</span><span style="font-style:italic">. 
Prove that </span><span style="font-style:italic">f</span><span style="font-style:italic"> is linear and that </span>Λ<span style="font-style:italic"> is continuous if and only if 
</span><span style="font-style:italic">f</span> <span style="font-style:italic"> is continuous. 
Also, </span>Λ<span style="font-style:italic"> is open if and only if </span><span style="font-style:italic">f</span><span style="font-style:italic"> is open.</span>
</p><p><span style="font-weight:bold">Proof.</span>
The equation Λ = <span style="font-style:italic">f</span> ∘ π has necessarily a unique solution, 
which is the binary relation 
</p><table class="display dcenter"><tr style="vertical-align:middle"><td class="dcell">
     

</td><td class="dcell"><table style="border-spacing:6px;border-collapse:separate;" class="cellpading0"><tr><td style="text-align:right;white-space:nowrap" ><a id="1_9: defintion of f."> </a>
<span style="font-style:italic">f</span> ≜{(π <span style="font-style:italic">x</span>, Λ <span style="font-style:italic">x</span>): <span style="font-style:italic">x</span> ∈ <span style="font-style:italic">X</span>} ⊂ <span style="font-style:italic">X</span>/<span style="font-style:italic">N</span> × <span style="font-style:italic">Y</span>.
</td><td style="text-align:left;white-space:nowrap" >&nbsp;</td><td style="text-align:right;white-space:nowrap" >&nbsp;</td><td style="text-align:left;white-space:nowrap" >&nbsp;</td><td style="text-align:right;white-space:nowrap" >&nbsp;</td><td style="text-align:left;white-space:nowrap" >&nbsp;</td><td style="text-align:right;white-space:nowrap" >&nbsp;</td><td style="text-align:left;white-space:nowrap" >&nbsp;</td><td style="text-align:right;white-space:nowrap" >&nbsp;</td><td style="text-align:left;white-space:nowrap" >&nbsp;</td><td style="text-align:right;white-space:nowrap" >    (15)</td></tr>
</table></td></tr>
</table><p>
To ensure that <span style="font-style:italic">f</span> is actually a mapping, simply remark that 
the linearity of Λ implies 
</p><table class="display dcenter"><tr style="vertical-align:middle"><td class="dcell">
     

</td><td class="dcell"><table style="border-spacing:6px;border-collapse:separate;" class="cellpading0"><tr><td style="text-align:right;white-space:nowrap" >    Λ <span style="font-style:italic">x</span> ≠ Λ  <span style="font-style:italic">x</span>′ ⇒ π <span style="font-style:italic">x</span>′ ≠ π <span style="font-style:italic">x</span>′.
</td><td style="text-align:left;white-space:nowrap" >&nbsp;</td><td style="text-align:right;white-space:nowrap" >&nbsp;</td><td style="text-align:left;white-space:nowrap" >&nbsp;</td><td style="text-align:right;white-space:nowrap" >&nbsp;</td><td style="text-align:left;white-space:nowrap" >&nbsp;</td><td style="text-align:right;white-space:nowrap" >&nbsp;</td><td style="text-align:left;white-space:nowrap" >&nbsp;</td><td style="text-align:right;white-space:nowrap" >&nbsp;</td><td style="text-align:left;white-space:nowrap" >&nbsp;</td><td style="text-align:right;white-space:nowrap" >    (16)</td></tr>
</table></td></tr>
</table><p>
It straightforwardly derives from (<a href="#1_9%3A%20defintion%20of%20f.">15</a>) that 
<span style="font-style:italic">f</span> inherits linearity from π and Λ. Now remark that 
</p><table class="display dcenter"><tr style="vertical-align:middle"><td class="dcell">
     

</td><td class="dcell"><table style="border-spacing:6px;border-collapse:separate;" class="cellpading0"><tr><td style="text-align:right;white-space:nowrap" >    π <span style="font-style:italic">x</span> = <span style="font-style:italic">N</span>   
(<span style="font-style:italic">f</span> linear)⇒ 
<span style="font-style:italic">f</span>(π <span style="font-style:italic">x</span>) = 0 
(<a href="#1_9%3A%20defintion%20of%20f.">15</a>)⇒ 
Λ <span style="font-style:italic">x</span> = 0 
⇒ π <span style="font-style:italic">x</span> = <span style="font-style:italic">N</span> 
</td><td style="text-align:left;white-space:nowrap" >&nbsp;</td><td style="text-align:right;white-space:nowrap" >&nbsp;</td><td style="text-align:left;white-space:nowrap" >&nbsp;</td><td style="text-align:right;white-space:nowrap" >&nbsp;</td><td style="text-align:left;white-space:nowrap" >&nbsp;</td><td style="text-align:right;white-space:nowrap" >&nbsp;</td><td style="text-align:left;white-space:nowrap" >&nbsp;</td><td style="text-align:right;white-space:nowrap" >&nbsp;</td><td style="text-align:left;white-space:nowrap" >&nbsp;</td><td style="text-align:right;white-space:nowrap" >    (17)</td></tr>
</table></td></tr>
</table><p>
and so conclude that <span style="font-style:italic">f</span> is also one-to-one.
Now assume <span style="font-style:italic">f</span> to be continuous. Then so is 
Λ = <span style="font-style:italic">f</span>∘ π , 
by (a) of [1.41]. 
Conversely, 
if Λ is continuous, then for each neighborhood <span style="font-style:italic">V</span> of 0<sub><span style="font-style:italic">Y</span></sub> 
there exists a neighborhood <span style="font-style:italic">U</span> of 0<sub><span style="font-style:italic">X</span></sub> such that
</p><table class="display dcenter"><tr style="vertical-align:middle"><td class="dcell">
     

</td><td class="dcell"><table style="border-spacing:6px;border-collapse:separate;" class="cellpading0"><tr><td style="text-align:right;white-space:nowrap" ><table class="display"><tr style="vertical-align:middle"><td class="dcell">    Λ(<span style="font-style:italic">U</span>) = <span style="font-style:italic">f</span></td><td class="dcell">⎛<br>
⎝</td><td class="dcell">π(<span style="font-style:italic">U</span>)</td><td class="dcell">⎞<br>
⎠</td><td class="dcell">
⊂ 
<span style="font-style:italic">V</span>.
</td></tr>
</table></td><td style="text-align:left;white-space:nowrap" >&nbsp;</td><td style="text-align:right;white-space:nowrap" >&nbsp;</td><td style="text-align:left;white-space:nowrap" >&nbsp;</td><td style="text-align:right;white-space:nowrap" >&nbsp;</td><td style="text-align:left;white-space:nowrap" >&nbsp;</td><td style="text-align:right;white-space:nowrap" >&nbsp;</td><td style="text-align:left;white-space:nowrap" >&nbsp;</td><td style="text-align:right;white-space:nowrap" >&nbsp;</td><td style="text-align:left;white-space:nowrap" >&nbsp;</td><td style="text-align:right;white-space:nowrap" >    (18)</td></tr>
</table></td></tr>
</table><p>
Since π is open (see (a) of [1.41]), π(<span style="font-style:italic">U</span>) is a neighborhood of 
<span style="font-style:italic">N</span>=0<sub><span style="font-style:italic">X</span>/<span style="font-style:italic">N</span></sub>: 
This is sufficient to establish that the linear mapping <span style="font-style:italic">f</span> is continuous.
If <span style="font-style:italic">f</span> is open, so is Λ = <span style="font-style:italic">f</span>∘ π, by (a) of [1.41]. 
Conversely, let 
</p><table class="display dcenter"><tr style="vertical-align:middle"><td class="dcell">
     

</td><td class="dcell"><table style="border-spacing:6px;border-collapse:separate;" class="cellpading0"><tr><td style="text-align:right;white-space:nowrap" >    <span style="font-style:italic">W</span> ≜ π (<span style="font-style:italic">V</span>) ⊂  <span style="font-style:italic">X</span>/<span style="font-style:italic">N</span>    (<span style="font-style:italic">V</span>  neighborhood of  0<sub><span style="font-style:italic">X</span></sub>) 
</td><td style="text-align:left;white-space:nowrap" >&nbsp;</td><td style="text-align:right;white-space:nowrap" >&nbsp;</td><td style="text-align:left;white-space:nowrap" >&nbsp;</td><td style="text-align:right;white-space:nowrap" >&nbsp;</td><td style="text-align:left;white-space:nowrap" >&nbsp;</td><td style="text-align:right;white-space:nowrap" >&nbsp;</td><td style="text-align:left;white-space:nowrap" >&nbsp;</td><td style="text-align:right;white-space:nowrap" >&nbsp;</td><td style="text-align:left;white-space:nowrap" >&nbsp;</td><td style="text-align:right;white-space:nowrap" >    (19)</td></tr>
</table></td></tr>
</table><p>
range over all neighborhoods of <span style="font-style:italic">N</span>, as Λ is kept open: So is 
</p><table class="display dcenter"><tr style="vertical-align:middle"><td class="dcell">
     

</td><td class="dcell"><table style="border-spacing:6px;border-collapse:separate;" class="cellpading0"><tr><td style="text-align:right;white-space:nowrap" ><table class="display"><tr style="vertical-align:middle"><td class="dcell">    Λ(<span style="font-style:italic">V</span>) = <span style="font-style:italic">f</span> </td><td class="dcell">⎛<br>
⎝</td><td class="dcell">π(<span style="font-style:italic">V</span>)</td><td class="dcell">⎞<br>
⎠</td><td class="dcell">= <span style="font-style:italic">f</span>(<span style="font-style:italic">W</span>).  
</td></tr>
</table></td><td style="text-align:left;white-space:nowrap" >&nbsp;</td><td style="text-align:right;white-space:nowrap" >&nbsp;</td><td style="text-align:left;white-space:nowrap" >&nbsp;</td><td style="text-align:right;white-space:nowrap" >&nbsp;</td><td style="text-align:left;white-space:nowrap" >&nbsp;</td><td style="text-align:right;white-space:nowrap" >&nbsp;</td><td style="text-align:left;white-space:nowrap" >&nbsp;</td><td style="text-align:right;white-space:nowrap" >&nbsp;</td><td style="text-align:left;white-space:nowrap" >&nbsp;</td><td style="text-align:right;white-space:nowrap" >    (20)</td></tr>
</table></td></tr>
</table><p>
The linear mapping <span style="font-style:italic">f</span> is then open. 
</p><p><br>
</p>
<!--TOC section id="sec5" Exercise 10. An open mapping theorem-->
<h2 id="sec5" class="section">1.3  Exercise 10. An open mapping theorem</h2><!--SEC END --><p>
<span style="font-style:italic">Suppose that X and Y are topological vector spaces,
</span>dim<span style="font-style:italic">Y</span> &lt; ∞<span style="font-style:italic">,
</span>Λ : <span style="font-style:italic">X</span> → <span style="font-style:italic">Y</span><span style="font-style:italic"> is linear, and </span>Λ(<span style="font-style:italic">X</span>) = <span style="font-style:italic">Y</span><span style="font-style:italic">.
</span></p><ol class="enumerate" type=1><li class="li-enumerate"><span style="font-style:italic">
</span><span style="font-style:italic">
Prove that </span>Λ<span style="font-style:italic"> is an open mapping.</span><span style="font-style:italic">
</span></li><li class="li-enumerate"><span style="font-style:italic">
Assume, in addition, that the null space of </span>Λ<span style="font-style:italic"> is closed, 
and prove that </span>Λ<span style="font-style:italic"> is continuous.
</span><span style="font-style:italic">
</span></li></ol><p><span style="font-weight:bold">Proof.</span>
</p><ol class="enumerate" type=1><li class="li-enumerate">

Let <span style="font-style:italic">e</span> range over a base of <span style="font-style:italic">Y</span>: 
For each <span style="font-style:italic">e</span>, there exists <span style="font-style:italic">x</span><sub><span style="font-style:italic">e</span></sub> in <span style="font-style:italic">X</span> such that 
Λ(<span style="font-style:italic">x</span><sub><span style="font-style:italic">e</span></sub>)=<span style="font-style:italic">e</span>, 
since Λ is onto.So,
<table class="display dcenter"><tr style="vertical-align:middle"><td class="dcell">
     

</td><td class="dcell"><table style="border-spacing:6px;border-collapse:separate;" class="cellpading0"><tr><td style="text-align:right;white-space:nowrap" ><table class="display"><tr style="vertical-align:middle"><td class="dcell"><a id="1_10_sum"> </a>
<span style="font-style:italic">y</span> = </td><td class="dcell"><table class="display"><tr><td class="dcell" style="text-align:center">&nbsp;</td></tr>
<tr><td class="dcell" style="text-align:center"><span style="font-size:xx-large">∑</span></td></tr>
<tr><td class="dcell" style="text-align:center"><span style="font-style:italic">e</span></td></tr>
</table></td><td class="dcell"> <span style="font-style:italic">y</span><sub><span style="font-style:italic">e</span></sub> Λ <span style="font-style:italic">x</span><sub><span style="font-style:italic">e</span></sub>    (<span style="font-style:italic">y</span>∈ <span style="font-style:italic">Y</span>).
</td></tr>
</table></td><td style="text-align:left;white-space:nowrap" >&nbsp;</td><td style="text-align:right;white-space:nowrap" >&nbsp;</td><td style="text-align:left;white-space:nowrap" >&nbsp;</td><td style="text-align:right;white-space:nowrap" >&nbsp;</td><td style="text-align:left;white-space:nowrap" >&nbsp;</td><td style="text-align:right;white-space:nowrap" >&nbsp;</td><td style="text-align:left;white-space:nowrap" >&nbsp;</td><td style="text-align:right;white-space:nowrap" >&nbsp;</td><td style="text-align:left;white-space:nowrap" >&nbsp;</td><td style="text-align:right;white-space:nowrap" >    (21)</td></tr>
</table></td></tr>
</table>
The sequence {<span style="font-style:italic">x</span><sub><span style="font-style:italic">e</span></sub>} is finite hence bounded: 
Given <span style="font-style:italic">V</span> a balanced neighborhood of the origin, 
there exists a positive scalar <span style="font-style:italic">s</span> such that 
<table class="display dcenter"><tr style="vertical-align:middle"><td class="dcell">
     

</td><td class="dcell"><table style="border-spacing:6px;border-collapse:separate;" class="cellpading0"><tr><td style="text-align:right;white-space:nowrap" >          <span style="font-style:italic">x</span><sub><span style="font-style:italic">e</span></sub> ∈ <span style="font-style:italic">s</span> <span style="font-style:italic">V</span>
</td><td style="text-align:left;white-space:nowrap" >&nbsp;</td><td style="text-align:right;white-space:nowrap" >&nbsp;</td><td style="text-align:left;white-space:nowrap" >&nbsp;</td><td style="text-align:right;white-space:nowrap" >&nbsp;</td><td style="text-align:left;white-space:nowrap" >&nbsp;</td><td style="text-align:right;white-space:nowrap" >&nbsp;</td><td style="text-align:left;white-space:nowrap" >&nbsp;</td><td style="text-align:right;white-space:nowrap" >&nbsp;</td><td style="text-align:left;white-space:nowrap" >&nbsp;</td><td style="text-align:right;white-space:nowrap" >    (22)</td></tr>
</table></td></tr>
</table>
for all <span style="font-style:italic">x</span><sub><span style="font-style:italic">e</span></sub>.
Combining this with (<a href="#1_10_sum">21</a>) shows that 
<table class="display dcenter"><tr style="vertical-align:middle"><td class="dcell">
     

</td><td class="dcell"><table style="border-spacing:6px;border-collapse:separate;" class="cellpading0"><tr><td style="text-align:right;white-space:nowrap" ><table class="display"><tr style="vertical-align:middle"><td class="dcell">          <span style="font-style:italic">y</span> ∈ </td><td class="dcell"><table class="display"><tr><td class="dcell" style="text-align:center">&nbsp;</td></tr>
<tr><td class="dcell" style="text-align:center"><span style="font-size:xx-large">∑</span></td></tr>
<tr><td class="dcell" style="text-align:center"><span style="font-style:italic">e</span></td></tr>
</table></td><td class="dcell"> Λ (<span style="font-style:italic">V</span>)    (<span style="font-style:italic">y</span>∈ <span style="font-style:italic">Y</span>: |<span style="font-style:italic">y</span><sub><span style="font-style:italic">e</span></sub>| &lt; <span style="font-style:italic">s</span><sup><span style="font-size:small">-</span> 1</sup>).
</td></tr>
</table></td><td style="text-align:left;white-space:nowrap" >&nbsp;</td><td style="text-align:right;white-space:nowrap" >&nbsp;</td><td style="text-align:left;white-space:nowrap" >&nbsp;</td><td style="text-align:right;white-space:nowrap" >&nbsp;</td><td style="text-align:left;white-space:nowrap" >&nbsp;</td><td style="text-align:right;white-space:nowrap" >&nbsp;</td><td style="text-align:left;white-space:nowrap" >&nbsp;</td><td style="text-align:right;white-space:nowrap" >&nbsp;</td><td style="text-align:left;white-space:nowrap" >&nbsp;</td><td style="text-align:right;white-space:nowrap" >    (23)</td></tr>
</table></td></tr>
</table>

</li><li class="li-enumerate">
Since <span style="font-style:italic">N</span> is closed, π continously maps <span style="font-style:italic">X</span> onto <span style="font-style:italic">X</span>/<span style="font-style:italic">N</span>, 
another topological (Hausdorff) vector space, see [1.41]. 
Now take <span style="font-style:italic">f</span> as in Exercise 9: 
Since Λ is onto, the first isomorphism theorem asserts that 
<span style="font-style:italic">f</span> is an isomorphism of <span style="font-style:italic">X</span>/<span style="font-style:italic">N</span> onto <span style="font-style:italic">Y</span>. 
Consequently, <span style="font-style:italic">X</span>/<span style="font-style:italic">N</span> has dimension <span style="font-style:italic">n</span>=dim<span style="font-style:italic">Y</span>. 
<span style="font-style:italic">f</span> is then an homeomorphism of 
<span style="font-style:italic">X</span>/<span style="font-style:italic">N</span>≡ <span style="font-weight:bold"><span style="font-style:italic">C</span></span><sup><span style="font-style:italic">n</span></sup> 
onto <span style="font-style:italic">Y</span>; see [1.21].
We have thus established that <span style="font-style:italic">f</span> is continuous: So is Λ = <span style="font-style:italic">f</span>∘ π.

</li></ol><p><br>

</p>
<!--TOC section id="sec6" Exercise 14. <span style="font-style:italic">D</span><sub><span style="font-style:italic">K</span></sub> equipped with other seminorms-->
<h2 id="sec6" class="section">1.4  Exercise 14. <span style="font-style:italic">D</span><sub><span style="font-style:italic">K</span></sub> equipped with other seminorms</h2><!--SEC END --><p>
<span style="font-style:italic">Put </span><span style="font-style:italic">K</span> =[0, 1]<span style="font-style:italic"> and define </span> <span style="font-style:italic">D</span><sub><span style="font-style:italic">K</span></sub><span style="font-style:italic"> as in Section 1.46. 
Show that the following three families of seminorms 
(where </span><span style="font-style:italic">n</span> = 0, 1, 2, …<span style="font-style:italic">) define the same topology on </span> <span style="font-style:italic">D</span><sub><span style="font-style:italic">K</span></sub><span style="font-style:italic">. 
If </span><span style="font-style:italic">D</span> = <span style="font-style:italic">d</span>/<span style="font-style:italic">dx</span><span style="font-style:italic">: 
</span></p><ol class="enumerate" type=1><li class="li-enumerate"><span style="font-style:italic">
</span><span style="font-style:italic">
</span>|| <span style="font-style:italic">D</span><sup><span style="font-style:italic">n</span></sup> <span style="font-style:italic">f</span> ||<sub>∞</sub>= sup{| <span style="font-style:italic">D</span><sup><span style="font-style:italic">n</span></sup> <span style="font-style:italic">f</span>(<span style="font-style:italic">x</span>)|: ∞&lt; <span style="font-style:italic">x</span>&lt; ∞}<span style="font-style:italic">
</span><span style="font-style:italic">
</span></li><li class="li-enumerate"><span style="font-style:italic">
</span>|| <span style="font-style:italic">D</span><sup><span style="font-style:italic">n</span></sup> <span style="font-style:italic">f</span> ||<sub>1</sub> =∫<sub>0</sub><sup>1</sup> |<span style="font-style:italic">D</span><sup><span style="font-style:italic">n</span></sup> <span style="font-style:italic">f</span>(<span style="font-style:italic">x</span>) | <span style="font-style:italic">d</span><span style="font-style:italic">x</span><span style="font-style:italic">
</span><span style="font-style:italic">
</span></li><li class="li-enumerate"><span style="font-style:italic">
</span>|| <span style="font-style:italic">D</span><sup><span style="font-style:italic">n</span></sup> <span style="font-style:italic">f</span> ||<sub>2</sub> = {∫<sub>0</sub><sup>1</sup> | <span style="font-style:italic">D</span><sup><span style="font-style:italic">n</span></sup> <span style="font-style:italic">f</span>(<span style="font-style:italic">x</span>) |<sup>2</sup> <span style="font-style:italic">d</span><span style="font-style:italic">x</span> 
}<sup>1/2</sup>.<span style="font-style:italic">
</span><span style="font-style:italic">
</span></li></ol><p><span style="font-weight:bold">Proof.</span> 
First, remark that 
</p><table class="display dcenter"><tr style="vertical-align:middle"><td class="dcell">
     

</td><td class="dcell"><table style="border-spacing:6px;border-collapse:separate;" class="cellpading0"><tr><td style="text-align:right;white-space:nowrap" ><a id="1_14_2"> </a>
|| <span style="font-style:italic">D</span><sup><span style="font-style:italic">n</span></sup> <span style="font-style:italic">f</span> ||<sub>1</sub> 
≤ 
|| <span style="font-style:italic">D</span><sup><span style="font-style:italic">n</span></sup> <span style="font-style:italic">f</span> ||<sub>2</sub> 
≤ 
|| <span style="font-style:italic">D</span><sup><span style="font-style:italic">n</span></sup> <span style="font-style:italic">f</span> ||<sub>∞</sub>&lt;∞ 
</td><td style="text-align:left;white-space:nowrap" >&nbsp;</td><td style="text-align:right;white-space:nowrap" >&nbsp;</td><td style="text-align:left;white-space:nowrap" >&nbsp;</td><td style="text-align:right;white-space:nowrap" >&nbsp;</td><td style="text-align:left;white-space:nowrap" >&nbsp;</td><td style="text-align:right;white-space:nowrap" >&nbsp;</td><td style="text-align:left;white-space:nowrap" >&nbsp;</td><td style="text-align:right;white-space:nowrap" >&nbsp;</td><td style="text-align:left;white-space:nowrap" >&nbsp;</td><td style="text-align:right;white-space:nowrap" >    (24)</td></tr>
</table></td></tr>
</table><p>
(the inequality on the left is a Cauchy-Schwarz one), 
since <span style="font-style:italic">K</span> has length 1. Next, start from 
</p><table class="display dcenter"><tr style="vertical-align:middle"><td class="dcell">
     

</td><td class="dcell"><table style="border-spacing:6px;border-collapse:separate;" class="cellpading0"><tr><td style="text-align:right;white-space:nowrap" ><table class="display"><tr style="vertical-align:middle"><td class="dcell"><a id="1_14_3"> </a>
<span style="font-style:italic">D</span><sup><span style="font-style:italic">n</span></sup> <span style="font-style:italic">f</span>(<span style="font-style:italic">x</span>) = </td><td class="dcell"><span style="font-size:xx-large">∫</span></td><td class="dcell"><table class="display"><tr><td class="dcell" style="text-align:left"><span style="font-style:italic">x</span></td></tr>
<tr><td class="dcell" style="text-align:left"><br>
<br>
</td></tr>
<tr><td class="dcell" style="text-align:left"><span style="font-size:small">-</span> ∞</td></tr>
</table></td><td class="dcell"> <span style="font-style:italic">D</span><sup><span style="font-style:italic">n</span>+1</sup><span style="font-style:italic">f</span>
</td></tr>
</table></td><td style="text-align:left;white-space:nowrap" >&nbsp;</td><td style="text-align:right;white-space:nowrap" >&nbsp;</td><td style="text-align:left;white-space:nowrap" >&nbsp;</td><td style="text-align:right;white-space:nowrap" >&nbsp;</td><td style="text-align:left;white-space:nowrap" >&nbsp;</td><td style="text-align:right;white-space:nowrap" >&nbsp;</td><td style="text-align:left;white-space:nowrap" >&nbsp;</td><td style="text-align:right;white-space:nowrap" >&nbsp;</td><td style="text-align:left;white-space:nowrap" >&nbsp;</td><td style="text-align:right;white-space:nowrap" >    (25)</td></tr>
</table></td></tr>
</table><p>
(which is true, since <span style="font-style:italic">f</span> has a bounded support) to obtain
</p><table class="display dcenter"><tr style="vertical-align:middle"><td class="dcell">
     

</td><td class="dcell"><table style="border-spacing:6px;border-collapse:separate;" class="cellpading0"><tr><td style="text-align:right;white-space:nowrap" ><table class="display"><tr style="vertical-align:middle"><td class="dcell"><a id="1_14_4"> </a>
</td><td class="dcell">⎪<br>
⎪</td><td class="dcell"><span style="font-style:italic">D</span><sup><span style="font-style:italic">n</span></sup> <span style="font-style:italic">f</span>(<span style="font-style:italic">x</span>)</td><td class="dcell">⎪<br>
⎪</td><td class="dcell">
≤ 
</td><td class="dcell"><span style="font-size:xx-large">∫</span></td><td class="dcell"><table class="display"><tr><td class="dcell" style="text-align:left"><span style="font-style:italic">x</span></td></tr>
<tr><td class="dcell" style="text-align:left"><br>
<br>
</td></tr>
<tr><td class="dcell" style="text-align:left"><span style="font-size:small">-</span>∞</td></tr>
</table></td><td class="dcell"> </td><td class="dcell">⎪<br>
⎪</td><td class="dcell"><span style="font-style:italic">D</span><sup><span style="font-style:italic">n</span>+1</sup><span style="font-style:italic">f</span> </td><td class="dcell">⎪<br>
⎪</td><td class="dcell">
≤ 
||<span style="font-style:italic">D</span><sup><span style="font-style:italic">n</span>+1</sup><span style="font-style:italic">f</span> ||<sub>1</sub> 
</td></tr>
</table></td><td style="text-align:left;white-space:nowrap" >&nbsp;</td><td style="text-align:right;white-space:nowrap" >&nbsp;</td><td style="text-align:left;white-space:nowrap" >&nbsp;</td><td style="text-align:right;white-space:nowrap" >&nbsp;</td><td style="text-align:left;white-space:nowrap" >&nbsp;</td><td style="text-align:right;white-space:nowrap" >&nbsp;</td><td style="text-align:left;white-space:nowrap" >&nbsp;</td><td style="text-align:right;white-space:nowrap" >&nbsp;</td><td style="text-align:left;white-space:nowrap" >&nbsp;</td><td style="text-align:right;white-space:nowrap" >    (26)</td></tr>
</table></td></tr>
</table><p>
hence
</p><table class="display dcenter"><tr style="vertical-align:middle"><td class="dcell">
     

</td><td class="dcell"><table style="border-spacing:6px;border-collapse:separate;" class="cellpading0"><tr><td style="text-align:right;white-space:nowrap" ><a id="1_14_5"> </a>
|| <span style="font-style:italic">D</span><sup><span style="font-style:italic">n</span></sup> <span style="font-style:italic">f</span> ||<sub>∞</sub>≤ 
|| <span style="font-style:italic">D</span><sup><span style="font-style:italic">n</span>+1</sup> <span style="font-style:italic">f</span> ||<sub>1</sub> .
</td><td style="text-align:left;white-space:nowrap" >&nbsp;</td><td style="text-align:right;white-space:nowrap" >&nbsp;</td><td style="text-align:left;white-space:nowrap" >&nbsp;</td><td style="text-align:right;white-space:nowrap" >&nbsp;</td><td style="text-align:left;white-space:nowrap" >&nbsp;</td><td style="text-align:right;white-space:nowrap" >&nbsp;</td><td style="text-align:left;white-space:nowrap" >&nbsp;</td><td style="text-align:right;white-space:nowrap" >&nbsp;</td><td style="text-align:left;white-space:nowrap" >&nbsp;</td><td style="text-align:right;white-space:nowrap" >    (27)</td></tr>
</table></td></tr>
</table><p>
Combining (<a href="#1_14_2">24</a>) with (<a href="#1_14_5">27</a>) yields
</p><table class="display dcenter"><tr style="vertical-align:middle"><td class="dcell">
     

</td><td class="dcell"><table style="border-spacing:6px;border-collapse:separate;" class="cellpading0"><tr><td style="text-align:right;white-space:nowrap" ><a id="1_14_6"> </a>
||<span style="font-style:italic">D</span> <span style="font-style:italic">f</span> ||<sub>1</sub> 
≤ 
⋯
≤
|| <span style="font-style:italic">D</span><sup><span style="font-style:italic">n</span></sup> <span style="font-style:italic">f</span> ||<sub>1</sub> 
≤ 
|| <span style="font-style:italic">D</span><sup><span style="font-style:italic">n</span></sup> <span style="font-style:italic">f</span> ||<sub>2</sub> 
≤ 
|| <span style="font-style:italic">D</span><sup><span style="font-style:italic">n</span></sup> <span style="font-style:italic">f</span> ||<sub>∞</sub>≤ 
|| <span style="font-style:italic">D</span><sup><span style="font-style:italic">n</span>+1</sup><span style="font-style:italic">f</span>||<sub>1</sub> 
≤ 
⋯ .
</td><td style="text-align:left;white-space:nowrap" >&nbsp;</td><td style="text-align:right;white-space:nowrap" >&nbsp;</td><td style="text-align:left;white-space:nowrap" >&nbsp;</td><td style="text-align:right;white-space:nowrap" >&nbsp;</td><td style="text-align:left;white-space:nowrap" >&nbsp;</td><td style="text-align:right;white-space:nowrap" >&nbsp;</td><td style="text-align:left;white-space:nowrap" >&nbsp;</td><td style="text-align:right;white-space:nowrap" >&nbsp;</td><td style="text-align:left;white-space:nowrap" >&nbsp;</td><td style="text-align:right;white-space:nowrap" >    (28)</td></tr>
</table></td></tr>
</table><p>
We now define 
</p><table class="display dcenter"><tr style="vertical-align:middle"><td class="dcell">
     

</td><td class="dcell"><table style="border-spacing:6px;border-collapse:separate;" class="cellpading0"><tr><td style="text-align:right;white-space:nowrap" ><a id="1_14_1"> </a>
<span style="font-style:italic">V</span><sub><span style="font-style:italic">n</span></sub><sup>(<span style="font-style:italic">i</span>)</sup></td><td style="text-align:left;white-space:nowrap" >       ≜ {<span style="font-style:italic">f</span>∈  <span style="font-style:italic">D</span><sub><span style="font-style:italic">K</span></sub>: || <span style="font-style:italic">f</span> ||<sub><span style="font-style:italic">i</span></sub> &lt;1/<span style="font-style:italic">n</span>}  (<span style="font-style:italic">i</span>=1,2,∞)</td><td style="text-align:right;white-space:nowrap" >&nbsp;</td><td style="text-align:left;white-space:nowrap" >&nbsp;</td><td style="text-align:right;white-space:nowrap" >&nbsp;</td><td style="text-align:left;white-space:nowrap" >&nbsp;</td><td style="text-align:right;white-space:nowrap" >&nbsp;</td><td style="text-align:left;white-space:nowrap" >&nbsp;</td><td style="text-align:right;white-space:nowrap" >&nbsp;</td><td style="text-align:left;white-space:nowrap" >&nbsp;</td><td style="text-align:right;white-space:nowrap" >    (29)</td></tr>
<tr><td style="text-align:right;white-space:nowrap" >
<span style="font-style:italic">B</span><sup>(<span style="font-style:italic">i</span>)</sup></td><td style="text-align:left;white-space:nowrap" >≜ {<span style="font-style:italic">V</span><sub><span style="font-style:italic">n</span></sub><sup>(<span style="font-style:italic">i</span>)</sup> : <span style="font-style:italic">n</span>=1, 2, 3, …}
</td><td style="text-align:right;white-space:nowrap" >&nbsp;</td><td style="text-align:left;white-space:nowrap" >&nbsp;</td><td style="text-align:right;white-space:nowrap" >&nbsp;</td><td style="text-align:left;white-space:nowrap" >&nbsp;</td><td style="text-align:right;white-space:nowrap" >&nbsp;</td><td style="text-align:left;white-space:nowrap" >&nbsp;</td><td style="text-align:right;white-space:nowrap" >&nbsp;</td><td style="text-align:left;white-space:nowrap" >&nbsp;</td><td style="text-align:right;white-space:nowrap" >    (30)</td></tr>
</table></td></tr>
</table><p>
so that (<a href="#1_14_6">28</a>) is mirrored in terms of neighborhood inclusions, 
as follows,
</p><table class="display dcenter"><tr style="vertical-align:middle"><td class="dcell">
     

</td><td class="dcell"><table style="border-spacing:6px;border-collapse:separate;" class="cellpading0"><tr><td style="text-align:right;white-space:nowrap" ><a id="1_14_7"> </a>
<span style="font-style:italic">V</span><sub>1</sub><sup>(1)</sup> 
⊃
⋯ 
⊃ 
<span style="font-style:italic">V</span><sub><span style="font-style:italic">n</span></sub><sup>(1)</sup>
⊃ 
<span style="font-style:italic">V</span><sub><span style="font-style:italic">n</span></sub><sup>(2)</sup> 
⊃ 
<span style="font-style:italic">V</span><sub><span style="font-style:italic">n</span></sub><sup>(∞)</sup> 
⊃ 
<span style="font-style:italic">V</span><sub><span style="font-style:italic">n</span>+1</sub><sup>(1)</sup> 
⊃ 
⋯ .
</td><td style="text-align:left;white-space:nowrap" >&nbsp;</td><td style="text-align:right;white-space:nowrap" >&nbsp;</td><td style="text-align:left;white-space:nowrap" >&nbsp;</td><td style="text-align:right;white-space:nowrap" >&nbsp;</td><td style="text-align:left;white-space:nowrap" >&nbsp;</td><td style="text-align:right;white-space:nowrap" >&nbsp;</td><td style="text-align:left;white-space:nowrap" >&nbsp;</td><td style="text-align:right;white-space:nowrap" >&nbsp;</td><td style="text-align:left;white-space:nowrap" >&nbsp;</td><td style="text-align:right;white-space:nowrap" >    (31)</td></tr>
</table></td></tr>
</table><p>
Since 
<span style="font-style:italic">V</span><sub><span style="font-style:italic">n</span></sub><sup>(<span style="font-style:italic">i</span>)</sup>⊃ <span style="font-style:italic">V</span><sub><span style="font-style:italic">n</span>+1</sub><sup>(<span style="font-style:italic">i</span>)</sup>, 
<span style="font-style:italic">B</span><sub><span style="font-style:italic">i</span></sub> is the local base of a topology τ<sub><span style="font-style:italic">i</span></sub>. 
But the chain (<a href="#1_14_7">31</a>) forces the τ<sub><span style="font-style:italic">i</span></sub>’s to be equals. 
To see that, choose a set <span style="font-style:italic">S</span> that is τ<sub>1</sub>-open at, say <span style="font-style:italic">a</span>: So, 
<span style="font-style:italic">V</span><sub><span style="font-style:italic">n</span></sub><sup>(1)</sup> ⊂ <span style="font-style:italic">S</span>−<span style="font-style:italic">a</span> 
for some <span style="font-style:italic">n</span>. Now <span style="font-style:italic">V</span><sub><span style="font-style:italic">n</span></sub><sup>(1)</sup> ⊃ <span style="font-style:italic">V</span><sub><span style="font-style:italic">n</span></sub><sup>(2)</sup> (see (<a href="#1_14_7">31</a>)) forces 
<span style="font-style:italic">V</span><sub><span style="font-style:italic">n</span></sub><sup>(2)</sup> ⊂ <span style="font-style:italic">S</span>−<span style="font-style:italic">a</span> , 
which implies that <span style="font-style:italic">S</span> is τ<sub>2</sub>-open at <span style="font-style:italic">a</span>.
Similarly, we deduce, still from (<a href="#1_14_7">31</a>), that 
</p><table class="display dcenter"><tr style="vertical-align:middle"><td class="dcell">
     

</td><td class="dcell"><table style="border-spacing:6px;border-collapse:separate;" class="cellpading0"><tr><td style="text-align:right;white-space:nowrap" >  τ<sub>2</sub>-open 
⇒ 
τ<sub>∞</sub>-open 
⇒ 
τ<sub>1</sub>-open.
</td><td style="text-align:left;white-space:nowrap" >&nbsp;</td><td style="text-align:right;white-space:nowrap" >&nbsp;</td><td style="text-align:left;white-space:nowrap" >&nbsp;</td><td style="text-align:right;white-space:nowrap" >&nbsp;</td><td style="text-align:left;white-space:nowrap" >&nbsp;</td><td style="text-align:right;white-space:nowrap" >&nbsp;</td><td style="text-align:left;white-space:nowrap" >&nbsp;</td><td style="text-align:right;white-space:nowrap" >&nbsp;</td><td style="text-align:left;white-space:nowrap" >&nbsp;</td><td style="text-align:right;white-space:nowrap" >    (32)</td></tr>
</table></td></tr>
</table><p>
So ends the proof.
</p><p><br>
</p>
<!--TOC section id="sec7" Exercise 16. Uniqueness of topology for test functions-->
<h2 id="sec7" class="section">1.5  Exercise 16. Uniqueness of topology for test functions</h2><!--SEC END --><p>
<span style="font-style:italic">
Prove that the topology of </span><span style="font-style:italic">C</span>(Ω)<span style="font-style:italic"> does not depend on the particular 
choice of </span>{<span style="font-style:italic">K</span><sub><span style="font-style:italic">n</span></sub>}<span style="font-style:italic">, as long as this sequence satisfies the conditions 
specified in section 1.44. Do the same for </span><span style="font-style:italic">C</span><sup>∞</sup>(Ω)<span style="font-style:italic"> (Section 1.46).</span>
</p>
<!--TOC paragraph id="sec8" Comment-->
<h5 id="sec8" class="paragraph">Comment</h5><!--SEC END --><p>This is an invariance property: 
The function test topology only depends on the existence of the 
supremum-seminorms <span style="font-style:italic">p</span><sub><span style="font-style:italic">n</span></sub>, then, eventually, 
only on the ambient space itself. 
This should then be regarded as a very part of the textbook [<a href="#FA">2</a>]
The proof consists in combining trivial consequences of the local base 
definition with a well-known result (<span style="font-style:italic">e.g.</span> [2.6] in [<a href="#BigRudin">1</a>]) 
about intersection of nonempty compact sets. </p>
<!--TOC paragraph id="sec9" Lemma 1-->
<h5 id="sec9" class="paragraph">Lemma 1</h5><!--SEC END --><p> Let <span style="font-style:italic">X</span> be a topological space with a countable local base 
{<span style="font-style:italic">V</span><sub><span style="font-style:italic">n</span></sub>: <span style="font-style:italic">n</span>=1, 2, 3, …}. 
If 
Ṽ<sub><span style="font-style:italic">n</span></sub> = <span style="font-style:italic">V</span><sub>1</sub> ∩ ⋯ ∩ <span style="font-style:italic">V</span><sub><span style="font-style:italic">n</span></sub>, 
then every subsequence 
{Ṽ<sub>ρ(<span style="font-style:italic">n</span>)</sub>} 
is a decreasing (<span style="font-style:italic">i.e.</span> Ṽ<sub>ρ(<span style="font-style:italic">n</span>)</sub> ⊃ Ṽ<sub>ρ(<span style="font-style:italic">n</span>+1)</sub>)
local base of <span style="font-style:italic">X</span>.
</p><p><span style="font-weight:bold">Proof.</span>
The decreasing property is trivial. Now remark that 
<span style="font-style:italic">V</span><sub><span style="font-style:italic">n</span></sub> ⊃ Ṽ<sub><span style="font-style:italic">n</span></sub>:
This shows that 
{Ṽ<sub><span style="font-style:italic">n</span></sub>} 
is a local base of <span style="font-style:italic">X</span>. Then so is 
{Ṽ<sub>ρ(<span style="font-style:italic">n</span>)</sub>},
since Ṽ<sub><span style="font-style:italic">n</span></sub> ⊃ Ṽ<sub>ρ(<span style="font-style:italic">n</span>)</sub>.
</p><p><br>

The following special case 
<span style="font-style:italic">V</span><sub><span style="font-style:italic">n</span></sub> = Ṽ<sub><span style="font-style:italic">n</span></sub> 
is one of the key ingredients:
</p>
<!--TOC paragraph id="sec10" Corollary 1 (special case <span style="font-style:italic">V</span><sub><span style="font-style:italic">n</span></sub> = Ṽ<sub><span style="font-style:italic">n</span></sub>)-->
<h5 id="sec10" class="paragraph">Corollary 1 (special case <span style="font-style:italic">V</span><sub><span style="font-style:italic">n</span></sub> = Ṽ<sub><span style="font-style:italic">n</span></sub>)</h5><!--SEC END --><p>
Under the same notations of Lemma 1, if {<span style="font-style:italic">V</span><sub><span style="font-style:italic">n</span></sub>} is a decreasing 
local base, then so is {<span style="font-style:italic">V</span><sub>ρ(<span style="font-style:italic">n</span>)</sub>}.
</p>
<!--TOC paragraph id="sec11" Corollary 2-->
<h5 id="sec11" class="paragraph">Corollary 2</h5><!--SEC END --><p>
If 
{<span style="font-style:italic">Q</span><sub><span style="font-style:italic">n</span></sub>} 
is a sequence of compact sets that satisfies the conditions specified 
in section 1.44, then every subsequence 
{<span style="font-style:italic">Q</span><sub>ρ(<span style="font-style:italic">n</span>)</sub>} 
also satisfies theses conditions.
Furthermore, if τ<sub><span style="font-style:italic">Q</span></sub> is the <span style="font-style:italic">C</span>(Ω)’s 
(respectively <span style="font-style:italic">C</span><sup>∞</sup>(Ω)’s) topology of the seminorms <span style="font-style:italic">p</span><sub><span style="font-style:italic">n</span></sub>, 
as defined in section 1.44 (respectively 1.46), then the seminorms 
<span style="font-style:italic">p</span><sub>ρ(<span style="font-style:italic">n</span>)</sub> 
define the same topology τ<sub><span style="font-style:italic">Q</span></sub>.
</p><p><span style="font-weight:bold">Proof.</span>Let <span style="font-style:italic">X</span> be <span style="font-style:italic">C</span>(Ω) topologized with the seminorms <span style="font-style:italic">p</span><sub><span style="font-style:italic">n</span></sub> 
(the case <span style="font-style:italic">X</span>=<span style="font-style:italic">C</span><sup>∞</sup>(Ω) is proved the same way).
If 
<span style="font-style:italic">V</span><sub><span style="font-style:italic">n</span></sub> = {<span style="font-style:italic">p</span><sub><span style="font-style:italic">n</span></sub> &lt; 1/<span style="font-style:italic">n</span>}, 
then 
{<span style="font-style:italic">V</span><sub><span style="font-style:italic">n</span></sub>} 
is a decreasing local base of <span style="font-style:italic">X</span>.
Moreover,
</p><table class="display dcenter"><tr style="vertical-align:middle"><td class="dcell">
     

</td><td class="dcell"><table style="border-spacing:6px;border-collapse:separate;" class="cellpading0"><tr><td style="text-align:right;white-space:nowrap" >    <span style="font-style:italic">Q</span><sub>ρ(<span style="font-style:italic">n</span>)</sub> 
⊂ 
∘<span style="font-style:italic">Q</span><sub>ρ(<span style="font-style:italic">n</span>) + 1</sub> 
⊂ 
<span style="font-style:italic">Q</span><sub>ρ(<span style="font-style:italic">n</span>) + 1</sub> 
⊂ 
<span style="font-style:italic">Q</span><sub>ρ(<span style="font-style:italic">n</span>+ 1)</sub>.
</td><td style="text-align:left;white-space:nowrap" >&nbsp;</td><td style="text-align:right;white-space:nowrap" >&nbsp;</td><td style="text-align:left;white-space:nowrap" >&nbsp;</td><td style="text-align:right;white-space:nowrap" >&nbsp;</td><td style="text-align:left;white-space:nowrap" >&nbsp;</td><td style="text-align:right;white-space:nowrap" >&nbsp;</td><td style="text-align:left;white-space:nowrap" >&nbsp;</td><td style="text-align:right;white-space:nowrap" >&nbsp;</td><td style="text-align:left;white-space:nowrap" >&nbsp;</td><td style="text-align:right;white-space:nowrap" >    (33)</td></tr>
</table></td></tr>
</table><p>
Thus,
</p><table class="display dcenter"><tr style="vertical-align:middle"><td class="dcell">
     

</td><td class="dcell"><table style="border-spacing:6px;border-collapse:separate;" class="cellpading0"><tr><td style="text-align:right;white-space:nowrap" >    <span style="font-style:italic">Q</span><sub>ρ(<span style="font-style:italic">n</span>)</sub> 
⊂ 
∘<span style="font-style:italic">Q</span><sub>ρ(<span style="font-style:italic">n</span>+ 1)</sub>.
</td><td style="text-align:left;white-space:nowrap" >&nbsp;</td><td style="text-align:right;white-space:nowrap" >&nbsp;</td><td style="text-align:left;white-space:nowrap" >&nbsp;</td><td style="text-align:right;white-space:nowrap" >&nbsp;</td><td style="text-align:left;white-space:nowrap" >&nbsp;</td><td style="text-align:right;white-space:nowrap" >&nbsp;</td><td style="text-align:left;white-space:nowrap" >&nbsp;</td><td style="text-align:right;white-space:nowrap" >&nbsp;</td><td style="text-align:left;white-space:nowrap" >&nbsp;</td><td style="text-align:right;white-space:nowrap" >    (34)</td></tr>
</table></td></tr>
</table><p>
In other words, 
<span style="font-style:italic">Q</span><sub>ρ(<span style="font-style:italic">n</span>)</sub> satisfies the conditions specified in section 1.44.
{<span style="font-style:italic">p</span><sub>ρ(<span style="font-style:italic">n</span>)</sub>}
then defines a topology τ<sub><span style="font-style:italic">Q</span><sub>ρ</sub></sub> for which 
{<span style="font-style:italic">V</span><sub>ρ(<span style="font-style:italic">n</span>)</sub>} 
is a local base. So, 
τ<sub><span style="font-style:italic">Q</span><sub>ρ</sub></sub> ⊂ τ<sub><span style="font-style:italic">Q</span></sub>.
Conversely, the above corollary asserts that 
{<span style="font-style:italic">V</span><sub>ρ(<span style="font-style:italic">n</span>)</sub>} 
is a local base of τ<sub><span style="font-style:italic">Q</span></sub>, which yields 
τ<sub><span style="font-style:italic">Q</span></sub>⊂ τ<sub><span style="font-style:italic">Q</span><sub>ρ</sub></sub>.
</p><p><br>

</p>
<!--TOC paragraph id="sec12" Lemma 2-->
<h5 id="sec12" class="paragraph">Lemma 2</h5><!--SEC END --><p> 
If a sequence of compact sets {<span style="font-style:italic">Q</span><sub><span style="font-style:italic">n</span></sub>} satisfies the conditions 
specified in section 1.44, then every compact set <span style="font-style:italic">K</span> lies in allmost all 
<span style="font-style:italic">Q</span><sub><span style="font-style:italic">n</span></sub><sup>∘</sup>, <span style="font-style:italic">i.e.</span> there exists <span style="font-style:italic">m</span> such that 
</p><table class="display dcenter"><tr style="vertical-align:middle"><td class="dcell">
     

</td><td class="dcell"><table style="border-spacing:6px;border-collapse:separate;" class="cellpading0"><tr><td style="text-align:right;white-space:nowrap" >    <span style="font-style:italic">K</span> ⊂ 
∘<span style="font-style:italic">Q</span><sub><span style="font-style:italic">m</span></sub> 
⊂ 
∘<span style="font-style:italic">Q</span><sub><span style="font-style:italic">m</span>+1</sub>
⊂
∘<span style="font-style:italic">Q</span><sub><span style="font-style:italic">m</span>+2</sub>
⊂
⋯.
</td><td style="text-align:left;white-space:nowrap" >&nbsp;</td><td style="text-align:right;white-space:nowrap" >&nbsp;</td><td style="text-align:left;white-space:nowrap" >&nbsp;</td><td style="text-align:right;white-space:nowrap" >&nbsp;</td><td style="text-align:left;white-space:nowrap" >&nbsp;</td><td style="text-align:right;white-space:nowrap" >&nbsp;</td><td style="text-align:left;white-space:nowrap" >&nbsp;</td><td style="text-align:right;white-space:nowrap" >&nbsp;</td><td style="text-align:left;white-space:nowrap" >&nbsp;</td><td style="text-align:right;white-space:nowrap" >    (35)</td></tr>
</table></td></tr>
</table><p><span style="font-weight:bold">Proof.</span>
The following definition
</p><table class="display dcenter"><tr style="vertical-align:middle"><td class="dcell">
     

</td><td class="dcell"><table style="border-spacing:6px;border-collapse:separate;" class="cellpading0"><tr><td style="text-align:right;white-space:nowrap" >    <span style="font-style:italic">C</span><sub><span style="font-style:italic">n</span></sub> ≜ <span style="font-style:italic">K</span> ∖ ∘<span style="font-style:italic">Q</span><sub><span style="font-style:italic">n</span></sub>    (<span style="font-style:italic">n</span>=1, 2, 3, …)
</td><td style="text-align:left;white-space:nowrap" >&nbsp;</td><td style="text-align:right;white-space:nowrap" >&nbsp;</td><td style="text-align:left;white-space:nowrap" >&nbsp;</td><td style="text-align:right;white-space:nowrap" >&nbsp;</td><td style="text-align:left;white-space:nowrap" >&nbsp;</td><td style="text-align:right;white-space:nowrap" >&nbsp;</td><td style="text-align:left;white-space:nowrap" >&nbsp;</td><td style="text-align:right;white-space:nowrap" >&nbsp;</td><td style="text-align:left;white-space:nowrap" >&nbsp;</td><td style="text-align:right;white-space:nowrap" >    (36)</td></tr>
</table></td></tr>
</table><p>
shapes {<span style="font-style:italic">C</span><sub><span style="font-style:italic">n</span></sub>} as a decreasing sequence of compact<sup><a id="text1" href="#note1">1</a></sup> 
sets. We now suppose (to reach a contradiction) that 
no <span style="font-style:italic">C</span><sub><span style="font-style:italic">n</span></sub> is empty 
and so conclude<sup><a id="text2" href="#note2">2</a></sup> 
that the <span style="font-style:italic">C</span><sub><span style="font-style:italic">n</span></sub>’s intersection contains a point that is not in any <span style="font-style:italic">Q</span><sub><span style="font-style:italic">n</span></sub><sup>∘</sup>. 
On the other hand, the conditions specified in [1.44] force the 
<span style="font-style:italic">Q</span><sub><span style="font-style:italic">n</span></sub><sup>∘</sup>’s collection 
to be an open cover.
This contradiction reveals that 
<span style="font-style:italic">C</span><sub><span style="font-style:italic">m</span></sub> = ∅, 
<span style="font-style:italic">i.e.</span> <span style="font-style:italic">K</span> ⊂ <span style="font-style:italic">Q</span><sub><span style="font-style:italic">m</span></sub><sup>∘</sup>, 
for some <span style="font-style:italic">m</span>.
Finally, 
</p><table class="display dcenter"><tr style="vertical-align:middle"><td class="dcell">
     

</td><td class="dcell"><table style="border-spacing:6px;border-collapse:separate;" class="cellpading0"><tr><td style="text-align:right;white-space:nowrap" >    <span style="font-style:italic">K</span>⊂ 
∘<span style="font-style:italic">Q</span><sub><span style="font-style:italic">m</span></sub>
⊂
<span style="font-style:italic">Q</span><sub><span style="font-style:italic">m</span></sub>
⊂
∘<span style="font-style:italic">Q</span><sub><span style="font-style:italic">m</span>+1</sub>
⊂
<span style="font-style:italic">Q</span><sub><span style="font-style:italic">m</span> +1</sub>
⊂
∘<span style="font-style:italic">Q</span><sub><span style="font-style:italic">m</span>+2</sub>
⊂
⋯.
</td><td style="text-align:left;white-space:nowrap" >&nbsp;</td><td style="text-align:right;white-space:nowrap" >&nbsp;</td><td style="text-align:left;white-space:nowrap" >&nbsp;</td><td style="text-align:right;white-space:nowrap" >&nbsp;</td><td style="text-align:left;white-space:nowrap" >&nbsp;</td><td style="text-align:right;white-space:nowrap" >&nbsp;</td><td style="text-align:left;white-space:nowrap" >&nbsp;</td><td style="text-align:right;white-space:nowrap" >&nbsp;</td><td style="text-align:left;white-space:nowrap" >&nbsp;</td><td style="text-align:right;white-space:nowrap" >    (37)</td></tr>
</table></td></tr>
</table><p> 
So ends the proof.
</p><p><br>

We are now in a fair position to establish the following:
</p>
<!--TOC paragraph id="sec13" Theorem-->
<h5 id="sec13" class="paragraph">Theorem</h5><!--SEC END --><p> 
The topology of <span style="font-style:italic">C</span>(Ω) does not depend on the particular choice of 
{<span style="font-style:italic">K</span><sub><span style="font-style:italic">n</span></sub>}, as long as this sequence satisfies the conditions 
specified in section 1.44. Neither does the topology of <span style="font-style:italic">C</span><sup>∞</sup>(Ω), 
as long as this sequence satisfies the conditions specified in section 1.44.
</p><p><span style="font-weight:bold">Proof.</span>With the second corollary’s notations,
τ<sub><span style="font-style:italic">K</span></sub> = τ<sub><span style="font-style:italic">K</span><sub>λ</sub></sub>,
for every subsequence {<span style="font-style:italic">K</span><sub>λ(<span style="font-style:italic">n</span>)</sub>}.
Similarly, let 
{<span style="font-style:italic">L</span><sub><span style="font-style:italic">n</span></sub>} 
be another sequence of compact subsets of Ω that satisfies 
the condition specified in [1.44], 
so that 
τ<sub><span style="font-style:italic">L</span></sub> = τ<sub><span style="font-style:italic">L</span><sub>κ</sub></sub>
for every subsequence {<span style="font-style:italic">L</span><sub>κ(<span style="font-style:italic">n</span>)</sub>}. 
Now apply the above Lemma 2 with <span style="font-style:italic">K</span><sub><span style="font-style:italic">i</span></sub> (<span style="font-style:italic">i</span>=1, 2, 3, …) and so conclude that 
<span style="font-style:italic">K</span><sub><span style="font-style:italic">i</span></sub> 
⊂ 
<span style="font-style:italic">L</span><sub><span style="font-style:italic">m</span><sub><span style="font-style:italic">i</span></sub></sub><sup>∘</sup> 
⊂ 
<span style="font-style:italic">L</span><sub><span style="font-style:italic">m</span><sub><span style="font-style:italic">i</span></sub>+1</sub><sup>∘</sup>
⊂
⋯
for some <span style="font-style:italic">m</span><sub><span style="font-style:italic">i</span></sub>. Finally, set λ<sub><span style="font-style:italic">i</span></sub> = <span style="font-style:italic">m</span><sub><span style="font-style:italic">i</span></sub> + <span style="font-style:italic">i</span> then obtain 
</p><table class="display dcenter"><tr style="vertical-align:middle"><td class="dcell">
     

</td><td class="dcell"><table style="border-spacing:6px;border-collapse:separate;" class="cellpading0"><tr><td style="text-align:right;white-space:nowrap" >    <a id="1_16. K subset interior L"> </a>
<span style="font-style:italic">K</span><sub><span style="font-style:italic">i</span></sub>
⊂ 
∘<span style="font-style:italic">L</span><sub>κ<sub><span style="font-style:italic">i</span></sub></sub>.
</td><td style="text-align:left;white-space:nowrap" >&nbsp;</td><td style="text-align:right;white-space:nowrap" >&nbsp;</td><td style="text-align:left;white-space:nowrap" >&nbsp;</td><td style="text-align:right;white-space:nowrap" >&nbsp;</td><td style="text-align:left;white-space:nowrap" >&nbsp;</td><td style="text-align:right;white-space:nowrap" >&nbsp;</td><td style="text-align:left;white-space:nowrap" >&nbsp;</td><td style="text-align:right;white-space:nowrap" >&nbsp;</td><td style="text-align:left;white-space:nowrap" >&nbsp;</td><td style="text-align:right;white-space:nowrap" >    (38)</td></tr>
</table></td></tr>
</table><p> 
Let us reiterate the above proof with <span style="font-style:italic">K</span><sub><span style="font-style:italic">n</span></sub> and <span style="font-style:italic">L</span><sub><span style="font-style:italic">n</span></sub> in exchanged roles 
then similarly find a subsequence {λ<sub><span style="font-style:italic">j</span></sub>: <span style="font-style:italic">j</span>=1, 2, 3, …} such that 
</p><table class="display dcenter"><tr style="vertical-align:middle"><td class="dcell">
     

</td><td class="dcell"><table style="border-spacing:6px;border-collapse:separate;" class="cellpading0"><tr><td style="text-align:right;white-space:nowrap" >  <a id="1_16. L subset interior K"> </a>
<span style="font-style:italic">L</span><sub><span style="font-style:italic">j</span></sub> ⊂ ∘<span style="font-style:italic">K</span><sub>λ<sub><span style="font-style:italic">j</span></sub></sub>
</td><td style="text-align:left;white-space:nowrap" >&nbsp;</td><td style="text-align:right;white-space:nowrap" >&nbsp;</td><td style="text-align:left;white-space:nowrap" >&nbsp;</td><td style="text-align:right;white-space:nowrap" >&nbsp;</td><td style="text-align:left;white-space:nowrap" >&nbsp;</td><td style="text-align:right;white-space:nowrap" >&nbsp;</td><td style="text-align:left;white-space:nowrap" >&nbsp;</td><td style="text-align:right;white-space:nowrap" >&nbsp;</td><td style="text-align:left;white-space:nowrap" >&nbsp;</td><td style="text-align:right;white-space:nowrap" >    (39)</td></tr>
</table></td></tr>
</table><p>
Combine 
(<a href="#1_16.%20K%20subset%20interior%20L">38</a>) with 
(<a href="#1_16.%20L%20subset%20interior%20K">39</a>) 
and so obtain
</p><table class="display dcenter"><tr style="vertical-align:middle"><td class="dcell">
     

</td><td class="dcell"><table style="border-spacing:6px;border-collapse:separate;" class="cellpading0"><tr><td style="text-align:right;white-space:nowrap" >    <span style="font-style:italic">K</span><sub>1</sub> 
⊂ 
∘<span style="font-style:italic">L</span><sub>κ<sub>1</sub></sub> 
⊂ 
<span style="font-style:italic">L</span><sub>κ<sub>1</sub></sub> 
⊂ 
∘<span style="font-style:italic">K</span><sub>λ<sub>κ<sub>1</sub></sub></sub>
⊂ 
<span style="font-style:italic">K</span><sub>λ<sub>κ<sub>1</sub></sub></sub>
⊂
∘<span style="font-style:italic">L</span><sub>κ<sub>λ<sub>κ<sub>1</sub></sub></sub></sub>
⊂
⋯, 
</td><td style="text-align:left;white-space:nowrap" >&nbsp;</td><td style="text-align:right;white-space:nowrap" >&nbsp;</td><td style="text-align:left;white-space:nowrap" >&nbsp;</td><td style="text-align:right;white-space:nowrap" >&nbsp;</td><td style="text-align:left;white-space:nowrap" >&nbsp;</td><td style="text-align:right;white-space:nowrap" >&nbsp;</td><td style="text-align:left;white-space:nowrap" >&nbsp;</td><td style="text-align:right;white-space:nowrap" >&nbsp;</td><td style="text-align:left;white-space:nowrap" >&nbsp;</td><td style="text-align:right;white-space:nowrap" >    (40)</td></tr>
</table></td></tr>
</table><p>
which means that the sequence 
<span style="font-style:italic">Q</span> = (
<span style="font-style:italic">K</span><sub>1</sub>, 
<span style="font-style:italic">L</span><sub>κ<sub>1</sub></sub>, 
<span style="font-style:italic">K</span><sub>λ<sub>κ<sub>1</sub></sub></sub>, 
…
)
satisfies the conditions specified in section 1.44. 
It now follows from the corollary 2 that 
</p><table class="display dcenter"><tr style="vertical-align:middle"><td class="dcell">
     

</td><td class="dcell"><table style="border-spacing:6px;border-collapse:separate;" class="cellpading0"><tr><td style="text-align:right;white-space:nowrap" >    τ<sub><span style="font-style:italic">K</span></sub> 
= 
τ<sub><span style="font-style:italic">K</span><sub>λ</sub></sub> 
= 
τ<sub><span style="font-style:italic">Q</span></sub> 
= 
τ<sub><span style="font-style:italic">L</span><sub>κ</sub></sub> 
= τ<sub><span style="font-style:italic">L</span></sub>.
</td><td style="text-align:left;white-space:nowrap" >&nbsp;</td><td style="text-align:right;white-space:nowrap" >&nbsp;</td><td style="text-align:left;white-space:nowrap" >&nbsp;</td><td style="text-align:right;white-space:nowrap" >&nbsp;</td><td style="text-align:left;white-space:nowrap" >&nbsp;</td><td style="text-align:right;white-space:nowrap" >&nbsp;</td><td style="text-align:left;white-space:nowrap" >&nbsp;</td><td style="text-align:right;white-space:nowrap" >&nbsp;</td><td style="text-align:left;white-space:nowrap" >&nbsp;</td><td style="text-align:right;white-space:nowrap" >    (41)</td></tr>
</table></td></tr>
</table><p> 
So ends the proof
</p><p><br>


</p>
<!--TOC section id="sec14" Exercise 17. Derivation in some non normed space-->
<h2 id="sec14" class="section">1.6  Exercise 17. Derivation in some non normed space</h2><!--SEC END --><p>
<span style="font-style:italic">In the setting of Section 1.46, prove that 
</span><span style="font-style:italic">f</span> ↦ <span style="font-style:italic">D</span><sup>α</sup><span style="font-style:italic">f</span><span style="font-style:italic"> 
is a continuous mapping of 
</span><span style="font-style:italic">C</span><sup>∞</sup>(Ω)<span style="font-style:italic"> into 
</span><span style="font-style:italic">C</span><sup>∞</sup>(Ω)<span style="font-style:italic"> and also of 
</span> <span style="font-style:italic">D</span><sub><span style="font-style:italic">K</span></sub><span style="font-style:italic"> into 
</span> <span style="font-style:italic">D</span><sub><span style="font-style:italic">K</span></sub><span style="font-style:italic">, for every multi-index </span>α<span style="font-style:italic">.
</span>
</p><p><span style="font-weight:bold">Proof.</span> 
In both cases, <span style="font-style:italic">D</span><sup>α</sup> is a linear mapping. 
It is then sufficient to establish continuousness at the origin.
We begin with the <span style="font-style:italic">C</span><sup>∞</sup>(Ω) case. <br>
<br>
Let <span style="font-style:italic">U</span> be an aribtray neighborhood of the origin.
There so exists <span style="font-style:italic">N</span> such that <span style="font-style:italic">U</span> contains
</p><table class="display dcenter"><tr style="vertical-align:middle"><td class="dcell">
     

</td><td class="dcell"><table style="border-spacing:6px;border-collapse:separate;" class="cellpading0"><tr><td style="text-align:right;white-space:nowrap" >    <span style="font-style:italic">V</span><sub><span style="font-style:italic">N</span></sub>= {
ϕ ∈ <span style="font-style:italic">C</span><sup>∞</sup>(Ω)
: 
max
{
| <span style="font-style:italic">D</span><sup>β</sup>ϕ(<span style="font-style:italic">x</span>) |
: 
| β | ≤ <span style="font-style:italic">N</span>, <span style="font-style:italic">x</span>∈ <span style="font-style:italic">K</span><sub><span style="font-style:italic">N</span></sub>
}
&lt; 1/<span style="font-style:italic">N</span>
}.
</td><td style="text-align:left;white-space:nowrap" >&nbsp;</td><td style="text-align:right;white-space:nowrap" >&nbsp;</td><td style="text-align:left;white-space:nowrap" >&nbsp;</td><td style="text-align:right;white-space:nowrap" >&nbsp;</td><td style="text-align:left;white-space:nowrap" >&nbsp;</td><td style="text-align:right;white-space:nowrap" >&nbsp;</td><td style="text-align:left;white-space:nowrap" >&nbsp;</td><td style="text-align:right;white-space:nowrap" >&nbsp;</td><td style="text-align:left;white-space:nowrap" >&nbsp;</td><td style="text-align:right;white-space:nowrap" >    (42)</td></tr>
</table></td></tr>
</table><p>
Now pick <span style="font-style:italic">g</span> in <span style="font-style:italic">V</span><sub><span style="font-style:italic">N</span>+|α|</sub>, so that
</p><table class="display dcenter"><tr style="vertical-align:middle"><td class="dcell">
     

</td><td class="dcell"><table style="border-spacing:6px;border-collapse:separate;" class="cellpading0"><tr><td style="text-align:right;white-space:nowrap" ><table class="display"><tr style="vertical-align:middle"><td class="dcell">    max
{
| <span style="font-style:italic">D</span><sup>γ</sup><span style="font-style:italic">g</span>(<span style="font-style:italic">x</span>) |
: 
| γ | ≤ <span style="font-style:italic">N</span>+| α |, 
<span style="font-style:italic">x</span>∈ <span style="font-style:italic">K</span><sub><span style="font-style:italic">N</span></sub>
}
&lt; </td><td class="dcell"><table class="display"><tr><td class="dcell" style="text-align:center">1</td></tr>
<tr><td class="hbar"></td></tr>
<tr><td class="dcell" style="text-align:center"><span style="font-style:italic">N</span></td></tr>
</table></td><td class="dcell">.
</td></tr>
</table></td><td style="text-align:left;white-space:nowrap" >&nbsp;</td><td style="text-align:right;white-space:nowrap" >&nbsp;</td><td style="text-align:left;white-space:nowrap" >&nbsp;</td><td style="text-align:right;white-space:nowrap" >&nbsp;</td><td style="text-align:left;white-space:nowrap" >&nbsp;</td><td style="text-align:right;white-space:nowrap" >&nbsp;</td><td style="text-align:left;white-space:nowrap" >&nbsp;</td><td style="text-align:right;white-space:nowrap" >&nbsp;</td><td style="text-align:left;white-space:nowrap" >&nbsp;</td><td style="text-align:right;white-space:nowrap" >    (43)</td></tr>
</table></td></tr>
</table><p>
(the fact that <span style="font-style:italic">K</span><sub><span style="font-style:italic">N</span></sub>⊂ <span style="font-style:italic">K</span><sub><span style="font-style:italic">N</span>+|α|</sub> was tacitely used).
The special case γ = β + α yields
</p><table class="display dcenter"><tr style="vertical-align:middle"><td class="dcell">
     

</td><td class="dcell"><table style="border-spacing:6px;border-collapse:separate;" class="cellpading0"><tr><td style="text-align:right;white-space:nowrap" ><table class="display"><tr style="vertical-align:middle"><td class="dcell">    sup
{
| <span style="font-style:italic">D</span><sup>β</sup><span style="font-style:italic">D</span><sup>α</sup><span style="font-style:italic">g</span>(<span style="font-style:italic">x</span>) |
: 
| β | ≤ <span style="font-style:italic">N</span>, 
<span style="font-style:italic">x</span>∈ <span style="font-style:italic">K</span><sub><span style="font-style:italic">N</span></sub>
}
&lt; </td><td class="dcell"><table class="display"><tr><td class="dcell" style="text-align:center">1</td></tr>
<tr><td class="hbar"></td></tr>
<tr><td class="dcell" style="text-align:center"><span style="font-style:italic">N</span></td></tr>
</table></td><td class="dcell">.
</td></tr>
</table></td><td style="text-align:left;white-space:nowrap" >&nbsp;</td><td style="text-align:right;white-space:nowrap" >&nbsp;</td><td style="text-align:left;white-space:nowrap" >&nbsp;</td><td style="text-align:right;white-space:nowrap" >&nbsp;</td><td style="text-align:left;white-space:nowrap" >&nbsp;</td><td style="text-align:right;white-space:nowrap" >&nbsp;</td><td style="text-align:left;white-space:nowrap" >&nbsp;</td><td style="text-align:right;white-space:nowrap" >&nbsp;</td><td style="text-align:left;white-space:nowrap" >&nbsp;</td><td style="text-align:right;white-space:nowrap" >    (44)</td></tr>
</table></td></tr>
</table><p>
We have just proved that
</p><table class="display dcenter"><tr style="vertical-align:middle"><td class="dcell">
     

</td><td class="dcell"><table style="border-spacing:6px;border-collapse:separate;" class="cellpading0"><tr><td style="text-align:right;white-space:nowrap" >    <span style="font-style:italic">g</span> ∈ <span style="font-style:italic">V</span><sub><span style="font-style:italic">N</span> + | α|</sub>
⇒ 
<span style="font-style:italic">D</span><sup>α</sup><span style="font-style:italic">g</span> ∈ <span style="font-style:italic">V</span><sub><span style="font-style:italic">N</span></sub>,
  
<span style="font-style:italic">i.e.</span> 
  
<span style="font-style:italic">D</span><sup>α</sup>(<span style="font-style:italic">V</span><sub><span style="font-style:italic">N</span>+|α|</sub>) ⊂ <span style="font-style:italic">V</span><sub><span style="font-style:italic">N</span></sub>.
</td><td style="text-align:left;white-space:nowrap" >&nbsp;</td><td style="text-align:right;white-space:nowrap" >&nbsp;</td><td style="text-align:left;white-space:nowrap" >&nbsp;</td><td style="text-align:right;white-space:nowrap" >&nbsp;</td><td style="text-align:left;white-space:nowrap" >&nbsp;</td><td style="text-align:right;white-space:nowrap" >&nbsp;</td><td style="text-align:left;white-space:nowrap" >&nbsp;</td><td style="text-align:right;white-space:nowrap" >&nbsp;</td><td style="text-align:left;white-space:nowrap" >&nbsp;</td><td style="text-align:right;white-space:nowrap" >    (45)</td></tr>
</table></td></tr>
</table><p>
The continuity of 
<span style="font-style:italic">D</span><sup>α</sup>: <span style="font-style:italic">C</span><sup>∞</sup>(Ω) → <span style="font-style:italic">C</span><sup>∞</sup>(Ω) 
is so established.
We now prove the second part, <span style="font-style:italic">i.e.</span> that <span style="font-style:italic">D</span><sup>α</sup>: <span style="font-style:italic">D</span><sub><span style="font-style:italic">K</span></sub> → <span style="font-style:italic">D</span><sub><span style="font-style:italic">K</span></sub> is continuous. <br>
<br>
Let <span style="font-style:italic">r</span> be a positive scalar. From now on, <span style="font-style:italic">V</span>(<span style="font-style:italic">r</span>) will denote the 
preimage in  <span style="font-style:italic">D</span><sub><span style="font-style:italic">K</span></sub> of the open disc <span style="font-style:italic">D</span>(<span style="font-style:italic">r</span>) <span style="font-style:italic">i.e.</span> </p><table class="display dcenter"><tr style="vertical-align:middle"><td class="dcell">
     

</td><td class="dcell"><table style="border-spacing:6px;border-collapse:separate;" class="cellpading0"><tr><td style="text-align:right;white-space:nowrap" >  <span style="font-style:italic">V</span>(<span style="font-style:italic">r</span>) ≜ {<span style="font-style:italic">f</span> ∈  <span style="font-style:italic">D</span><sub><span style="font-style:italic">K</span></sub>: | <span style="font-style:italic">D</span><sup>α</sup><span style="font-style:italic">f</span> | &lt; <span style="font-style:italic">r</span>}.
</td><td style="text-align:left;white-space:nowrap" >&nbsp;</td><td style="text-align:right;white-space:nowrap" >&nbsp;</td><td style="text-align:left;white-space:nowrap" >&nbsp;</td><td style="text-align:right;white-space:nowrap" >&nbsp;</td><td style="text-align:left;white-space:nowrap" >&nbsp;</td><td style="text-align:right;white-space:nowrap" >&nbsp;</td><td style="text-align:left;white-space:nowrap" >&nbsp;</td><td style="text-align:right;white-space:nowrap" >&nbsp;</td><td style="text-align:left;white-space:nowrap" >&nbsp;</td><td style="text-align:right;white-space:nowrap" >    (46)</td></tr>
</table></td></tr>
</table><p>
Similarly, we define <span style="font-style:italic">A</span>(<span style="font-style:italic">r</span>) as the preimage of <span style="font-style:italic">D</span>(<span style="font-style:italic">r</span>) in <span style="font-style:italic">C</span><sup>∞</sup>(Ω), 
<span style="font-style:italic">i.e.</span> </p><table class="display dcenter"><tr style="vertical-align:middle"><td class="dcell">
     

</td><td class="dcell"><table style="border-spacing:6px;border-collapse:separate;" class="cellpading0"><tr><td style="text-align:right;white-space:nowrap" >  <span style="font-style:italic">A</span>(<span style="font-style:italic">r</span>) ≜ {<span style="font-style:italic">f</span> ∈ <span style="font-style:italic">C</span><sup>∞</sup>(Ω): | <span style="font-style:italic">D</span><sup>α</sup><span style="font-style:italic">f</span> | &lt; <span style="font-style:italic">r</span> }.
</td><td style="text-align:left;white-space:nowrap" >&nbsp;</td><td style="text-align:right;white-space:nowrap" >&nbsp;</td><td style="text-align:left;white-space:nowrap" >&nbsp;</td><td style="text-align:right;white-space:nowrap" >&nbsp;</td><td style="text-align:left;white-space:nowrap" >&nbsp;</td><td style="text-align:right;white-space:nowrap" >&nbsp;</td><td style="text-align:left;white-space:nowrap" >&nbsp;</td><td style="text-align:right;white-space:nowrap" >&nbsp;</td><td style="text-align:left;white-space:nowrap" >&nbsp;</td><td style="text-align:right;white-space:nowrap" >    (47)</td></tr>
</table></td></tr>
</table><p>
The continuousness of 
<span style="font-style:italic">D</span><sup>α</sup> in <span style="font-style:italic">C</span><sup>∞</sup>(Ω) 
implies that <span style="font-style:italic">A</span>(<span style="font-style:italic">r</span>) is open in <span style="font-style:italic">C</span><sup>∞</sup>(Ω). Moreover, 
<span style="font-style:italic">B</span>(<span style="font-style:italic">r</span>) = <span style="font-style:italic">A</span>(<span style="font-style:italic">r</span>) ∩   <span style="font-style:italic">D</span><sub><span style="font-style:italic">K</span></sub>. 
<span style="font-style:italic">B</span>(<span style="font-style:italic">r</span>) is then open in  <span style="font-style:italic">D</span><sub><span style="font-style:italic">K</span></sub>, for all <span style="font-style:italic">r</span>. So ends the proof.
</p><p><br>



</p><!--BEGIN NOTES chapter-->
<hr class="footnoterule"><dl class="thefootnotes"><dt class="dt-thefootnotes">
<a id="note1" href="#text1">1</a></dt><dd class="dd-thefootnotes"><div class="footnotetext">
See (b) of 2.5 of [<a href="#BigRudin">1</a>].
</div></dd><dt class="dt-thefootnotes"><a id="note2" href="#text2">2</a></dt><dd class="dd-thefootnotes"><div class="footnotetext">
The intersection of a decreasing sequence of nomempty Hausdorff compact sets 
is nonempty. This is a corollary of 2.6 of [<a href="#BigRudin">1</a>].
</div></dd></dl>
<!--END NOTES-->
<!--TOC chapter id="sec15" References-->
<h1 id="sec15" class="chapter">References</h1><!--SEC END --><dl class="thebibliography"><dt class="dt-thebibliography">
<a id="BigRudin">[1]</a></dt><dd class="dd-thebibliography">
Walter Rudin.
<em>Real and Complex Analysis</em>.
McGraw-Hill, 1986.</dd><dt class="dt-thebibliography"><a id="FA">[2]</a></dt><dd class="dd-thebibliography">
Walter Rudin.
<em>Functional Analysis</em>.
McGraw-Hill, 1991.</dd></dl><!--CUT END -->
<!--HTMLFOOT-->
<!--ENDHTML-->
<!--FOOTER-->
<hr style="height:2"><blockquote class="quote"><em>This document was translated from L<sup>A</sup>T<sub>E</sub>X by
</em><a href="http://hevea.inria.fr/index.html"><em>H</em><em><span style="font-size:small"><sup>E</sup></span></em><em>V</em><em><span style="font-size:small"><sup>E</sup></span></em><em>A</em></a><em>.</em></blockquote></body>
</html>
