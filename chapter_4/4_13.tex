%!TEX root = /Volumes/HD_2/Rudin/Rudin_DM.tex
{ \CMUCS 
\begin{enumerate}
\renewcommand{\labelenumi}{(\alph{enumi})}
\item Suppose $T\in \mathscr{B}(X,Y), T_n \in  \mathscr{B}(X,Y)\, $ for $n=1,\, 2, \, 3,\, \dots$, each $T_n\,$ has finite-dimensional range, and $\lim \| T-T_n\|=0\, $. Prove that $T$ is compact.
\item Assume $Y$ is a Hilbert space, and prove the converse of (a): Every compact $T\in  \mathscr{B}(X,Y)$ can be approximated in the operator norm by operators with finite-dimensional ranges. Hint: In a Hilbert space there are linear projections of norm 1 onto any closed subspace. (See theorems 5.16, 12.4.)
\end{enumerate}
}
%: a
\paragraph{PROOF.} Since each $T_n$ is compact, (a) follows from (c) of [4.18]. Besides, we take the opportunity to alternatively prove that the compact operators subspace is norm closed.\\
\\
Reset every $T_n$ as a compact operator. Let $\{x_{\,0}^{\,i}:\, i\in \N\}$ be in $U$ the open unit ball of $X$. Since $T_1$ is compact, $\{x^i_0\}$ contains a subsequence $\{x^{\,i}_{\,1}:\, i\in \N\}$ such that $\{T_1x^{i}_{\,1}\}$ converges to a point $y_{\,1}$ of $Y$. The same reasoning can be recursively applied to $T_{n}$ and $\{x^{\,i}_{\, n-1}\}\subset U$ so that $\{T_{n\,}x^{\,i}_{\, n}\}$ tends to some $y_{n}$ of $Y$, as $\{x^{\,i}_{\, n}\}$ is a subsequence of $\{x^{\,i}_{\,n-1}\}$. Then
\begin{align}\label{4_13_a_0}
T_{n\,} x^{i}_p \,\underset{i\to \infty}{\longrightarrow}\,y_n\quad (p\> n=1,\, 2,\, 3,\, \dotsb )\quad .
\end{align}
Applied with $\{x^{\,i}_{\,n}:\, (n,i\,)\in\N^2\}$, a Cantor's diagonal process therefore provides a subsequence $\{\tilde{x}_{\, j}:\, j\in \N\} $ such that
\begin{align}\label{4_13_a_1}
&T_{j\,} \tilde{x}_k \underset{k\to \infty}{\longrightarrow}y_{j}\quad ;\\
&T_{j\,} \tilde{x}_{\, j} \underset{j\to \infty}{\longrightarrow} y_{j}\quad .
\end{align}
%Since $\|T-T_k\|\underset{k\to \infty}{\longrightarrow}0$, (\ref{4_13_a_1}) implies that
%\begin{align}\label{4_13_a_2}
%T \tilde{x}_k  -y_k  \underset{k\to \infty}{\longrightarrow}0\quad.
%\end{align}
We now easily obtain
\begin{align}\label{4_13_a_3}
\| T_{j\,} \tilde{x}_{j\,}  -T_{\, k} \tilde{x}_{\, k}\| \<
\| T_{j\,} \tilde{x}_{j\,}- y_{j\,}\| +
\| y_{j\,} -T_{j\,} \tilde{x}_{k\,}\| +
\| T_{j\,} - T_{ k\,}\|   \underset{k> j\to \infty}{\longrightarrow}0\quad .
\end{align}
$\{T_{ j\,} \tilde{x}_{\, j} \}$ is then a Cauchy sequence. So is $\{T \,\tilde{x}_{\, j}\}$, since $\| T-T_{j}\,\| \to 0$. On the other hand, $Y$ is complete: (a) is then proved and we now establish the counterpart in a Hilbert space.\\
\\
%: b
Fix $\eps$ as a positive scalar. Since $T$ is compact, $Y$ contains a finite set $C$ such that 
\begin{align}\label{4_13_b_1}
T(U\,)\subset \bigcup_{c\in C} B(c,\, \eps)\quad .
\end{align}
As a Hilbert space, $Y$ contains a \textsl{maximal orthonormal set} (or \textsl{Hilbert basis}) $M$. This implies that $\text{span}(M)$ is dense in $Y$; \cf 4.18 \& 4.22 of \cite{Big_Rudin}. The finiteness of $C$ forces $M$ to enclose a finite set $S$ so that 
\begin{align}\label{4_13_b_2}
\forall c\in C, \, \exists  s(c\,) \in \text{span} (S\,):\, \|c - s(c\,)\| <\eps\quad .
\end{align}
Let $x$ be in $U$. It follows from (\ref{4_13_b_1}) that 
\begin{align}\label{4_13_b_3}
\|Tx - c_x \| < \eps
\end{align}
for some $c_x$ of $C$. We now combine (\ref{4_13_b_2}) and (\ref{4_13_b_3}) to obtain
\begin{align}
\|Tx - s(c_x) \| \<  \| Tx - c_x \| + \| c_x - s(c_x)\| < 2\eps
\end{align}
As a finite-dimensional subspace, $\text{span}(S\,)$ is closed (see footnote 4, Exercise 1.10). We so obtain
\begin{align}
Y=\text{span}(S\,)\oplus \text{span}(S\,)^\bot \quad ,
\end{align}
by [12.4]. There so exists a unique projection projection $\pi=\pi(\eps)$ of $Y$ onto itself (see [5.6] for the definition) such that
\begin{align}
\pi (Y) = \text{span}(S\,)\, ,\,  \,  (I-\pi)(Y\,) = \text{span}(S\,)^\bot\quad .
\end{align}
It is easily checked that $\pi$ has norm $1$. Moreover,
\begin{align}
\pi s = s \quad (s\in\text{span}(S\,))\quad .
\end{align}
Thus,
\begin{align}
(I-\pi) (Tx\,)= (I-\pi ) (Tx - s(c_x)) \quad (x\in U\,)\quad .
\end{align}
Then,
\begin{align}
\|( I- \pi )(T x) \| \< \| I- \pi \| \, \| Tx -s(c_x)\| < 4 \eps  \quad (x\in U)\quad 
\end{align}
(the fact that $\pi$ has norm $1$ is hidden in the right side inequality). We have just so proved that 
\begin{align}
\| T-\pi\circ T\, \| \in \underset{\eps \sim 0}{O}(\eps) \quad . 
\end{align}
That is particularly true whether $\eps=\eps_0,\, \eps_1,\, \eps_2,\, \dots ,\, \eps_n \underset{n\to \infty}{\longrightarrow} 0\,$. Let so $T_n$ be $ \pi(\eps_n) \circ T\,$ and conclude that these (compact) operators approximate $T$ in the desired fashion, \ie
\begin{align}
\| T-T_n \| \underset{n \to \infty}{\longrightarrow} 0\quad .
\end{align}
\QED
