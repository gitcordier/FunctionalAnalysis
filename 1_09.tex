\textit{Suppose
\begin{enumerate}
  \item $X$ and $Y$ are topological vector spaces,
  \item $\Lambda: X\to Y$ is linear.
  \item $N$ is a closed subspace of $X$,
  \item$\pi: X\to X/N$ is the quotient map, and
  \item $\Lambda x=0$ for every $x\in N$.
\end{enumerate}
Prove that there is a unique $f:X/N\to Y$ which satisfies 
%
  $\Lambda=f\circ \pi$, 
%
that is, 
%
  $\Lambda x=f(\pi (x))$ for all $x\in X$. 
%
Prove that $f$ is linear and that $\Lambda$ is continuous if and only if 
%
  $f $ is continuous. 
%
Also, $\Lambda$ is open if and only if $f$ is open.}
%
\begin{proof}
The equation $\Lambda = f \circ \pi$ has necessarily a unique solution, 
which is the binary relation 
%
  \begin{align}\label{1_9: defintion of f.}
     f \Def\set{(\pi x, \Lambda x)}{x \in X} \subset X/N \times Y.
    \end{align}
%
To ensure that $f$ is actually a mapping, simply remark that 
the linearity of $\Lambda$ implies 
%
  \begin{align}
    %\forall ( x,  x') \in X^2: 
    %
    \Lambda x \neq \Lambda  x' \then \pi x' \neq \pi x'.
  \end{align}
%
It straightforwardly derives from (\ref{1_9: defintion of f.}) that 
$f$ inherits linearity from $\pi$ and $\Lambda$. Now remark that 
%
  \begin{align}
    \pi x = N   
      \citethen{f\text{ linear}} 
    f(\pi x) = 0 
      \citethen{\ref{1_9: defintion of f.}} 
    \Lambda x = 0 
      \then \pi x = N 
\end{align}
and so conclude that $f$ is also one-to-one.
%
Now assume $f$ to be continuous. Then so is 
%
  $\Lambda = f\circ \pi $, 
% 
by (a) of [1.41]. 
%
Conversely, 
%
if $\Lambda$ is continuous, then for each neighborhood $V$ of $0_Y$ 
there exists a neighborhood $U$ of $0_X$ such that
%
  \begin{align}
    \Lambda(U) = f\left(\pi(U)\right) 
      \subset 
    V.
  \end{align}
%
Since $\pi$ is open (see (a) of [1.41]), $\pi(U)$ is a neighborhood of 
%
  $N=0_{X/N}$: 
%s
This is sufficient to establish that the linear mapping $f$ is continuous.
%
If $f$ is open, so is $\Lambda = f\circ \pi$, by (a) of [1.41]. 
%
Conversely, let  
%
  \begin{align}
    W \Def \pi (V) \subset  X/N \quad (V \text{ neighborhood of } 0_X) 
  \end{align}
%
run through the neighborhoods of $N$, as $\Lambda$ is kept open: So is 
%
  \begin{align}
    \Lambda(V) = f \left(\pi(V)\right) = f(W).  
  \end{align}
%
The linear mapping $f$ is then open. 
\end{proof}


