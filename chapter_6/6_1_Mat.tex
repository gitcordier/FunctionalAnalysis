We will actually prove more by showing that $\mathscr{D}(\Omega)$ is separable for each nonempty open subset $\Omega\,$ of $\R^n\,$. 
\begin{proof}
The following is split in three parts. The first one is about the above requested result: %
That was our first part. We now go further by proving the separability of $\mathscr{D}(\Omega)\,$. To do so, we keep $(\alpha,\, j\,)$ in $\N^n\times \N\,$. Remark that $S$ encloses $\text{supp}(D^{\, \alpha}f\,) $: according to the first part, there exists a sequence $\{ P_{\alpha,\, j}:\, j\in \N \}  \subset\R [X_1,\dotsc ,\, X_n]$ such that 
\begin{align}\label{6_1_4}
\|  D^{\, \alpha} f-  \psi P_{\alpha,\, j \,} \|_\infty \underset{j\infty}{\longrightarrow} 0 \quad . 
\end{align}
Now let $m$ range over $\{1,\, 2,\,3,\, \dots\}$ and set $W_{m,\, j\,}$ in $\mathscr{D}(\Omega)$ as follows
\begin{align}\label{6_1_5a}
D^{\,\minus \alpha} \phi \in \mathscr{D}(\Omega):\, D^{\,\alpha} D^{\,\minus \alpha} \phi= \phi\quad .
\end{align}
%\begin{align}\label{6_1_5c}
%D^{\,\minus \alpha} \phi(x\,) \Def 
%\underset{ \alpha_1 \text{ time(s)}  }{
                                                         % \underbrace{ \int_{\minus \infty}^{x_1} \dotsc\int_{\minus\infty}^{x_1}}
%                                                          }
% \,\,\dotsc\,
% \underset{ \alpha_n \text{ time(s)}  }{
                                                         % \underbrace{ \int_{\minus \infty}^{x_n} \dotsc\int_{\minus\infty}^{x_n}}
                                                        %  } 
                                                       %  \,\, \phi \quad (x\in \R^n)\quad .
                                                        %  \end{align}
\begin{align}\label{6_1_5}
W_{m,\, j\,}(x\,)\Def D^{\,\minus (m,\dotsc,\,m)} ( \psi P_{(m,\dotsc, \, m),\, j\,})
                                                          \end{align}
By (\ref{6_1_4}), there exists a natural number $k(\!m)$ such that 
\begin{align}\label{6_1_6}
\|D^{(m,\dotsc,\, m)} (f   -W_{m,\, j})   \|_\infty < 1/m  \quad (j\geq k(\!m) )\quad  .
\end{align}
 Assume without loss of generality that $S$ has diameter $1$, so that (\ref{6_1_6}) yields
\begin{align}\label{6_1_7}
\|D^{\, \lambda } (f   -W_{m,\, k(\!m)}  )  \|_\infty < 1/m \quad ( \lvert \lambda \rvert \leq m )\quad ,
%\max \{  \lvert D^{\, \lambda}(f- W_{m,\, k(\!m) })(x\,)\rvert :\, x\in S,\, \lvert \lambda\rvert \leq m\,\} < 1/m \quad 
\end{align}
by the mean value theorem. In other words $(\text{remark that }\text{supp} (f   -W_{m,\, k(\!m)}) \subseteq S\,)$,  
\begin{align}\label{6_1_8}
f- W_{m,\, k(\!m)}  \in U_m \Def \{ \phi \in \mathscr{D}_S :\, \| \phi \|_m< 1/m \} \supseteq  U_{m+1}\supseteq  \dotsb \quad( m=1,\, 2,\, 3,\, \dots) \quad  .
\end{align}
Pick $W$ in $\beta$ (see (b) of [6.3]): $W\cap\mathscr{D}_S$ contains a neighbourhood of $0$. Hence $W$ contains some $U_{m\,}$, for $m$ sufficiently large. Thus 
\begin{align}\label{6_1_9}
W_{m,\, k(\!m) } \underset{m\infty} {\longrightarrow} f  \quad  (\text{in } \mathscr{D}(\Omega))\quad .
\end{align}
We have so established that the $W_{m,\, k(\!m)}$'s family is dense in $\mathscr{D}(\Omega)$. We now aim to disclose a  countable set $\tilde{W}\,$ that has the same property.\\
\\
Choose $\delta $ in $\R_+$ and fetch any $W_{m,\, k(\!m)}$. Let $X$ be $(X_1,\dotsc,\, X_n)$ and express $P_{(m,\dotsc,\, m\,),\, k(\!m)}$ as 
\begin{align}\label{6_1_10}
P(X\,)=\sum_{\lvert \gamma \rvert \leq d} p_\gamma \cdot X^{\,\gamma} \quad  .
\end{align}
Since $\bar{\Q}=\R$, $\Q[X\,]$ hosts some ${\Q(X\,)=\sum_{\lvert \gamma \rvert \leq d} q_\gamma \cdot X^{\,\gamma}}$ such that $\lvert p_\gamma - q_\gamma\rvert < \delta\,$ for all $\gamma$. Thus,
\begin{align}\label{6_1_11}
\lvert P(x\,)-Q(x\,) \,\rvert \leq   \sum_{\lvert \gamma \rvert \leq d} \lvert p_\gamma-q_\gamma\rvert\, \lvert x\,\rvert^{\lvert \gamma\rvert} \leq \delta \sum_{l  \leq d} \binom{l+n-1}{n-1}\, \| \, x\, \|^{l}_\infty \quad (x\in \R^n)\quad .
\end{align}
Since $S$ is bounded, we so obtain
\begin{align}\label{6_1_12}
 \|\psi (P- Q) \|_\infty  \in O(\delta) \quad .
\end{align}
Now define $\tilde{W}_m\,$ in terms of $Q$ as $W_{m,\, k(\!m)}$ was defined in terms of $P$, and consider the integrations made in (\ref{6_1_5}): each $D^{\,\lambda} \tilde{W}_m\,\, (\lvert \lambda \rvert\leq m)$ can be obtained from some of them. So (\ref{6_1_12}) yields
\begin{align}\label{6_1_13}
 \| D^{\,\lambda} (W_{m,\, k(\!m)} - \tilde{W}_m ) \|_\infty  \in O(\delta)\quad (\lvert \lambda \rvert\leq m) \quad   . 
\end{align}
To be more specific, these $\lambda$'s only exist in finite  amount, so the big O can be assumed to be the same for all them. Since $\delta$ was arbitrary, combining  (\ref{6_1_9}) with (\ref{6_1_13}) establishes the density of the all $\tilde{W}_m$'s family $\tilde{W}$.\\
\\
Furthermore, each member of $\tilde{W}$ is only made of two ingredients: $\psi$ and a polynomial of $\Q[X]$. The mapping $\psi$ is attached to some $K_i$ and $\Q[X]$ inherits countableness from $\Q$. Note that the ``integrations packs" of (\ref{6_1_5}) only exist in countable amount. %
Our $\tilde{W}$ is then countable. %
\end{proof}












