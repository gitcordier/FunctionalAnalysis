\renewcommand{\labelenumi}{(\alph{enumi})} 
{\it  
\begin{enumerate}
\item Prove that a set $E\subset \mathcal{D}(\Omega)$ is bounded if and only if 
\begin{align*}
\sup \{\lvert\Lambda \phi\rvert:\, \phi \in E\,\}< \infty
\end{align*} 
for every $\Lambda \in \mathcal{D}(\Omega)$.
\item Suppose $\{\phi_j\}$ is a sequence in $\mathcal{D}(\Omega)$ such that $\{\Lambda \phi_j\}$ is a bounded sequence of numbers, for every $\Lambda \in \mathcal{D}'(\Omega)$. Prove that some subsequence of $\{\phi_j\}$ converges, in the topology of $\mathcal{D}(\Omega)$.
\item Suppose $\{\Lambda _j\}$ is a sequence in $\mathcal{D}'(\Omega)$ such that $\{\Lambda_{j\,} \phi\}$ is bounded, for every $\phi \in \mathcal{D}(\Omega)$. Prove that some subsequence of $\{\Lambda _j\}$ converges in $\mathcal{D}'(\Omega)$ and that the convergence is uniform on every bounded subset of $\mathcal{D}(\Omega)$. Hint: By the Banach-Steinhaus theorem, the restrictions of the $\Lambda_j$ to $\mathcal{D}_K$ are equicontinuous. Apply Ascoli's theorem.
\end{enumerate}}
\begin{proof} 
%: (a)
Since $\mathcal{D}(\Omega)$ is a locally convex space (see (b) of [6.4]), [3.18] states that $E\,$ is bounded if and only if it is weakly bounded. That is (a). \\
\\
%: (b) 
To prove (b), we first use (a) to conclude that $E= \{\phi_j:\, j\in \N\}$ is bounded: so is $\overline{E}$. %
By (c) of [6.5], there exists some $ \mathcal{D}_K$ that contains $\overline{E}$. %
Since $ \mathcal{D}_K$ has the Heine-Borel property (see [1.46]), $\overline{E}$ is $\tau_K$-compact. %
Apply [A4] with the metrizable space $\mathcal{D}_K$ (see [1.46]) to conclude that $\overline{E}$ has a $\tau_K$ limit point. %
It then follows from (b) of [6.5] that (b) holds.
\end{proof}