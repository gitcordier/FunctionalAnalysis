%
\renewcommand{\labelenumi}{(\alph{enumi})} 
{\it 
\begin{enumerate}
\item Suppose that $c_m=\exp\{\minus (m!)!\}$, $m=0,\, 1,\, 2,\, \dots\, $. Does the series
\begin{align*}{
\sum_{m=0}^\infty c_m (D^m\phi)(0)
}\end{align*}
converges for every $\phi\in C^{\,\infty} (R)$?
\item Let $\Omega$ be open in $\R^n$, suppose $\Lambda_i\in \mathscr{D}'(\Omega)$, and suppose that all $\Lambda_i$ have their supports in some fixed compact $K\subseteq \Omega$. Prove that the sequence $\{\Lambda_i\}$ cannot converge in $\mathscr{D}'(\Omega)$ unless the orders of the $\Lambda_j$ are bounded. Hint: Use the Banach-Steinhaus theorem.
\item Can the assumption about the supports be dropped in (b)?
\end{enumerate}}
\begin{proof} The answer is: no. Let us establish this assertion. Assume, to reach a contradiction, that the above series converges for every smooth $\phi:\, \R\to\,\C\,$. \\
\\
The sequence $\{c_m\, (D^m\phi)(0)\}$ so converges to $0$. Nevertheless, it is proved in [1.46] that $C^{\,\infty}(\Omega)$ is not locally bounded. In other words, it is always possible to excavate a $\phi\,$ for which the magnitude of the $m$-th derivative at $0$ is as large as we please\footnote{indeed [1.46] provides sufficient tools for constructive proof of this; see the $\phi_j-\check{\phi}_j$ involved in (\ref{2_3_phicheck}).}, \eg greater than $1/c_m$.
 A desired contradiction is then reached. We now prove (b), again by contradiction.\\
\\
To do so we assume $\{\Lambda_j\}$ to converge to some $\Lambda$ of $\mathscr{D}'(\Omega)$ and we let $Q$ run through the compact sets of $\Omega\,$. Next, we define
\begin{align}{\label{6_6_1}
S(T,\, Q\,)\Def \{N\in \N, \, \exists C\in \R_+:\, \lvert T\phi\, \rvert \leq C\, \| \phi \|_N \,\text{ for all }\phi \text{ of } \mathscr{D}_Q \}\quad (T\in \mathscr{D}(\Omega))\quad .
}\end{align}
Such subset of $\N$ has a minimum $\omega(T,\, Q)$. The following value
\begin{align}{\label{6_6_2}
\omega (T\,) \Def \max\{ \omega(T,\, Q\,): \, Q\subseteq \Omega\, , \,\, Q \text{ compact }\}\leq \infty
}\end{align}
is then the order of $T$. Assume, to reach a contradiction, that, after passage to a subsequence,
\begin{align}{\label{6_6_3}
\omega (\Lambda_{j\,},\, Q_j\,) =j\quad (j=1,\, 2,\, 3,\, \dots)
}\end{align}
for some compact $Q=Q_j\,$. By (a) of [6.24], $Q_j$ cuts $\text{supp}\Lambda_j\,$, say in $p_j\,$. Since $K$ encloses $\text{supp}\Lambda_j\,$, $\{p_j\}$ tends, after passage to a subsequence, to some $p$ of $K\,$.
Choose a positive scalar $r$ so that 
\begin{align}{\label{6_6_5}
\overline{B}(p,\, r)\Def \{ x\in \R^n:\, \lvert x-p\,\rvert\leq r\}\subseteq \Omega \quad .
}\end{align}
Such closed ball $\overline{B}(p,\, r)$ is a compact subset of $\Omega$. By (b) of [6.5] (which refers to [1.46])  $\mathscr{D}_{\overline{B}(p,\, r)}$ is then a Fréchet space. It now follows from [2.6] that $\{\Lambda_j\}$ is equicontinuous on $\mathscr{D}_{\overline{B}(p,\, r)}\,$. There so exists\footnote{For more details, see Exercise 2.3.} a nonnegative integer $N$ such that 
\begin{align}{\label{6_6_6}
\lvert \Lambda \phi \,\rvert \leq C\, \| \phi \|_N \quad (\phi \in \mathscr{D}_{\overline{B}(p,\, r)})
}\end{align}
for some positive constant $C$. On the other hand, $\overline{B}(p,\, r)$ contains almost all the $p_j$'s. Hence
\begin{align}{\label{6_6_7} 
\lvert \Lambda_N\,\phi \,\rvert >  C\, \| \phi \|_N 
}\end{align}
for some $\phi$ of $\mathscr{D}_{\overline{B}(p,r)}\,$. (b) is then established.\\
\\
To prove (c), we introduce a sequence $\{x_m:\, m\in \Z\}$ of $ \Omega$ that has no limit point. Let $\{ \alpha_m:\, m\in \Z\}$ be in $\N$ and so define\footnote{As $\Omega=\R\,$, the case $\alpha_m= m$ is the ``counterpart" of the series of (a) and the case $(x_m,\, \alpha_m)= (m,\, 0)$ is the \textsl{Dirac comb}.}
\begin{align}{
\Lambda:\, \mathscr{D}(\Omega) \to &  \, \C   \qquad\qquad\qquad\qquad .\\
 \phi \mapsto & \,  \sum_{m=\minus \infty}^\infty (D^{\,\alpha_m}\phi)(x_m)  \nonumber
}\end{align}
$\Lambda$ belongs to $\mathscr{D}'(\Omega)$, since $\{x_m\}$ has no limit point. Next, we easily check that
\begin{align}{
\Lambda_j:\, \mathscr{D}(\Omega) \to   & \, \C  \qquad\qquad\qquad\qquad(j\in \N)\\
 \phi \mapsto &  \,  \sum_{\lvert m\rvert \leq j} (D^{\,\alpha_m}\phi)(x_m) \nonumber
}\end{align}
is also a distribution and that $\{\Lambda_j\}$ tends to $\Lambda$ in $\mathscr{D}'(\Omega)$. Nevertheless, no $\Lambda_j$'s can have common support because $\{x_m\}$ has no limit point. Our assumption can therefore be dropped.\end{proof}














