\textit{
Prove that the topology of $C(\Omega)$ does not depend on the particular 
choice of $\{K_{n}\}$, as long as this sequence satisfies the conditions 
specified in section 1.44. Do the same for $C^\infty(\Omega)$ (Section 1.46).}
%
\paragraph{Comment}This is an invariance property: 
The function test topology only depends on the existence of the 
supremum-seminorms $|f|_N = \max{|f|}$, then, eventually, 
only on the ambient space itself. 
This should then be regarded as a very part of the textbook \cite{FA}
%
The proof consists in combining trivial consequences of the local base 
definition with the well-known result (\eg [2.6] in \cite{BigRudin}) 
about intersection of nonempty compact sets. 

\paragraph{Lemma} Let $X$ be a topological space with a countable local base 
$\set{V_N}{\counting{N}}$. 
If 
%
  $\tilde{V}_{N} = V_1 \cuts \cdots \cuts V_N$, 
%
then every subsequence 
% 
  $\singleton{\tilde{V}_{\rho(N)}}$ 
%
is a also a {decreasing} (\ie 
%
  $\tilde{V}_{\rho(N)} \contains \tilde{V}_{\rho(N+1)}$)
%
local base of $X$.
%
\begin{proof}
The decreasing property is trivial. Now remark that 
%
  $V_N \contains \tilde{V}_{N} \contains \tilde{V}_{N+1}$:
%
The left inclusion shows that 
%
  $\singleton{\tilde{V}_{N}}$ 
% 
is a local base of $X$. Then so is 
%
  $\singleton{\tilde{V}_{\rho(N)}}$,
% 
since $\tilde{V}_{N} \contains \tilde{V}_{\rho(N)}$.
\end{proof}
%
The following special case 
%
  $V_{N} = \tilde{V}_{N}$ 
% 
is one of the key ingredients:
\paragraph{Corollary 1 (special case)}
With the same notations, if $\singleton{V_{N}}$ is a decreasing local base, 
then so is $\singleton{V_{\rho(N)}}$.
%
\paragraph{Corollary 2}
If 
%
  $\singleton{Q_N}$ 
%
is a sequence of compacts that satisfies the conditions specified 
in section 1.44, then every subsequence 
%
  $\singleton{Q_{\rho(N)}}$ 
%
also satisfies theses conditions.
%
Furthermore, if $\tau_{Q}$ is the $C(\Omega)$'s 
(respectively $C^\infty (\Omega)$'s) topology of the seminorms $p_{N}$, 
as defined in section 1.44 (respectively 1.46), then the seminorms 
%
  $p_{\rho(N)}$ 
%
define the same topology $\tau_{Q}$.
%
\begin{proof}%
%
Let $X$ be $C(\Omega)$ topologized with the seminorms $p_{N}$ 
(the case $X=C^\infty(\Omega)$ is proved the same way).
%
If 
  %
    $V_{N} = \singleton{p_{N} < 1/N}$, 
  %
then 
  %
    $\singleton{V_{N}}$ 
  %
is a decreasing local base of $X$.
%
Moreover,
% 
  \begin{align}
    Q_{\rho(N)} 
      \subset 
    \interior{Q}_{\rho(N) + 1} 
      \subset 
    Q_{\rho(N) + 1} 
      \subset 
    Q_{\rho(N+ 1)},
  \end{align}
% 
and this yields
%
  \begin{align}
    Q_{\rho(N)} 
      \subset 
    \interior{Q}_{\rho(N+ 1)}.
  \end{align}
%
In other words, 
%
  $Q_{\rho(N)}$ satisfies the conditions specified in section 1.44.
%
%
  $\singleton{p_{\rho(N)}}$
% 
then defines a topology $\tau_{Q_\rho}$ for which  
% 
  $\singleton{V_{\rho(N)}}$ 
%
is a local base. So, 
% 
  $\tau_{Q_\rho} \subset \tau_{Q}$.
%
Conversely, the above corollary turns 
%
  $\singleton{V_{\rho(N)}}$ 
%
into a local base of $\tau_{Q}$. Hence  
%
  $\tau_{Q}\subset \tau_{Q_\rho}$.
%
\end{proof}
\noindent We are now in a fair position to establish the following:
%\newpage
\paragraph{Theorem} 
The topology of $C(\Omega)$ does not depend on the particular choice of 
$\{K_{n}\}$, as long as this sequence satisfies the conditions 
specified in section 1.44. Neither does the topology of $C^\infty (\Omega)$, 
as long as this sequence satisfies the conditions specified in section 1.46.
%
\begin{proof}%
With the second corollary's notations,
% 
  $\tau_{K} = \tau_{K_\kappa}$,
%
for every subsequence $\singleton{K_{\kappa(n)}}$.
% 
Similarly, let 
%
  $\singleton{L_n}$ 
% 
be a sequence of compact subsets of $\Omega$ that satisfies 
the condition specified in [1.44], 
so that 
%
  $\tau_{L} = \tau_{L_\lambda}$
%
for every subsequence $\singleton{L_{\lambda(n)}}$. 
%
The following definition
%
  \begin{align}
    C_{i, j} \Def K_i \setminus \interior{L_j} \quad (\counting{i, j})
  \end{align}
%
shapes $\set{C_{i, j}}{\counting{j}}$ as a decreasing sequence of compacts.
%
We now suppose (to reach a contradiction) that 
% 
  no $C_{i, j}$ is empty 
% 
and so conclude that 
% 
  %$\bigcap_{j=1}^\infty C_{i, j}$ 
  $C_{i, 1} \cap C_{i, 2} \cap \cdots$
%
contains a point that is not in any $\text{int}(L_j)$. 
But the conditions specified in [1.44] force 
% 
  $\set{\text{int}(L_j)}{\counting{j}}$ 
%
to be an open cover.
% 
This contradiction reveals that 
%
  $C_{i,\,j} = C_{i, \,j+1} = C_{i, \,j+2} = \cdots =\emptyset$
%  
for some $j=j\up{i}$. We then define  
%
  $\lambda_i = i + j\up{i}$, 
% 
so that
%
  \begin{align}
    %
    \label{1_16. K subset interior L}
    %
    K_i \setminus \interior{L}_{\lambda_i} = \emptyset, \quad \ie \quad
    K_i 
      \subset 
    \interior{L}_{\lambda_i}.
  \end{align} 
%
Let us reiterate the above proof with $K_n$ and $L_n$ in exchanged roles 
then similarly find a subsequence $\set{\kappa_j}{\counting{j}}$ such that 
%
  \begin{align}
  %
  \label{1_16. L subset interior K}
  %
    L_j \subset \interior{K}_{\kappa_j}
  \end{align}
%
Combine 
%
  (\ref{1_16. K subset interior L}) with 
  (\ref{1_16. L subset interior K}) 
%
and so obtain
%
  \begin{align}
    K_1 
      \subset 
    \interior{L}_{\lambda_1} 
      \subset 
    L_{\lambda_1} 
      \subset 
    \interior{K}_{\kappa\circ\lambda_1}
      \subset 
    K_{\kappa\circ\lambda_1}
      \subset
    \interior{L}_{\lambda_{\kappa\circ\lambda_1}}
      \subset
    \cdots
  \end{align}
%
Thus the sequence 
%
  $Q = (
    K_1, 
    L_{\lambda_1}, 
    K_{\kappa\circ\lambda_1}, 
    L_{\lambda_{\kappa\circ\lambda_1}},
    \dots
  )$
%
satisfies the conditions specified in section 1.44. 
It now follows from the corollary 2 that 
%
  \begin{align}  
    \tau_{K} 
    = 
      \tau_{K_\kappa} 
    = 
      \tau_{Q} 
    = 
      \tau_{L_\lambda} 
    = \tau_{L}.
  \end{align} 
%
So ends the proof
\end{proof}
% END
