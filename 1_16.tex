%!TEX root = /Volumes/HD_2/Rudin/Rudin_DM.tex
\textit{
Prove that the topology of $C(\Omega)$ does not depend on the particular 
choice of $\{K_{n}\}$, as long as this sequence satisfies the conditions 
specified in section 1.44. Do the same for $C^\infty(\Omega)$ (Section 1.46).}
%
\paragraph{Lemma} Let $X$ be a topological space with a countable local base 
$\set{V_N}{\counting{N}}$. 
If 
%
  $\tilde{V}_N = V_1 \cuts \cdots \cuts V_N$, 
%
then every subsequence 
% 
  $\singleton{\tilde{V}_{\rho(N)}}$ 
%
is a also a {decreasing} (\ie 
%
  $\tilde{V}_{\rho(N)} \contains \tilde{V}_{\rho(N+1)}$)
%
local base of $X$.
%
\begin{proof}
The decreasing property is trivial. Now remark that 
%
  $V_N \contains \tilde{V}_N \contains \tilde{V}_{N+1}$:
%
The left inclusion shows that 
%
  $\singleton{\tilde{V}_N}$ 
% 
is a local base of $X$. Then so is 
%
  $\singleton{\tilde{V}_{\rho(N)}}$,
% 
since $\tilde{V}_N \contains \tilde{V}_{\rho(N)}$.
\end{proof}
%
\paragraph{Corollary}
If 
%
  $\singleton{Q_N}$ 
%
is a sequence of compacts that satisfies the conditions specified 
in section 1.44, then every subsequence 
%
  $\singleton{Q_{\rho(N)}}$ 
%
also satisfies theses conditions.
%
Furthermore, if $\tau^{Q}$ is the $C(\Omega)$'s 
(respectively $C^\infty (\Omega)$) topology of the seminorms $p^{Q}_N = p_N$, 
as defined in section 1.44 (respectively 1.46), then the seminorms 
%
  $p^{Q}_{\rho(N)}$ 
%
define the same topology $\tau^{Q}$.
%
\begin{proof}%
%
Let $X$ be $C(\Omega)$ topologized with the seminorms $p^{Q}_N$ 
(the case $X=C^\infty(\Omega)$ is proved the same way).
%
If 
  %
    $V^{Q}_N = \singleton{p^{Q}_N < 1/N}$, 
  %
then 
  %
    $\singleton{V^{Q}_N}$ 
  %
is a decreasing local base of $X$.
%
Moreover,
% 
  \begin{align}
    Q_{\rho(N)} 
      \subset 
    \interior{Q}_{\rho(N) + 1} 
      \subset 
    Q_{\rho(N) + 1} 
      \subset 
    Q_{\rho(N+ 1)},
  \end{align}
% 
and this yields
%
  \begin{align}
    Q_{\rho(N)} 
      \subset 
    \interior{Q}_{\rho(N+ 1)}.
  \end{align}
%
In other words, 
%
  $Q_{\rho(N)}$ satisfies the conditions specified in section 1.44.
%
%
  $\singleton{p^{Q}_{\rho(N)}}$
% 
then defines a topology $\tau^{Q_\rho}$ for which  
% 
  $\singleton{V^{Q}_{\rho(N)}}$ 
%
is a local base. So, 
% 
  $\tau^{Q_\rho} \subset \tau^{Q}$.
%
Conversely, the Lemma turns 
%
  $\singleton{V^{Q}_{\rho(N)}}$ 
%
into a local base of $\tau^{Q}$. Hence  
%
  $\tau^{Q}\subset \tau^{Q_\rho}$.
%
\end{proof}
\paragraph{Theorem} 
The topology of $C(\Omega)$ does not depend on the particular choice of 
$\{K_{n}\}$, as long as this sequence satisfies the conditions 
specified in section 1.44. Neither does the topology of $C^\infty (\Omega)$, 
as long as this sequence satisfies the conditions specified in section 1.46.
%
\begin{proof}%
With the Corollary's notations,
% 
  $\tau^{K} = \tau^{K_\kappa}$,
%
for every subsequence $\singleton{K_{\kappa(n)}}$.
% 
Similarly, let 
%
  $\singleton{L_n}$ 
% 
be a sequence of compact subsets of $\Omega$ that satisfies 
the condition specified in [1.44], 
so that 
%
  $\tau^{L} = \tau^{L_\lambda}$
%
for every subsequence $\singleton{L_{\lambda(n)}}$. 
%
The following definition
%
  \begin{align}
    C_{i, j} \Def K_i \setminus \interior{L_j} \quad (\counting{i, j})
  \end{align}
%
turns $\set{C_{i, j}}{\counting{j}}$ into a decreasing sequence of compacts.
%
We now suppose (to reach a contradiction) that 
% 
  no $C_{i,j}$ is empty 
% 
and so conclude that 
% 
  $\bigcap_{j=1}^\infty C_{i, j}$ 
%
contains a point that is not in any $L_j$. 
But the conditions specified in [1.44] force 
% 
  $\singleton{\interior{L}_j}$ 
%
to be an open cover.
% 
This contradiction reveals that $C_{i,j}, C_{i, j+1}, C_{i, j+2}, \dots,$  
are actually empty for some $j=j\up{i}$. We then define  
%
  $\lambda(i) = i + j\up{i}$, 
% 
so that
%
  \begin{align}
    %
    \label{1_16. K subset interior L}
    %
    K_i 
      \subset 
    \interior{L}_{\lambda(i)}.
  \end{align} 
%
Let us reiterate the above proof with $K_n$ and $L_n$ in exchanged roles 
then similarly find a subsequence $\set{\kappa(j)}{\counting{j}}$ such that 
%
  \begin{align}
  %
  \label{1_16. L subset interior K}
  %
    L_j \subset \interior{K}_{\kappa(j)}
  \end{align}
%
Combine 
%
  (\ref{1_16. K subset interior L}) with 
  (\ref{1_16. L subset interior K}) 
%
and so obtain
%
  \begin{align}
    K_1 
      \subset 
    \interior{L}_{\lambda(1)} 
      \subset 
    L_{\lambda(1)} 
      \subset 
    \interior{K}_{\kappa\circ\lambda(1)}
      \subset 
    K_{\kappa\circ\lambda(1)}
      \subset
    \interior{L}_{\lambda\circ\kappa\circ\lambda(1)}
      \subset
    \cdots
  \end{align}
%
Thus the sequence 
%
  $Q = (
    K_1, 
    L_{\lambda(1)}, 
    K_{\kappa\circ\lambda(1)}, 
    L_{\lambda\circ\kappa\circ\lambda(1)},
    \dots
  )$

satisfies the conditions specified in section 1.44. 
It now follows from the Corollary that 
%
  \begin{align}  
    \tau^{K} 
    = 
      \tau^{K_\kappa} 
    = 
      \tau^{Q} 
    = 
      \tau^{L_\lambda} 
    = \tau^{L}.
  \end{align} 
%
So ends the proof
\end{proof}
% END
