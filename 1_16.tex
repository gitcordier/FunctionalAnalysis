\textit{
Prove that the topology of $C(\Omega)$ does not depend on the particular 
choice of $\singleton{K_n}$, as long as this sequence satisfies the conditions 
specified in section 1.44. Do the same for $C^\infty(\Omega)$ (Section 1.46).}
%
\paragraph{Comment}This is an invariance property: 
The function test topology only depends on the existence of the 
supremum-seminorms $p_n$, then, eventually, 
only on the ambient space itself. 
This should then be regarded as a very part of the textbook \cite{FA}
%
The proof consists in combining trivial consequences of the local base 
definition with a well-known result (\eg [2.6] in \cite{BigRudin}) 
about intersection of nonempty compact sets. 

\paragraph{Lemma 1} Let $X$ be a topological space with a countable local base 
$\set{V_n}{\counting{n}}$. 
If 
%
  $\tilde{V}_{n} = V_1 \cuts \cdots \cuts V_n$, 
%
then every subsequence 
% 
  $\singleton{\tilde{V}_{\rho(n)}}$ 
%
is a decreasing (\ie 
%
  $\tilde{V}_{\rho(n)} \contains \tilde{V}_{\rho(n+1)}$)
%
local base of $X$.
%
\begin{proof}
The decreasing property is trivial. Now remark that 
%
  $V_n \contains \tilde{V}_{n}$:
%
This shows that 
%
  $\singleton{\tilde{V}_{n}}$ 
% 
is a local base of $X$. Then so is 
%
  $\singleton{\tilde{V}_{\rho(n)}}$,
% 
since $\tilde{V}_{n} \contains \tilde{V}_{\rho(n)}$.
\end{proof}
%
\noindent The following special case 
%
  $V_{n} = \tilde{V}_{n}$ 
% 
is one of the key ingredients:
%: COROLLARY 1 OF LEMMA 1-----------------------------------------------------%
\paragraph{Corollary 1 (special case $V_{n} = \tilde{V}_{n}$)}
Under the same notations of Lemma 1, if $\singleton{V_{n}}$ is a decreasing 
local base, then so is $\singleton{V_{\rho(n)}}$.
%
%: COROLLARY 2 OF LEMMA 1-----------------------------------------------------%
\paragraph{Corollary 2}
If 
%
  $\singleton{Q_n}$ 
%
is a sequence of compact sets that satisfies the conditions specified 
in section 1.44, then every subsequence 
%
  $\singleton{Q_{\rho(n)}}$ 
%
also satisfies theses conditions.
%
Furthermore, if $\tau_{Q}$ is the $C(\Omega)$'s 
(respectively $C^\infty (\Omega)$'s) topology of the seminorms $p_{n}$, 
as defined in section 1.44 (respectively 1.46), then the seminorms 
%
  $p_{\rho(n)}$ 
%
define the same topology $\tau_{Q}$.
%
\begin{proof}%
%
Let $X$ be $C(\Omega)$ topologized with the seminorms $p_{n}$ 
(the case $X=C^\infty(\Omega)$ is proved the same way).
%
If 
  %
    $V_{n} = \singleton{p_{n} < 1/n}$, 
  %
then 
  %
    $\singleton{V_{n}}$ 
  %
is a decreasing local base of $X$.
%
Moreover,
% 
  \begin{align}
    Q_{\rho(n)} 
      \subset 
    \interior{Q}_{\rho(n) + 1} 
      \subset 
    Q_{\rho(n) + 1} 
      \subset 
    Q_{\rho(n+ 1)}.
  \end{align}
% 
Thus,
%
  \begin{align}
    Q_{\rho(n)} 
      \subset 
    \interior{Q}_{\rho(n+ 1)}.
  \end{align}
%
In other words, 
%
  $Q_{\rho(n)}$ satisfies the conditions specified in section 1.44.
%
%
  $\singleton{p_{\rho(n)}}$
% 
then defines a topology $\tau_{Q_\rho}$ for which  
% 
  $\singleton{V_{\rho(n)}}$ 
%
is a local base. So, 
% 
  $\tau_{Q_\rho} \subset \tau_{Q}$.
%
Conversely, the above corollary asserts that 
%
  $\singleton{V_{\rho(n)}}$ 
%
is a local base of $\tau_{Q}$, which yields  
%
  $\tau_{Q}\subset \tau_{Q_\rho}$.
%
\end{proof}
%: LEMMA 2 -------------------------------------------------------------------%
\paragraph{Lemma 2} 
If a sequence of compact sets $\singleton{Q_n}$ satisfies the conditions 
specified in section 1.44, then every compact set $K$ lies in allmost all 
%
  $Q^\circ_n$, \ie
%
there exists $m$ such that 
%
  \begin{align}
    K \subset 
    \interior{Q}_m 
      \subset 
    \interior{Q}_{m+1}
      \subset
    \interior{Q}_{m+2}
      \subset
    \cdots.
  \end{align}
%
\begin{proof}
The following definition
%
  \begin{align}
    C_n \Def K \setminus \interior{Q}_n \quad (\counting{n})
  \end{align}
%
shapes $\singleton{C_n}$ as a decreasing sequence of compact\footnote{
  See (b) of 2.5 of \cite{BigRudin}.
} 
sets. We now suppose (to reach a contradiction) that 
% 
  no $C_n$ is empty 
% 
and so conclude\footnote{
  The intersection of a decreasing sequence of nomempty Hausdorff compact sets 
  is nonempty. This is a corollary of 2.6 of \cite{BigRudin}.
} 
that the $C_n$'s intersection contains a point that is not in any $Q^\circ_n$. 
On the other hand, the conditions specified in [1.44] force the 
% 
  $Q^\circ_n$'s collection  
%
to be an open cover.
% 
This contradiction reveals that 
%
  $C_m = \emptyset$, 
    \ie 
  $K \subset Q^\circ_m$, 
%  
for some $m$.
%
Finally,  
  \begin{align}
    K\subset 
    \interior{Q}_m
      \subset
    Q_m
      \subset
    \interior{Q}_{m+1}
      \subset
    Q_{m +1}
      \subset
    \interior{Q}_{m+2}
      \subset
    \cdots.
  \end{align}  
So ends the proof.
\end{proof}
%: THEOREM -------------------------------------------------------------------%
\noindent We are now in a fair position to establish the following:
%\newpage
\paragraph{Theorem} 
The topology of $C(\Omega)$ does not depend on the particular choice of 
$\singleton{K_n}$, as long as this sequence satisfies the conditions 
specified in section 1.44. Neither does the topology of $C^\infty (\Omega)$, 
as long as this sequence satisfies the conditions specified in section 1.44.
%
\begin{proof}%
With the second corollary's notations,
% 
  $\tau_{K} = \tau_{K_\lambda}$,
%
for every subsequence $\singleton{K_{\lambda(n)}}$.
% 
Similarly, let 
%
  $\singleton{L_n}$ 
% 
be another sequence of compact subsets of $\Omega$ that satisfies 
the condition specified in [1.44], 
so that 
%
  $\tau_{L} = \tau_{L_\kappa}$
%
for every subsequence $\singleton{L_{\kappa(n)}}$. 
%
Now apply the above Lemma 2 with $K_i$ ($\counting{i}$) and so conclude that  
%
  $K_i 
    \subset 
  L^\circ_{m_i} 
    \subset 
  L^\circ_{m_{i}+1}
    \subset
  \cdots$
%  
for some $m_i$. The special case $n= m_i + i$ is then 
%
  \begin{align}
    %
    \label{1_16. K subset interior L}
    K_i
      \subset 
    \interior{L}_{\kappa_i}
      \quad 
    (\kappa_i \Def m_i + i)
  \end{align} 
%
Let us reiterate the above proof with $K_n$ and $L_n$ in exchanged roles 
then similarly find a subsequence $\set{\lambda_j}{\counting{j}}$ such that 
%
  \begin{align}
  %
  \label{1_16. L subset interior K}
  %
    L_j \subset \interior{K}_{\lambda_j}
  \end{align}
%
Combine 
%
  (\ref{1_16. K subset interior L}) with 
  (\ref{1_16. L subset interior K}) 
%
and so obtain
%
  \begin{align}
    K_1 
      \subset 
    \interior{L}_{\kappa_1} 
      \subset 
    L_{\kappa_1} 
      \subset 
    \interior{K}_{\lambda_{\kappa_1}}
      \subset 
    K_{\lambda_{\kappa_1}}
      \subset
    \interior{L}_{\kappa_{\lambda_{\kappa_1}}}
      \subset
    \cdots, 
  \end{align}
%
which means that the sequence 
%
  $Q = (
    K_1, 
    L_{\kappa_1}, 
    K_{\lambda_{\kappa_1}}, 
    %L_{\kappa_{\lambda_{\kappa_1}}},
    \dots
  )$
%
satisfies the conditions specified in section 1.44. 
It now follows from the corollary 2 that 
%
  \begin{align}  
    \tau_{K} 
    = 
      \tau_{K_\lambda} 
    = 
      \tau_{Q} 
    = 
      \tau_{L_\kappa} 
    = \tau_{L}.
  \end{align} 
%
So ends the proof
\end{proof}
% END
