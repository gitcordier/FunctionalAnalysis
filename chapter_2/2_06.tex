\textit{
Define the Fourier coefficient $\hat{f}(n)$ of a function %
%
  $f\in L^2(T)$ ($T$ is the unit circle) %
%
by
%
  \begin{align*}
  \hat{f}(n) 
    = 
  \frac{1}{2\pi} \int_{\minus \pi}^{\pi} %
  %
    f(e^{i\theta}) e^{\minus i n\theta} d\theta
  %
  \end{align*}
%
for all $n\in \Z$ (the integers). Put 
%
  \begin{align*}
    \Lambda_n f =\sum_{k=\minus n }^{n} \hat{f}(k).
  \end{align*}
%
Prove that %
  $\set{f \in {L}^2(T)}{\underset{n\infty}{\lim}\,\Lambda_n f \text{ exists}} $ %
%
is a dense subspace of $L^2(T)$ of the first category. %
}
%
\begin{proof}Let $(f,g)$ denote the inner product of $f$ and $g$ in $L^2(T)$, %
so that 
  \begin{align}
    \Lambda_n f = \sum_{k=\minus n}^n (f, e_k). 
  \end{align}
%
Moreover, as a finite sum of linear functional $f \mapsto (f, e_k)$, %
$\Lambda_n$ is continuous. 
Now remark that 
% 
  \begin{align}
    P \Def \text{span}\{ e_k: k\in \Z\} 
      \subset 
    F \Def \set{
        f \in {L}^2(T)
      }{
        \lim_{n\infty} \Lambda_n f \text{ exists}
    } \subset {L}^2(T), 
  \end{align}
%
then conclude that %
%
  \begin{align}
    {L}^2(T)= \overline{P} = \overline{ F},
  \end{align}
%
with the help of the Fejér's theorem\footnote{
%
  See 4.25 of \cite{BigRudin}.}. %
%
We have so established that $F$ is dense in $L^2(T)$. %
We now prove the second part. %
\newline\newline\noindent
%
Obviously, $\sum_{k\in \Z} \| e_k\|^2$ diverges. %
Thus, the theorem \citeresultFA{12.6} forces one of the series %
%
  \begin{align}
    \sum_{k\in \Z} (g, e_k) \quad (g \in L^2(T))
  \end{align}
%
to diverge. %
The subspace $F$ is then a proper subset of ${L}^2(T)$. %
It now follows from (b) of \citeresultFA{2.7} that %
%
  $F$ is not of the second category. %
%
\end{proof}









