%!TEX root = /Volumes/HD_2/Rudin/Rudin_DM.tex
\textit{
Put $K=[-1,1]$; define $\D_{K}$ as in section 1.46 
(with $\R$ in place of $\R^{n}$). 
Supose $\{f_{n}\}$ is a sequence of Lebesgue integrable functions such that 
%
  $\Lambda\phi 
    = 
  \underset{n \to \infty}{\lim} \int_{\minus 1}^1 f_{n}(t)\phi(t)dt$
%
exists for every $\phi\in\D_{K}$. 
Show that $\Lambda$ is a continuous linear functional on $\D_{K}$. 
Show that there is a positive integer $p$ and a number $M<\infty$ such that 
  \begin{align}
    \left\lvert 
      \int_{\minus 1}^1 f_n (t)\phi (t) dt\
    \right\rvert
    \leq 
    M \|D^{p} \phi \|_\infty
  \nonumber
  \end{align}
for all $n$.
For example, if $f_{n}(t)=n^{3}t$ on $[\minus 1/n, 1/n]$ and $0$ elsewhere, 
show that this can be done with $p=1$. 
Construct an example where it can be done with $p=2$ but not with $p=1$.}
%
%
\renewcommand{\labelenumi}{(\roman{enumi})}%
%
\newline\newline\noindent
We will also consider the case $p=0$. The following version of %
the mean value theorem will be of a great deal of help.
%
\paragraph{Lemma}\label{2.3 Lemma}
If 
%
  $\phi\in\D_{[a, b]}$, then %
%
%
  \begin{align}\label{2.3. Mean value inequality.}
    \| D^\alpha \phi \|_\infty 
      \leq 
    \| D^{p} \phi \|_\infty \left(\frac{\lambda}{2}\right)^{p-\alpha}
      \quad (\alpha = 0, 1, \dots, p) 
  \end{align}
%
at every order $\integers{p}$; %
where $\lambda$ is the length $\magnitude{b-a}$. %
%
\begin{proof}
Let $x_0$ be in %
% 
    $(a, b)$ .
% 
We first consider the case $x_0 \leq c = (a+b)/2$: %
The mean value theorem asserts that there exists %
%
  $x_{1}$ ($a <x_{1} < x_{0} $), 
%
such that 
%
  \begin{align}
    \phi(x_0) =\phi(x_0) - \phi(a)=  D\phi(x_{1})(x_{0} -a).
  \end{align}
%
Since every %
%
  $D^p\phi$ lies in $\D_{[a, b]}$, %
%
a straightforward proof by induction shows that there exists a partition % 
%
  $a < \cdots < x_p < \cdots < x_0 $ 
%
such that 
  \begin{align}
    \phi(x_0) & = D^0 \phi(x_0) \\ 
              & = D^1\phi(x_{1})(x_{0} - a ) \\
            % & = D^2\phi(x_{2})(x_{0} - a )(x_{1} - a)\\
              & = \cdots \nonumber\\
              & = D^p\phi(x_{p})(x_{0} - a)\cdots(x_{p-1}-a),
  \end{align}
%
for all $p$. %
More compactly, 
%
  \begin{align}
      D^\alpha \phi (x_0) = D^p\phi(x_p) \prod_{k= \alpha}^{p-1}(x_k - a);
      %(x_\alpha -a )\cdots (x_{p-1} -a).
  \end{align}
%
which yields, 
%
  \begin{align}\label{2.3. Mean Value (1).}
    \lvert D^\alpha \phi(x) \rvert
      \leq 
    \| D^{p} \phi \|_\infty 
    \left(\frac{\lambda}{2}\right)^{p-\alpha}
    \quad (x \in [a, c])
  \end{align}
%
The case $x_0 \geq c$ outputs a ``reversed'' result, with %
%
  $b > \cdots > x_p > \cdots > x_0$ %
%
and $x_k -b$ playing the role of $x_k-a$: %
So, %
%
  \begin{align}\label{2.3. Mean Value (2).}
    \lvert D^\alpha \phi(x) \rvert
      \leq 
    \| D^{p} \phi \|_\infty 
    \left(\frac{\lambda}{2}\right)^{p-\alpha} 
    \quad (x \in [c, b]).
  \end{align}
%
Finally, we combine %
%
  (\ref{2.3. Mean Value (1).}) with %
  (\ref{2.3. Mean Value (2).}) %
%
and so obtain %
%
  \begin{align}
    \| D^\alpha \phi \|_\infty 
      \leq 
    \| D^{p} \phi \|_\infty 
    \left(\frac{\lambda}{2}\right)^{p-\alpha}.
  \end{align}
%
\end{proof}
%: PROOF OF THE STATEMENT ----------------------------------------------------%
%/ FIRST PART ----------------------------------------------------------------%
%
\begin{proof}
We first consider %
%
  $C_0(\R)$ %
%
topologized by the supremum norm. %
Given a Lebesgue integrable function $u$, we put %
  \begin{align}
    \bra{u}\ket{\phi} 
      \Def 
    \int_\R u\phi  
      \quad(\phi \in C_0(\R)).
  \end{align}
%
The following inequalities % 
%
  \begin{align}\label{2.3  g  bounded operator (1).}
    %
    \left\lvert\bra{u}\ket{\phi}\right\rvert 
      \leq 
    \int_\R \left\lvert u\phi \right\rvert 
      \leq
    \norm{L^1}{u}
      \quad(\norm{\infty}{\phi} \leq 1)
  \end{align} 
%
imply that every linear functional 
%
  \begin{align}
    \bra{u}: C_0(\R) &\to \C \\
             \phi    &\mapsto \bra{u}\ket{\phi} \nonumber
  \end{align}
is bounded on the open unit ball. It is therefore continuous; %
%
  see \citeresultFA{1.18}. %
%
Conversely, ${u}$ can be identified with %
%
  $\bra{u}$, %
%
since $u$ is determined (a.e) by the integrals %
%
  $\bra{u}\ket{\phi}$. %
%
In the Banach spaces terminology, 
% 
  $u$ is then (identified with) a linear \textit{bounded}\hspace{2pt}\footnote{
  %
    see \citeresultFA{1.32, 4.1}
  } %
% %
operator $\bra{u}$, of norm %
%
  \begin{align}\label{2.3  g  bounded operator (2).}
    \sup\set{
      \magnitude{\bra{u}\ket{\phi}}
    }{ 
      \norm{\infty}{\phi} \leq 1
    } =  \norm{L^1}{u}.
  \end{align} 
%
Note that, in the latter equality, $\leq \norm{L^1}{u}$ comes from %
%
  (\ref{2.3  g  bounded operator (1).}), %
%
as the converse comes from the Stone-Weierstrass theorem\footnote{
%
  See 7.26 of \cite{BabyRudin}.
}. %
%
We now consider the special cases %
%
  $u = g_n$ %
  %($\counting{n}$), %
%
, where $g_n$ is %
%
  \begin{align}
    g_n : \R & \to \R\\
           x & \mapsto \begin{cases} 
  %%%%%%%%%%%%%%%%%%%%%%%%%%%%%%%%%%
  n^3 x & \left(x \in    \left[\minus \frac{1}{n}, \frac{1}{n}\right]\right)\\
  0     & \left(x \notin \left[\minus \frac{1}{n}, \frac{1}{n}\right]\right).
  %%%%%%%%%%%%%%%%%%%%%%%%%%%%%%%%%%
  \end{cases}\nonumber
  \end{align}
%
First, remark that $g_n(x) \tendsto{n}{\infty} 0$ ($x \in \R$), %
as the sequence $\singleton{g_n}$ fails to converge in $C_0(\R)$ %
%
  (since $g_n(1/n)= n^2 \geq 1$),  %
%
and also in $L^1$ %
%
  (since $\int_\R \lvert g_n \rvert = n^2 \longrightarrow{\infty}$)
. %
Nevertheless, we will show that %
%
  the $\bra{g_n}$ %
%
converge pointwise\footnote{
%
  See \citeresultFA{3.14} for a definition of the related topology. %
%
} on $\D_K$ %
\ie %
there exists a $\tau_K$-continuous linear form $\Lambda$ such that %
%
  \begin{align}\label{2.3. Convergence in D_K, not vague.}
    \bra{g_n}\ket{\phi} \tendsto{n}{\infty} \Lambda \phi, %
  \end{align}
%
where $\phi$ ranges over $\D_K$. %
%
We now prove (\ref{2.3  g  bounded operator (2).}) in the special cases %
%
  $u = g_n$. %
%
To do so, we fetch %
%
  $\phi^{+}_{1}, \dots, \phi^{+}_{j}, \dots,$ from $C_K^\infty(\R)$. %
%
More specifically, %
%
  \begin{enumerate}
    \item{
      $\phi^{+}_{j} = 1$ on $[e^{\minus j}, 1-e^{\minus j}]$; 
    }%
    \item{
      $\phi^{+}_{j} = 0$ on $\R\setminus [-1, 1]$;%
    }
    \item{
      $0 \leq \phi^{+}_{j} \leq 1$ on $\R$; %
    }
  \end{enumerate}
%
see \citeresultFA{[1.46]} for a possible construction of those $\phi^{+}_{j}$. 
%
Let %
%
  $\phi^{-}_{1}, \dots, \phi^{-}_{j}, \dots,$ %
%
mirror the $\phi^{+}_{j}$, in the sense that %
%
  $\phi^{-}_{j}(x) = \phi^{+}_{j}(\minus x)$, %
%
so that %
%
  \begin{enumerate}\addtocounter{enumi}{3}
    \item{
      $\phi_{j} \Def \phi^{+}_{j} - \phi^{-}_{j}$ is odd, as $g_n$ is;
    }
    \item{
      every $\phi_{j}$ is in $C_K^\infty(\R)$;
    }
    \item{
      The sequence $\singleton{\phi_j}$ converges (pointwise) to %
      $1_{[0, 1]} - 1_{[\minus 1, 0]}$, and $\lvert\phi_{j} \rvert \leq 1$. %
    }
  \end{enumerate}
%
Thus, with the help of the Lebesgue's convergence theorem, 
%
  \begin{align}\label{2.3. Norm in the dual equals norm L1 (1).}
      \langle{g_n}\lvert{\phi_j}\rangle
      = 
    2 \int_{0}^1 g_n(t) \phi^{+}_{j}(t) d t
      \tendsto{j}{\infty} 
    2 \int_{0}^1 g_n(t) d t
      = \norm{L^1}{g_n} = n.
  \end{align}
%
Finally, %
%
  \begin{align}\label{2.3. Norm in the dual equals norm L1 (2).}
    \norm{L^1}{g_n}
      \citegeq{\ref{2.3  g  bounded operator (2).}}
    \sup\set{
    \magnitude{\bra{g_n}\ket{\phi}}
    }{
      \norm{N}{\phi} \leq 1
    } 
      \citegeq{\ref{2.3. Norm in the dual equals norm L1 (1).}} 
    \norm{L^1}{g_n};
  \end{align}  %
%
which is the desired result. %
%
So, in terms of boundedness constants: %
Given $n$, there exists $C_{n} < \infty$ such that 
  %
    \begin{align}\label{2.3. Optimal bound.}
      \left\lvert 
        \bra{g_n}\ket{\phi} 
      \right\rvert 
        \leq 
      C_{n} \quad (\norm{\infty}{\phi} \leq 1); 
    \end{align}
  %
see (\ref{2.3  g  bounded operator (1).}). 
Furthermore, % 
%
  $\norm{L^1}{g_n}$ is actually the best, \ie lowest, possible $C_{n}$; see %
  %
    (\ref{2.3. Norm in the dual equals norm L1 (2).}). %
  %
But, on the other hand, %
%
  (\ref{2.3. Norm in the dual equals norm L1 (1).}) %
%
shows that there exists a subsequence %%
%
  $\singleton{\langle{g_n}\lvert{\phi_{\rho(n)}}\rangle}$ %
%
such that %
%
  $\langle{g_n}\lvert{\phi_{\rho(n)}}\rangle$ %
% 
is greater than, say, $\sqrt{n} $; as $\norm{\infty}{\phi_{\rho(n)}} = 1$. %
%
Consequently, there is no bound $M$ such that %%
%
  \begin{align}\label{2.3. Uniform bound for the supremum norm.}
    \left\lvert 
      \bra{g_n}\ket{\phi} 
    \right\rvert 
      \leq M
    \quad (\norm{\infty}{\phi} \leq 1; \counting{n}).
  \end{align}
%
In other words, the $g_n$ have no \textit{uniform bound} in ${L^1}$, %
  \ie %
the collection of all continous linear mappings %
%
  $\bra{g_n}$ %
%
is not equicontinous %
%
  (see discussion in \citeresultFA{2.6}). %
%
As a consequence, %
%
  the $\bra{g_n}$ %
%
do not converge pointwise (or ``vaguely'', in Radon measure context): %
A vague (\ie pointwise) convergence would be (by definition) %
%
  \begin{align}
    \bra{g_n}\ket{\phi} \tendsto{n}{\infty} \Lambda \phi %
    \quad (\phi \in C_0(\R)) %
  \end{align}
%
for some $\Lambda\in C_0(\R)^\ast$, which would make %
%
  (\ref{2.3. Uniform bound for the supremum norm.})
%
hold; see \citeresultFA{2.6, 2.8}. %
%
This by no means says that the %
%
  $\bra{g_n}$ %
%
do not converge pointwise, in a relevant space, to some $\Lambda$ (see %
%
  (\ref{2.3. Convergence in D_K, not vague.}). %
%
\newline\newline\noindent
% END

% SECOND PART -----------------------------------------------------------------
From now on, unless the contrary is explicitely stated, %
we asume that $\phi$ only denotes an element of $C_K^\infty(\R)$. 
Let $f_n$ be a Lebesgue integrable function such that %
%
  \begin{align}\label{2.3. Convergence on D_K (1).}
    \Lambda \phi = \underset{n \to \infty}{\lim} 
      \int_{K} f_n\phi 
      %
      \quad (\phi \in C_K^\infty(\R)).
  \end{align}
%
for some linear form $\Lambda$. %
%
Since $\phi$ vanishes outside $K$, we can suppose without loss of generality %
that the support of $f_n$ lies in $K$. So, %
%
  (\ref{2.3. Convergence on D_K (1).}) %
% 
can be restated as follows, 
%
  \begin{align}
    \Lambda \phi = \underset{n \to \infty}{\lim} 
      \bra{f_n}\ket{\phi} 
      %
      \quad (\phi \in C_K^\infty(\R)).
  \end{align}
%
Let $K_1, K_2, \dots, $ be compact sets that satisfy the conditions 
specified in \citeresultFA{1.44}. %
$\D_K$ is $C_K^\infty(\R)$ topologized by the related seminorms %
%
  $p_1, p_2, \dots$; see \citeresultFA{1.46, 6.2} and Exercise 1.16.
%
We know that $K\subset K_m$ for some index $m$ %
(see Lemma 2 of Exercise 1.16): From now on, we only consider the indices 
$N \geq m$, so that%
%
  \renewcommand{\labelenumi}{(\alph{enumi})}%
  %
  \begin{enumerate}
    \item{
      $p_N(\phi) = \norm{N}{\phi} \Def \max 
      \set{\lvert D^{\alpha}\phi(x) \rvert}{\alpha \leq N, x \in \R}$, %
      for $\phi \in \D_K$;
    }
    \item{
      The collection of the sets %
      $V_N = \set{ \phi \in \D_K}{ \norm{N}{\phi} < 2^{\minus N}}$ %
      is a (decreasing) local base of $\tau_K$, the subspace topology of $\D_K$; %
      see \citeresultFA{6.2} for a more complete discussion.
    }
  \end{enumerate}
  %
  \renewcommand{\labelenumi}{(\roman{enumi})}
  %
%
Let us specialize (\ref{2.3  g  bounded operator (1).}) with %
%
  $u=f_n$ and $ \phi \in V_m$ %
%
then conclude that $\bra{f_n}$ is bounded by $\norm{L^1}{f_n}$ on $V_m$: %
Every linear functional $\bra{f_n}$ is therefore $\tau_K$-continuous; see %
%
  \citeresultFA{1.18}. \newline\newline\noindent
%
To sum it up: %
%
  \begin{enumerate}
    \item{$\D_K$, equipped the topology $\tau_K$, is a Fréchet space %
      (see \citeresultFA{section 1.46})};
    %
    \item{Every linear functional $\bra{f_n}$ is continuous %
      with respect to this topology;}
    %
    \item{
      $\bra{f_n}\ket{\phi} \tendsto{n}{\infty} \Lambda \phi$ for all $\phi$, 
        \ie 
      $ \Lambda-\bra{f_n} \tendsto{n}{\infty}0$.
    }
  \end{enumerate}
%
With the help of \citeresultFA{[2.6] and [2.8]}, we conclude that %
%
  $\Lambda$ is continuous 
%
and that the sequence
%
  $\singleton{\bra{f_n}}$ %
%
is equicontinuous. 
%
So is the sequence %
%
  $\singleton{\Lambda - \bra{f_n}}$, %
%
since addition is continuous.
%
There so exists $i, j$ such that, for all $n$, 
%
  \begin{align}
   %
    \left\lvert 
      \Lambda \phi 
    \right\rvert
      < 
    1/2 & \quad \text{if }\phi\in V_i, \\
    %
    \left\lvert 
      \Lambda\phi - \bra{f_n}\ket{\phi} 
    \right\rvert  
      < 
    1/2 & \quad \text{if } \phi\in V_j.
    %
  \end{align}
%
Choose $p = \max\singleton{i, j}$, so that $V_p = V_i \cap V_j$: %
The latter inequalities imply that %
%
  \begin{align}
    %
    \left\lvert \bra{f_n}\ket{\phi} \right\rvert  & \leq 
      \magnitude{
        \Lambda\phi 
        - \bra{f_n }\ket{\phi} 
      }
      + 
      \magnitude{\Lambda \phi}
      %
     < 1 \
     %
       \quad\text{if } \phi\in V_p. %
    %
  \end{align}
%
Now remark that every $\psi =\psi[\mu, \phi]$, where %
%
  \begin{align}
     \psi[\mu, \phi] \Def 
      \begin{cases}
        ({1}/{\mu \cdot 2^{p} \norm{p}{\phi}}) \phi & (\phi \neq 0, \mu > 1)\\
        0                                          & (\phi   =  0, \mu > 1),  
      \end{cases}
  \end{align}
%
keeps in $V_p$. Finally, % 
it is clear that each below statement implies the following one.
%
  \begin{align}
    \lvert \langle {f_n}  | \psi \rangle \rvert &  
    < 1 \\
      %
    \left\lvert \bra{f_n}\ket{\phi} \right\rvert &  
    < 2^{p}  \norm{p}{\phi} \cdot \mu \\
    %
    \magnitude{\bra{f_n}\ket{\phi}} & 
    \leq 2^{p} \norm{p}{\phi} \\
    %
     \magnitude{\bra{f_n}\ket{\phi}} &
     \leq 2^{p} \singleton{
       \norm{\infty}{D^0 \phi} 
       + 
         \cdots 
      + 
        \norm{\infty}{D^p \phi}
      }.
    %
  \end{align}
  Finally, with the help of (\ref{2.3. Mean value inequality.}), 
  \begin{align}
    \label{2.3. Bound with p and M (theoritical).}
     \magnitude{\bra{f_n}\ket{\phi}} &
       \leq 
      2^{p} (p+1) \norm{\infty}{D^p \phi} .
  \end{align}
%
The first part is so proved, with \textit{some} $p$ and $ M= 2^{p}(p+1)$. %
\newline\newline\noindent
%
% END
We now come back to the special case $f_n = g_n$ (see the first part). %
From now on, $f_n(x) = n^3 x$ on $[\minus 1/n, 1/n]$, $0$ elsewhere. %
Actually, we will prove that 
%
  \renewcommand{\labelenumi}{(\alph{enumi})}
  %
  \begin{enumerate}
    \item{
      %
        $\Lambda\phi 
          = 
        \underset{n \to \infty}{\lim} \int_{\minus 1}^1 f_{n}(t)\phi(t)dt$ 
        %
      exists for every $\phi\in\D_{K}$;
    }
    \item{
      A \textit{uniform} bound %
      %
        $\lvert\bra{f_n}\ket{\phi}\rvert \leq M \| D^p \phi \|$ %
        ($\counting{n}$) %
      %
      exists for all those $f_n$, with $p=1$ as the smallest possible $p$.
    }
  \end{enumerate}
  %
  \renewcommand{\labelenumi}{(\roman{enumi})}
%
%
Bear in mind that %
%
  $K \subset K_m$ %
%
and shift the $K_N$'s indices by $\minus m$, so that %
%
  $K_{m+1}$ becomes $K_1$, $K_{m+2}$ becomes $K_2$, and so on. %
%
The resulting topology $\tau_K$ remains unchanged (see Exercise 1.16). %
%
We let $\phi$ keep running on $\D_K$ and so define %
%
  \begin{align}
    \Beta_{n}(\phi) 
      & 
        \Def  \max\set{\magnitude{\phi(x)}}{x \in [\minus 1/n, 1/n]}, \\
    %
    \Delta_{n}(\phi) 
      & 
        \Def  \max\set{\magnitude{\phi(x) - \phi(0)}}{x \in [\minus 1/n, 1/n]}. 
  \end{align}
%
The mean value asserts that 
%
  \begin{align}\label{2.3. Mean value inequality (concrete).}
    \magnitude{ \phi(1/n) - \phi(\minus 1/n)} 
      \leq 
    \Beta_{n}(\phi')\magnitude{1/n - ({\minus}1/n)} 
      = 
    \frac{2}{n} \Beta_{n}(\phi')
    %
  \end{align}.
%
An integration by parts shows that %
%
  \begin{align}\label{2.3. Integration by parts.}
    \bra{f_n}\ket{\phi}
      & =  
        \left[ \frac{n^3 t^2}{2} \phi(t) \right]^{1/n}_{\minus 1/n}
        - \frac{n^3}{2} \int_{\minus 1/n}^{1/n} t^{2} \phi'(t)dt\\ 
      %
      \label{2.3. Equality from integration by parts.}
      & =  
        \frac{n}{2}\left(\phi(1/n) - \phi(\minus 1/n)\right)
        - \frac{n^3}{2} \int_{\minus 1/n}^{1/n} t^{2} \phi'(t)dt.
      %
  \end{align}
Combine %
%
  (\ref{2.3. Mean value inequality (concrete).}) %
%
with %
%
  (\ref{2.3. Integration by parts.}) %
%
and so obtain %
%
  \begin{align}
    \magnitude{\bra{f_n}\ket{\phi}}
    & \leq  
      \frac{n}{2}\magnitude{\phi(1/n) - \phi(\minus 1/n)}
        + 
      \frac{n^3}{2}\int_{\minus1/n}^{1/n} t^{2}\lvert\phi'(t)\rvert dt
      \\
    %
      & \leq  
        \Beta_{n}(\phi') 
          + 
        \frac{n^3}{2} \Beta_{n}(\phi') \int_{\minus1/n}^{1/n} t^{2} dt \\
      %
      \label{2.3. Bound with p and M (M_n).}
      & \leq 
      \frac{4}{3} \Beta_{n}(\phi') \\
      \label{2.3. Bound with p and M (concrete).}
      & \leq 
      \frac{4}{3} \| \phi' \|_\infty;
      %
  \end{align}
%
Futhermore, %
%
  (\ref{2.3. Bound with p and M (M_n).}) %
% 
gives a hint about the convergence of $f_n$: Since %
%
  $\Beta_{n}(\phi')$ tends to $\phi'(0)$, %
%
we may expect that %
%
  $f_n$ tends to $\frac{4}{3}\phi'(0)$. %
%
This is actually true: A straightforward computation shows that %
%
  \begin{align}
    \bra{f_n}\ket{\phi} - \frac{4}{3}\phi'(0)
      \citeq{\ref{2.3. Equality from integration by parts.}} 
    \frac{\phi(1/n) - \phi(\minus 1/n)}{1/n - (\minus 1/n)}
    - \phi'(0)
    - \frac{n^3}{2} \int_{\minus 1/n}^{1/n} (\phi' - \phi'(0)) t^{2} dt.
  \end{align}
%
So, 
%
  \begin{align}
    \magnitude{\bra{f_n}\ket{\phi} -\frac{4}{3} \phi'(0)} 
      & \leq 
    \magnitude{
      \frac{\phi(1/n) - \phi(\minus 1/n)}{1/n - (\minus 1/n)}
      - \phi'(0)
    }
    + \frac{1}{3}\Delta_{n}(\phi') 
    % 
    \tendsto{n}{\infty} 0.
  \end{align}
%
We have just proved that 
%
  \begin{align}\label{2.3. (f_n) converges to Dirac (1).}
    \bra{f_n}\ket{\phi} \tendsto{n}{\infty} \Lambda\phi 
      = 
    \frac{4}{3}\phi'(0) 
    %
    \quad (\phi\in \mathscr{D}_K).
  \end{align}
%
In other words, 
%
  \begin{align}
    \bra{f_n} \tendsto{n}{\infty}\Lambda = \minus \frac{4}{3}\delta' ,
  \end{align}
%
where $\delta$ is the \textit{Dirac measure} and 
$\delta', \delta'', \dots, $ its derivatives ; see \citeresultFA{6.1 and 6.9}.
%
\newline\newline\noindent
It follows from the previous part that $\Lambda$ is $\tau_K$-continuous. %
Moreover, we have a bound %
%
  \begin{align}
    \lvert \bra{f_n}\ket{\phi} \rvert \leq M \| D^p \phi \|_\infty 
    \quad (\counting{n}), 
  \end{align}
%
with $p=1$ and $M= \frac{4}{3}$ (which is a concrete version of %
%
  (\ref{2.3. Bound with p and M (concrete).})%
%
). Furthermore, we have already spotlighted a sequence %
%
  \begin{align}
    \set{
      \langle{f_n} \lvert \phi_{\rho(n)} \rangle
    }{
      \| \phi_{\rho(n)} \|_\infty \leq 1, \counting{n}
    }
  \end{align}
%
that is not bounded. We then restate  %
%
  (\ref{2.3. Uniform bound for the supremum norm.}) %
%
in a more precise fashion: %
There is no constant $M$ such that %
%
  \begin{align}
    \lvert \langle f_n \lvert \phi \rangle \rvert \leq M 
    \quad (\phi\in C_K^\infty(\R), \| \phi \|_\infty \leq 1).
  \end{align}
%
The previous bound of $\bra{f_n}$ %
%
  - see (\ref{2.3. Bound with p and M (concrete).}), 
%
is therefore the best possible one, \ie % 
% 
  % 
    $p=1$ is the smallest possible $p$ and, given $p=1$,  %
  %
    $M=\frac{4}{3}$ is the smallest possible $M$ %
    (to see that, compare %
      (\ref{2.3. Bound with p and M (M_n).}) with 
      (\ref{2.3. (f_n) converges to Dirac (1).})%
    ); which is (b). %
%
\newline\newline\noindent
%
%\newline\newline\noindent
%END

$f_n$ is not differentiable on $\R$, so we give %
%
  $\bra{f_n}$ %
%
a \textit{derivative}
%
  \footnote{
    See \citeresultFA{6.1} 
    for a further discussion.
  }, %
as follows %
%
  \begin{align}
    \bra{f_n}': \D_K &\to \C\\
                    \phi &\mapsto \minus \bra{f_n}\ket{\phi'}; \nonumber
  \end{align}
%
It has been proved that every $\bra{f_n}$ is continuous. %
So is %
%
  \begin{align}
    D : \D_K & \to \D_K \\
        \phi & \mapsto \phi'; \nonumber
  \end{align}
see  Exercise 1.17. %
$\bra{f_n}'$ is therefore continuous. Now apply %
%
  (\ref{2.3. (f_n) converges to Dirac (1).}). %
%
with $\phi'$ and so obtain %
%
  \begin{align}
    \minus \bra{f_n}\ket{\phi'} &\tendsto{n}{\infty} \frac{4}{3} \phi''(0)
    \quad(\phi \in \D_K), \nonumber
  \end{align}
%
\ie %
%
  \begin{align}
    \bra{f_n}' &\tendsto{n}{\infty} \frac{4}{3} \delta''. 
  \end{align}
%
It follows from %
%
  (\ref{2.3. Bound with p and M (concrete).}) %
%
that, for some positive constant $C$, 
%
  \begin{align}
  \lvert \bra{f_n}\ket{\phi'} \rvert 
    \leq 
  C \| \phi''\|_\infty \quad (\counting{n}).
  \end{align}
%
It is therefore possible to uniformly bound 
%
  $\bra{f_n}'$ %
%
with respect to a norm %
%
  $\|D^p\cdot\|_\infty$, %
%
namely $\|D^2\cdot\|$. 
%
Then arises a question: %
%
  Is $2$ the smallest $p$? 
%
The answer is: Yes. 
%
To show this, we first assume, to reach a contradiction, that %
%
  there exists a positive constant $M$ such that
%
  \begin{align}
    \lvert \langle f_n \lvert \phi' \rangle \rvert 
      \leq 
    C \| \phi' \|_\infty 
    \quad (\counting{n}).
  \end{align}
%
Define %
%
  \begin{align}
    \Phi_{j}(x) = \int_{\minus 1}^x \phi_j.
  \end{align}
%
The oddness of $\phi_j$ forces %
% 
  $\Phi_{j}$ to vanish outside $[\minus 1, 1]$. 
%
$\phi_{j}$ is then in $\D_K$. So, under our assumption, 
%
  \begin{align}
    \lvert 
      \langle{f_n}\lvert \Phi_j' \rangle
    \rvert 
      \leq 
    M \| \Phi_{j}' \|
    \quad (\counting{n}); 
  \end{align}
%
which is %
%
  \begin{align}
    \lvert
      \langle{f_n}\lvert{\phi_j}\rangle
    \rvert 
      \leq 
    M 
    \quad (\counting{n}).
  \end{align}
%
We have thus reached a contradiction (again with the sequence %
%
  $\singleton{\langle f_n\lvert \phi_{\rho(n)} \rangle}$) %
%
and so conclude that there is no constant $M$ such that %
%
  \begin{align}
    \lvert 
      \langle \lvert f_n \phi'\rangle
    \rvert 
      \leq 
    M \| \phi' \|_\infty
    \quad (\counting{n}).
  \end{align}
%
Finally, assume, to reach a contradicton, that %
there exists a constant $M$ such that 
%
  \begin{align}
    \lvert \langle f_n \lvert \phi' \rangle \rvert 
      \leq 
    M \| \phi \|_\infty.
  \end{align}
%
The mean value theorem (see (\ref{2.3. Mean value inequality.})) asserts that %
%
  \begin{align}
    \lvert \langle f_n \lvert \phi' \rangle \rvert 
      \leq 
    M \| \phi \|_\infty \leq C \frac{\lambda}{2}\| \phi' \|; 
  \end{align}
%
which is, again, a desired contradiction. So ends the proof.


\end{proof}
%\renewcommand{\labelenumi}{\alph{enumi}.}%
\renewcommand{\labelenumi}{$(\textit{\alph{enumi}})$}%