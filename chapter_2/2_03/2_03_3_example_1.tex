We now come back to the special case $f_n = g_n$ (see the first part). %
From now on, $f_n(x) = n^3 x$ on $[\minus 1/n, 1/n]$, $0$ elsewhere. %
Actually, we will prove that 
%
  \renewcommand{\labelenumi}{(\alph{enumi})}
  %
  \begin{enumerate}
    \item{
      %
        $\Lambda\phi 
          = 
        \underset{n \to \infty}{\lim} \int_{\minus 1}^1 f_{n}(t)\phi(t)dt$ 
        %
      exists for every $\phi\in\D_{K}$;
    }
    \item{
      A \textit{uniform} bound %
      %
        $\lvert\bra{f_n}\ket{\phi}\rvert \leq M \| D^p \phi \|$ %
        ($\counting{n}$) %
      %
      exists for all those $f_n$, with $p=1$ as the smallest possible $p$.
    }
  \end{enumerate}
  %
  \renewcommand{\labelenumi}{(\roman{enumi})}
%
%
Bear in mind that %
%
  $K \subset K_m$ %
%
and shift the $K_N$'s indices by $\minus m$, so that %
%
  $K_{m+1}$ becomes $K_1$, $K_{m+2}$ becomes $K_2$, and so on. %
%
The resulting topology $\tau_K$ remains unchanged (see Exercise 1.16). %
%
We let $\phi$ keep running on $\D_K$ and so define %
%
  \begin{align}
    \Beta_{n}(\phi) 
      & 
        \Def  \max\set{\magnitude{\phi(x)}}{x \in [\minus 1/n, 1/n]}, \\
    %
    \Delta_{n}(\phi) 
      & 
        \Def  \max\set{\magnitude{\phi(x) - \phi(0)}}{x \in [\minus 1/n, 1/n]}. 
  \end{align}
%
The mean value asserts that 
%
  \begin{align}\label{2.3. Mean value inequality (concrete).}
    \magnitude{ \phi(1/n) - \phi(\minus 1/n)} 
      \leq 
    \Beta_{n}(\phi')\magnitude{1/n - ({\minus}1/n)} 
      = 
    \frac{2}{n} \Beta_{n}(\phi')
    %
  \end{align}.
%
An integration by parts shows that %
%
  \begin{align}\label{2.3. Integration by parts.}
    \bra{f_n}\ket{\phi}
      & =  
        \left[ \frac{n^3 t^2}{2} \phi(t) \right]^{1/n}_{\minus 1/n}
        - \frac{n^3}{2} \int_{\minus 1/n}^{1/n} t^{2} \phi'(t)dt\\ 
      %
      \label{2.3. Equality from integration by parts.}
      & =  
        \frac{n}{2}\left(\phi(1/n) - \phi(\minus 1/n)\right)
        - \frac{n^3}{2} \int_{\minus 1/n}^{1/n} t^{2} \phi'(t)dt.
      %
  \end{align}
Combine %
%
  (\ref{2.3. Mean value inequality (concrete).}) %
%
with %
%
  (\ref{2.3. Integration by parts.}) %
%
and so obtain %
%
  \begin{align}
    \magnitude{\bra{f_n}\ket{\phi}}
    & \leq  
      \frac{n}{2}\magnitude{\phi(1/n) - \phi(\minus 1/n)}
        + 
      \frac{n^3}{2}\int_{\minus1/n}^{1/n} t^{2}\lvert\phi'(t)\rvert dt
      \\
    %
      & \leq  
        \Beta_{n}(\phi') 
          + 
        \frac{n^3}{2} \Beta_{n}(\phi') \int_{\minus1/n}^{1/n} t^{2} dt \\
      %
      \label{2.3. Bound with p and M (M_n).}
      & \leq 
      \frac{4}{3} \Beta_{n}(\phi') \\
      \label{2.3. Bound with p and M (concrete).}
      & \leq 
      \frac{4}{3} \| \phi' \|_\infty;
      %
  \end{align}
%
Futhermore, %
%
  (\ref{2.3. Bound with p and M (M_n).}) %
% 
gives a hint about the convergence of $f_n$: Since %
%
  $\Beta_{n}(\phi')$ tends to $\phi'(0)$, %
%
we may expect that %
%
  $f_n$ tends to $\frac{4}{3}\phi'(0)$. %
%
This is actually true: A straightforward computation shows that %
%
  \begin{align}
    \bra{f_n}\ket{\phi} - \frac{4}{3}\phi'(0)
      \citeq{\ref{2.3. Equality from integration by parts.}} 
    \frac{\phi(1/n) - \phi(\minus 1/n)}{1/n - (\minus 1/n)}
    - \phi'(0)
    - \frac{n^3}{2} \int_{\minus 1/n}^{1/n} (\phi' - \phi'(0)) t^{2} dt.
  \end{align}
%
So, 
%
  \begin{align}
    \magnitude{\bra{f_n}\ket{\phi} -\frac{4}{3} \phi'(0)} 
      & \leq 
    \magnitude{
      \frac{\phi(1/n) - \phi(\minus 1/n)}{1/n - (\minus 1/n)}
      - \phi'(0)
    }
    + \frac{1}{3}\Delta_{n}(\phi') 
    % 
    \tendsto{n}{\infty} 0.
  \end{align}
%
We have just proved that 
%
  \begin{align}\label{2.3. (f_n) converges to Dirac (1).}
    \bra{f_n}\ket{\phi} \tendsto{n}{\infty} \Lambda\phi 
      = 
    \frac{4}{3}\phi'(0) 
    %
    \quad (\phi\in \mathscr{D}_K).
  \end{align}
%
In other words, 
%
  \begin{align}
    \bra{f_n} \tendsto{n}{\infty}\Lambda = \minus \frac{4}{3}\delta' ,
  \end{align}
%
where $\delta$ is the \textit{Dirac measure} and 
$\delta', \delta'', \dots, $ its derivatives ; see \citeresultFA{6.1 and 6.9}.
%
\newline\newline\noindent
It follows from the previous part that $\Lambda$ is $\tau_K$-continuous. %
Moreover, we have a bound %
%
  \begin{align}
    \lvert \bra{f_n}\ket{\phi} \rvert \leq M \| D^p \phi \|_\infty 
    \quad (\counting{n}), 
  \end{align}
%
with $p=1$ and $M= \frac{4}{3}$ (which is a concrete version of %
%
  (\ref{2.3. Bound with p and M (concrete).})%
%
). Furthermore, we have already spotlighted a sequence %
%
  \begin{align}
    \set{
      \langle{f_n} \lvert \phi_{\rho(n)} \rangle
    }{
      \| \phi_{\rho(n)} \|_\infty \leq 1, \counting{n}
    }
  \end{align}
%
that is not bounded. We then restate  %
%
  (\ref{2.3. Uniform bound for the supremum norm.}) %
%
in a more precise fashion: %
There is no constant $M$ such that %
%
  \begin{align}
    \lvert \langle f_n \lvert \phi \rangle \rvert \leq M 
    \quad (\phi\in C_K^\infty(\R), \| \phi \|_\infty \leq 1).
  \end{align}
%
The previous bound of $\bra{f_n}$ %
%
  - see (\ref{2.3. Bound with p and M (concrete).}), 
%
is therefore the best possible one, \ie % 
% 
  % 
    $p=1$ is the smallest possible $p$ and, given $p=1$,  %
  %
    $M=\frac{4}{3}$ is the smallest possible $M$ %
    (to see that, compare %
      (\ref{2.3. Bound with p and M (M_n).}) with 
      (\ref{2.3. (f_n) converges to Dirac (1).})%
    ); which is (b). %
%
\newline\newline\noindent
%
%\newline\newline\noindent
%END
