We first consider %
%
  $C_0(\R)$ %
%
topologized by the supremum norm. %
Given a Lebesgue integrable function $u$, we put %
  \begin{align}
    \bra{u}\ket{\phi} 
      \Def 
    \int_\R u\phi  
      \quad(\phi \in C_0(\R)).
  \end{align}
%
The following inequalities % 
%
  \begin{align}\label{2.3  g  bounded operator (1).}
    %
    \left\lvert\bra{u}\ket{\phi}\right\rvert 
      \leq 
    \int_\R \left\lvert u\phi \right\rvert 
      \leq
    \norm{L^1}{u}
      \quad(\norm{\infty}{\phi} \leq 1)
  \end{align} 
%
imply that every linear functional 
%
  \begin{align}
    \bra{u}: C_0(\R) &\to \C \\
             \phi    &\mapsto \bra{u}\ket{\phi} \nonumber
  \end{align}
%
is bounded on the open unit ball. It is therefore continuous; %
%
  see \citeresultFA{1.18}. %
%
Conversely, ${u}$ can be identified with %
%
  $\bra{u}$, %
%
since $u$ is determined (a.e) by the integrals %
%
  $\bra{u}\ket{\phi}$. %
%
In the Banach spaces terminology, 
% 
  $u$ is then (identified with) a linear \textit{bounded}\hspace{2pt}\footnote{
  %
    see \citeresultFA{1.32, 4.1}
  } %
% %
operator $\bra{u}$, of norm %
%
  \begin{align}\label{2.3  g  bounded operator (2).}
    \sup\set{
      \magnitude{\bra{u}\ket{\phi}}
    }{ 
      \norm{\infty}{\phi} = 1
    } 
    =  \norm{L^1}{u}.
  \end{align} 
%
Note that, in the latter equality, $\leq \norm{L^1}{u}$ comes from %
%
  (\ref{2.3  g  bounded operator (1).}), %
%
as the converse comes from the Stone-Weierstrass theorem\footnote{
%
  See 7.26 of \cite{BabyRudin}.
}. %
%
We now consider the special cases %
%
  $u = g_n$ %
  %($\counting{n}$), %
%
, where $g_n$ is %
%
  \begin{align}
    g_n : \R & \to \R\\
           x & \mapsto \begin{cases} 
  %%%%%%%%%%%%%%%%%%%%%%%%%%%%%%%%%%
  n^3 x & \left(x \in    \left[\minus \frac{1}{n}, \frac{1}{n}\right]\right)\\
  0     & \left(x \notin \left[\minus \frac{1}{n}, \frac{1}{n}\right]\right).
  %%%%%%%%%%%%%%%%%%%%%%%%%%%%%%%%%%
  \end{cases}\nonumber
  \end{align}
%
First, remark that $g_n(x)$ $\longrightarrow 0$, %
as the sequence $\singleton{g_n}$ fails to converge in $C_0(\R)$ %
%
  (since $g_n(1/n)= n^2 \geq 1$),  %
%
and also in $L^1$ %
%
  (since $\int_\R \lvert g_n \rvert = n^2 \longrightarrow{\infty}$)
. %
Nevertheless, we will show that %
%
  the $\bra{g_n}$ %
%
converge pointwise\footnote{
%
  See \citeresultFA{3.14} for a definition of the related topology. %
%
} on $\D_K$ %
\ie %
there exists a $\tau_K$-continuous linear form $\Lambda$ such that %
%
  \begin{align}\label{2.3. Convergence in D_K, not vague.}
    \bra{g_n}\ket{\phi} \tendsto{n}{\infty} \Lambda \phi, %
  \end{align}
%
where $\phi$ ranges over $\D_K$. %
%
We now prove (\ref{2.3  g  bounded operator (2).}) in the special cases %
%
  $u = g_n$. %
%
To do so, we fetch %
%
  $\phi^{+}_{1}, \dots, \phi^{+}_{j}, \dots,$ from $C_K^\infty(\R)$. %
%
More specifically, %
%
  \begin{enumerate}
    \item{
      $\phi^{+}_{j} = 1$ on $[e^{\minus j}, 1-e^{\minus j}]$; 
    }%
    \item{
      $\phi^{+}_{j} = 0$ on $\R\setminus [-1, 1]$;%
    }
    \item{
      $0 \leq \phi^{+}_{j} \leq 1$ on $\R$; %
    }
  \end{enumerate}
%
see \citeresultFA{[1.46]} for a possible construction of those $\phi^{+}_{j}$. 
%
Let %
%
  $\phi^{-}_{1}, \dots, \phi^{-}_{j}, \dots,$ %
%
mirror the $\phi^{+}_{j}$, in the sense that %
%
  $\phi^{-}_{j}(x) = \phi^{+}_{j}(\minus x)$, %
%
so that %
%
  \begin{enumerate}\addtocounter{enumi}{3}
    \item{
      $\phi_{j} \Def \phi^{+}_{j} - \phi^{-}_{j}$ is odd, as $g_n$ is;
    }
    \item{
      every $\phi_{j}$ is in $C_K^\infty(\R)$;
    }
    \item{
      The sequence $\singleton{\phi_j}$ converges (pointwise) to %
      $1_{[0, 1]} - 1_{[\minus 1, 0]}$, and $\norm{\infty}{\phi_{j}} = 1$. %
    }
  \end{enumerate}
%
Thus, with the help of the Lebesgue's convergence theorem, 
%
  \begin{align}\label{2.3. Norm in the dual equals norm L1 (1).}
      \langle{g_n}\lvert{\phi_j}\rangle
      = 
    2 \int_{0}^1 g_n(t) \phi^{+}_{j}(t) d t
      \tendsto{j}{\infty} 
    2 \int_{0}^1 g_n(t) d t
      = \norm{L^1}{g_n} = n.
  \end{align}
%
Finally, %
%
  \begin{align}\label{2.3. Norm in the dual equals norm L1 (2).}
    \norm{L^1}{g_n}
      \citeleq{\ref{2.3. Norm in the dual equals norm L1 (1).}} 
    \sup\set{
    \magnitude{\bra{g_n}\ket{\phi}}
    }{
      \norm{\infty}{\phi} = 1
    }
      \citeleq{\ref{2.3  g  bounded operator (2).}}
    \norm{L^1}{g_n};
  \end{align}  %
%
which is the desired result. %
%
So, in terms of boundedness constants: %
Given $n$, there exists $C_{n} < \infty$ such that 
  %
    \begin{align}\label{2.3. Optimal bound.}
      \left\lvert 
        \bra{g_n}\ket{\phi} 
      \right\rvert 
        \leq 
      C_{n} \quad (\norm{\infty}{\phi} = 1); 
    \end{align}
  %
see (\ref{2.3  g  bounded operator (1).}). 
Furthermore, % 
%
  $\norm{L^1}{g_n}$ is actually the best, \ie lowest, possible $C_{n}$; see %
  %
    (\ref{2.3. Norm in the dual equals norm L1 (2).}). %
  %
But, on the other hand, %
%
  (\ref{2.3. Norm in the dual equals norm L1 (1).}) %
%
shows that there exists a subsequence %%
%
  $\singleton{\langle{g_n}\lvert{\phi_{\rho(n)}}\rangle}$ %
%
such that %
%
  $\langle{g_n}\lvert{\phi_{\rho(n)}}\rangle$ %
% 
is greater than, say, $n -0.01$, as $\norm{\infty}{\phi_{\rho(n)}} = 1$. %
%
Consequently, there is no bound $M$ such that %%
%
  \begin{align}\label{2.3. Uniform bound for the supremum norm.}
    \left\lvert 
      \bra{g_n}\ket{\phi} 
    \right\rvert 
      \leq M
    \quad (\norm{\infty}{\phi} = 1; \counting{n}).
  \end{align}
%
In other words, the $g_n$ have no \textit{uniform bound} in ${L^1}$, %
  \ie %
the collection of all continous linear mappings %
%
  $\bra{g_n}$ %
%
is not equicontinous %
%
  (see discussion in \citeresultFA{2.6}). %
%
As a consequence, %
%
  the $\bra{g_n}$ %
%
do not converge pointwise (or ``vaguely'', in Radon measure context): %
A vague (\ie pointwise) convergence would be (by definition) %
%
  \begin{align}
    \bra{g_n}\ket{\phi} \tendsto{n}{\infty} \Lambda \phi %
    \quad (\phi \in C_0(\R)) %
  \end{align}
%
for some $\Lambda\in C_0(\R)^\ast$, which would make %
%
  (\ref{2.3. Uniform bound for the supremum norm.})
%
hold; see \citeresultFA{2.6, 2.8}. %
%
This by no means says that the %
%
  $\bra{g_n}$ %
%
do not converge pointwise, in a relevant space, to some $\Lambda$ (see %
%
  (\ref{2.3. Convergence in D_K, not vague.}). %
%
\newline\newline\noindent
% END
