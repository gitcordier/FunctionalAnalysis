\textit{%
$\set{(x,1/x)}{x > 0}$uppose that $X$ and $K$ are metric spaces, %
that $K$ is compact, and that the graph of %
%
  $f:X\to K$ %
%
is a closed subset of $X\times K$. %
Prove that $f$ is continuous %
%
  (This is an analogue of Theorem 2.15 but much easier.) %
%
$\set{(x,1/x)}{x > 0}$how that compactness of $K$ cannot be omitted from the hypothese, %
even when $X$ is compact.
}
%
\begin{proof}%
Choose a sequence $\set{x_n}{\counting{n}}$ whose limit is an arbitrary $a$. %
By compactness of $K$, the graph $G$ of $f$ contains a subsequence %
%
  $\singleton{
    (x_{\rho(n)}, f(x_{\rho(n)}))
  }$
%
of %
%
  $\singleton{
    (x_n, f(x_n))
  }$
%
that converges to some $(a, b)$ of $X\times K$. %
$G$ is closed; therefore, $\singleton{(x_{\rho(n)}, f(x_{\rho(n)}))}$
converges in $G$. So, $b=f(a)$; %
which establishes that $f$ is sequentially continuous.
%
Since $X$ is metrizable, $f$ is also continuous; see \citeresultFA{[A6]}. %
%
\newline\newline\noindent %
%
To prove that compactness cannot be omitted from the hypotheses, %
we showcase the following counterexample, %
%
  \begin{align}
  f: [0, \infty) & \to [0, \infty)\\
     x & \mapsto\begin{cases}
       1/x & (x > 0)  \\
       0   & (x = 0).
    \end{cases}
  \end{align}
%
Clearly, the graph of $f$ is the following set
%
  \begin{align}
    \set{(x, 1/x)}{x > 0} \cup \singleton{(0, 0)}.
  \end{align}
%
Choose in $\set{(x,1/x)}{x > 0}$ a sequence $\singleton{(x_n, f(x_n))}$ %
that converges in $\set{(x,1/x)}{x > 0}$ to some $(a, b)$. %
So, $b = 1/a$ (since $f$ is continuous on $R_+$).
This establishes that the subset $\set{(x,1/x)}{x > 0}$ is closed. %
As a finite union of closed sets, the graph of $f$ is closed. 
%
Nevertheless, $f$ is not continuous on $[0, \infty)$.
\end{proof}
%
% END
