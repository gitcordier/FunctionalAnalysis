\textit{%
Suppose that $X$ and $K$ are metric spaces, %
that $K$ is compact, and that the graph of %
%
  $f:X\to K$ %
%
is a closed subset of $X\times K$. %
Prove that $f$ is continuous %
%
  (This is an analogue of Theorem 2.15 but much easier.) %
%
Show that compactness of $K$ cannot be omitted from the hypothese, %
even when $X$ is compact.
}
%
\begin{proof}%
Choose a sequence $\set{x_n}{\counting{n}}$ whose limit is an arbitrary $a$. %
By compactness of $K$, the graph $G$ of $f$ contains a subsequence %
%
  $\singleton{
    (x_{\rho(n)}, f(x_{\rho(n)}))
  }$
%
of %
%
  $\singleton{
    (x_n, f(x_n))
  }$
%
that converges to some $(a, b)$ of $X\times K$. %
$G$ is closed; therefore, $\singleton{(x_{\rho(n)}, f(x_{\rho(n)}))}$
converges in $G$. So, $b=f(a)$; %
which establishes that $f$ is sequentially continuous.
%
Since $X$ is metrizable, $f$ is also continuous; see \citeresultFA{[A6]}. %
So ends the proof.
%
\newline\newline\noindent %
%
To show that compactness cannot be omitted from the hypotheses, %
we showcase the following counterexample, %
%
  \begin{align}
  f: [0, \infty) & \to [0, \infty)\\
     x & \mapsto\begin{cases}
       1/x & (x > 0)  \\\nonumber
       0   & (x = 0).
    \end{cases}
  \end{align}
%
Clearly, $f$ has a discontinuity at $0$. %
Nevertheless the graph $G$ of $f$ is closed. %
To see that, first remark that %
%
  \begin{align}
    G = \set{(x, 1/x)}{x > 0} \cup \singleton{(0, 0)}.
  \end{align}
%
Next, let %
%
  $\singleton{(x_n, 1/x_n)}$ %
%
be a sequence in %
%
  $G_+ = \set{(x, 1/x)}{x > 0}$ %
%
that converges to $(a, b)$. %
To be more specific: %
$a=0$ contradicts the boundedness of $\singleton{(x_n,1/x_n)}$: %
$a$ is necessarily positive and $b = 1/a$, %
since $x\mapsto 1/x$ is continuous on $R_+$.
This establishes that $(a, b) \in G_+$, hence the closedness $G_+$. %
Finally, we conclude that $G$ is closed, as a finite union of closed sets.
%
\end{proof}
%
% END
