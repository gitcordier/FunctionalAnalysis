\textit{%
Prove that a bilinear mapping is continuous %
if it is continuous at the origin $(0, 0)$.
}
\begin{proof}
Let $(X_1, X_2, Z)$ be topological spaces %
and $B$ a bilinear mapping %
%
  \begin{align}
    B: X_1 \times X_2 \to Z 
  \end{align}
%
From now on, $x=(x_1, x_2)$ denotes an arbitrary element of %
%
  $X_1\times X_2$. %
%
We henceforth assume that $B$ is continuous at the origin %
%
  $(0, 0)$ of $X_1\times X_2$, \ie  %
%
given an arbitrary balanced open subset $W$ of $Z$, %
there exists in $X_i$ ($i=1, 2$) a balanced open subset $U_i$ such that %
%
  \begin{align}
    B(U_1 \times U_2) \subset W .
  \end{align}
%
Let $\nu_i(x)$ denote any scalar that is greater than %
%
  $\mu_i(x_i) = \inf\set{r> 0}{x_i \in r \cdot U_i}$. %
%
So, 
%
  \begin{align}
    B(x_1, x_2) 
    %& = B(
    %    \nu(x)_1 \nu(x)_1^{\minus 1} \cdot x_1, 
    %    \nu(x)_2 \nu(x)_2^{\minus 1} \cdot x_2) \\
      & = 
        \nu_1(x)\nu_2(x) 
            \cdot 
        B\left(
          \nu_1(x)^{\minus 1} x_1 , 
          \nu_2(x)^{\minus 1} x_2
        \right) \\
      & \in 
    \nu_1(x) \nu_2(x) \cdot B(U_1 \times U_2) \\
    \label{2_10. Bound.}
      & \subset 
    \nu_1(x) \nu_2(x) \cdot W.
  \end{align}
%% 
Now pick $p=(p_1, p_2)$ in $X_1\times X_2$: %
%
It directly follows from (\ref{2_10. Bound.}) that %
%
  \begin{align}
    B(p_1, p_2) - B(x_1, x_2) 
      = B(p_1, p_2 - x_2) + B(p_1 -x_1, x_2 -p_2) + B(p_1-x_1,p_2) \\
      \label{2.10. In multiple of W.}
       \in  
         \nu_1(p)\nu_2(p-x)\cdot W + 
         \nu_1(p-x)\nu_2(x-p) \cdot W  + 
         \nu_1(p-x)\nu_2(p) \cdot W.
  \end{align}
%
Let us henceforth assume that %
%
  \begin{align}
    p_i -x_i \in [\mu_1(p) + \mu_2(p) + 2]^{\minus 1} \cdot U_i;   
  \end{align}
%
which yields %
%
  \begin{align}
    \mu_i(p_i -x_i) 
      \leq 
   [\mu_1(p) + \mu_2(p) + 2]^{\minus 1}.
  \end{align}
%
Finally, combine the special case %
%
  \begin{align}
    \nu_i(p -x) & =   [\mu_1(p) + \mu_2(p) + 1]^{\minus 1}, \\ 
    \nu_i(p)    & =\,\,\mu_1(p) + \mu_2(p) + 1
  \end{align}
%
with (\ref{2.10. In multiple of W.}) and so obtain %%
%
  \begin{align}
    B(p_1, p_2) - B(x_1, x_2)  \in W + W+ W.
  \end{align}
%
$W$ being arbitrary, %
we have so established the continuousness of $B$ at $(p_1, p_2)$. %
Since $(p_1, p_2)$ is also arbitrary, the proof is complete.
\end{proof}
%
%END