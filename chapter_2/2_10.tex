\textit{%
Prove that a bilinear mapping is continuous %
if it is continuous at the origin $(0, 0)$.
}
\begin{proof}
Let $(X_1, X_2, Z)$ be topological spaces %
and $B$ a bilinear mapping %
%
  \begin{align}
    B: X_1 \times X_2 \to Z 
  \end{align}
%
From now on, $x_i$ ($i=1, 2$) denotes an arbitrary element of %
%
  $X_i$. %
%
We henceforth assume that $B$ is continuous at the origin %
%
  $(0, 0)$ of $X_1\times X_2$, \ie  %
%
given an arbitrary balanced open subset $W$ of $Z$, %
there exists in $X_i$ a balanced open subset $U_i$ such that %
%
  \begin{align}
    B(U_1 \times U_2) \subset W .
  \end{align}
%
Let $\ceil{x_i}$ denote any scalar that is greater than %
%
  $\mu_i(x_i) = \inf\set{\alpha > 0}{x_i \in \alpha \cdot U_i}$. %
%
So, 
%
  \begin{align}
    B(x_1, x_2) 
    %& = B(
    %    \ceil{x_1} \ceil{x_1}^{\minus 1} \cdot x_1, 
    %    \ceil{x_2} \ceil{x_2}^{\minus 1} \cdot x_2) \\
      & = 
        \ceil{x_1}\ceil{x_2} 
            \cdot 
        B\left(
          \ceil{x_1}^{\minus 1}x , 
          \ceil{x_2}^{\minus 1} x_2
        \right) \\
      & \in 
    \ceil{x_1} \ceil{x_2} \cdot B(U_1 \times U_2) \\
    \label{2_10. Bound.}
      & \subset 
    \ceil{x_1} \ceil{x_2} \cdot W.
  \end{align}
%% 
To prove that $B$ is continuous, %
we pick $(p_1, p_2)$ an arbitrary element of $X_1\times X_2$. %
From now on, %
%
  $\ceil{p_i} < \mu_i(p_i) + 1$. 
%
It directly follows from (\ref{2_10. Bound.}) that %
%
  \begin{align}
    B(p_1, p_2) - B(x_1, x_2) 
      = B(p_1, p_2 - x_2) + B(p_1 -x_1, x_2 -p_2) + B(p_1-x_1,p_2) \\
      \label{2.10. In multiple of W.}
       \in  
         \ceil{p_1}\ceil{p_2 - x_2}\cdot W + 
         \ceil{p_1 - x_1}\ceil{x_2-p_2} \cdot W  + 
         \ceil{p_1-x_1}\ceil{p_2} \cdot W 
  \end{align}
%
We now request that $x_i$ keeps in a specific neighborhood of $p_i$, namely %
%
\begin{align}
  x_i -p_i \in \frac{1}{\mu_1(p_1) + \mu_2(p_2) + 2} \cdot U_i,   %
  \end{align}
%
so that 
%
\begin{align}
  \mu_i(x_i -p_i) \leq \frac{1}{\mu_1(p_1) + \mu_2(p_2) + 2}. %
  \end{align}
%
From now on, $\ceil{x_i - p_i}$ is bounded as follows  %
%
  \begin{align}
    \mu_i(x_i - p_i)< \ceil{x_i - p_i} < \frac{1}{\mu_1(p_1) + \mu_2(p_2) + 1}.
  \end{align}
%
Finally, within such boundaries, (\ref{2.10. In multiple of W.}) yields %
%
  \begin{align}
    B(p_1, p_2) - B(x_1, x_2) \in W + W + W 
    %\quad (x_i - p_i \in \frac{1}{\mu_1(p_1) + \mu_2(p_2) + 2} \cdot U_i)
  \end{align}
%
$W$ being arbitrary, %
we have so established the continuousness of $B$ at $(p_1, p_2)$. %
Since $(p_1, p_2)$ is also arbitrary, the proof is complete.
\end{proof}
%
%END