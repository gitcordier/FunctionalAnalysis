% % % % % % % % % % % % % % % % % % % % % % % % % % % % % % % % % % % % % % % % % % % % % % % % % % % % % % % % % % % % % % % %
% FunctionalAnalysis 
% 2_10.tex
% 
% encoding: UTF-8 
% EOL: LF
%
% format: LaTeX
% indent: spaces (2)
% width: 127
% % % % % % % % % % % % % % % % % % % % % % % % % % % % % % % % % % % % % % % % % % % % % % % % % % % % % % % % % % % % % % % %
\textit{%
Prove that a bilinear mapping is continuous if it is continuous at the origin $(0, 0)$.
}
%
\begin{proof}
Let $B: X_1 \times X_2 \to Z$ be a bilinear mapping that is continuous at $(0,0)$, where $X_i$\;\,$\bigl(i\in \{1, 2\}\bigr)$ %
and $Z$ are topological vector spaces. This implies that, for any balanced open $W$, $X_i$ contains a balanced open $U_i$ %
such that %
%
\begin{align}
  B(U_1 \times U_2) \subset W. 
\end{align}
%
Let $a_i$ be in $X_i$. Therefore, $a_i \in r_i U_i$ for some positive $r_i$. We now choose $b_i$ in %
$a_i + (1 + r_1  + r_2)^{\minus 1} U_i$. Hence 
%
\begin{align}
  B(b_1, b_2) - B(a_1,  a_2) & = B(b_1 - a_1, b_2) + B(a_1, b_2) - B(a_1, a_2)\\
  & = B(b_1 - a_1, b_2) + B(a_1, b_2 - a_2)\\
  & = \underbrace{B(b_1 - a_1, b_2 - a_2)}_{\in \frac{1}{(1 + r_1  + r_2)^2}W} 
    + \underbrace{B(b_1 - a_1, a_2)}_{\in \frac{r_2}{1 + r_1 + r_2}W}
    + \underbrace{B(a_1, b_2 - a_2)}_{\in \frac{r_1}{1 + r_1 + r_2}W}\\
  & \in W + W  + W.
\end{align}
%
We conclude that $B$ is continuous at every $(a_1, a_2)$, since $W$ was arbitrary. 
\end{proof}
%END