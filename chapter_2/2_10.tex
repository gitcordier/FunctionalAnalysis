% % % % % % % % % % % % % % % % % % % % % % % % % % % % % % % % % % % % % % % % % % % % % % % % % % % % % % % % % % % % % % % %
% FunctionalAnalysis 
% 2_10.tex
% 
% encoding: UTF-8 
% EOL: LF
%
% format: LaTeX
% indent: spaces (2)
% width: 127
% % % % % % % % % % % % % % % % % % % % % % % % % % % % % % % % % % % % % % % % % % % % % % % % % % % % % % % % % % % % % % % %
\textit{%
Prove that a bilinear mapping is continuous if it is continuous at the origin $(0, 0)$.
}
%
\begin{proof}
Let $B: X_1 \times X_2 \to Z$ be a bilinear mapping that is continuous at $(0,0)$, where $X_i$ and $Z$ are topological vector %
spaces. This implies that, for any balanced open set $W$, $X_i$ contains a balanced open set $U_i$ such that %
%
\begin{align}
  B(U_1 \times U_2) \subset W. 
\end{align}
%
Pick $a_i$ from $ X_i$, and then choose $r_i > 1/2$ large enough so that $a_i \in r_i U_i$. The vector $b_i - a_i$ is now %
restricted to $(r_1 + r_2)^{\minus 1} U_i$. Hence %
%
\begin{align}
  B(b_1, b_2) - B(a_1,  a_2) & = B(b_1 - a_1, b_2) + B(a_1, b_2) - B(a_1, a_2)\\
  & = B(b_1 - a_1, b_2) + B(a_1, b_2 - a_2)\\
  & = B(b_1 - a_1, b_2 - a_2) + B(b_1 - a_1, a_2) + B(a_1, b_2 - a_2)\\
  & \subset \frac{1}{(r_1 + r_2)^2}W  + \frac{r_2}{r_1 + r_2}W + \frac{r_1}{r_1 + r_2}W\\
  & \subset W + W  + W 
\end{align}
%
We conclude that $B$ is continuous at every $(a_1, a_2)$, since $W$ was arbitrary. This concludes the proof. 
\end{proof}
%END