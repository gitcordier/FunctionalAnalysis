\textit{%
Prove that a bilinear mapping is continuous %
if it is continuous at the origin $(0, 0)$.
}
\begin{proof}
Let $(X_1, X_2, Z)$ be topological spaces %
and $B$ a bilinear mapping %
%
\begin{align}
  B: X_1 \times X_2 \to Z. 
\end{align}
%
From now on, $x=(x_1, x_2)$ denotes an arbitrary element of %
%
  $X_1\times X_2$. %
%
We henceforth assume that $B$ is continuous at the origin %
%
  $(0, 0)$ of $X_1\times X_2$, \ie  %
%
given an arbitrary \textbf{balanced} open subset $W$ of $Z$, %
there exists in $X_i$ ($i=1, 2$) a \textbf{balanced} open subset $U_i$ %
such that %
%
\begin{align}
  B(U_1 \times U_2) \subset W .
\end{align}
%
In such context, %
%
  $\lambda_i(x) $ is chosen greater than %
  $\mu_i(x_i)=\inf\set{r> 0}{x_i \in r \cdot U_i}$; %
%
see [1.33] of \cite{FA} for further reading about the %
\textit{Minkowski functionals} $\mu$. %
%
In other words, $x_i$ lies in $\lambda_i(x) U_i$, since $U_i$ is balanced. %
%
Thus, % 
%
\begin{align}
  B(x_1, x_2) 
    & = 
      \lambda_1(x)\lambda_2(x) \cdot B(
        x_1/ \lambda_1(x), 
        x_2/ \lambda_2(x)) \\
    & \in 
  \lambda_1(x) \lambda_2(x) \cdot B(U_1 \times U_2) \\
    & \subset 
  \lambda_1(x) \lambda_2(x) \cdot W. \label{2_10. Bound.}
\end{align}
%
Pick $p=(p_1, p_2)$ in $X_1\times X_2$,  %
and let $ q=(q_1, q_2)$ range over $X \times Y$, as a first step: %
It directly follows from (\ref{2_10. Bound.}) that %
%
\begin{align}
  B(p) - B(q) &= B(p_1, p_2 - q_2) + B(p_1, q_2) - B(q_1, q_2)\\
  &= B(p_1, p_2 - q_2) + B(p_1-q_1, q_2)  \\
  &= B(p_1, p_2 - q_2) + B(p_1 -q_1, q_2 -p_2) + B(p_1-q_1,p_2) \\
  &\in  
     \lambda_1(p)   \lambda_2(p-q) W + 
     \lambda_1(p-q) \lambda_2(q-p) W + 
     \lambda_1(p-q) \lambda_2(p)   W.\label{2.10. In multiple of W.}
\end{align}
%
We now restrict $q$ to a particular neighborhood of $p$. %
More specifically, 
%
\begin{align}
  p_i - q_i \in \frac{1}{\mu_1(p_1) + \mu_2(p_2) + 2} U_i; 
\end{align}
%
which implies %
%
\begin{align}
  \mu_i(q_i -p_i) = \mu_i(p_i -q_i) \leq \frac{1}{\mu_1(p_1)+\mu_2(p_2)+2} 
\end{align}
(the equality at the left is valid, since $U_i = -U_i$). %
The special case %
\begin{align}
  \lambda_i(p)   &\triangleq  \mu_1(p_1) + \mu_2(p_2) + 1, \\
  \lambda_i(p-q) &\triangleq  \frac{1}{\mu_1(p_1) + \mu_2(p_2) + 1} %
    \triangleq  \lambda_i(q-p) 
\end{align}
implies that %
%
\begin{align}
  B(p) - B(q)  \in W + W+ W, 
\end{align}    
%
since $W$ is balanced. %
%
$W$ being arbitrary, we have so established the continuousness of $B$ at %
arbitrary $p$; which achieves the proof.
\end{proof}
%END