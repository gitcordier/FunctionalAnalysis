\textit{
Define the Fourier coefficient $\hat{f}(n)$ of a function %
%
  $f\in L^2(T)$ ($T$ is the unit circle) %
%
by
%
  \begin{align*}
  \hat{f}(n) 
    = 
  \frac{1}{2\pi} \int_{\minus \pi}^{\pi} %
  %
    f(e^{i\theta}) e^{\minus i n\theta} d\theta
  %
  \end{align*}
%
for all $n\in \Z$ (the integers). Put 
%
  \begin{align*}
    \Lambda_n f =\sum_{k=\minus n }^{n} \hat{f}(k).
  \end{align*}
%
Prove that %
  $\set{f \in {L}^2(T)}{\underset{n\infty}{\lim}\,\Lambda_n f \text{ exists}} $ %
%
is a dense subspace of $L^2(T)$ of the first category. %
}
%
\begin{proof}%
Let $f(\theta)$ stand for $f(e^{i\theta})$, so that %
%
  $L^2(T)$ is identified with a closed subset of $L^2([\minus \pi, \pi])$, %
%
hence the inner product %
%
  \begin{align}\label{2.06. Inner product.}
    \hat{f}(n) 
      = 
    (f,e_n) 
      =
    \frac{1}{2\pi} \int_{\minus \pi}^\pi f(\theta)e^{\minus in\theta}d\theta.
  \end{align}
%
We believe it is customary to write %
\begin{align}\label{2.06. Rewriting of Lambda_n.}
  \Lambda_n (f) = (f, e_{\minus n}) + \cdots + (f, e_n).
\end{align}
Moreover, a well known (and easy to prove) result is % 
%
  \begin{align}\label{2.06. Orthonormality.}
    (e_{n}, e_{n'}) = [n=n'], \text{\ie } %
    \set{e_n}{n\in\Z} \text{ is an orthormal subset of } L^2(T).
  \end{align}
%
For the sake of brevity, we assume the isometric ($\equiv$) %
identification %
%
  $L^2\equiv (L^2)^\ast$. %
%
So,  %
%
  \begin{align}\label{2.06. Norm of Lambda_n.}
    \norm{}{\Lambda_n}^2
      \citeq{\ref{2.06. Rewriting of Lambda_n.}}
    \norm{}{e_{\minus n}  + \cdots + e_n}^2 
      \citeq{\ref{2.06. Orthonormality.}} 
    \norm{}{e_{\minus n }}^2 + \cdots + \norm{}{e_{n}}^2
      \citeq{\ref{2.06. Orthonormality.}}
    2n+1.
  \end{align}
%
%$\Lambda_n$ is therefore a bounded linear functional, of norm $\sqrt{2n+1}$. %
We now assume, to reach a contradiction, that %
%
  \begin{align}
    B \Def \set{f \in {L}^2(T)}{
      %{\underset{n\infty}
      \sup\set{
        \magnitude{\Lambda_n \,f}}{
        \counting{n}
      } < \infty
  } %
  \end{align}
%
is of the second category. %
So, the Banach-Steinhaus theorem \citeresultFA{2.5} asserts that the sequence %
%
  $\singleton{\Lambda_n}$ %
%
is norm-bounded; which is a desired contradiction, since %
%
  \begin{align}
    \norm{}{\Lambda_n} \citeq{
      \ref{2.06. Norm of Lambda_n.}
    } \sqrt{2n + 1} \tendsto{n}{\infty}\infty.
%
  \end{align}
%
We have just established that $B$ is actually of the first category; %
and so is its subset %
%
  L=$\set{f\in {L}^2(T)}{\lim_{n\longrightarrow\infty}\Lambda_n f \text{ exists}}$. %
%.
We now prove that $L$ is nevertheless dense in $L^2(T)$. %
To do so, we let $P$ be $\text{span}\set{e_k}{k\in Z} $, %
the collection of the trignometric polynomials %
%
  $p(\theta)= \sum \lambda_k e^{ik\theta}$: %
%
Combining %
%
  (\ref{2.06. Rewriting of Lambda_n.}) with %
  (\ref{2.06. Orthonormality.}) %
%
shows that %
$\Lambda_n(p)= \sum \lambda_k$ for almost all $n$. %
Thus, 
%
  \begin{align}\label{2.06. Basic inclusions.}
    P \subset L \subset L^2(T).
  \end{align}
%
We know from the Fejér theorem (the Lebesgue variant) that %
$P$ is dense in $L^2(T)$. We then conclude, with the help of %
%
  (\ref{2.06. Basic inclusions.}), %
%
that %
%
  \begin{align}
    L^2(T) = \overline{P} = \overline{L}.
  \end{align}
%
%It is now clear that $\overline{L} = L^2(T)$. 
So ends the proof
\end{proof}


