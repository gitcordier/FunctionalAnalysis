\textit{%
Let X be the normed space of all real polynomials in one variable, with %
%
  \begin{equation*}
    \norm*{f}=\int_0^1\lvert f(t)\rvert\ dt.
\end{equation*}
%
Put  %
  $B(f, g)=\int _0^1 f(t)g(t) dt $, %
%
and show that $B$ is a bilinear continuous functional on $X\times X$ %
which is separately continuous but is not continuous.
}
\begin{proof} Let $f$ denote the first variable, $g$ the second one. %
Remark that %
%
  \begin{align}\label{magnitude of B(f, g)}
    \magnitude*{B(f, g)} 
      & < \norm*{f} \cdot \max_{[0,1]} \magnitude*{g} ; 
  \end{align}
%
which is sufficient (\citeFA{1.18}) to assert that any %
%
  $f \mapsto B(f, g)$ 
%
is continuous. The continuity of all %
%
  $g \mapsto B(f,g)$ %
%
follows (Define $C(g, f) = B(f, g)$ and proceed as above). %
Suppose, to reach a contradiction, that $B$ is continuous. %
Thus, there exists a positive $M$ such that, 
%
  \begin{equation}\label{magnitude of B(f, g) H_c}
    \magnitude*{B(f, g)} < M \norm*{f}\norm*{g}.
  \end{equation}
%
Put %
%
  \begin{equation}\label{definition of f_n}
    f_n(x)\Def 2 \sqrt{n}\cdot x^{n}\in \R[x] \qquad\qquad (\counting{n}), 
  \end{equation}
%
so that 
%
  \begin{equation}\label{norm of f_n}
    \norm*{f_n} = \frac{2 \sqrt{n}}{n+1} \tendsto{n}{\infty}0.
  \end{equation}
%
On the other hand,
%
  \begin{equation}\label{norm of B(f_n, g_n)}
    B(f_n, f_n)= \frac{4 n}{2n+1} > 1.
  \end{equation}
%
Finally, we combine %
%
  (\ref{norm of B(f_n, g_n)}) and %
  (\ref{magnitude of B(f, g) H_c}) with %
  (\ref{norm of f_n}) %
%
and so obtain
\begin{equation}
 1 < B(f_n, f_n) <  M \norm*{f_n}^2  \tendsto{n}{\infty} 0.
\end{equation}
Our continuousness assumption is then contradicted. So ends the proof.
\end{proof}
