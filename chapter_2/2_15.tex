\textit{%
Suppose $X$ is an F-space and $Y$ is a subspace of $X$ %
whose complement is of the first category. %
Prove that $Y=X$. }Hint: \textit{%
%
  $Y$ must intersect $x+Y$ for every $x\in X$. %
%
}
%
\begin{proof} Assume $Y$ is a subgroup of $X$. %
Under our assumptions, %
there exists a sequence $\set{E_n}{\counting{n}}$ of $X$ such that %
%
  \renewcommand{\labelenumi}{(\roman{enumi})} 
  \begin{enumerate}
    \item ${(\overline{E}_n)}^\circ=\emptyset ;$
    \item $X\setminus Y = \displaystyle{\bigcup_{n=1}^\infty  E_n}$.
  \end{enumerate}
%
By (i), the complement $V_n$ of $\overline{E}_n$ is a dense open set. %
Now pick $x$ in $X$: $x + V_n$ is dense open as well, 
since the translation by $x$ is an homeomorphism of $X$. %
%
Note that the density is also a special case of \citeresultFA{1.3 (b)}, as follows, %
%
\begin{align}\label{2_15_2}
  X = x + X \subseteq \overline{x + V_n}.
\end{align} 
%
We now apply the Baire's theorem twice to establish that %
%
\begin{enumerate}
  \item every intersection $W_n = V_n \cap (x+V_n)$ is dense in $X$;
  \item so is the intersection $\displaystyle{\bigcap_{n=1}^\infty W_n}$.
\end{enumerate}
%
Bear in mind that every dense subset of $X$ is nonempty and remark that %
\begin{align}\label{2_15_3}
  \bigcap_{n=1}^\infty W_n \subseteq \bigcap_{n=1}^\infty V_n \subseteq Y
\end{align} 
holds, since %
%
  $X\setminus Y \subseteq \bigcup_{n=1}^\infty \overline{E}_n$. %
%
Now pick $w$ in $\bigcap_{n=1}^\infty W_n$, so that %
$w$ is an element of $Y$ (by (\ref{2_15_3})) that lies in every $W_n$. %
The key ingredient is that $W_n$ a subset of $x + V_n$. %
Hence %
%
\begin{align}
  w \in  x + V_n. 
\end{align}
%
Finally, 
\begin{align}
  w - x \in \bigcap_{n=1}^\infty V_n \subseteq Y; 
\end{align}
%
again with (\ref{2_15_3}).
It is now clear that $x + Y$ cuts $Y$ in $x + (w - x ) = w$, which establishes that $x$ lies is $Y - Y$. %
As a conclusion, 
\begin{align}
  X \subseteq Y,  
\end{align}
%
since $x$ was arbitrary and that $Y$ is a subgroup. %
So ends the proof.
%
%
%  \begin{align}\label{2_15_4}
%    (x+S)\cap S\neq\emptyset.
%  \end{align}
%
%Moreover, it follows from (ii) that %
%
%  $X\setminus Y \subseteq \bigcup_n \closure{E}_n$, \ie 
%  $Y \contains S$. %
%Combined with (\ref{2_15_4}), this shows that $x+Y$ cuts $Y$. 
%Therefore, our arbitrary $x$ is an element of the subgroup $Y$. %
%We have thus established that $X \subseteq Y$, which achieves the proof.
\end{proof}
\renewcommand{\labelenumi}{(\alph{enumi})} 
%
% END
