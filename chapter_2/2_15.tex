\textit{%
Suppose $X$ is an F-space and $Y$ is a subspace of $X$ %
whose complement is of the first category. %
Prove that $Y=X$. }Hint: \textit{%
%
  $Y$ must intersect $x+Y$ for every $x\in X$. %
%
}
%
\begin{proof} Assume $Y$ is a subgroup of $X$. %
Under our assumptions, %
there exists a sequence $\set{E_n}{\counting{n}}$ of $X$ such that %
%
  \renewcommand{\labelenumi}{(\roman{enumi})} 
  \begin{enumerate}
    \item ${(\overline{E}_n)}^\circ=\emptyset ;$
    \item $X\setminus Y = \displaystyle{\bigcup_{n=1}^\infty  E_n}$.
  \end{enumerate}
%
By (i), the complement $V_n$ of $\overline{E}_n$ is a dense open set. %
Since $X$ is an F-space, it follows from the Baire's theorem that %
%
  the intersection $S$ of the $V_n$'s is dense in $X$:  %
%
So is $x+S$ ($x\in X$). To see that, remark that %
%
  \begin{align}\label{2_15_2}
    X = x + \overline{S} \subseteq \overline{x + S}
  \end{align} 
%
follows from \citeresultFA{1.3 (b)}. %
Since $S$ and $x+S$ are both dense open subsets of $X$, %
the Baire's theorem asserts that %
%
  \begin{align}
    \overline{(x+S)\cap S} = X.
  \end{align} 
%
Thus, 
%
  \begin{align}\label{2_15_4}
    (x+S)\cap S\neq\emptyset.
  \end{align}
%
Moreover, it follows from (ii) that %
%
  $X\setminus Y \subseteq \bigcup_n \closure{E}_n$, \ie 
  $Y \contains S$. %
Combined with (\ref{2_15_4}), this shows that $x+Y$ cuts $Y$. 
Therefore, our arbitrary $x$ is an element of the subgroup $Y$. %
We have thus established that $X \subseteq Y$, which achieves the proof.
\end{proof}
\renewcommand{\labelenumi}{(\alph{enumi})} 
%
% END
