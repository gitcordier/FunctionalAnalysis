% % % % % % % % % % % % % % % % % % % % % % % % % % % % % % % % % % % % % % % % % % % % % % % % % % % % % % % % % % % % % % % %
% FunctionalAnalysis 
% 2_15.tex
% 
% encoding: UTF-8 
% EOL: LF
%
% format: LaTeX
% indent: spaces (2)
% width: 127
% % % % % % % % % % % % % % % % % % % % % % % % % % % % % % % % % % % % % % % % % % % % % % % % % % % % % % % % % % % % % % % %
\textit{Suppose $X$ is an F-space and $Y$ is a subspace of $X$ whose complement is of the first category. Prove that $Y=X$. }%
Hint: \textit{$Y$ must intersect $x+Y$ for every $x\in X$. }%
%
\begin{proof}%
% % % % % % % % % % % % % % % % % % % % % % % % % % % % % % % % % % % % % % % % % % % % % % % % % % % % % % % % % % % % % % % %
% A. Assumptions.
% % % % % % % % % % % % % % % % % % % % % % % % % % % % % % % % % % % % % % % % % % % % % % % % % % % % % % % % % % % % % % % %
Take $E_n$ and $V_n$ such that %
%
\renewcommand{\labelenumi}{(\roman{enumi})} 
\begin{enumerate}
  \item $X = V_n \cup \overline{E}_n$ \quad ($V_n \cap \overline{E}_n = \emptyset)$;
  \item $X = \overline{V}_n$; 
  \item $ X\setminus Y = \bigcup_{n=1}^\infty E_n$
\end{enumerate}
\renewcommand{\labelenumi}{(\alph{enumi})} 
%
for all positive integers $n$. First, let $x$ be an arbitrary element of $X$. We note that $x + V_n$ is open and dense 
as well (because%
\footnote{
  This is also a special case of \citeresultFA{1.3 (b)}, since %
  $X = x + X \subset \overline{x + V_n}$.
}
the translation by $x$ is a homeomorphism of $X$ onto $X$).
%
% % % % % % % % % % % % % % % % % % % % % % % % % % % % % % % % % % % % % % % % % % % % % % % % % % % % % % % % % % % % % % % %
% B. Where Baire's theorem is involved.
% % % % % % % % % % % % % % % % % % % % % % % % % % % % % % % % % % % % % % % % % % % % % % % % % % % % % % % % % % % % % % % %
Next, apply Baire's theorem twice to establish that %
%
\begin{enumerate}
  \item every intersection $W_n = V_n \cap [x+V_n]$ is dense in $X$;
  \item so is the nonempty intersection $\bigcap_{n=1}^\infty W_n$. 
\end{enumerate}
%
Moreover, the intersection $\bigcap_{n=1}^\infty V_n$ avoids every $E_n$. To sum up, %
%
\begin{align}\label{2_15_1}
  w \citein{b} \bigcap_{n=1}^\infty W_n \citesubset{a} \bigcap_{n=1}^\infty V_n \citesubset{iii} Y 
\end{align} 
%
for some $w = w(x)$. %
% % % % % % % % % % % % % % % % % % % % % % % % % % % % % % % % % % % % % % % % % % % % % % % % % % % % % % % % % % % % % % % %
% C. Key ingredient! 
% % % % % % % % % % % % % % % % % % % % % % % % % % % % % % % % % % % % % % % % % % % % % % % % % % % % % % % % % % % % % % % %
Note that $w$ also lies in every $x + V_n$, by (a). Hence %
%
\begin{align}\label{2_15_2}
  w - x \in \bigcap_{n=1}^\infty V_n \citesubset{\ref{2_15_1}} Y. 
\end{align}
%
% % % % % % % % % % % % % % % % % % % % % % % % % % % % % % % % % % % % % % % % % % % % % % % % % % % % % % % % % % % % % % % %
% D. Put B and C together then conclude. 
% % % % % % % % % % % % % % % % % % % % % % % % % % % % % % % % % % % % % % % % % % % % % % % % % % % % % % % % % % % % % % % %
Finally, combining (\ref{2_15_1}) with (\ref{2_15_2}) yields %
%
\begin{align}
  x = w - (w - x) \in Y - Y = Y,
\end{align}
%
because $Y$ is a subgroup of $X$. This establishes that %
%
\begin{align}
  X \subset Y, 
\end{align}
%
since $x$ was arbitrary.
\end{proof}
%
% END
