% % % % % % % % % % % % % % % % % % % % % % % % % % % % % % % % % % % % % % % % % % % % % % % % % % % % % % % % % % % % % % % %
% FunctionalAnalysis 
% 2.03.tex
% 
% encoding: UTF-8 
% EOL: LF
%
% format: LaTeX
% indent: spaces (2)
% width: 127
% % % % % % % % % % % % % % % % % % % % % % % % % % % % % % % % % % % % % % % % % % % % % % % % % % % % % % % % % % % % % % % %
\textit{Put $K=\closedInterval*{\minus 1}{1}$; define $\D_{K}$ as in Section 1.46 (with $\R$ in place of $\R^n$). Suppose %
$\{f_{n}\}$ is a sequence of Lebesgue integrable functions such that %
$\Lambda(\phi) = \underset{n \to \infty}{\lim} \int_{\minus 1}^1 f_{n}(t)\phi(t) \diff{t}$ exists for every $\phi\in\D_{K}$. %
Show that $\Lambda$ is a continuous functional on $\D_{K}$. Show that there is a positive integer $p$ and a number $M<\infty$ %
such that
%
\begin{equation*}
  \magnitude[\Bigg]{\int_{\minus 1}^1 f_n (t)\phi (t) \diff{t}}\leq M \norm*[\infty]{D^p\phi}
\end{equation*}
%
for all $n$. For example, if $f_{n}(t)=n^{3}t$ on $\closedInterval*{\minus 1}{1}$ and $0$ elsewhere, show that this can be %
done with $p=1$. Construct an example where it can be done with $p=2$ but not with $p=1$.}
%
\begin{proof}
% % % % % % % % % % % % % % % % % % % % % % % % % % % % % % % % % % % % % % % % % % % % % % % % % % % % % % % % % % % % % % % %
% First part: Introduction, f_n as a bounded and signed Radon measure.
% % % % % % % % % % % % % % % % % % % % % % % % % % % % % % % % % % % % % % % % % % % % % % % % % % % % % % % % % % % % % % % %
Equipped with the supremum norm, $\Continuous_K=C(\R)\cap \set{u}{\supp{u} \subset K}$ is the copy of $C_c(K)$ in $C(\R)$. %
Each density $f_n$ is then identified modulo $N(\beta_\R)$ with the following Radon measure%
%
\begin{align}
  \mu_n: \Continuous_K &\to \C \\
  u &\mapsto \int_{\minus 1}^{1} f_n(t)u(t)\diff{t}; \nonumber
\end{align}
%
see [\ref{notations:Radon-measures}] and [\ref{notations:Lebesgue-integration}] for definitions and notations. Every %
$\mu_n$ is continuous since $\norm*{\mu_n} = \norm*[1]{f_n}$ is finite; \cf \citeFA{6.19}. Note that the dual space %
$\Continuous_K^\ast$ is a Banach space as well, by \citeFA{4.1}. Here, we consider only pointwise convergence, %
which is weaker than norm convergence. Define
%
\begin{equation}
  \Lambda_n \Def \restriction{\mu_n}{\D_K},
\end{equation}
%
and assume the pointwise convergence of the sequence $\singleton{\Lambda_n}$, that is 
%
\begin{equation}\label{2.3:assumed-pointwise-convergence}
  \Lambda_n(\phi) \tendsto{n}{\infty} \Lambda(\phi) \qquad(\phi \in \D_K).
\end{equation}
%
We show that the linear form $\Lambda$ is continuous in the topology of $\D_K$. We also construct a particular sequence %
$\{\Lambda_n\}$ whose pointwise limit $\Lambda$ is not norm-bounded. By contraposition, Theorem \citeFA{2.8} implies that %
pointwise convergence on $\D_K$ does not extend to $\Continuous_K$. This conclusion also follows from the following bounds:
%
\begin{align}
  \big\lvert \Lambda_n(\phi) \big\rvert &\leq M \norm*[\infty]{\phi'} \label{2.3:norm-p-bound1},\\
  \magnitude*{\Lambda_n(\phi')} &\leq M \norm*[\infty]{\phi''} \label{2.3:norm-p-bound2}.
\end{align}
%
Combined with the lack of boundedness at order $p=0$, \cf \eqref{2.3:norm-p-bound1}, the contraposition of Theorem %
\citeFA{2.6} implies that $\{\Lambda_n\}$ does not converge pointwise on $\Continuous_K$. In Radon measure theory, pointwise %
convergence is known as \emph{vague convergence}. The rest of the current proof has been split into the following paragraphs,
%
\begin{description}
  \item[Continuity of $\Lambda$:]{%
    The sequence $\singleton{\Lambda_n}$ has pointwise limit $\Lambda \in \D_K^\ast$.
  }
  \item[Existence of the uniform bound $\magnitude*{\Lambda_n(\phi)}\leq M\norm*{D^p \phi}$:]{%
    $\singleton{\Lambda_n}$ is equicontinuous, hence $\magnitude*{\Lambda_n(\phi)} \leq M \norm*{D^p \phi}$.
  }
  \item[Existence of $\singleton{\Lambda_n}$ with optimal uniform bound $\magnitude*{\Lambda_n(\phi)}\leq M \norm*{D \phi}$:]{
    In this example, no similar bound in $\norm*{\phi}$ exists. We show that $\Lambda$ does not extend to a Radon measure. 
  }
  \item[Case of a uniform bound in $\norm*{D^2\phi}$:]{%
    Where the smallest possible $p$ is $2$}.
\end{description}
%
We first prove that $\displaystyle{\lim_{n\to \infty}{\restriction{\Lambda_n}{\D_K}}}$ is continuous in $\D_K$'s topology.
% % % % % % % % % % % % % % % % % % % % % % % % % % % % % % % % % % % % % % % % % % % % % % % % % % % % % % % % % % % % % % % %
% Second part: The limit Lambda is continuous 
% % % % % % % % % % % % % % % % % % % % % % % % % % % % % % % % % % % % % % % % % % % % % % % % % % % % % % % % % % % % % % % %
\paragraph{Continuity of $\Lambda$.} We equip $\D_K$ with derivative norms %
$\norm*[N]{\phi}\Def \norm*[\infty]{\phi} + \norm*[\infty]{D\phi} + \cdots + \norm*[\infty]{D^N\phi}$ for all $\phi \in \D_K$ %
and all $\counting{N}$. The induced topology $\tau_K$ of $\D_K$ is the weakest topology that makes all norms %
$\norm*[N]{\cdot}$ continuous, \cf \citeFA{1.46, 6.2} and Exercise [1.16]. Equivalently, the collection of all convex %
balanced sets %
%
\begin{equation}\label{2.3:local-base-of-DK}
  V_N \Def \set[\Big]{\phi}{\phi \in \D_K, \norm*[N]{\phi}< 1/N}
\end{equation}
%
forms a local base of $\tau_K$, \cf \citeFA{6.2}. Note that $\norm*[N]{\phi} < 1$ implies $\norm*[\infty]{\phi} < 1 $: %
Every $\Lambda_n$ is then $\tau_K$-continuous by \citeFA{1.18}, since $\magnitude*{\Lambda_n}$ is bounded by %
$\norm*{\Lambda_n}$ on $V_1$. In summary: %
%
\begin{enumerate}
  \item{$\D_K$, equipped with the topology $\tau_K$, is a Fréchet space; see \citeFA{1.46}}.
    %
  \item{Every functional $\Lambda_n$ is $\tau_K$-continuous.}
    %
  \item{
      $\Lambda_n(\phi) \to \Lambda(\phi)$ pointwise on $\D_K$ (our premise).
    }
\end{enumerate}
%
The hypotheses of the Banach-Steinhaus theorem are then satisfied, which supports the following existence proof.
% % % % % % % % % % % % % % % % % % % % % % % % % % % % % % % % % % % % % % % % % % % % % % % % % % % % % % % % % % % % % % % %
% Third part: The sequence {Lambda_n} is equicontinuous, hence the uniform bound |Lambda_n .| < M |.| .
% % % % % % % % % % % % % % % % % % % % % % % % % % % % % % % % % % % % % % % % % % % % % % % % % % % % % % % % % % % % % % % %
\paragraph{Existence of the uniform bound $\magnitude*{\Lambda_n(\phi)}\leq M\norm*{D^p \phi}$.} By \citeFA{2.6, 2.8}, %
the equicontinuous sequence $\singleton{\Lambda_n}$ converges pointwise to a continuous $\Lambda$. Furthermore, the %
equicontinuity of $\singleton{\Lambda_n}$ ensures that all $\magnitude*{\Lambda_n}$ remain below $1$ on a common balanced %
neighborhood $V_p$. Hence 
%
\begin{equation}
  \frac{1}{p}\cdot\frac{\phi}{\norm*[p]{\phi} + \epsilon} \in V_p \qquad (\epsilon > 0).
\end{equation}
%
This gives $\magnitude*{\Lambda_n(\phi)}<  p\cdot \Bigl\{\norm*[p]{\phi} + \epsilon \Bigr\}$, %
which reduces to $\magnitude*{\Lambda_n(\phi)}\leq  p\norm*[p]{\phi}$. Use [\ref{annex:mean-value-with-derivatives}] with %
$I, D, \dots, D^p$ to output %
%
\begin{equation}\label{2.3:bound-from-derivative}
  \magnitude*{\Lambda_n(\phi)} \leq p(p+1) \norm*[\infty]{D^p \phi}.
\end{equation}
%
This completes the first part of the proof, with some $p$ and a positive constant $M = M(p)$. We now construct an example of 
$\singleton{\Lambda_n}$.
% % % % % % % % % % % % % % % % % % % % % % % % % % % % % % % % % % % % % % % % % % % % % % % % % % % % % % % % % % % % % % % %
% Fourth part: Counterexample
% % % % % % % % % % % % % % % % % % % % % % % % % % % % % % % % % % % % % % % % % % % % % % % % % % % % % % % % % % % % % % % %
\paragraph{Existence of $\singleton{\Lambda_n}$ with optimal uniform bound $\magnitude*{\Lambda_n(\phi)}\leq M \norm*{D\phi}$.}
Let $\psi$ be a smooth even mapping that equals $1$ on $\closedInterval*{\minus 1/2}{1/2}$, vanishes outside %
$\closedInterval*{\minus 1}{1}$, and satisfies $0\leq \psi \leq 1$ on $\R$. $\psi$ can be obtained from the general %
construction in \citeFA{1.46}. Alternatively, take the derivative of $\phi$ from Lemma %
[\ref{lemma-derivative-not-bounded-by-magnitude}], with $\tau=1/2$, $\omega=2$, and $A=1$. %
We set 
%
\begin{equation}
  f_n(t) \Def n^3t \Iverson[\Big]{\magnitude*{t} \leq 1/n} \qquad(t \in \R).
\end{equation}
%
Under the identification $C_c(K) \equiv \Continuous_K$, $\mu_n$ reads as the difference of two positive Radon measures %
$\mu_n^+$ and $\mu_n^-$, since %
%
\begin{equation}
  \mu_n(u) = \underbrace{n^3\int_{0}^{1/n} t u(t) \diff{t}}_{\mu_n^+(u)} 
    - \underbrace{n^3\int_{\minus 1/n}^0 \minus t u(t) \diff{t}}_{\mu_n^-(u)} \qquad (u \in \Continuous_K).
\end{equation} 
%
Thus, $\mu_n$ is a signed Radon measure, whose compact support $\closedInterval*{\minus 1}{1}$ shrinks to $\singleton{0}$ as %
$n \to \infty$. We see that $\norm*{\mu_n} \leq \norm*{\mu_n^+} + \norm*{\mu_n^-}$. However, the sequence $\singleton{\mu_n}$ %
is not uniformly bounded, since 
%
\begin{equation}\label{2.3:norm-of-lambda-n}
  \norm*{\mu_n} = \norm*[1]{f_n} = n = \norm*{\mu_n^+}  + \norm*{\mu_n^-}. 
\end{equation}
%
These equalities are justified by \citeFA{6.19}, but the logistic function %
$\sigma_\lambda:t\mapsto 1/\bigl(1+\exp(\minus\lambda t)\bigr)$ provides a direct proof of them. Indeed, this function is a %
standard smooth approximation of $H:t\mapsto \Iverson*{t\geq 0}$, the Heaviside step function; see [\ref{annex-Dirac}]. %
Lebesgue's dominated convergence theorem implies
%
\begin{equation}
  \mu_n\bigl(u\cdot(\,\underbrace{2 \sigma_\lambda - 1}_{\text{odd}}\,)\bigr) 
  = 2n^3 \int_0^{1/n} \underbrace{t\, \bigl(2\sigma_\lambda(t) - 1\bigr)}_{\text{even}} \diff{t} 
  \tendsto{\lambda}{\infty} n \qquad(n \geq 2). 
\end{equation}
%
We first observe that this $\{\mu_n\}$ cannot converge vaguely: this would, by Theorem \citeFA{2.6}, imply %
$\sup_n{\norm*{\mu_n}}\ < \infty$, which would contradict \eqref{2.3:norm-of-lambda-n}. However, we investigate further %
to establish weaker convergence and boundedness in $\tau_K$. The mean value theorem implies that, given $\phi \in \D_K$, 
%
\begin{equation}
  \phi(1/n)- \phi(\minus 1/n) = \frac{2}{n}\phi'(t_n)
\end{equation}
%
for some $t_n = t_n(\phi)\in \openInterval*{\minus 1/n}{1/n}$. Moreover, integration by parts yields %
%
\begin{align}
  \mu_n(\phi) = \Lambda_n(\phi) & = \evalbar[\bigg]{\frac{n^3}{2}t^2 \phi(t)}{\minus 1/n}{1/n} 
      - \frac{n^3}{2} \int_{\minus 1/n}^{1/n} t^2 \phi'(t) \diff{t}\\
  & = \frac{n}{2}\bigl(\phi(1/n) - \phi(\minus 1/n)\bigr)
      - \frac{n^3}{2} \int_{\minus 1/n}^{1/n} t^2 \phi'(t) \diff{t}\\
  & = \phi'(t_n) 
      - \frac{n^3}{2} \int_{\minus 1/n}^{1/n} t^2 \phi'(t) \diff{t}.\label{2.3:integration-by-part}
\end{align}
%
Note that when $\phi' = 1$ in a neighborhood of $0$, \eg,  for $\phi(t)= \psi(t) \cdot t$, the latter equality reduces to %
%
\begin{equation}
  \Lambda_n(\phi) = 1 - \frac{n^3}{2} \int_{\minus 1/n}^{1/n} t^2 \diff{t} = \frac{2}{3}.
\end{equation}
%
This suggests that continuity of $\phi'$ dictates $\Lambda_n(\phi) \to\frac{2}{3}\phi'(0)$. We establish this convergence %
in two steps. First, 
%
\begin{align}
  \Lambda_n(\phi) - \frac{2}{3}\phi'(0) & = \phi'(t_n) - \phi'(0) - \frac{n^3}{2} \int_{\minus 1/n}^{1/n} t^2 \phi'(t)  \diff{t}
    + \phi'(0) \underbrace{\frac{n^3}{2} \int_{\minus 1/n}^{1/n} t^{2} \diff{t}}_{1/3} \\
  & = \phi'(t_n) 
    - \phi'(0)
  - \frac{n^3}{2} \int_{\minus 1/n}^{1/n} t^2 \bigl(\phi'(t) - \phi'(0)\bigr)  \diff{t}.
\end{align}
%
Next, taking absolute values gives %
%
\begin{equation}
  \magnitude*{\Lambda_n(\phi) - \frac{2}{3}\phi'(0)} \leq \magnitude*{\phi'(t_n) - \phi'(0)} 
  + \frac{1}{3}\max_{\closedInterval*{\minus 1}{1}}{\magnitude*{\phi' - \phi'(0)}} \tendsto{n}{\infty} 0.
\end{equation}
%
As a result, %
%
\begin{equation}\label{2.3:convergence-to-dirac}
  \Lambda_n(\phi) \tendsto{n}{\infty} \minus \frac{2}{3}\delta'(\phi) \qquad (\phi\in \D_K), 
\end{equation}
%
where $\delta': \phi \mapsto \minus \phi'(0)$ is the \emph{derivative} of the \textit{Dirac measure} %
$\delta: \phi \mapsto \phi(0)$; see \citeFA{6.1, 6.9} and [\ref{annex-Dirac}]. The previous reasoning shows that 
the limit $\Lambda = \minus \frac{2}{3}\delta'$ is $\tau_K$-continuous. In addition, \eqref{2.3:integration-by-part} %
justifies
%
\begin{equation}
  \magnitude*{\Lambda_n(\phi)} \leq \magnitude*{\phi'(t_n)} 
  + \frac{1}{3} \max_{\closedInterval*{\minus 1}{1}}{\magnitude*{\phi'}}.
\end{equation}
%
A simpler bound is
%
\begin{equation}\label{2.3:upper-bound}
  \magnitude*{\Lambda_n(\phi)} \leq \frac{4}{3} \norm*[\infty]{\phi'}. 
\end{equation}
%
This is a concrete instance of \eqref{2.3:bound-from-derivative}, with $p=1$ and $M=4/3$. To establish %
\eqref{2.3:norm-p-bound1}, it suffices to prove that no reduction to order $p=0$ is possible. To do so, we first assume, %
to reach a contradiction, that there exists $M$ such that 
%
\begin{equation}\label{2.3:upper-bound-assumption}
  \magnitude*{\Lambda_n(\phi)} \leq M \norm*[\infty]{\phi} \qquad (\phi \in \D_K, \, \counting{n}).
\end{equation}
%
Next, we choose %
\begin{equation}\label{2.3:zn-definition}
  \phi_n \Def \psi \cdot \tilde{\phi}_n,
\end{equation}
%
where $\tilde{\phi}_n$ is $\phi$ from Lemma [\ref{lemma-derivative-not-bounded-by-magnitude}] with %
$\tau= 1/n = 1/ \omega = 1/A$. Hence $\norm*[\infty]{\phi_n} < 2$. In contrast, for $n \geq 4$, $\Lambda_n (\phi_n)$ is 
%
\begin{equation} 
  \frac{n}{2} \bigl(\,\underbrace{\tilde{\phi}_n(1/n)}_{1} 
  - \underbrace{\tilde{\phi}_n(\minus 1 /n)}_{\minus 1}\,\bigr) 
  - \underbrace{\frac{n^3}{2} \int_{\minus 1/n}^{1/n} t^2 \tilde{\phi}'(t) \diff{t}}_{\frac{1}{3}n} = \frac{2}{3}n.
\end{equation}
%
Our assumption \eqref{2.3:upper-bound-assumption} then reads as
% 
\begin{equation}\label{2.3:upper-bound-contradiction2}
  \frac{2}{3}n \leq 2M < \infty. 
\end{equation}
%
The behavior $\frac{2}{3}n \to\infty$ provides the desired contradiction. Combining \eqref{2.3:convergence-to-dirac} with %
\eqref{2.3:upper-bound-contradiction2} gives%
%
\begin{equation}
  \magnitude*{\Lambda(\phi_n)} \geq \magnitude*{\Lambda_n (\phi_n)}  - \magnitude*{ \Lambda_n(\phi_n) - \Lambda(\phi_n)}
  \tendsto{n}{\infty} \infty.
\end{equation}
%
This shows that $\Lambda$ has no continuity bound for $K$ because $\norm*[\infty]{\phi_n} < 2$ while %
$\magnitude*{\Lambda_n (\phi_n)}\to \infty$. Therefore, $\Lambda$ cannot be extended as a Radon measure; %
see [\ref{notations:Radon-measures}], equation \eqref{notations:definition-of-Radon-measure}. A direct way to see this is to %
pick $\phi_\omega$ from Lemma [\ref{lemma-derivative-not-bounded-by-magnitude}], so that 
%
\begin{equation}
  \Lambda(\psi\cdot \phi_\omega) = \frac{2}{3} \omega \tendsto{\omega}{\infty} \infty 
\end{equation}
%
contrasts with $\norm*[\infty]{\psi\cdot \phi_\omega}=1$. Thus, we have exhibited a sequence of Radon measures $\mu_n$ that 
%
\begin{enumerate} 
  \item{
    does not converge vaguely to any Radon measure,
  }
  \item{but converges pointwise on $\D_K$ to $\Lambda \in \D_K^\ast$, in the specific $\D_K$'s topology; %
    see \eqref{2.3:local-base-of-DK}, \eqref{2.3:convergence-to-dirac}, and \eqref{2.3:upper-bound}.
  }
\end{enumerate}
%
We now propose a second example, with $p=2$ as the smallest admissible order.
%
% % % % % % % % % % % % % % % % % % % % % % % % % % % % % % % % % % % % % % % % % % % % % % % % % % % % % % % % % % % % % % % %
% Fifth part: Second Counterexample
% % % % % % % % % % % % % % % % % % % % % % % % % % % % % % % % % % % % % % % % % % % % % % % % % % % % % % % % % % % % % % % %
\paragraph{Case of a uniform bound in $\norm*{D^2\phi}$.} We define the \emph{derivative}
%
\begin{align}
  \Lambda_n': \D_K &\to \C\\
  \phi &\mapsto \minus \Lambda_n(\phi'); \nonumber
\end{align}
%
see \citeFA{6.1}. We have proved that every $\Lambda_n$ is continuous. So is the derivative operator in $\D_K$; see Exercise %
[1.17]. Therefore, $\Lambda_n'$ is continuous. Applying \eqref{2.3:convergence-to-dirac} with $\phi'$ yields %
%
\begin{equation}
  \Lambda_n'(\phi) \tendsto{n}{\infty}  \minus \frac{2}{3} \phi''(0).
\end{equation}
%
Furthermore, Theorem \citeFA{2.8} implies that the limit $\minus \frac{2}{3} \phi''(0)$ is $\tau_K$ continuous. 
Additionally, it follows from \eqref{2.3:upper-bound} that the bound \eqref{2.3:norm-p-bound2} is %
%
\begin{equation}
  \magnitude*{\Lambda_n'(\phi)} \leq \frac{4\,}{3} \norm*[\infty]{\phi''}.
\end{equation}
%
To prove this, it suffices to show that $2$ is the smallest suitable $p$. First, we assume, to reach a contradiction, that
%
\begin{equation}
  \magnitude*{\Lambda_n(\phi')} \leq M \norm*[\infty]{\phi'} \qquad (\phi \in \D_K, \, \counting{n}).
\end{equation}
%
Next, let $\Phi_n$ be the antiderivative of $\phi_n$ that vanishes at $\minus 1$; see \eqref{2.3:zn-definition}. The oddness %
of $\Phi_n$ ($\psi$ is even) implies that $\supp{\Phi_n} \subset \closedInterval*{\minus 1}{1}$. So, under our assumption, 
%
\begin{equation}
  \magnitude*{\Lambda_n'(\Phi_n)} = \magnitude*{\Lambda_n(\Phi_n')} \leq M \norm*[\infty]{\Phi_n'}. 
\end{equation}
%
Equivalently,
% 
\begin{equation}
  \magnitude*{\Lambda_n(\phi_n)} \leq M \norm*[\infty]{\phi_n}, 
\end{equation}
%
which has already been disproved. To reach one final contradiction, assume that there exists $M = M(0)$ so that 
%
\begin{equation}
  \magnitude*{\Lambda_n'(\phi)}  \leq M \norm*[\infty]{\phi} \qquad(\phi \in \D_K, \, \counting{n}).
\end{equation}
%
Lemma [\ref{lemma-derivative-not-bounded-by-magnitude}] implies that %
%
\begin{equation}
  \magnitude*{\Lambda_n'(\phi)}  \leq M \norm*[\infty]{\phi'}.
\end{equation}
%
This contradiction concludes the proof.
\end{proof}
% END
%