% % % % % % % % % % % % % % % % % % % % % % % % % % % % % % % % % % % % % % % % % % % % % % % % % % % % % % % % % % % % % % % %
% FunctionalAnalysis 
% 2_09.tex
% 
% encoding: UTF-8 
% EOL: LF
%
% format: LaTeX
% indent: spaces (2)
% width: 127
% % % % % % % % % % % % % % % % % % % % % % % % % % % % % % % % % % % % % % % % % % % % % % % % % % % % % % % % % % % % % % % %
\textit{
Suppose $X,Y,Z$ are Banach spaces and 
%
\begin{equation*}
  B:X\times Y \to Z
\end{equation*}
%
is bilinear and continuous. Prove that there exists $M<\infty$ such that 
%
\begin{equation*}
  \norm*{B(x, y)}
  \leq 
  M \norm*{x}\norm*{y} \qquad (x\in X, y\in Y).
\end{equation*}
%
Is completeness needed here?}
%
\begin{proof}%
% % % % % % % % % % % % % % % % % % % % % % % % % % % % % % % % % % % % % % % % % % % % % % % % % % % % % % % % % % % % % % % %
% First part: proof. 
% % % % % % % % % % % % % % % % % % % % % % % % % % % % % % % % % % % % % % % % % % % % % % % % % % % % % % % % % % % % % % % %
Completeness is not required. To show this, we only assume that $X$, $Y$, and $Z$ are normed spaces. Since $B$ is continuous, 
there exists $r > 0$ such that 
%
\begin{equation}\label{2.9:bound-for-B}
  \norm*{B(x, y)} < 1
\end{equation}
%
whenever $\norm*{x}+\norm*{y} < r$. Now, for general $(x, y) \in X\times Y$ and for all $0< r - s<\min (1, r)$, we substitute:%
%
\begin{align}
  x_s &\Def \frac{s}{2} \cdot \frac{x}{\norm*{x} + r-s} \qquad (\text{for } x),\\
  y_s &\Def \frac{s}{2} \cdot \frac{y}{\norm*{y} + r-s} \qquad (\text{for } y),
\end{align}
%
in \eqref{2.9:bound-for-B}. A straightforward computation shows that, at fixed $(x, y)$, 
%
\begin{equation}
  \norm*{B(x, y)}= \frac{4}{s^2} \Bigl(\norm*{x} + r-s \Bigr) \Bigl(\norm*{y} + r-s\Bigr) \underbrace{\norm*{B(x_s, y_s)}}_{<1}
  < \frac{4}{s^2} \Bigl(\norm*{x}\norm*{y} + O(r-s) \Bigr)
  \tendsto{s}{r} \frac{4}{r^{2}}\norm*{x}\norm*{y};
\end{equation}
% 
which establishes the existence of a bound $M=4/r^2$. To illustrate this result, we now turn to a concrete example.

% % % % % % % % % % % % % % % % % % % % % % % % % % % % % % % % % % % % % % % % % % % % % % % % % % % % % % % % % % % % % % % %
% Second part: example.
% % % % % % % % % % % % % % % % % % % % % % % % % % % % % % % % % % % % % % % % % % % % % % % % % % % % % % % % % % % % % % % %
Take $X= Y= Z= C_c(\R)$ equipped with $\norm*[\infty]{\cdot}$, which is noncomplete; see %
\citebook{5.4.4}{AnalyseIII}. Nevertheless, the following bilinear mapping 
%
\begin{align}
  B: C_c(\R)^{2} & \to  C_c(\R) \\
  (f, g) & \mapsto f \cdot g \nonumber
\end{align}
%
is continuous. In addition, $B$ has continuity bound $M=1$, since 
%
\begin{equation}\label{2.9:example-bound-M}
  \norm*[\infty]{f\cdot g} \leq \norm*[\infty]{f} \cdot \norm*[\infty]{g}.
\end{equation}
%
To prove the continuity of $B$ at every $(f, g)$, fix $\epsilon > 0$ and let $r$ satisfy
%
\begin{equation}\label{2.9:example-r-and-epsilon}
  0 < r < \min\Bigl(1, \frac{\epsilon}{1 + \norm*[\infty]{f} + \norm*[\infty]{g}}\Bigr). 
\end{equation}
%
Next, choose $(u, v) \in C_c(\R)^2$ such that 
%
\begin{equation}\label{2.9:example-distance-smaller-than-r}
  \underbrace{\norm*[\infty]{u}- \norm*[\infty]{f} \leq \norm*[\infty]{f-u} }_{\text{reversed triangle inequality}} 
  + \norm*[\infty]{g-v} 
  < r. 
\end{equation}
%
Hence 
%
\begin{align}
  % 
  {\lVert\hspaceNorm \overbrace{(f-u) g + u (g-v)}^{fg - uv} \hspaceNorm\rVert}_\infty
    & \leq \norm*[\infty]{f-u} \cdot \norm*[\infty]{g} + \norm*[\infty]{u} \cdot \norm*[\infty]{g-v} 
    && \bigl(\text{by the triangle inequality and } \eqref{2.9:example-bound-M} \bigr) \\
  %
    & < r \cdot \norm*[\infty]{g} + \left(r + \norm*[\infty]{f}\right) \cdot r 
    && \bigl(\text{since } \norm*{u} \leq r + \norm*{f} \text{, see } \eqref{2.9:example-distance-smaller-than-r}\bigr)\\
  % 
    & < \epsilon \frac{r + \norm*[\infty]{f} + \norm*[\infty]{g}}{ 1+ \norm*[\infty]{f} + \norm*[\infty]{g}} 
    && \bigl(\text{by } \eqref{2.9:example-r-and-epsilon}\bigr)\\
  %
    & < \epsilon. 
    && \left(\text{since } r < 1 \text{, see } \eqref{2.9:example-r-and-epsilon}\right)
\end{align}
%
In conclusion,
%
\begin{equation}
  \norm*[\infty]{B(f,g) - B(u, v)} < \epsilon.
\end{equation}
%
Since $\epsilon$ was arbitrary, the continuity of $B$ is established. We have thus constructed a continuous bilinear %
$B: X^2 \to X$ over a noncomplete normed space $X$. This concludes the proof.
\end{proof}
% END
% 