From now on, unless the contrary is explicitely stated, %
we asume that $\phi$ only denotes an element of $C_K^\infty(\R)$. 
Let $f_n$ be a Lebesgue integrable function such that %
%
  \begin{align}\label{2.3. Convergence on D_K (1).}
    \Lambda \phi = \underset{n \to \infty}{\lim} 
      \int_{K} f_n\phi 
      %
      \quad (\phi \in C_K^\infty(\R)).
  \end{align}
%
for some linear form $\Lambda$. %
%
Since $\phi$ vanishes outside $K$, we can suppose without loss of generality %
that the support of $f_n$ lies in $K$. So, %
%
  (\ref{2.3. Convergence on D_K (1).}) %
% 
can be restated as follows, 
%
  \begin{align}
    \Lambda \phi = \underset{n \to \infty}{\lim} 
      \bra{f_n}\ket{\phi} 
      %
      \quad (\phi \in C_K^\infty(\R)).
  \end{align}
%
Let $K_1, K_2, \dots, $ be compact sets that satisfy the conditions 
specified in \citeresultFA{1.44}. %
$\D_K$ is $C_K^\infty(\R)$ topologized by the related seminorms %
%
  $p_1, p_2, \dots$; see \citeresultFA{1.46, 6.2} and Exercise 1.16.
%
We know that $K\subseteq K_m$ for some index $m$ %
(see Lemma 2 of Exercise 1.16): From now on, we only consider the indices 
$N \geq m$, so that%
%
  \renewcommand{\labelenumi}{(\alph{enumi})}%
  %
  \begin{enumerate}
    \item{
      $p_N(\phi) = \norm{N}{\phi} \Def \max 
      \set{\lvert D^{\alpha}\phi(x) \rvert}{\alpha \leq N, x \in \R}$, %
      for $\phi \in \D_K$;
    }
    \item{
      The collection of the sets %
      $V_N = \set{ \phi \in \D_K}{ \norm{N}{\phi} < 2^{\minus N}}$ %
      is a (decreasing) local base of $\tau_K$, the subspace topology of $\D_K$; %
      see \citeresultFA{6.2} for a more complete discussion.
    }
  \end{enumerate}
  %
  \renewcommand{\labelenumi}{(\roman{enumi})}
  %
%
Let us specialize (\ref{2.3  g  bounded operator (1).}) with %
%
  $u=f_n$ and $ \phi \in V_m$ %
%
then conclude that $\bra{f_n}$ is bounded by $\norm{L^1}{f_n}$ on $V_m$: %
Every linear functional $\bra{f_n}$ is therefore $\tau_K$-continuous; see %
%
  \citeresultFA{1.18}. \newline\newline\noindent
%
To sum it up: %
%
  \begin{enumerate}
    \item{$\D_K$, equipped the topology $\tau_K$, is a Fréchet space %
      (see \citeresultFA{section 1.46})};
    %
    \item{Every linear functional $\bra{f_n}$ is continuous %
      with respect to this topology;}
    %
    \item{
      $\bra{f_n}\ket{\phi} \tendsto{n}{\infty} \Lambda \phi$ for all $\phi$, 
        \ie 
      $ \Lambda-\bra{f_n} \tendsto{n}{\infty}0$.
    }
  \end{enumerate}
%
With the help of \citeresultFA{[2.6] and [2.8]}, we conclude that %
%
  $\Lambda$ is continuous 
%
and that the sequence
%
  $\singleton{\bra{f_n}}$ %
%
is equicontinuous. 
%
So is the sequence %
%
  $\singleton{\Lambda - \bra{f_n}}$, %
%
since addition is continuous.
%
There so exists $i, j$ such that, for all $n$, 
%
  \begin{align}
   %
    \left\lvert 
      \Lambda \phi 
    \right\rvert
      < 
    1/2 & \quad \text{if }\phi\in V_i, \\
    %
    \left\lvert 
      \Lambda\phi - \bra{f_n}\ket{\phi} 
    \right\rvert  
      < 
    1/2 & \quad \text{if } \phi\in V_j.
    %
  \end{align}
%
Choose $p = \max\singleton{i, j}$, so that $V_p = V_i \cap V_j$: %
The latter inequalities imply that %
%
  \begin{align}
    %
    \left\lvert \bra{f_n}\ket{\phi} \right\rvert  & \leq 
      \magnitude{
        \Lambda\phi 
        - \bra{f_n }\ket{\phi} 
      }
      + 
      \magnitude{\Lambda \phi}
      %
     < 1 \
     %
       \quad\text{if } \phi\in V_p. %
    %
  \end{align}
%
Now remark that every $\psi =\psi[\mu, \phi]$, where %
%
  \begin{align}
     \psi[\mu, \phi] \Def 
      \begin{cases}
        ({1}/{\mu \cdot 2^{p} \norm{p}{\phi}}) \phi & (\phi \neq 0, \mu > 1)\\
        0                                          & (\phi   =  0, \mu > 1),  
      \end{cases}
  \end{align}
%
keeps in $V_p$. Finally, % 
it is clear that each below statement implies the following one.
%
  \begin{align}
    \lvert \langle {f_n}  | \psi \rangle \rvert &  
    < 1 \\
      %
    \left\lvert \bra{f_n}\ket{\phi} \right\rvert &  
    < 2^{p}  \norm{p}{\phi} \cdot \mu \\
    %
    \magnitude{\bra{f_n}\ket{\phi}} & 
    \leq 2^{p} \norm{p}{\phi} \\
    %
     \magnitude{\bra{f_n}\ket{\phi}} &
     \leq 2^{p} \singleton{
       \norm{\infty}{D^0 \phi} 
       + 
         \cdots 
      + 
        \norm{\infty}{D^p \phi}
      }.
    %
  \end{align}
  Finally, with the help of (\ref{2.3. Mean value inequality.}), 
  \begin{align}
    \label{2.3. Bound with p and M (theoritical).}
     \magnitude{\bra{f_n}\ket{\phi}} &
       \leq 
      2^{p} (p+1) \norm{\infty}{D^p \phi} .
  \end{align}
%
The first part is so proved, with \textit{some} $p$ and $ M= 2^{p}(p+1)$. %
\newline\newline\noindent
%
% END