In order to construct the second requested example, we give %
%
  ${f_n}$ %
%
a \textit{derivative}%
%
  \footnote{
    See \citeresultFA{6.1} 
    for a further discussion.
  } %
${f_n}'$, as follows %
%
  \begin{align}
    {f_n}': \D_K &\to \C\\
                    \phi &\mapsto \minus \bra{f_n}\ket{\phi'}. \nonumber
  \end{align}
%
It has been proved that every $\bra{f_n}$ is continuous. %
So is %
%
  \begin{align}
    D : \D_K & \to \D_K \\
        \phi & \mapsto \phi'; \nonumber
  \end{align}
see  Exercise 1.17. %
${f_n}'$ is therefore continuous. Now apply %
%
  (\ref{2.3. (f_n) converges to Dirac (1).}) %
%
with $\phi'$ and so obtain %
%
  \begin{align}
    \minus \bra{f_n}\ket{\phi'} &\tendsto{n}{\infty} \frac{4}{3} \phi''(0)
    \quad(\phi \in \D_K), \nonumber
  \end{align}
%
\ie %
%
  \begin{align}
    {f_n}' &\tendsto{n}{\infty} \frac{4}{3} \delta''. 
  \end{align}
%
It follows from %
%
  (\ref{2.3. Bound with p and M (concrete).}) %
%
that, 
%
  \begin{align}
  \lvert \bra{f_n}\ket{\phi'} \rvert 
    \leq 
  \frac{4}{3} \norm{\infty}{ \phi''} \quad (\counting{n}).
  \end{align}
%
It is therefore possible to uniformly bound 
%
  ${f_n}'$ %
%
with respect to a norm %
%
  $\norm{\infty}{D^p\cdot}$, %
%
namely $\norm{\infty}{D^2\cdot}$. 
%
Then arises a question: %
%
  Is $2$ the smallest $p$? 
%
The answer is: Yes. 
%
To show this, we first assume, to reach a contradiction, that %
%
  there exists a positive constant $M$ such that
%
  \begin{align}
    \lvert \langle f_n \lvert \phi' \rangle \rvert 
      \leq 
    M \norm{\infty}{\phi'} 
    \quad (\counting{n}).
  \end{align}
%
Define %
%
  \begin{align}
    \Phi_{j}(x) = \int_{\minus 1}^x \phi_j.
  \end{align}
%
The oddness of $\phi_j$ forces %
% 
  $\Phi_{j}$ to vanish outside $[\minus 1, 1]$: %
%
$\phi_{j}$ is therefore in $\D_K$. So, under our assumption, 
%
  \begin{align}
    \lvert 
      \langle{f_n}\lvert \Phi_j' \rangle
    \rvert 
      \leq 
    M \norm{\infty}{\Phi_{j}'}
    \quad (\counting{n}); 
  \end{align}
%
which is %
%
  \begin{align}
    \lvert
      \langle{f_n}\lvert{\phi_j}\rangle
    \rvert 
      \leq 
    M 
    \quad (\counting{n}).
  \end{align}
%
We have thus reached a contradiction (again with the sequence %
%
  $\singleton{\langle f_n\lvert \phi_{\rho(n)} \rangle}$) %
%
and so conclude that there is no constant $M$ such that %
%
  \begin{align}
    \lvert 
      \langle \lvert f_n \phi'\rangle
    \rvert 
      \leq 
    M \norm{\infty}{\phi'}
    \quad (\counting{n}).
  \end{align}
%
Finally, assume, to reach a contradicton, that %
there exists a constant $M$ such that 
%
  \begin{align}
    \lvert \langle f_n \lvert \phi' \rangle \rvert 
      \leq 
    M \norm{\infty}{\phi}.
  \end{align}
%
The mean value theorem (see (\ref{2.3. Mean value inequality.})) asserts that %
%
  \begin{align}
    \lvert \langle f_n \lvert \phi' \rangle \rvert 
      \leq 
    M \norm{\infty}{\phi} 
      \leq 
    M \norm{\infty}{\phi'}; 
  \end{align}
%
which is, again, a desired contradiction. So ends the proof.

