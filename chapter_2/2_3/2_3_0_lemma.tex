If 
%
  $\phi\in\D_{[a, b]}$, then 
%
  \begin{align}\label{2.3. mean value inequality.}
    \norma{\infty}{D^k \phi} \leq \norma{\infty}{D^p \phi} \left(\frac{b-a}{2}\right)^{p-k} .
  \end{align}
%
for all $k \leq p$ in $\N$.
\begin{proof}
Choose $a < x_0 \leq (a+b)/2$ first. By the mean value theorem, there exists $x_1 \in \openinterval{a}{x_0}$ such that % 
%
  \begin{align}
    \phi(x_0) = \phi(x_0) - \phi(a)=  D\phi(x_{1})(x_{0} - a).
  \end{align}
%
Repeating the same reasoning for $D\phi, D^2 \phi, \dots, D^p \phi \in \D_{[a, b]}$ yields % 
%
\begin{align}
  \phi(x_0) & = D^0 \phi(x_0) \\ 
            & = D^1\phi(x_{1})(x_0 - a) \\
            & = D^2\phi(x_{2})(x_1 - a)(x_0 - a) \\
            &  \nonumber \mspace{10mu} \vdots \\
            & = D^p\phi(x_{p})(x_{p-1} - a) \cdots (x_0-a),
\end{align}
%
for some $x_1, \dots ,x_p \in \openinterval{a}{x_0}$. Hence %
%
\begin{align}\label{2.3. inequality}
  \magnitude{\phi(x_0)} \leq \norma{\infty}{D^p \phi} \left(\frac{b-a}{2}\right)^p .
\end{align}
%
Similarly, if $b > x_0 \geq (a+b)/2$ (interchanging the roles of $a$ and $b$), the same inequality (\ref{2.3. inequality}) holds. %
So, %
%
\begin{align}\label{2.3. mean value inequality, case k=0}
  \magnitude{\phi(x_0)} \leq \norma{\infty}{D^p \phi} \left(\frac{b-a}{2}\right)^p \quad ( a < x_0 < b ), 
\end{align}
%
which establishes \eqref{2.3. mean value inequality.} when $k=0$. %
Finally, applying \eqref{2.3. mean value inequality, case k=0} to $D^k \phi $ in place of $\phi$ gives % 
%
\begin{align}
  \norma{\infty}{D^k \phi} \leq \norma{\infty}{D^p \phi} \left(\frac{b-a}{2}\right)^{p-k}
\end{align}
%
for all $0 \leq k \leq p$. %
\end{proof}