% a4paper = ISO 216 standard. 
% Don't forget to revert to default if you want the US Letter instead.
\documentclass[a4paper,12pt,titlepage,openany, leqno]{book}

%: MATH PACKAGE USED
%\usepackage{graphicx}

\usepackage{amsmath}
\usepackage{
physics, 
amssymb, 
mathrsfs, 
amsthm, 
mathspec, 
fouridx, 
%%stix
%%tensor
theorem, 
tikz
}
\usetikzlibrary{arrows,snakes,backgrounds,patterns,matrix,shapes,fit,calc,shadows,plotmarks}

%\usetikzlibrary{shapes.geometric}
%\usetikzlibrary{arrows.meta,arrows}
%\usepackage{tlatex}
% TEXs LOGOS! 
%\usepackage{dtklogos}

%\usepackage[bottom=3cm,top=2cm]{geometry}
% Metric system, sorry;)
\usepackage[left=3cm,top=2.5cm,right=3cm,bottom=2.5cm]{geometry}
%: COLOR
\usepackage{array}
%\usepackage[dvipsnames]{xcolor}
%\usepackage[]{color}
%\usepackage{framed}
\usepackage{colortbl}

%: FOR XELATEX. SKIP IF ERROR MESSAGE
\usepackage{xltxtra,xunicode}
%: for HEVEA- Uncomment before using XeTeX
%\usepackage[utf8]{inputenc}
\usepackage[xetex]{hyperref}
%\usepackage{hyperref}
%\usepackage{listings}
\hypersetup{%
    pdfborder = {0 0 0},
    colorlinks,
    citecolor=red,
    filecolor=green,
    linkcolor=black,
    urlcolor=blue%cyan!50!black!90
}
\defaultfontfeatures{Mapping=tex-text}

% FONTS: PERSONNAL CHOICE- CAN BE CHANGED!
\newfontfamily{\CMUCS}{CMU Classical Serif}
\newfontfamily{\CMU}{CMU Serif}
\newfontfamily{\CMUSS}{CMU Sans Serif}
\newfontfamily{\fw}{CMU Typewriter Text Light}
%\newfontfamily{\Arabic}{Al Bayan}
% On mac OS:
%\newfontfamily{\Didot}{Didot}
% Alternatively, for all Unix systems:
%\newfontfamily{\Didot}{Theano Didot}
%\usepackage{mathspec}
\def\mainfont{CMU Serif}
\setmainfont{\mainfont}

\setmathfont(Latin)[Uppercase=Regular,Lowercase=Regular]{\mainfont}
\setmathfont(Greek)[Uppercase=Regular,Lowercase=Regular]{\mainfont}
\setmathrm{\mainfont}
\setmathbb{\mainfont}
%\setmathit{\mainfont}
\setmathtt{CMU Typewriter Text Light}
%\setmathbf{CMU Serif}


%: TYPOGRAPHIC CONVENTIONS, XETEX
\usepackage{polyglossia}
%\selectbackgroundlanguage[variant=usmax]{english}
\setdefaultlanguage[variant=usmax]{english}

%
%\usepackage{glossaries}
%\makeglossaries


%\newglossaryentry{maths}
%{
%    name=mathematics,
%    description={Mathematics is what mathematicians do}
%}
%: ----------------------------PRIMITIVES------------------------------------- 
\newcommand{\insmall}[1]{\text{\small{#1}}}
% Some acceptable minus sign.
\def\minus{\insmall{-}}

% The field C and the usual subsets R, Q, Z, N. 
\newcommand\usualSet[1]{{\mathbf #1}}
\def\C{\usualSet{C}}
\def\R{\usualSet{R}}
\def\Q{\usualSet{Q}}
\def\Z{\usualSet{Z}}
\def\N{\usualSet{N}}
\def\K{\usualSet{K}}

% "Defined as equals to", alternative to := .
\def\Def{\triangleq}

% The usual scalar field = C, most of the time.
\def\field{\C}

% Counting numbers
\newcommand\counting[1]{#1  =  1, 2, 3, \dots}
% integers (nonnegative)
\newcommand\integers[1]{#1=0, 1, 2, \dots}
\newcommand{\D}{{\mathscr D}}

% Indices: upper, lower
\newcommand{\up}[1]{^{(#1)}}
\newcommand{\low}[1]{_{#1}}
\newcommand{\upnw}[2]{\fourIdx{#2}{}{}{}#1}
\newcommand{\downsw}[2]{\fourIdx{}{#2}{}{}#1}
\newcommand{\scriptatleft}[3]{\fourIdx{#2}{#3}{}{}#1}
\newcommand{\diagscript}[3]{\fourIdx{#2}{}{}{#3}#1}
%\tensor*[^x]{V}{_k}

% function, relation
\newcommand{\function}[1]{\mathtt{#1}}
\newcommand{\relation}[2]{{#1}_{#2}}
\newcommand{\f}[2]{#1(#2)}
\newcommand{\id}[1]{\text{id}_{#1}}
%\newcommand{\DeclareMathOperator}[1]{\text{#1}}

% Sets
%\renewcommand{\notin}{\tiny{\not\in}}
%\renewcommand{\ni}{\small \ni}
\DeclareMathOperator\opcard{card}
\newcommand{\card}[1]{\opcard(#1)}
%
\def\contains{\supseteq }
\def\cuts{\cap}
\DeclareMathOperator\cvxhull{co}
\newcommand{\co}[1]{\cvxhull(#1)}
\newcommand{\set}[2]{\{#1: #2\}}
\newcommand{\singleton}[1]{\{#1\}}
\newcommand{\interior}[1]{\overset{\circ}{#1}}
\newcommand{\closure}[1]{\overline{#1}}


% Arithmetics
\newcommand{\ceil}[1]{\lceil #1 \rceilf}
% Analysis
%\newcommand\magnitude[1]{\left\lvert\, #1 \,\right\rvert}
\newcommand\magnitude[1]{\abs{#1}}
\newcommand{\norma}[2]{\norm{#2}_{#1}}
%\renewcommand{\norm}[2]{\norm{#2}_{#1}}
\def\weakstar{\text{weak}^\ast\text{-}}
% Topology
\newcommand{\localbase}[1]{\mathscr #1}

%Iverson bracket
\newcommand{\boolean}[1]{\left[\,#1\,\right]}

% limits
\newcommand{\tendsto}[2]{\underset{#1\to #2}{\longrightarrow}}

% Variables
\newcommand{\varit}[1]{\mathit{#1}}
\def\vart{\varit{t}}
%Usual terms
\def\ie{\textit{i.e.} }
\def\eg{\textit{e.g.} }
\def\cf{\textit{cf.\,}}
\def\iif{{\bf iff} }
\def\wlg{{\bf wlg }} % Without loss og generality
\def\then{\Rightarrow}
\def\therefore{\Rightarrow}
\def\since{\Leftarrow}

% Citations
\newcommand{\citehere}[2]{\overset{(#1)}{#2}}
\newcommand{\citeq}[1]{\citehere{#1}{=}}
%\newcomand{\citeineq}[1]{\citehere{#1}{=}}
\newcommand{\citeleq}[1]{\citehere{#1}{\leq}}
\newcommand{\citegeq}[1]{\citehere{#1}{\geq}}
\newcommand{\citeleast}[1]{\citehere{#1}{<}}
\newcommand{\citegreater}[1]{\citehere{#1}{>}}

\newcommand{\citesubset}[1]{\citehere{#1}{\subset}}
\newcommand{\citesupset}[1]{\citehere{#1}{\supseteq }}

\newcommand{\citethen}[1]{\citehere{#1}{\Rightarrow}}
%\newcommand{\citesince}[1]{\citehere{#1}{\Leftarrow}}
\newcommand{\citeresult}[2]{#1 of #2}
\newcommand{\citeresultFA}[1]{\citeresult{#1}{\cite{FA}}}

\newcommand{\citin}[1]{\citehere{#1}{\in}}
% Misc
\newcommand{\underbarwithindex}[2]{\underline{#1}\,\!_{#2}}
\newcommand{\dy}[1]{{\function{dyadic}}(#1)}
\def\ddy{{\function{decay}}}

%--------------END OF PRIMITIVES--------------------------------
\def\ROOT{./}
\def\TITLE{
  Solutions to some exercises from Walter Rudin's \textit{Functional Analysis}
}
\def\EMAIL{}
\def\AUTHOR{gitcordier}

\begin{document}
%\begin{abstract}
%    \input{\ROOT/abstract.tex}
%\end{abstract}
\title{\TITLE}
\author{\AUTHOR}
\date{\today}
\maketitle

% FORMAT ENUMERATION (DEFAULT. OPTIONS: ALPH, ARABIC, ROMAN,…)
\renewcommand{\labelenumi}{$(\textit{\alph{enumi}}\,)$}
% IF LANG = @fr
%\renewcommand{\chaptername}{Chapitre}
%
% CHAPTER  NAME :
\frontmatter
\tableofcontents
%\clearpage

%\printglossary

\renewcommand{\labelenumi}{\arabic{enumi}.}
%\addcontentsline{toc}{chapter}{Notations and conventions}
\chapter{Notations and Conventions}
\section*{Logic}
\begin{enumerate}
\item{{\bf Halmos' iff.} \iif is a short for ``if and only if".}
\item{{\bf Definitions (of values) with $\Def$.} Given variables %
$\varit{a}$ and $\varit{b}$, %
$a\Def b$ means that $\varit{a}$ is defined as equal to $\varit{b}$.}
\item{{\bf $\equiv$.} $a\equiv b$ means that there exists a ``natural'' %
bijection $\to$ that maps $a$ to $b$; which let us identify $a$ with $b$. %
In a metric space context, $a\equiv b$ means that $\to$ is isometric.}
\item{{\bf Definitions (formul\ae).} Definitions use the \iif format. %
In other words, every definition has a ``only if''. %
}
\item{{\bf Iverson notation.} Given a boolean expression $\Phi$, %
$\boolean{\Phi}$ returns the truth value of $\Phi$, encoded as follows, %
%
  \begin{align} \nonumber
    \boolean{\Phi}\Def 
    \begin{cases}
      0 & \quad\quad \text{if } \Phi \text{ is false;} \\
      1 & \quad\quad \text{if } \Phi \text{ is true.}
    \end{cases}
  \end{align}

For example, $\boolean{1 > 0} = 1$ but $\boolean{ \sqrt{2} \in \Q} = 0$.
}
\end{enumerate}
%\section{Vector spaces}
%\begin{enumerate}
%  \item{If $X$ is a vector space of base $B$ and $e$ is an element of B, }
%\end{enumerate}
\section*{Topological vector spaces}
\begin{enumerate}
\item{{\bf Product space}}
\item {\bf Scalar field.} The usual (complete) scalar field is $\C$. %
A property, \eg linearity, that is true on $\C$ is also true on $\R$. %
The complex case is then a {\it special case} of the real one. %
Sometimes, this specialization is not purely formal. %
For example, theorem 12.7 of \cite{FA} asserts that, in a Hilbert space $H$ %
equipped with the inner product $\bra{\,\cdot\,}\ket{\,\cdot\,}$, %
every nonzero linear continuous operator $T$ ``breaks orthogonality'', %
in the sense that there always exists $x=x(T)$ in $H$ that satisfies %
%
  $\bra{Tx}\ket{x} \neq 0$. %
%
The proof of this theorem strongly depends on the complex field. %
Actually, a real counterpart does not exists. %
To see that, consider the $90^\circ$ rotations of the euclidian plane. %
%
Nevertheless, {\it unless the contrary is explicitely mentioned}, %
the exension to the real case will always be obvious. %
So, taking $\C$ as the scalar field shall mean %
%
\begin{quote}{\it %
Instead of letting the scalar field undefined, we choose $\C$ for the sake of %
expessivity. But considering $\R$ instead of %
$\C$ would actually make no difference here
}. %
\end{quote}
%
\item {\bf Finite dimensional spaces}. %
It may be customary to identify, without loss of generality, %
any vector space of finite dimension $n$ with $\C^n$. % 
%
Such identification is relevant in the sense that all vector spaces that %
share common dimension $n$ are actually topological vector spaces that are %
homeomorphic each others. \\
\\
To see that, let $\mathit{Y}$ run through all $n$-dimensional subspaces. 
For instance, $Y$ be $\C^n$, %
or the set of all complex %
polynomials of degree smaller than $n$, %
or the set of complex matrices of size $(n/d, d)$ ($d |n)$. \\
\\
It is easy to get an isomorphism $f$ of $\C^n$ onto $Y$. %
To do so, let us pick a base $F_Y$ of $Y$: %
There so exists a one-to-one mapping of $F_{\C^n}$ onto $ F_Y$, %
and such mapping extends to an isomorphism $f: \C^n \to Y$. \\
\\
Additionally, $Y$ is a vector subspace of some topological space %
$(X, \tau_X)$. %
To see that, assume that no such $X\neq Y$ exists then equipp $X=Y$ %
with the norm %
%
\begin{align}
  \| \, \|_{Y, 2}: Y & \to \R \\
            y & \mapsto \| f^{\,\minus 1} (y)\|_2 \nonumber, 
\end{align}
%
where $\| \, \|_2$ is the Euclidian norm of $\C^n$. %
%
As a consequence, $Y$ is always a topological vector space of topology %
%
\begin{align}\label{inherited norm}
  \tau_Y \triangleq \{ Y\cap U: U \in \tau_X\}, 
\end{align}
%
for some $\tau_X$. %
%
We now use the Section 1.21 of \cite{FA} to conclude that %
$f$ is more specifically an homeomorphism of $Y$ onto $Y\subseteq X$. %
%
Given two copies $(\mathit{X}_i, \mathit{Y}_i, \mathit{f}_i)$, $(i=1, 2)$ of the variable %
$(\mathit{X}, \mathit{Y}, \mathit{f}\,)$, %
we then obtain the following commutative diagram, %
%
\begin{equation}
\begin{tikzpicture}[-stealth,
  label/.style = { font=\footnotesize }]
  \matrix (m)
    [
      matrix of math nodes,
      row sep    = 4em,
      column sep = 4em
    ]
    {
         & \C^n &     \\
      Y_1&      & Y_2 \\
    };
  \path (m-2-1) edge node [above,label]  {$f_1^{-1}$} (m-1-2);
  \path (m-1-2) edge node [above,label]  {$f_2$} (m-2-3);
  \path (m-2-1) edge node [above,label]  {$\varphi$} (m-2-3);
\end{tikzpicture}
\end{equation}
%
It is now clear that all $Y$ are homeomorphic each other. %
The special case $Y_1 = Y_2$ means that $\tau_{Y_1} = \tau_{Y_2}$. 
In other words, each vector space $Y$ has a unique topology $\tau_Y$; %
the embedding $Y\subseteq X$ no longer matters. % 
% 
$\tau_y$ is therefore induced by $\|\,\|_{Y, 2}$, which implies that %
$Y$ is normable. $Y$ can now be seen as a normed space, %
of arbitray norm $\|\,\|_{Y}$. %
%
Finally, we reach the equivalence of all such norms $\| \, \|_Y$, %
in the sense that %
%
\begin{align}
  A \| y_1\|_{Y_1} \leq \| \varphi(y_1) \|_{Y_2} \leq B\| y_1 \|_{Y_1} \quad (y_1 \in Y_1)
\end{align}
%
for positive constants $\mathit{A}, \mathit{B}$. %
For instance, choose %
%
  $B = \sup\{\| \varphi(y_1) \|_{Y_2}: \| y \|_{Y_1} = 1\}$. %
%
\end{enumerate}
% END


\mainmatter
%\part{Content}
\renewcommand{\thechapter}{\arabic{chapter}}
\renewcommand{\thesection}{\arabic{section}}
\renewcommand{\thesubsection}{\arabic{subsection}}
\chapter{Topological Vector Spaces}
%\section{Exercise 1. Basic results}
%%:1
\renewcommand{\labelenumi}{(\alph{enumi})} 
\textit{Suppose $X$ is a vector space. All sets mentioned below are understood 
  to be subsets of $X$. Prove the following statements from the axioms 
  as given as in section 1.4.
\begin{enumerate}
\item{If $x,\,y\in X$ there is a unique $z\in X$ such that $x+z=y$.}
\item{ $0\cdot x=0=\alpha\cdot 0 \quad (\alpha\in\field, x\in X)$.}
\item{ $2A\subseteq A+A$.}
\item{ $A$ is convex if and only if $(s+t)A=sA+tA$ %
  for all positive scalars $s$ and $t$.}
\item{ Every union (and intersection) of balanced sets is balanced.}
\item{ Every intersection of convex sets is convex.}
\item{ If $\Gamma$ is a collection of convex sets that is totally ordered by 
  set inclusion, then the union of all members of $\Gamma$ is convex.}
\item{ If $A$ and $B$ are convex, so is $A+B$.}
\item{ If $A$ and $B$ are balanced, so is $A+B$.}
\item{ Show that parts (f\,), (g) and (h) hold with subspaces in place of 
  convex sets.}
\end{enumerate}
}
%
\renewcommand{\labelenumi}{(\alph{enumi})} 
\begin{proof}
\begin{enumerate}
%: (a)
\item Such property only depends on the group structure of $X$: Each $x$ in
$X$ has an opposite $\minus x$. Let $x'$ be any opposite of $x$, so that
${x-x=0}=x+x'$. %
Thus, $\minus x +x -x =  \minus x + x + x' $, %
which is equivalent to $\minus x = x'$. So is established the uniqueness of %
$\minus x $. %
%
It is now clear that $x+z=y$ \iif $z=\minus x +y$, %
which asserts both the existence and the uniqueness of $z$.
%: (b)
\item Remark that %
%
\begin{align}
  0\cdot x & =(0+0)\cdot x=0\cdot x+0\cdot x \\
           & =(0+0)\cdot x=0 +0\cdot x 
\end{align} 
%
then conclude from (a) that $0\cdot x=0$. So, %
\begin{align} \label{inverse of x}
  0=0\cdot x=(1-1)\cdot x &=x+(\minus 1)\cdot x
  \Rightarrow \minus 1\cdot x= \minus x.
\end{align}
%
Finally, %
%
\begin{align}
  \alpha\cdot 0\overset{(\ref{inverse of x})}{=}
  \alpha\cdot (x+(\minus 1\cdot x))
  = \alpha \cdot x + \alpha \cdot (\minus 1) \cdot x 
  = (\alpha-\alpha )  \cdot x =0\cdot x = 0,
\end{align}
%
which proves (b).
%
%: (c)
\item Remark that 
%
\begin{align}
  2x =(1+1) x = x + x
\end{align}
%
for every $x$ in $X$, and so conclude that %
%
\begin{align}\label{double lies in sum}
  2A = \{2x: x\in A \} 
  = \{x + x: x \in A \} 
  \subseteq \{ x + y : (x,\,y) \in A^2 \} 
  = A+A
\end{align}
%
for all subsets $A$ of $X$; which proves (c). %
%: (d)
\item If $A$ is convex, then %
%
\begin{align}
  A \subseteq \frac{s}{s+t} A + \frac{t}{s+t} A \subseteq A;
\end{align}
%
which is %
%
\begin{align}
  sA + tA = (s+t)A.
\end{align}
%
Conversely, the special case $s+t=1$ is %
%
\begin{align}
  sA + (1-s)A = A.
\end{align}
%
The latter extends to $s=0$, since %
%
\begin{align}
  0A + A \overset{(b)}{=}\{0\}+A=A.
\end{align}
%
The extension to $s=1$ is analogously established %
(or simply use the fact that $+$ is commutative!).
So ends the proof. %
%: (e)
\item Let $A$ range over $B$ a collection of balanced subsets, so that %
%
\begin{align}
  \alpha \bigcap B \subseteq  \alpha A \subseteq A \subseteq \bigcup B 
\end{align}
%
for all scalars $\alpha$ of magnitude $\leq 1$. %
The inclusion $\alpha \bigcap B \subseteq A$ establishes the first part. %
Now remark that %
%
\begin{align}
  \alpha A  \subseteq \bigcup {B} 
\end{align}
%
implies %
%
\begin{align}
  \alpha \bigcup {B} \subseteq \bigcup {B};
\end{align}
which achieves the proof. %
%
%: (f)
\item Let $A$ range over $C$ a collection of convex subsets, so that %
%
\begin{align}
  (s+t) \bigcap C \subseteq s\bigcap C + t\bigcap C \subseteq  sA + tA 
  \overset{(d)}{\subseteq} (s+t)A
\end{align}
%
for all positives scalars $\mathit{s}$, $\mathit{t}$. %
Inclusions at both extremities force %
%
\begin{align}
  s\bigcap C  + t\bigcap C = (s+t) \bigcap C.
\end{align}
%
We now conclude from (d) that the intersection of $C$ is convex. %
So ends the proof.
%: (g)
\item Skip all trivial cases %
%
  $\Gamma = \emptyset$, %
  $\{ \emptyset\}$, %
  $\{\{x\}\}$, %
  $\{\emptyset$, %
  $\{x\}\}$ %
%
then pick $x_1, x_2$ in $\bigcup \Gamma$, 
so that each $x_i$ ($i=1, 2)$ lies in some $C_i \in \Gamma$. %
%
Since $\Gamma$ is totally ordered by set inclusion, we henceforth assume %
without loss of generality that $C_1$ is a subset of $C_2$. %
%
So, $x_1, x_2$ are now elements of the convex set $C_2$. %
Every convex combination of our $x_i$'s is then in %
$C_2 \subseteq \bigcup \Gamma$. Hence (g). %
%
%: (h)
\item Simply remark that 
%
\begin{align}
  s (A+B) + t (A+B) = s A+ t A +s B +t B = (s+t)(A+B)
\end{align}
%
for all positive scalars $\mathit{s}$ and $\mathit{t}$, %
then conclude from (d) that $A + B$ is convex. %
%: (i) 
\item Given any $\alpha$ from the closed unit disc, %
%
\begin{align}
\alpha(A+B)=\alpha A+ \alpha B \subseteq A+B. 
\end{align}
%
There is no more to prove: $A+B$ is balanced. %
%: (j)
\item Our proof will be based on the following lemma, %
%
\renewcommand{\labelenumii}{(\roman{enumii})}
\textit{
\begin{quote}
If $S$ is nonempty, then each of the following three properties
\begin{enumerate}
\item $S$ is a vector subspace of $X$;
\item $S$ is convex balanced such that $S + S = S$;
\item $S$ is convex balanced such that $\lambda S=S\quad (\lambda > 0)$
\end{enumerate}
implies the other two.
\end{quote}
}
%
To prove the lemma, let $\mathit{S}$ %
run through all nonempty subsets of $X$. %
First, assume that (i) holds: Clearly, every $S$ is convex balanced. %
Moreover, $S+S \subseteq S$.  Conversely, $S = S + \{0\} \subseteq S + S $; %
which establishes (ii). %
%
Next, assume (only) (ii): A proof by induction shows that %
%
\begin{align}\label{induction nS}
  nS = (n-1)  S + S = S + S = S \quad (n=1,2,3, \dots)
\end{align}
%
with the help of (b) and (d). %
Pick $\lambda >0$ then choose $n$ so large that $1 < n \lambda < n^2$. %
Thus, %
%
\begin{align}
  nS \overset{(\ref{induction nS})}{\subseteq} S 
  \subseteq n\,\lambda S 
  \subseteq  n^2 S, 
\end{align}
%
since $S$ is balanced. %
For instance, set %
%
  $n = \lceil{1/\lambda}\rceil + \lceil{\lambda}\rceil$. %
%
Dividing the latter inclusions by $n$ shows that %
%
\begin{align} 
  S \subseteq \lambda S \subseteq nS \overset{(\ref{induction nS})}{\subseteq} S,
\end{align}
%
which is (iii). Finally, dropping (ii) in favor of (iii) leads to %
%
\begin{align} 
  \alpha  S +\beta  S 
  \overset{(a)}{=} |\alpha | S + | \beta | S 
  \overset{(d)}{=} (|\alpha | + | \beta| )S 
  \overset{(iii)}{=} S 
  %
  \quad (|\alpha| + |\beta| > 0);
\end{align}
%
where the equality at the left holds as $S$ is balanced. %
%
Moreover (under the sole assumption that $S$ is balanced), %
this extends to $|\alpha| + |\beta| = 0$, as follows, %
%
\begin{align} 
  \alpha S + \beta S  = 0S + 0S\overset{(b)}{=} \{0\} 
  \overset{(b)}{=} 0S \subseteq S.
\end{align}
%
Hence (i), which achieves the lemma's proof. %
We will now offer a straightforward proof of (j). \\
\\
Let $V$ be a collection of vector spaces of $X$, %
of intersection $I$ and union $U$. 
%
First, remark that every member of $V$ is convex balanced: %
So is $I$ (combine (e) with (f)). %
%
Next, let $\mathit{Y}$ range over $V$, so that %
%
\begin{align}
  I + I \subseteq Y + Y \subseteq  Y; 
\end{align}
%
which yields
%
\begin{align}
  I + I = I 
\end{align}
%
(the fact that $I  = I + \{0\} \subseteq I + I$ was tacitely used). %
%
It now follows from the lemma's (ii) $\Rightarrow$ (i) that %
$I$ is a vector subspace of $X$. %
%
Now temporarily assume that $S$ is totally ordered by set inclusion: %
Combining (e) with (g) establishes that $U$ is convex balanced. %
%
To show that $U$ is more specifically a vector subspace, %
we first remark that such total order implies that either %
$Z \subseteq Y$ or $Y \subseteq Z$, as $\mathit{Z}$ ranges over $V$. %
A straightforward consequence is that 
%
\begin{align}
  Y \subseteq Y + Z  \subseteq Y\cup Z.
\end{align}
%
Another one is that $Y \cup Z$ ranges over $V$ as well. %
Combined with the latter inclusions, this leads to %
%
\begin{align}
  U \subseteq U  + U \subseteq U.
\end{align}
%
It then follows from the lemma's (ii) $\Rightarrow$ (i) that %
$U$ is a vector subspace of $X$. %
%
Finally, let $\mathit{A},\mathit{B}$ run through all vector subspaces of $X$: %
Combining (h) with (i) proves that $A+B$ is convex balanced as well. %
%
Furthermore, %
%
\begin{align}
  A + B \overset{(i) \Rightarrow (ii)}{=} (A + A) + (B + B) = (A + B) + (A + B),
\end{align}
% 
where the equality at the right holds as $X$ is an abelian group. %
We now conclude from (ii) that any $A+B$ is a vector subspace of $X$. %
%
So ends the proof. %
\end{enumerate}
\end{proof}
% END

%\section{Exercise 2. Convex hull}
%\textit{The convex hull of a set $A$ in a vector space $X$ is the set of all %
convex combinations of members of $A$, that is the set of all sums %
%
  $t_1 x_1 +\cdots +t_n x_n$ %
%
in which $x_i \in A,\, t_i \geq 0$, $\sum t_i = 1$; $n$ is arbitrary. 
%
Prove that the convex hull of a set $A$ is convex and that is the intersection 
of all convex sets that contain $A$.}
%
\begin{proof} The convex hull of a set $S$ will be denoted by $\co{S}$. %
Remark that $S \subset \co{S}$ %
(to see that, take $t_1 = 1$ for each $x_1$ in $S$) and that %
$\co{A} \subset \co{B}$ where $A \subset B$ (obvious).
\\
Our proof will directly derive from the following lemma, 
\renewcommand{\labelenumi}{(\roman{enumi})} 
\begin{quote}
\textit{Let $S$ be a subset of a vector space $X$: Its convex hull $\co{S}$ %
is convex and the following statements %
\begin{enumerate}
  \item $S$ is convex; %
  \item{
    %
      $s_1 S + \dots + s_n S = (s_1 + \cdots + s_n) S$ %
      %
      for all positive scalar variables $\mathit{s_1}, \dots, \mathit{s_n}$; %
  }
  \item{
    %
    $\mathit{t_1} S + \dots + \mathit{t_n} S = S$ %
    %
    for all positive scalar variables $\mathit{s_1}, \dots, \mathit{s_n}$ %
    such that %
    %
    $s_1 + \cdots + s_n = 1$; %
  }
  \item $\co{S} = S$
\end{enumerate}
are equivalent.}
\end{quote}
%
More specifically, %
%
our proof of the second part will only depend on (i) $\Rightarrow$ (iv). \\
\\
From now on, we skip the trivial case $S=\emptyset$ %
then only consider nonempty sets. %
To prove the first part, let $\mathit{a}$, $\mathit{b}$ %
run through the convex combination(s) of $S$, so that %
%
  $a = t_1 x_1 + \cdots + t_n x_n$ and %
  $b = t_{n+1} x_{n+1} + \cdots + t_{n+p} x_{n+p}$ %
%
for some $(\mathit{t_i}, \mathit{x_i})$. %
%
Every sum $sa + (1-s)b $ ($0\leq s \leq 1$) is then a convex combination of %
$\mathit{x_1}, \dots, \mathit{x_{n+p}}$, since %
%
\begin{align}
  sa + (1-s)b = \sum_{i=1}^n st_i x_i + \sum_{i=n+1}^{n+p} (1-s)t_i x_i 
\end{align}
%
and
\begin{align}
  \sum_{i=1}^n st_i + \sum_{i=n+1}^{n+p} (1-s)t_i &= 
  s\sum_{i=1}^n t_i +(1-s) \sum_{i=n+1}^{n+p} t_i  = 1.
\end{align}
%
In terms of sets $S$, this reads %
\begin{align}
  s\co{S} + (1-s)\co{S} \subset \co{S}; 
\end{align}
which was our fist goal. %
We now aim at the equivalence %
%
(i) $\Rightarrow$ $\cdots$ $\Rightarrow$ (iv) $\Rightarrow$ (i): %
%
An easy proof by induction makes the implication (i) $\Rightarrow$ (ii) %
directly come from (d) of the above exercise 1, chapter 1. %
%
(iii) is a special case of (ii),
and the implication (iii) $\Rightarrow$ (iv) derives from the definition of %
the convex hull. %
%
We now close the chain with (iv) $\Rightarrow$ (i), %
by remarking that $S$ is convex whether $S = \co{S}$. %
%  
The lemma being proved, let us establish the second part. %
To do so, start from $F\ni\co{A}$ then possibly enrich $F$ the following way: %
\begin{align}
  B \in F \Rightarrow B \text{ is convex and contains }A.
\end{align}
Note that our definition of $F$ is weaker than the primary assumption %
``[$F$ only encompasses] \textit{all convex sets that contain A}", %
which is the special case %
%
\begin{align}
  B \in F \Leftrightarrow B \text{ is convex and contains }A.
\end{align}
%
In any case, the key ingredient is that $\co{A} \in F$ implies %
%
\begin{align}
 \co{A} \supset \bigcap_{B \in F} B.
\end{align}
%
Conversely, the next formula %
%
\begin{align}
  \co{A} \subset \co{B} \overset{(i) \Rightarrow (iv)}{=} B \quad (B \in F) 
\end{align}
%
is valid and implies %
\begin{align}
  \co{A} \subset \bigcap_{B \in F} B. 
\end{align}
%
So ends the proof
\end{proof}
%
\newpage
\section{Exercise 7. Metrizability \& number theory}
%\section{Exercise 7. Metrizability \& number theory}
\textit{
Let be X the vector space of all complex functions on the unit interval 
$[0, 1]$, topologized by the family of seminorms 
%
  \begin{align}
    p_{x}(f)=|f(x)| \quad (0\leq x\leq 1).\nonumber
  \end{align}
%
This topology is called the topology of pointwise convergence. 
Justify this terminology.
Show that there is a sequence $\{f_n\}$ in X such that (a) $\{f_n\}$ converges 
to $0$ as $n \to\infty$, but (b) if $\{γ_n\}$ is any sequence of scalars such 
that $γ_n\to\infty$ then $\{γ_nf_n\}$ does not converge to $0$. 
(Use the fact that the collection of all complex sequences converging to $0$ 
has the same cardinality as $[0, 1]$.)
This shows that metrizability cannot be omited in (b) of Theorem 1.28.
}
\begin{proof}
Our justification consists in proving that $\tau$-convergence and pointwise 
convergence are the same one. 
%
To do so, remark first that the family of the seminorms $p_{x}$ is separating.
By [1.37], the collection $\mathscr{B}$ of all finite intersections 
of the sets 
%
  \begin{align}
    V\up{(x, k} 
      \Def 
    \singleton{p_x < 2^{\minus k}} 
      \quad 
    (x \in [0, 1], k \in \N)
  \end{align}
%
is then a local base for a topology $\tau$ on $X$. Given 
%
  $\set{f_n}{\counting{n}}$, 
%
we set
%
\newcommand\off[1]{\function{off}(#1)}
%
\begin{align}
  \off{U} \Def \sum_{n=1}^\infty [f_n \notin U] \quad (U\in\tau),
\end{align}
%
with the convention $\off{U}=\infty$ whether the sum has no finite support. 
So, 
%
  \begin{align}
    %
    \label{Inequality boolean series}
    %
    \sum_{i=1}^m \off{U\up{i}} 
      = 
    \sum_{n=1}^\infty \sum_{i=1}^m [f_n \notin U\up{i}]
      \geq 
    \off{U\up{1} \cuts \cdots \cuts U\up{m}}
  \end{align}
%
We first assume that $\singleton{f_n}$ $\tau$-converges to some $f$ in $X$, \ie
%
  \begin{align}
    \off{f+V} < \infty \quad(V \in \mathscr{B}).
  \end{align}
%
The special cases $V=V\up{x, k}$ mean the pointwise convergence of 
$\singleton{f_n}$. 
%
Conversely, assume that $\singleton{f_n}$ does not $\tau$-converges to any $g$ 
in $X$, \ie 
%
  \begin{align}
    %
    \label{Divergence}
    %
    \forall g \in X, \exists V\up{g} \in \localbase{B}: 
      \off{g+V\up{g}} = \infty. 
    %
  \end{align}
%
Given $g$, $V\up{g}$ is then an intersection
%
  $
    V\up{x\up{1}, k\up{1}} \cap \cdots \cap 
    V\up{x\up{m}, k\up{m}} 
  $.
%
%So, (\ref{1.7 Inequality boolean series}) implies 
Thus
%
  \begin{align}
    \sum_{i=1}^m \off{g + V\up{x\up{i}, k\up{i}}} 
      %
        \citegeq{\ref{Inequality boolean series}} 
      %
    \off{g + V\up{g}} 
    % 
      \citeq{\ref{Divergence}} 
    %
    \infty .
  \end{align}
%
One of the sum $\off{g + V\up{x\up{i}, k\up{i}}}$ must then be $\infty$. 
This implies that convergence of $f_n$ to $g$ fails at point $x_i$.
$g$ being arbitrary, we so conclude that $f_n$ does not converge pointwise.
We have just proved that 
%
  $\tau$-convergence is a rewording of pointwise convergence.
%
% SECOND PART
We now aim to prove the second part.
%
From now on, 
%
  $\varit{k}$, $\varit{n}$ and $\varit{p}$ 
%
run on $\N_+$. Let $\dy{x}$ be the usual dyadic expansion of a real number $x$, 
so that $\dy{x}$ is an aperiodic binary sequence \iif $x$ is irrational. 
%
Define
%
\begin{align}
  %
  \label{f_n(x) definition}
  %
  f_n(x) 
    \Def 
  \begin{cases}
      2^{-\sum_{k= 1}^{n} {\dy{x}_{-k}}} & (x \in [0, 1]\setminus \Q )\\
      0                                            & (x \in [0,1]\cap \Q)
    \end{cases}
\end{align}
%
so that $f_n(x) \tendsto{n}{\infty} 0$, %
%
and take scalars 
%
  $\gamma_n$ 
%
such that $\tendsto{n}{\infty} \infty$, \ie 
%
  at fixed $p$, $\gamma_{n}$ is greater than $2^{p}$ for almost all $n$.
%
Next, choose $n\up{p}$ among those \textit{almost all} $n$ that are 
large enough to satisfy 
%
  \begin{align}
    n\up{p-1} - n\up{p-2} < n\up{p}- n\up{p-1} 
  \end{align}
%
(start with $n\up{\minus 1} = n\up{0} = 0$) and so obtain 
%
  \begin{align}
    2^p < \gamma_{n\up{p}}:\, 
    %
      0< n\up{p} - n\up{p-1}\tendsto{p}{\infty} \infty.
    %
  \end{align}
%
The indicator $\chi$ of 
%
  $\{n\up{1}, n\up{2}, \dots\}$
%
is then aperiodic, \ie 
%
  \def\xgamma{x^{(\gamma)}}
  \begin{align}
    \xgamma 
      \Def
    \sum_{k=1}^\infty \chi_k 2^{\minus k} 
      %\notin \Q
  \end{align}
is irrational. Consequently,
%
  \begin{align}
    \dy{\xgamma}_{\minus k} &= \chi_{k}.
  \end{align}
%
We now easily see that
%
  \begin{align}
    \chi_1 + \cdots + \chi_{n\up{p}} = p, 
  \end{align}
%
which, combined with (\ref{f_n(x) definition}), yields
%
  \begin{align}
    f_{n\up{p}}(\xgamma) = 2^{\minus p}.
  \end{align}
%
Finally,
%
  \begin{align}
    \gamma_{n\up{p}} f_{n\up{p}}(\xgamma) > 1.
  \end{align}
%
We have so established that the subsequence 
%
  $\singleton{\gamma_{n\up{p}}f_{n\up{p}}}$ 
% 
does not tend pointwise to $0$, hence neither does the whole sequence 
%
  $\singleton{\gamma_{n}f_n}$.
%
In other words, (b) holds, which is in violent contrast with [1.28]: 
$X$ is then not metrizable. So ends the proof.
\end{proof}
% END

\newpage
\section{Exercise 9. Quotient map}
\textit{Suppose
\begin{enumerate}
  \item $X$ and $Y$ are topological vector spaces,
  \item $\Lambda: X\to Y$ is linear.
  \item $N$ is a closed subspace of $X$,
  \item$\pi: X\to X/N$ is the quotient map, and
  \item $\Lambda x=0$ for every $x\in N$.
\end{enumerate}
Prove that there is a unique $f:X/N\to Y$ which satisfies 
%
  $\Lambda=f\circ \pi$, 
%
that is, 
%
  $\Lambda x=f(\pi (x))$ for all $x\in X$. 
%
Prove that $f$ is linear and that $\Lambda$ is continuous if and only if 
%
  $f $ is continuous. 
%
Also, $\Lambda$ is open if and only if $f$ is open.}
%
\begin{proof}
The equation $\Lambda = f \circ \pi$ has necessarily a unique solution, 
which is the binary relation 
%
  \begin{align}\label{1_9: defintion of f.}
     f \Def\set{(\pi x, \Lambda x)}{x \in X} \subset X/N \times Y.
    \end{align}
%
To ensure that $f$ is actually a mapping, simply remark that 
the linearity of $\Lambda$ implies 
%
  \begin{align}
    %\forall ( x,  x') \in X^2: 
    %
    \Lambda x \neq \Lambda  x' \then \pi x' \neq \pi x'.
  \end{align}
%
It straightforwardly derives from (\ref{1_9: defintion of f.}) that 
$f$ inherits linearity from $\pi$ and $\Lambda$. Now remark that 
%
  \begin{align}
    \pi x = N   
      \citethen{f\text{ linear}} 
    f(\pi x) = 0 
      \citethen{\ref{1_9: defintion of f.}} 
    \Lambda x = 0 
      \then \pi x = N 
\end{align}
and so conclude that $f$ is also one-to-one.
%
Now assume $f$ to be continuous. Then so is 
%
  $\Lambda = f\circ \pi $, 
% 
by (a) of [1.41]. 
%
Conversely, 
%
if $\Lambda$ is continuous, then for each neighborhood $V$ of $0_Y$ 
there exists a neighborhood $U$ of $0_X$ such that
%
  \begin{align}
    \Lambda(U) = f\left(\pi(U)\right) 
      \subset 
    V.
  \end{align}
%
Since $\pi$ is open (see (a) of [1.41]), $\pi(U)$ is a neighborhood of 
%
  $N=0_{X/N}$: 
%s
This is sufficient to establish that the linear mapping $f$ is continuous.
%
If $f$ is open, so is $\Lambda = f\circ \pi$, by (a) of [1.41]. 
%
Conversely, let  
%
  \begin{align}
    W \Def \pi (V) \subset  X/N \quad (V \text{ neighborhood of } 0_X) 
  \end{align}
%
range over all neighborhoods of $N$, as $\Lambda$ is kept open: So is 
%
  \begin{align}
    \Lambda(V) = f \left(\pi(V)\right) = f(W).  
  \end{align}
%
The linear mapping $f$ is then open. 
\end{proof}



\newpage
\section{Exercise 10. An open mapping theorem}
%\section{1.10 Exercise 10. An open mapping theorem}
\textit{Suppose that X and Y are topological vector spaces,
%
  $\dim Y < \infty$,
%
$\Lambda : X \to Y$ is linear, and $\Lambda(X) = Y$.
%
  \begin{enumerate}
    \item{
      Prove that $\Lambda$ is an open mapping.}
    \item{
      Assume, in addition, that the null space of $\Lambda$ is closed, 
      and prove that $\Lambda$ is continuous.
    }
  \end{enumerate}
  %
}
%
\begin{proof}
  \begin{enumerate}
    \item{
      Let $e$ range over a base of $Y$: 
      For each $e$, there exists $x_e$ in $X$ such that 
      %
        $\Lambda(x_e)=e$, 
      % 
      since $\Lambda$ is onto.So,
      %
        \begin{align}\label{1_10_sum}
          y = \sum_{e} y_e \Lambda x_e \quad (y\in Y).
        \end{align}
      %
      The sequence $\singleton{x_e}$ is finite hence bounded: 
        Given $V$ a balanced neighborhood of the origin, 
        there exists a positive scalar $s$ such that  
      %
        \begin{align}
          x_e \in s V
        \end{align}
      %
      for all $x_e$.
      Combining this with (\ref{1_10_sum}) shows that 
      %
        \begin{align}
          y \in \sum_e \Lambda (V) \quad (y\in Y: |y_e| < s^{\minus 1}).
        \end{align}
      %
  }
    \item{
      Since $N$ is closed, $\pi$ continously maps $X$ onto $X/N$, 
      another topological (Hausdorff) vector space, see [1.41]. 
      %
      Now take $f$ as in Exercise 9: 
      Since $\Lambda$ is onto, the first isomorphism theorem asserts that 
      %
        $f$ is an isomorphism of $X/N$ onto $Y$. 
      %
      Consequently, $X/N$ has dimension $n=\dim Y$. 
      $f$ is then an homeomorphism of 
      %
        $X/N\equiv \C^{n}$ 
      %
      onto $Y$; see [1.21].
      We have thus established that $f$ is continuous: So is $\Lambda = f\circ \pi$.
    }
  \end{enumerate}
\end{proof}
\newpage
\section{Exercise 14. $\mathscr{D}_K$ equipped with other seminorms}
\textit{Put $K =[0, 1]$ and define $\D_K$ as in Section 1.46. 
Show that the following three families of seminorms 
  (where $n = 0, 1, 2, \dots$) define the same topology on $\D_K$. 
If $D = d/dx$: 
%
  \begin{enumerate}
    \item{
      $\| D^n f \|_\infty = \sup\set{\left| D^n f(x)\right|}{\infty< x< \infty}$
    }
    \item{
      $\| D^n f \|_1 =\int_0^1 \left|D^n f(\varit{x}) \right| d\varit{x}$
    }
    \item{
      $\| D^n f \|_2 = \left\{
        \int_0^1 | D^n f(\varit{x}) |^2  d\varit{x} 
      \right\}^{1/2}.$
    }
  \end{enumerate}
  %
}
%
\begin{proof} 
First, remark that  
%
  \begin{align}\label{1_14_2}
    \| D^n f \|_1 
      \leq 
    \| D^n f \|_2 
      \leq 
    \| D^n f \|_\infty 
      <\infty 
  \end{align}
%
(the inequality on the left is a Cauchy-Schwarz one), 
since $K$ has length $1$. Next, start from 
%
  \begin{align}\label{1_14_3}
    D^n f(x) = \int_{\minus \infty}^x D^{n+1}f
  \end{align}
%
(which is true, since $f$ has a bounded support) to obtain
  \begin{align}\label{1_14_4}
    \left| D^n f(x)\right|
      \leq 
    \int_{\minus\infty}^x \left| D^{n+1}f \right| 
      \leq 
    \|D^{n+1}f \|_1 
  \end{align}
%
hence
%
  \begin{align}\label{1_14_5}
    \| D^n f \|_\infty 
      \leq 
    \| D^{n+1} f \|_1 .
  \end{align}
%
Combining (\ref{1_14_2}) with (\ref{1_14_5}) yields
%
  \begin{align}\label{1_14_6}
    \|D f \|_1 
      \leq 
    \cdots
      \leq
    \| D^n f \|_1 
      \leq 
    \| D^{n} f \|_2 
      \leq 
    \| D^{n} f \|_\infty 
      \leq 
    \| D^{n+1}f\|_1 
      \leq 
    \cdots .
  \end{align}
%
We now define 
  \begin{align}\label{1_14_1}
    V\up{i}_n &       \Def \set{f\in \D_K}{\| f \|_i <1/n}\quad(i=1,2,\infty)\\
    \mathscr{B}\up{i}&\Def \set{V\up{i}_n }{\counting{n}}
  \end{align}
so that (\ref{1_14_6}) is mirrored in terms of neighborhood inclusions, 
as follows,
%
  \begin{align}\label{1_14_7}
    V\up{1}_1 
      \supset
    \cdots 
      \supset 
    V\up{1}_n
      \supset 
    V\up{2}_n 
      \supset 
    V\up{\infty}_n 
      \supset 
    V\up{1}_{n+1} 
      \supset 
    \cdots .
  \end{align}
%
Since 
  $V\up{i}_n\supset V\up{i}_{n+1}$, 
$\mathscr{B}\up{i}$ is the local base of a topology $\tau_i$. 
But the chain (\ref{1_14_7}) forces the $\tau_i$'s to be equals. 
To see that, choose a set $S$ that is $\tau_1$-open at, say $a$: So, 
%
  $V\up{1}_n \subset S-a$  
%
for some $n$. Now $V\up{1}_n \supset V\up{2}_n$ (see (\ref{1_14_7})) forces  
%
  $V\up{2}_n \subset S-a$ , 
%
which implies that $S$ is $\tau_2$-open at $a$.
Similarly, we deduce, still from (\ref{1_14_7}), that 
\begin{align}
  \tau_2\text{-open} 
    \then 
  \tau_\infty\text{-open} 
    \then 
  \tau_1\text{-open}.
\end{align}
So ends the proof.
\end{proof}






\newpage
\section{Exercise 16. Uniqueness of topology for test functions}
\textit{
Prove that the topology of $C(\Omega)$ does not depend on the particular 
choice of $\singleton{K_n}$, as long as this sequence satisfies the conditions 
specified in section 1.44. Do the same for $C^\infty(\Omega)$ (Section 1.46).}
%
\paragraph{Comment}This is an invariance property: 
The function test topology only depends on the existence of the 
supremum-seminorms $p_n$, then, eventually, 
only on the ambient space itself. 
This should be regarded as a very part of the textbook \cite{FA}
%
The proof consists in combining trivial consequences of the local base 
definition with a well-known result (\eg [2.6] in \cite{BigRudin}) 
about intersection of nonempty compact sets. 

\paragraph{Lemma 1} {\it %
Let $X$ be a topological space with a countable local base %
$\mathit{\set{V_n}{\counting{n}}}$. 
If 
%
  $\mathit{\tilde{V}_{n} = V_1 \cuts \cdots \cuts V_n}$, 
%
then every subsequence 
% 
  $\mathit{\singleton{\tilde{V}_{\rho(n)}}}$ 
%
is a decreasing (\ie 
%
  $\mathit{\tilde{V}_{\rho(n)} \contains \tilde{V}_{\rho(n+1)}}$)
%
local base of $\varit{X}$.
}
%
\begin{proof}
The decreasing property is trivial. Now remark that 
%
  $V_n \contains \tilde{V}_{n}$:
%
This shows that 
%
  $\singleton{\tilde{V}_{n}}$ 
% 
is a local base of $X$. Then so is 
%
  $\singleton{\tilde{V}_{\rho(n)}}$,
% 
since $\tilde{V}_{n} \contains \tilde{V}_{\rho(n)}$.
\end{proof}
%
\noindent The following special case 
%
  $V_{n} = \tilde{V}_{n}$ 
% 
is one of the key ingredients:
%: COROLLARY 1 OF LEMMA 1-----------------------------------------------------%
\paragraph{Corollary 1 (special case $V_{n} = \tilde{V}_{n}$)}
{\it Under the same notations of Lemma 1, if $\mathit{\singleton{V_{n}}}$ %
is a decreasing local base, then so is $\mathit{\singleton{V_{\rho(n)}}}$.}
%
%: COROLLARY 2 OF LEMMA 1-----------------------------------------------------%
\paragraph{Corollary 2}{\it %
If 
%
  $\mathit{\mathit{\singleton{Q_n}}}$ 
%
is a sequence of compact sets that satisfies the conditions specified 
in section 1.44, then every subsequence 
%
  $\mathit{\singleton{Q_{\rho(n)}}}$ 
%
also satisfies theses conditions.
%
Furthermore, if $\mathit{\tau_{Q}}$ is the $\mathit{C(\Omega)}$'s 
(respectively $\mathit{C^\infty (\Omega)}$'s) topology of the seminorms %
$\mathit{p_{n}}$, 
as defined in section 1.44 (respectively 1.46), then the seminorms 
%
  $\mathit{p_{\rho(n)}}$ 
%
define the same topology $\mathit{\tau_{Q}}$. %
}
%
\begin{proof}%
%
Let $X$ be $C(\Omega)$ topologized by the seminorms $p_{n}$ 
(the case $X=C^\infty(\Omega)$ is proved the same way).
%
If 
  %
    $V_{n} = \singleton{p_{n} < 1/n}$, 
  %
then 
  %
    $\singleton{V_{n}}$ 
  %
is a decreasing local base of $X$.
%
Moreover,
% 
  \begin{align}
    Q_{\rho(n)} 
      \subset 
    \interior{Q}_{\rho(n) + 1} 
      \subset 
    Q_{\rho(n) + 1} 
      \subset 
    Q_{\rho(n+ 1)}.
  \end{align}
% 
Thus,
%
  \begin{align}
    Q_{\rho(n)} 
      \subset 
    \interior{Q}_{\rho(n+ 1)}.
  \end{align}
%
In other words, 
%
  $Q_{\rho(n)}$ satisfies the conditions specified in section 1.44.
%
%
  $\singleton{p_{\rho(n)}}$
% 
then defines a topology $\tau_{Q_\rho}$ for which  
% 
  $\singleton{V_{\rho(n)}}$ 
%
is a local base. So, 
% 
  $\tau_{Q_\rho} \subset \tau_{Q}$.
%
Conversely, the above corollary asserts that 
%
  $\singleton{V_{\rho(n)}}$ 
%
is a local base of $\tau_{Q}$, which yields  
%
  $\tau_{Q}\subset \tau_{Q_\rho}$.
%
\end{proof}
%: LEMMA 2 -------------------------------------------------------------------%
\paragraph{Lemma 2}{ \it \label{1.16 Lemma 2}
If a sequence of compact sets $\mathit{\singleton{Q_n}}$ satisfies the conditions 
specified in section 1.44, then every compact set $K$ lies in allmost all 
%
  $\mathit{Q^{\,\circ}_n}$, \ie
%
there exists $\varit{m}$ such that 
%
  \begin{align}\mathit{
    K \subset 
    \interior{Q}_m 
      \subset 
    \interior{Q}_{m+1}
      \subset
    \interior{Q}_{m+2}
      \subset
    \cdots.}
  \end{align}
}
%
\begin{proof}
The following definition
%
  \begin{align}
    C_n \Def K \setminus \interior{Q}_n 
  \end{align}
%
shapes $\singleton{C_n}$ as a decreasing sequence of compact\footnote{
  See (b) of 2.5 of \cite{BigRudin}.
} 
sets. We now suppose (to reach a contradiction) that 
% 
  no $C_n$ is empty 
% 
and so conclude\footnote{
  In every Hausdorff space, the intersection of a decreasing sequence of %
  nomempty compact sets is nonempty. %
  This is a corollary of 2.6 of \cite{BigRudin}.
} 
that the $C_n$'s intersection contains a point that is not in any $Q^\circ_n$. 
On the other hand, the conditions specified in [1.44] force the 
% 
  $Q^\circ_n$'s collection  
%
to be an open cover.
% 
This contradiction reveals that 
%
  $C_m = \emptyset$, 
    \ie 
  $K \subset Q^\circ_m$, 
%  
for some $m$.
%
Finally,  
%
  \begin{align}
    K\subset 
    \interior{Q}_m
      \subset
    Q_m
      \subset
    \interior{Q}_{m+1}
      \subset
    Q_{m +1}
      \subset
    \interior{Q}_{m+2}
      \subset
    \cdots.
  \end{align}  
%
\end{proof}
%: THEOREM -------------------------------------------------------------------%
\noindent We are now in a fair position to establish the following:
%\newpage
\paragraph{Theorem}{\it 
The topology of %
%
$\mathit{C(\Omega)}$ %
%
does not depend on the particular choice of %
%
$\mathit{\singleton{K_n}}$, %
%
as long as this sequence satisfies the conditions specified in section 1.44. %
Neither does the topology of %
%
$\mathit{C^{\,\infty} (\Omega)}$, %
%
as long as this sequence satisfies the conditions specified in section 1.44.
}
%
\begin{proof}%
With the second corollary's notations,
% 
  $\tau_{K} = \tau_{K_\lambda}$,
%
for every subsequence $\singleton{K_{\lambda(n)}}$.
% 
Similarly, let 
%
  $\singleton{L_n}$ 
% 
be another sequence of compact subsets of $\Omega$ that satisfies 
the condition specified in [1.44], 
so that 
%
  $\tau_{L} = \tau_{L_\kappa}$
%
for every subsequence $\singleton{L_{\kappa(n)}}$. 
%
Now apply the above Lemma 2 with $K_i$ ($\counting{i}$) and so conclude that  
%
  $K_i 
    \subset 
  L^\circ_{m_i} 
    \subset 
  L^\circ_{m_{i}+1}
    \subset
  \cdots$
%  
for some $m_i$. In particular, the special case $\kappa_i = m_i + i$ is 
%
  \begin{align}
    %
    \label{1_16. K subset interior L}
    K_i
      \subset 
    \interior{L}_{\kappa_i}.
  \end{align} 
%
Let us reiterate the above proof with $K_n$ and $L_n$ in exchanged roles 
then similarly find a subsequence $\set{\lambda_j}{\counting{j}}$ such that 
%
  \begin{align}
  %
  \label{1_16. L subset interior K}
  %
    L_j \subset \interior{K}_{\lambda_j}
  \end{align}
%
Combine 
%
  (\ref{1_16. K subset interior L}) with 
  (\ref{1_16. L subset interior K}) 
%
and so obtain
%
  \begin{align}
    K_1 
      \subset 
    \interior{L}_{\kappa_1} 
      \subset 
    L_{\kappa_1} 
      \subset 
    \interior{K}_{\lambda_{\kappa_1}}
      \subset 
    K_{\lambda_{\kappa_1}}
      \subset
    \interior{L}_{\kappa_{\lambda_{\kappa_1}}}
      \subset
    \cdots, 
  \end{align}
%
which means that the sequence 
%
  $Q = (
    K_1, 
    L_{\kappa_1}, 
    K_{\lambda_{\kappa_1}}, 
    %L_{\kappa_{\lambda_{\kappa_1}}},
    \dots
  )$
%
satisfies the conditions specified in section 1.44. 
It now follows from the corollary 2 that 
%
  \begin{align}  
    \tau_{K} 
    = 
      \tau_{K_\lambda} 
    = 
      \tau_{Q} 
    = 
      \tau_{L_\kappa} 
    = \tau_{L}.
  \end{align} 
%
So ends the proof
\end{proof}
% END

\newpage
\section{Exercise 17. Derivation in some non normed space}
\textit{In the setting of Section 1.46, prove that 
  %
    $f \mapsto D^{\alpha}f$ 
  %
is a continuous mapping of 
%
  $C^{\infty}\left(\Omega\right)$ into 
  $C^{\infty}\left(\Omega\right)$ and also of 
  $\D_{K}$ into 
  $\D_{K}$, for every multi-index $\alpha$.
%
}
\begin{proof} 
In both cases, $D^\alpha$ is a linear mapping. 
It is then sufficient to establish continuousness at the origin.
%
We begin with the $C^\infty\left(\Omega\right)$ case. \\
\\
Let $U$ be an aribtray neighborhood of the origin.
There so exists $N$ such that $U$ contains
%
  \begin{align} 
    V_{N}= \left\{
      \phi \in C^\infty\left(\Omega\right): 
      \max\set{
        | D^\beta\phi (x) |
      }{
        \magnitude{ \beta } \leq N, x\in K_N
      }
    < 1/N
    \right\}.
  \end{align}
%
Now pick $g$ in $V_{N+\magnitude{\alpha}}$, so that
%
  \begin{align}
    \max
    \set{
      \magnitude{ D^\gamma g\left(x\right) }
    }{
      \magnitude{ \gamma | \leq N+| \alpha }, 
      x\in K_{N}
    }
    < \frac{1}{N}.
  \end{align}
%
(the fact that $K_N\subset K_{N+\magnitude{\alpha}}$ was tacitely used).
%
The special case $\gamma = \beta + \alpha$ yields
%‡
\begin{align}
    \max
    \set{
      | D^\beta D^\alpha g(x)|
    }{
      \magnitude{\beta} \leq N, 
      x\in K_{N}
    }
    < \frac{1}{N}.
  \end{align}
%
We have just proved that
%
  \begin{align}\label{1.17. inclusion}
    g \in V_{N + \magnitude{\alpha}}
      \then 
    D^\alpha g \in V_{N},
      \quad
      \ie
      \quad
    D^\alpha \left(V_{N+\magnitude{\alpha}}\right) \subset V_N.
  \end{align}
%
The continuity of 
  $D^{\alpha}: C^\infty \left(\Omega\right) \to C^\infty \left(\Omega\right)$ 
is so established. \\\\
%
%
To prove the continuousness of the restriction 
%
  $D^\alpha \lvert_{\D_K}: \D_K \to \D_K$, % 
%
%
we first remark the collection of the  
%
  ${V_N \cap \D_K}$ % 
%
is a local base of the subspace topology of $\D_K$.
%
%
  $V_{N+\alpha} \cap \D_K$ % 
%
is then a neighborhood of $0$ in this topology. %
Furthermore, 
%
  \begin{align}
    %
    D^\alpha \lvert_{\D_K} \left(V_{N+\magnitude{\alpha}} \cap \D_K\right) 
    % 
    & = 
      D^\alpha\left(V_{N+\magnitude{\alpha}} \cap \D_K\right) \\
      %
    & \subset
      D^\alpha\left(V_{N+\magnitude{\alpha}}\right) 
        \cap 
      D^\alpha\left(\D_K\right) \\
      %
    & \subset 
      V_N 
        \cap 
      \D_K
        %
          \quad (\text{see }\ref{1.17. inclusion})
        %
    %
  \end{align}
%
So ends the proof.
\end{proof}

\chapter{Completeness}
%
\setcounter{section}{2}
%

\section{Exercise 3. An equicontinous sequence of measures}
%
\textit{
Put $K=[-1,1]$; define $\D_{K}$ as in section 1.46 
(with $\R$ in place of $\R^{n}$). 
Supose $\{f_{n}\}$ is a sequence of Lebesgue integrable functions such that 
%
  $\Lambda\phi 
    = 
  \underset{n \to \infty}{\lim} \int_{\minus 1}^1 f_{n}(t)\phi(t)dt$
%
exists for every $\phi\in\D_{K}$. 
Show that $\Lambda$ is a continuous linear functional on $\D_{K}$. 
Show that there is a positive integer $p$ and a number $M<\infty$ such that 
  \begin{align}
    \left\lvert 
      \int_{\minus 1}^1 f_n (t)\phi (t) dt\
    \right\rvert
    \leq 
    M \norma{\infty}{D^{p}} 
  \nonumber
  \end{align}
for all $n$.
For example, if $f_{n}(t)=n^{3}t$ on $[\minus 1/n, 1/n]$ and $0$ elsewhere, 
show that this can be done with $p=1$. 
Construct an example where it can be done with $p=2$ but not with $p=1$.}
%
%
\renewcommand{\labelenumi}{(\roman{enumi})}%
%
\newline\newline\noindent
We will also consider the case $p=0$. Since all supports of %
%
  $\phi, \phi', \phi'', \dots, $ are in $K$, %
%
we make a specialization of the mean value theorem: %
%
\paragraph{Lemma}\label{2.3 Lemma}
 \input{\ROOT/chapter_2/2_3/2_3_0_lemma.tex}
%: PROOF OF THE STATEMENT ----------------------------------------------------%
%/ FIRST PART ----------------------------------------------------------------%
%
\begin{proof}
 \input{\ROOT/chapter_2/2_3/2_3_1_radon_measures.tex}
% SECOND PART -----------------------------------------------------------------
 \input{\ROOT/chapter_2/2_3/2_3_2_uniform_bound.tex}
 \input{\ROOT/chapter_2/2_3/2_3_3_example_1.tex}
 \input{\ROOT/chapter_2/2_3/2_3_4_example_2.tex}
\end{proof}
%\renewcommand{\labelenumi}{\alph{enumi}.}%
\renewcommand{\labelenumi}{$(\textit{\alph{enumi}})$}%
\newpage
%
\setcounter{section}{5}
%
\section{Exercise 6. Fourier series may diverge at $0$}
\textit{
Define the Fourier coefficient $\hat{f}(n)$ of a function %
%
  $f\in L^2(T)$ ($T$ is the unit circle) %
%
by
%
  \begin{align*}
  \hat{f}(n) 
    = 
  \frac{1}{2\pi} \int_{\minus \pi}^{\pi} %
  %
    f(e^{i\theta}) e^{\minus i n\theta} d\theta
  %
  \end{align*}
%
for all $n\in \Z$ (the integers). Put 
%
  \begin{align*}
    \Lambda_n f =\sum_{k=\minus n }^{n} \hat{f}(k).
  \end{align*}
%
Prove that %
  $\set{f \in {L}^2(T)}{\underset{n\infty}{\lim}\,\Lambda_n f \text{ exists}} $ %
%
is a dense subspace of $L^2(T)$ of the first category. %
}
%
\begin{proof}%
Let $f(\theta)$ stand for $f(e^{i\theta})$, so that %
%
  $L^2(T)$ is identified with a closed subset of $L^2([\minus \pi, \pi])$, %
%
hence the inner product %
%
  \begin{align}\label{2.06. Inner product.}
    \hat{f}(n) 
      = 
    (f,e_n) 
      =
    \frac{1}{2\pi} \int_{\minus \pi}^\pi f(\theta)e^{\minus in\theta}d\theta.
  \end{align}
%
We believe it is customary to write %
\begin{align}\label{2.06. Rewriting of Lambda_n.}
  \Lambda_n (f) = (f, e_{\minus n}) + \cdots + (f, e_n).
\end{align}
Moreover, a well known (and easy to prove) result is % 
%
  \begin{align}\label{2.06. Orthonormality.}
    (e_{n}, e_{n'}) = [n=n'], \text{\ie } %
    \set{e_n}{n\in\Z} \text{ is an orthormal subset of } L^2(T).
  \end{align}
%
For the sake of brevity, we assume the isometric ($\equiv$) %
identification %
%
  $L^2\equiv (L^2)^\ast$. %
%
So,  %
%
  \begin{align}\label{2.06. Norm of Lambda_n.}
    \norm{}{\Lambda_n}^2
      \citeq{\ref{2.06. Rewriting of Lambda_n.}}
    \norm{}{e_{\minus n}  + \cdots + e_n}^2 
      \citeq{\ref{2.06. Orthonormality.}} 
    \norm{}{e_{\minus n }}^2 + \cdots + \norm{}{e_{n}}^2
      \citeq{\ref{2.06. Orthonormality.}}
    2n+1.
  \end{align}
%
%$\Lambda_n$ is therefore a bounded linear functional, of norm $\sqrt{2n+1}$. %
We now assume, to reach a contradiction, that %
%
  \begin{align}
    B \Def \set{f \in {L}^2(T)}{
      %{\underset{n\infty}
      \sup\set{
        \magnitude{\Lambda_n \,f}}{
        \counting{n}
      } < \infty
  } %
  \end{align}
%
is of the second category. %
So, the Banach-Steinhaus theorem \citeresultFA{2.5} asserts that the sequence %
%
  $\singleton{\Lambda_n}$ %
%
is norm-bounded; which is a desired contradiction, since %
%
  \begin{align}
    \norm{}{\Lambda_n} \citeq{
      \ref{2.06. Norm of Lambda_n.}
    } \sqrt{2n + 1} \tendsto{n}{\infty}\infty.
%
  \end{align}
%
We have just established that $B$ is actually of the first category; %
and so is its subset %
%
  L=$\set{f\in {L}^2(T)}{\lim_{n\longrightarrow\infty}\Lambda_n f \text{ exists}}$. %
%.
We now prove that $L$ is nevertheless dense in $L^2(T)$. %
To do so, we let $P$ be $\text{span}\set{e_k}{k\in Z} $, %
the collection of the trignometric polynomials %
%
  $p(\theta)= \sum \lambda_k e^{ik\theta}$: %
%
Combining %
%
  (\ref{2.06. Rewriting of Lambda_n.}) with %
  (\ref{2.06. Orthonormality.}) %
%
shows that %
$\Lambda_n(p)= \sum \lambda_k$ for almost all $n$. %
Thus, 
%
  \begin{align}\label{2.06. Basic inclusions.}
    P \subseteq L \subseteq L^2(T).
  \end{align}
%
We know from the Fejér theorem (the Lebesgue variant) that %
$P$ is dense in $L^2(T)$. We then conclude, with the help of %
%
  (\ref{2.06. Basic inclusions.}), %
%
that %
%
  \begin{align}
    L^2(T) = \overline{P} = \overline{L}.
  \end{align}
%
%It is now clear that $\overline{L} = L^2(T)$. 
So ends the proof
\end{proof}



\newpage
%
\setcounter{section}{8}
%
\section{Exercise 9. Boundedness without closedness}
\textit{
Suppose $X,Y,Z$ are Banach spaces and 
%
\begin{align*}
  B:X\times Y \to Z
\end{align*}
%
is bilinear and continuous. Prove that there exists $M<\infty$ such that 
%
\begin{align*}
  \norm{}{B(x, y)}
  \leq 
  M \norm{}{x}\norm{}{x} \quad (x\in X, y\in Y).
\end{align*}
%
Is completeness needed here?}
\begin{proof}%
% % % % % % % 
% FIRST PART %
% % % % % % % 
The answer is: No. To prove this, we only assume that %
%
  $\mathit{X}$, $\mathit{Y}$, $\mathit{Z}$ %
%
are normed spaces. %
%
Since $B$ is continous at the origin, there exists a positive $r$ %
such that %
%
\begin{align}\label{bound for B}
  \norm{}{x} + \norm{}{y} < r \Rightarrow %
  \norm{}{B(x, y)} < 1.
\end{align}
%
%
Given nonzero $\mathit{x}, \mathit{y}$, let $s$ range over $]0, r[$, so that the folllowing bound %
%
\begin{align}
  \norm{}{B(x, y)} = \frac{4\norm{}{x}\norm{}{y}}{s^2} %
    \left\|
      B \left(\frac{s}{2\norm{}{x}}x, \frac{s}{2\norm{}{y}}y \right)
    \right\|
  \overset{(\ref{bound for B})}{<} \frac{4 \norm{}{x} \norm{}{y}}{s^2}
  %\quad (x\neq 0, y \neq 0)
\end{align}
is effective. % 
It is now obvious that %%
%
\begin{align}
    B(x, y) \leq %
    \frac{\!\!\!4\!}{\,s^{\,2}}\norm{}{x}\norm{}{y} 
    \tendsto{s}{r}
    \frac{\!\!\!4\!}{\,r^{\,2}}\norm{}{x}\norm{}{y}
    \quad ((x, y) \in X \times Y); 
\end{align}
%
which achieves the proof.\\\\ %
% % % % % % % %
% SECOND PART %
% % % % % % % %
As a concrete example, choose %
%
  $X = Y = Z = C_c(\R)$, %
%
topologized by the supremum norm. %
%
$C_c(\R)$ is not complete %
%
(see 5.4.4 of \cite{AnalyseIII}), %
%
nevertheless the bilinear product %
%
\begin{align}
  \begin{aligned}
    B: &&& C_c(\R)^{2} & \to      &&& C_c(\R) \\\nonumber 
      &&& (f, g)       & \mapsto  &&& f \cdot g
  \end{aligned}
\end{align}
%
is bounded (since %
    $\norm{\infty}{f\cdot g} \leq \norm{\infty}{f}\cdot \norm{\infty}{g}$%
), %
and continuous. To show this, pick a positive scalar %
$\epsilon $ smaller than $1$, provided any $(f, g)$. Next, define %
%
\begin{align}
  r \triangleq \frac{\epsilon}{1 + \norm{\infty}{f} + \norm{\infty}{g}} < 1. %
\end{align}
%
We now restrict $(u, v)$ to a particular neighborhood of $(f, g)$. %
More specifically,  
\begin{align}
  \norm{\infty}{f-u} + \norm{\infty}{g-v} < r. %
\end{align}
%
Next, remark that %
%
  $\norm{\infty}{u} \leq r + \norm{\infty}{f}$ %
%
and so obtain (bear in mind that $r < 1$)%
%
  \begin{align}
    \norm{\infty}{fg - uv} 
      & =
    \norm{\infty}{(f-u) \cdot g + u \cdot (g-v)} \\
      & \leq
    \norm{\infty}{f-u} \cdot \norm{\infty}{g} + 
    \norm{\infty}{u}   \cdot \norm{\infty}{g-v} \\
      & < r \cdot \norm{\infty}{g} + (r + \norm{\infty}{f}) \cdot r \\
      & < r \cdot (r + \norm{\infty}{f} + \norm{\infty}{g})  \\
      & < \epsilon.
  \end{align}
%
Since $\epsilon$ was arbitrary, it is now established that $B$ continuous %
at every $(f, g)$.
\end{proof}
% END

\newpage
%
\section{Exercise 10. Continuousness of bilinear mappings}
\textit{%
Prove that a bilinear mapping is continuous %
if it is continuous at the origin $(0, 0)$.
}
\begin{proof}
Let $(X_1, X_2, Z)$ be topological spaces %
and $B$ a bilinear mapping %
%
  \begin{align}
    B: X_1 \times X_2 \to Z 
  \end{align}
%
From now on, $x=(x_1, x_2)$ denotes an arbitrary element of %
%
  $X_1\times X_2$. %
%
We henceforth assume that $B$ is continuous at the origin %
%
  $(0, 0)$ of $X_1\times X_2$, \ie  %
%
given an arbitrary balanced open subset $W$ of $Z$, %
there exists in $X_i$ ($i=1, 2$) a balanced open subset $U_i$ such that %
%
  \begin{align}
    B(U_1 \times U_2) \subset W .
  \end{align}
%
Let $\nu_i(x)$ denote any scalar that is greater than %
%
  $\mu_i(x_i) = \inf\set{r> 0}{x_i \in r \cdot U_i}$. %
%
So, 
%
  \begin{align}
    B(x_1, x_2) 
    %& = B(
    %    \nu(x)_1 \nu(x)_1^{\minus 1} \cdot x_1, 
    %    \nu(x)_2 \nu(x)_2^{\minus 1} \cdot x_2) \\
      & = 
        \nu_1(x)\nu_2(x) 
            \cdot 
        B\left(
          \nu_1(x)^{\minus 1} x_1 , 
          \nu_2(x)^{\minus 1} x_2
        \right) \\
      & \in 
    \nu_1(x) \nu_2(x) \cdot B(U_1 \times U_2) \\
    \label{2_10. Bound.}
      & \subset 
    \nu_1(x) \nu_2(x) \cdot W.
  \end{align}
%% 
Now pick $p=(p_1, p_2)$ in $X_1\times X_2$: %
%
It directly follows from (\ref{2_10. Bound.}) that %
%
  \begin{align}
    B(p_1, p_2) - B(x_1, x_2) 
      = B(p_1, p_2 - x_2) + B(p_1 -x_1, x_2 -p_2) + B(p_1-x_1,p_2) \\
      \label{2.10. In multiple of W.}
       \in  
         \nu_1(p)\nu_2(p-x)\cdot W + 
         \nu_1(p-x)\nu_2(x-p) \cdot W  + 
         \nu_1(p-x)\nu_2(p) \cdot W.
  \end{align}
%
Let us henceforth assume that %
%
  \begin{align}
    p_i -x_i \in [\mu_1(p) + \mu_2(p) + 2]^{\minus 1} \cdot U_i;   
  \end{align}
%
which yields %
%
  \begin{align}
    \mu_i(p_i -x_i) 
      \leq 
   [\mu_1(p) + \mu_2(p) + 2]^{\minus 1}.
  \end{align}
%
Finally, combine the special case %
%
  \begin{align}
    \nu_i(p -x) & =   [\mu_1(p) + \mu_2(p) + 1]^{\minus 1}, \\ 
    \nu_i(p)    & =\,\,\mu_1(p) + \mu_2(p) + 1
  \end{align}
%
with (\ref{2.10. In multiple of W.}) and so obtain %%
%
  \begin{align}
    B(p_1, p_2) - B(x_1, x_2)  \in W + W+ W.
  \end{align}
%
$W$ being arbitrary, %
we have so established the continuousness of $B$ at $(p_1, p_2)$. %
Since $(p_1, p_2)$ is also arbitrary, the proof is complete.
\end{proof}
%
%END
\newpage
%
\setcounter{section}{11}
%
\section{Exercise 12. A bilinear mapping that is not continuous}
\textit{%
Let X be the normed space of all real polynomials in one variable, with %
%
  \begin{align*}
    \norm{}{f}=\int_0^1\lvert f(t)\rvert\ dt.
\end{align*}
%
Put  %
  $B(f, g)=\int _0^1 f(t)g(t) dt $, %
%
and show that $B$ is a bilinear continuous functional on $X\times X$ %
which is separately but not continuous.
}
\begin{proof} Let $f$ denote the first variable, $g$ the second one. %
Remark that %
%
  \begin{align}\label{magnitude of B(f, g)}
    \magnitude{B(f, g)} 
      & < \norm{}{f} \cdot \max_{[0,1]} \magnitude{g} ; 
  \end{align}
%
which is sufficient (\citeresultFA{1.18}) to assert that any %
%
  $f \mapsto B(f, g)$ 
%
is continuous. The continuity of all %
%
  $g \mapsto B(f,g)$ %
%
follows (Put $C(g, f) = B(f, g)$ and proceed as above). %
Suppose, to reach a contradiction, that $B$ is continuous. %
There so exists a positive $M$ such that, 
%
  \begin{align}\label{magnitude of B(f, g) H_c}
    \magnitude{B(f, g)} < M \norm{}{f}\norm{}{g}.
  \end{align}
%
Put %
%
  \begin{align}\label{definition of f_n}
    f_n(x)\Def 2 \sqrt{n}\cdot x^{n}\in \R[x] \quad\quad (\counting{n}), 
  \end{align}
%
so that 
%
  \begin{align}\label{norm of f_n}
    \norm{}{f_n} = \frac{2 \sqrt{n}}{n+1} \tendsto{n}{\infty}0.
  \end{align}
%
On the other hand,
%
  \begin{align}\label{norm of B(f_n, g_n)}
    B(f_n, f_n)= \frac{4 n}{2n+1} > 1.
  \end{align}
%
Finally, we combine %
%
  (\ref{norm of B(f_n, g_n)}) and %
  (\ref{magnitude of B(f, g) H_c}) with %
  (\ref{norm of f_n}) %
%
and so obtain
\begin{align}
 1 < B(f_n, f_n) <  M \norm{}{f_n}^2  \tendsto{n}{\infty} 0.
\end{align}
Our continuousness assumption is then contradicted. So ends the proof.
\end{proof}

\newpage
%
\setcounter{section}{14}
%
\section{Exercise 15. Baire's cut}
\textit{%
Suppose $X$ is an F-space and $Y$ is a subspace of $X$ %
whose complement is of the first category. %
Prove that $Y=X$. }Hint: \textit{%
%
  $Y$ must intersect $x+Y$ for every $x\in X$. %
%
}
%
\begin{proof} Assume $Y$ is a subgroup of $X$. %
Under our assumptions, %
there exists a sequence $\set{E_n}{\counting{n}}$ of $X$ such that %
%
  \renewcommand{\labelenumi}{(\roman{enumi})} 
  \begin{enumerate}
    \item ${(\overline{E}_n)}^\circ=\emptyset ;$
    \item $X\setminus Y = \displaystyle{\bigcup_{n=1}^\infty  E_n}$.
  \end{enumerate}
%
By (i), the complement $V_n$ of $\overline{E}_n$ is a dense open set. %
Now pick $x$ in $X$: $x + V_n$ is dense open as well, 
since the translation by $x$ is an homeomorphism of $X$. %
%
Note that the density is also a special case of \citeresultFA{1.3 (b)}, as follows, %
%
\begin{align}\label{2_15_2}
  X = x + X \subseteq \overline{x + V_n}.
\end{align} 
%
We now apply the Baire's theorem twice to establish that %
%
\begin{enumerate}
  \item every intersection $W_n = V_n \cap (x+V_n)$ is dense in $X$;
  \item so is the intersection $\displaystyle{\bigcap_{n=1}^\infty W_n}$.
\end{enumerate}
%
Bear in mind that every dense subset of $X$ is nonempty and remark that %
\begin{align}\label{2_15_3}
  \bigcap_{n=1}^\infty W_n \subseteq \bigcap_{n=1}^\infty V_n \subseteq Y
\end{align} 
holds, since %
%
  $X\setminus Y \subseteq \bigcup_{n=1}^\infty \overline{E}_n$. %
%
Now pick $w$ in $\bigcap_{n=1}^\infty W_n$, so that %
$w$ is an element of $Y$ (by (\ref{2_15_3})) that lies in every $x + V_n$. %
Hence %
%
\begin{align}
  w - x \in \bigcap_{n=1}^\infty V_n \subseteq Y; 
\end{align}
%
again with (\ref{2_15_3}).
It is now clear that $x + Y$ cuts $Y$ in $x + (w - x ) = w$, which establishes that $x$ lies is $Y - Y$. %
As a conclusion, 
\begin{align}
  X \subseteq Y,  
\end{align}
%
since $x$ was arbitrary and that $Y$ is a subgroup. %
So ends the proof.
%
%
%  \begin{align}\label{2_15_4}
%    (x+S)\cap S\neq\emptyset.
%  \end{align}
%
%Moreover, it follows from (ii) that %
%
%  $X\setminus Y \subseteq \bigcup_n \closure{E}_n$, \ie 
%  $Y \contains S$. %
%Combined with (\ref{2_15_4}), this shows that $x+Y$ cuts $Y$. 
%Therefore, our arbitrary $x$ is an element of the subgroup $Y$. %
%We have thus established that $X \subseteq Y$, which achieves the proof.
\end{proof}
\renewcommand{\labelenumi}{(\alph{enumi})} 
%
% END

\newpage
%
\section{Exercise 16. An elementary closed graph theorem}
\textit{%
$\set{(x,1/x)}{x > 0}$uppose that $X$ and $K$ are metric spaces, %
that $K$ is compact, and that the graph of %
%
  $f:X\to K$ %
%
is a closed subset of $X\times K$. %
Prove that $f$ is continuous %
%
  (This is an analogue of Theorem 2.15 but much easier.) %
%
$\set{(x,1/x)}{x > 0}$how that compactness of $K$ cannot be omitted from the hypothese, %
even when $X$ is compact.
}
%
\begin{proof}%
Choose a sequence $\set{x_n}{\counting{n}}$ whose limit is an arbitrary $a$. %
By compactness of $K$, the graph $G$ of $f$ contains a subsequence %
%
  $\singleton{
    (x_{\rho(n)}, f(x_{\rho(n)}))
  }$
%
of %
%
  $\singleton{
    (x_n, f(x_n))
  }$
%
that converges to some $(a, b)$ of $X\times K$. %
$G$ is closed; therefore, $\singleton{(x_{\rho(n)}, f(x_{\rho(n)}))}$
converges in $G$. So, $b=f(a)$; %
which establishes that $f$ is sequentially continuous.
%
Since $X$ is metrizable, $f$ is also continuous; see \citeresultFA{[A6]}. %
%
\newline\newline\noindent %
%
To prove that compactness cannot be omitted from the hypotheses, %
we showcase the following counterexample, %
%
  \begin{align}
  f: [0, \infty) & \to [0, \infty)\\
     x & \mapsto\begin{cases}
       1/x & (x > 0)  \\
       0   & (x = 0).
    \end{cases}
  \end{align}
%
Clearly, the graph of $f$ is the following set
%
  \begin{align}
    \set{(x, 1/x)}{x > 0} \cup \singleton{(0, 0)}.
  \end{align}
%
Let $\singleton{(x_n, f(x_n))}$ be a sequence in $\set{(x,f(x))}{x > 0}$ %
that converges to, say, $(a, b)$. %
Clearly, $a\neq 0$, since $\singleton{(x_n, f(x_n))}$ is bounded. %
So, $a > 0$ and $b = 1/a$ (since $f$ is continuous on $R_+$).
This establishes that the subset $\set{(x,1/x)}{x > 0}$ is closed. %
As a finite union of closed sets, the graph of $f$ is closed. 
%
Nevertheless, $f$ is not continuous on $[0, \infty)$.
\end{proof}
%
% END

\chapter{Convexity}
\section{Exercise 3. }
\textit{%
Suppose X is a real vector space (without topology). %
Call a point $x_0\in A \subset X$ an internal point of $A$ if %
$A- x_0$ is an absorbing set. %
%
%\renewcommand{\labelenumi}{(\alph{enumi})}
%
  \begin{enumerate}
    \item{%
      Suppose $A$ and $B$ are disjoint convex sets in $X$, %
      and $A$ has an internal point. %
      Prove that there is a nonconstant linear functional $\Lambda$ such that %
      $\Lambda(A)\cap \Lambda(B)$ contains at most one point. %
      (The proof is similar to that of Theorem 3.4) %
    }%
    \item{%
      Show (with $X=\R^2$, for example) that it may not possible to have %
      $\Lambda (A)$ and $\Lambda (B)$ disjoint, under the hypotheses of (a). %
    }
  \end{enumerate} 
}
%
\begin{proof}%
Take $A$ and $B$ as in (a); the trivial case $B=\emptyset$ is discarded. %
Since $A-x_0$ is absorbing, so is its convex superset %
%
  $C= A-B - x_0 + b_0$ $(b_0 \in B)$. %
%
Note that $C$ contains the origin. %
Let $p$ be the Minkowski functional of $C$. Since $A$ and $B$ are disjoint, %
$b_0-x_0$ is not in $C$, hence $p(b_0-x_0) \geq1$. %
We now proceed as in the proof of the Hahn-Banach theorem \citeresultFA{3.4} %
to establish the existence of a linear functional %
%
  $\Lambda: X\to  \R$ such that %
%
  \begin{align}
    \Lambda \leq p %
  \end{align}
%
and %
%
  \begin{align}
    \Lambda(b_0-x_0) =1.
  \end{align}%
%
Then %
%
  \begin{align}
    \Lambda a -\Lambda b + 1 =\Lambda (a-b+ b_0-x_0) \leq p (a-b+ b_0-x_0) \leq 1\quad (a\in A, b \in B).
  \end{align}
%
Hence %
%
  \begin{align}\label{3_3_3}
    \Lambda a \leq \Lambda b .
  \end{align}
%
We now prove that $\Lambda (A) \cap \Lambda (B) $ contains at most one point. %
Suppose, to reach a contradiction, that this intersection contains %
%
  $y_1$ and $y_2$. %
%
There so exists %
%
  $(a_i, b_i)$ %%
%
in $A\times B$ ($i=1, 2$) such that %
%
  \begin{align}\label{3_3_4}
    \Lambda a_i= \Lambda b_i = y_i .
  \end{align}
%
Assume without loss of generality that $y_1< y_2$. Then, %
%\begin{align}
%y_1=\Lambda b_1 < \Lambda \left( \frac{1}{2} a_1\right ) + \Lambda \left( \frac{1}{2} a_2\right ) = \frac{1}{2} (y_1+y_2)  \quad .
%\end{align}
%
  \begin{align}\label{3_3_5}
    2\cdot y_1= \Lambda b_1+  \Lambda b_1 < \Lambda ( a_1 + a_2) = (y_1+y_2)  \quad .
  \end{align}
%
Remark that $a_3= \frac{1}{2} (a_1+ a_2)$ lies in the convex set $A$. %
This implies %
%
  \begin{align}
  \Lambda b_1 \overset{(\ref{3_3_5})}{<}  \Lambda a_3 \overset{(\ref{3_3_3})}{\leq} \Lambda b_1\quad ;
  \end{align}
%
which is a desired contradiction. (a) is so proved and we now deal with (b). %
%
\newline\newline\noindent
%
From now on, the space $X$ is $\R^2$. Fetch %
%
   \begin{align}
     S_1 & \Def \set{(x, y )\in \R^2}{x\leq 0, y \geq 0}, \\
     S_2 & \Def \set{(x, y )\in \R^2}{x>0, y > 0}, \\
     A   & \Def S_1\cup S_2, \\
     B   & \Def X\setminus A.
   \end{align}
%
Pick $(x_i, y_i)$ in $S_i$. %
Let $t$ range over the unit interval, and so obtain %
%
  \begin{align}
    t\cdot  \left(\begin{array}{c}x_1 \\y_1\end{array}\right)+
    (1-t)  \cdot \left(\begin{array}{c}x_2 \\ y_2\end{array}\right)= 
    \left(\begin{array}{c}t\cdot x_1+(1-t)\cdot x_2 \\t\cdot y_1 + (1-t)\cdot y_2\end{array}\right)
    \in \R\times \R_+\subset A.
  \end{align}
%
Thus, every segment that has an extremity in $S_1$ and the other one in $S_2$ %
lies in $A$. %
Moreover, each $S_i$ is convex. We can now conclude that $A$ is so. %
The convexity of $B$ is proved in the same manner. Furthermore, %
$A$ hosts a non degenerate triangle, \ie $A^{\circ}$ is nonempty\footnote{%
%
  For a immediate proof of this, remark that a triangle boundary is %
  compact/closed and apply [1.10] or 2.5 of \cite{BigRudin}.
}: %
$A$ contains an internal point. %
%
\newline\newline\noindent
%
Let $L$ be a vector line of $\R^2$. %
In other words, $L$ is the null space of a linear functional %
%
  $\Lambda: \R^2\to \R$ %
%
(to see this, take some nonzero $u$ in $L^\bot$ and set %
%
  $\Lambda x= (x,u)$ %
%
for all $x$ in $\R^2$). One easily checks that both $A$ and $B$ cut $L$. %
Hence %
%
  \begin{align}
    \Lambda (L)=\{0\}\subset \Lambda (A)\cap \Lambda (B)\neq\emptyset\quad.
    \end{align}
%
So ends the proof.
\end{proof}

\newpage
\section{Exercise 11. Meagerness of the polar}
\textit{\noindent
Let $X$ be an infinite-dimensional Fréchet space. %
Prove that $X^\ast$, with its $\weakstar$topology, %
is of the first category in itself.
}
\newline\newline\noindent
This is actually a consequence of the below lemma,  %
which we prove first. %
The proof that $X^\ast$ is of the first category in itself comes right after, %
as a corollary.%
%
\paragraph{Lemma.}{\it %
f $\varit{X}$ is an infinite dimensional topological vector space whose dual %
%
  $\mathit{X^{\,\ast}}$ %
%
separates points on $X$, then the polar
%
  \begin{align}
    \varit{K}_A\Def \{ \Lambda \in \mathit{X^\ast}:\, \magnitude{\Lambda} \leq \mathrm{1} \text{ on } A\}
  \end{align} 
%
of any absorbing subset $\varit{A}$ is a $\varit{\weakstar}$closed set that has empty interior.
}
%
\begin{proof}%
Let $x$ range over $X$. The linear form %
%
  $\Lambda \mapsto \Lambda x$ %
%
is $\weakstar$continuous; see \citeresultFA{3.14}. %
Therefore, %
%
  $P_x= \set{\Lambda \in X^\ast}{\magnitude{\Lambda x} \leq 1}$ %
%
is $\weakstar$ closed: %
%In particular, every $P_a$ ($a \in A$) is $\weakstar$closed:  %
As the intersection of $\set{P_a}{a\in A}$, %
$K_A$ is also a $\weakstar$closed set. %
We now prove the second half of the statement. % %
%
\newline\newline\noindent
%
From now on, $X$ is assumed to be endowed with its weak topology: %
$X$ is then locally convex, but its dual space is still %
%
  $X^\ast$ (see \citeresultFA{3.11}). %
%
Put %
%
  \begin{align}
    W_{F, x} \Def \bigcap_{x\in F} \set{\Lambda \in X^\ast}{\magnitude{\Lambda x} < r_x} \quad\quad (r_x > 0)
  \end{align}
%
where $F$ runs through the nonempty finite subsets of $X$. %
%
Clearly, the collection of all such $W$ is a local base of $X^\ast$. %
Pick one of those $W$ and remark that the following subspace %
%
  \begin{align}
    M \Def \text{span}(F)
  \end{align}
%
is finite dimensional. %
Assume, to reach a contradiction, that $A\subset M$. %
So, every $x$ lies in $t_xM=M$ for some $t_x>0$, since $A$ is absorbing. %
As a consequence, $X$ is the finite dimensional space $M$, %
which is a desired contradiction.
%
We have just established that $A\not\subset M$: %
Now pick $\varit{a}$ in $A\setminus M$ and so conclude that %f
%
  \begin{align}
    b\Def \frac{a}{t_a}\in A
  \end{align}
%
Remark that $b\notin M$ (otherwise, $a = t_a b \in t_a M=M$ would hold) %
and that $M$, as a finite dimensional space, is closed %
(see \citeresultFA{1.21 (b)} for a proof): %
By the Hahn-Banach theorem \citeresultFA{3.5}, %
there exists $\Lambda_a$ in $X^\ast$ such that 
%
  \begin{align}\label{3.11. Polar_4.}
    \Lambda_a b > 2
  \end{align}
%
and
%
  \begin{align}\label{3.11. Polar_4bis.}
    \Lambda_a (M) = \singleton{0}.
  \end{align}
%
The latter equality implies that $\Lambda_a$ vanishes on $F$; %
hence $\Lambda_a$ is an element of $W$. %
On the other hand, given an arbitrary $\Lambda \in K_A$, %
the following inequalities  %
%
  \begin{align}
    \magnitude{\Lambda_a b + \Lambda b} 
      \geq 
    2 - \magnitude{\Lambda b} 
      >
    1.
  \end{align}
%
show that $\Lambda + \Lambda_a$ is not in $K_A$. %
%
We have thus proved that
%
  \begin{align}
  \Lambda + W\not\subset K_A.
  \end{align}
%
Since $W$ and $\Lambda$ are both arbitrary, this achieves the proof. %
\end{proof}
%
%\newline\newline
\noindent
We now give a proof of the original statement. %
%
\paragraph{Corollary.}%
{\it If $\varit{X}$ is an infinite-dimensional Fréchet space, %
then $\mathit{X^\ast}$ is meager in itself.
}
%
\begin{proof}%
From now on, $X^\ast$ is only endowed with its $\weakstar$topology. %
Let $d$ be an invariant distance that is compatible with the topology of $X$, %
so that the following sets
%
  \begin{align}
    B_n \Def \set{x\in X}{d(0, x) <  1/n}\quad\quad (\counting{n}) 
  \end{align}
%
form a local base of $X$. %
%
If $\Lambda$ is in $X^\ast$, then %
%
  \begin{align}
    \magnitude{\Lambda} \leq  m \text{ on } B_n
  \end{align}
%
for some $(n, m) \in \singleton{1, 2, 3, \dots}^2$; see \citeresultFA{1.18}. %
%
Hence, $X^\ast$ is the countable union of all % 
%
  \begin{align}\label{3.11. Countable union.}
    m\cdot K_n \quad\quad (\counting{m,n}), 
  \end{align}
%
where $K_n$ is the polar of $B_n$. %
Clearly, showing that every $m\cdot K_n$ is nowhere dense %
is now sufficient. %
To do so, we use the fact that $X^\ast$ separates points; %
see \citeresultFA{3.4}. % 
As a consequence, the above lemma implies %
%
  \begin{align}
    \left({\overline{K}_n}\right)^\circ = \left({{K}_n}\right)^\circ=\emptyset. 
  \end{align}
%
Since the multiplication by $m$ is an homeomorphism (see \citeresultFA{1.7}), %
this is equivalent to %
%
  \begin{align}\label{3.11. Nowhere dense.}
    \left(\,{\overline{m\cdot K_n}}\, \right)^\circ 
      = 
    %\left(\, m \cdot \overline{{{K}_n}} \, \right)^\circ
    %  = 
    %\left(m \cdot K_n \right)^\circ 
    %  = 
    m\cdot \left({{K}_n}\right)^\circ 
      = 
    \emptyset.
  \end{align}
%
So ends the proof.% 
\end{proof}
\newpage

%\backmatter
%\part{Annex}
%\renewcommand\thechapter{\Alph{chapter}}
%\setcounter{chapter}{1}
%\chapter{Additional results}
%\newcounter{annex}
%\setcounter{annex}{1}
%\renewcommand\thesection{\thechapter.\arabic{annex}}
%\section{Number theory}
%\input{\ROOT/Annex_number_theory.tex}
\bibliographystyle{plain}
\bibliography{bibliography}{}
\addcontentsline{toc}{chapter}{Bibliography}
\end{document}


%Made on \XeTeX
