% a4paper = ISO 216 standard. 
% Don't forget to revert to default if you want the US Letter instead.
\documentclass[a4paper,10pt,titlepage,openany,leqno]{book}

% --------------------------- MATH PACKAGES ----------------------------
%\usepackage{graphicx}
%\usepackage{fontspec} % XeTeX only, now replaced by fontspec + unicode-math
\usepackage{amsmath, amssymb} 
\usepackage{
physics, 
amssymb, 
amsfonts, 
mathrsfs, 
amsthm, 
mathtools, 
% mathspec, % DEPRECATED: XeTeX-specific, removed for LuaLaTeX
fouridx, 
%%stix
%%tensor 
%tikz 
}
%\usetikzlibrary{arrows,snakes,backgrounds,patterns,matrix,shapes,fit,calc,shadows,plotmarks}
\usepackage{caption}
\captionsetup{labelsep=period}
\usepackage{diagbox} % For diagonal cells
%\usetikzlibrary{shapes.geometric}
%\usetikzlibrary{arrows.meta,arrows}
%\usepackage{tlatex}
% TEXs LOGOS! 
%\usepackage{dtklogos}

\usepackage[left=1.9cm,right=1.32cm,top=1.9cm,bottom=3.67cm]{geometry}

% --------------------------- COLOR ----------------------------
\usepackage{array}
%\usepackage[dvipsnames]{xcolor}
%\usepackage[]{color}
%\usepackage{framed}
\usepackage{colortbl}

% --------------------------- FONT + UNICODE-MATH ----------------------------
\usepackage{fontspec}

% 1. OPTION ADDED HERE: [math-style=upright]
% This forces Latin (x, y) and Greek letters to be upright by default.
\usepackage[math-style=upright]{unicode-math} 

% --------------------------- HYPERREF ----------------------------
\usepackage{hyperref} 
\hypersetup{
    pdfborder = {0 0 0},
    colorlinks,
    citecolor=red,
    filecolor=green,
    linkcolor=black,
    urlcolor=blue
}

% --------------------------- TEXT FONTS ----------------------------
\setmainfont[
  ItalicFont=NewCM10-BookItalic.otf,
  BoldFont=NewCM10-Bold.otf,       % There is no "Book Bold", we use standard Bold
  BoldItalicFont=NewCM10-BoldItalic.otf
]{NewCM10-Book.otf}

\setsansfont[
  ItalicFont=NewCMSans10-BookOblique.otf,
  BoldFont=NewCMSans10-Bold.otf,
  BoldItalicFont=NewCMSans10-BoldOblique.otf
]{NewCMSans10-Book.otf}

\setmonofont[
  Scale=MatchLowercase, % Good practice to keep mono readable against serif
  ItalicFont=NewCMMono10-BookItalic.otf,
  BoldFont=NewCMMono10-Bold.otf,
  BoldItalicFont=NewCMMono10-BoldOblique.otf
]{NewCMMono10-Book.otf}

%\newfontfamily\fw{NewCMMono10-Book.otf}
%\newfontfamily\CMUCS{NewCMMono10-Book.otf}
% --------------------------- MATH FONTS ----------------------------
\setmathfont{NewCMMath-Book.otf}
%\setmathfont[range=\mathtt]{NewCMMono10-Book.otf}
 
 %cmuntt.otf}
% --------------------------- TYPOGRAPHIC CONVENTIONS ----------------------------
\usepackage{polyglossia}
\setdefaultlanguage[variant=usmax]{english}

% --------------------------- PRIMITIVES ----------------------------
\newcommand{\insmall}[1]{\text{\small{#1}}}
%\newcommand{\minus}{\insmall{-}}
% Use \AtBeginDocument to ensure unicode-math doesn't overwrite it

\AtBeginDocument{
  \renewcommand{\minus}{%
    \mathbin{%
      \mathchoice
        {\scalebox{0.8}{$\text{\bf -}$}} % Display and Text style
        {\scalebox{0.8}{$\text{\bf -}$}} % Script style
        {\scalebox{0.5}{$\text{\bf -}$}} % Script-script style
        {\scalebox{0.8}{$\text{\bf -}$}} % Default fallback
    }%
  }
}

% The field C and the usual subsets R, Q, Z, N.
% NOTE: \mathbf produces bold italic in unicode-math, switch to \mathbb for standard notation
\newcommand\usualSet[1]{\mathbf{#1}} 
\def\C{\usualSet{C}}
\def\R{\usualSet{R}}
\def\Q{\usualSet{Q}}
\def\Z{\usualSet{Z}}
\def\N{\usualSet{N}}
\def\K{\usualSet{K}}

\def\Def{\triangleq}
\def\defIFF{\quad \text{\bf iff} \quad}

\newcommand\counting[1]{#1=1, 2, 3, \dots}
\newcommand\integers[1]{#1=0, 1, 2, \dots}
\newcommand\naturals[1]{#1=0, 1, 2, \dots}

\newcommand{\up}[1]{^{(#1)}}
\newcommand{\low}[1]{_{#1}}
\newcommand{\upnw}[2]{\fourIdx{#2}{}{}{}#1}
\newcommand{\downsw}[2]{\fourIdx{}{#2}{}{}#1}
\newcommand{\scriptatleft}[3]{\fourIdx{#2}{#3}{}{}#1}
\newcommand{\diagscript}[3]{\fourIdx{#2}{}{}{#3}#1}

\newcommand{\function}[1]{\mathtt{#1}}
\newcommand{\relation}[2]{{#1}_{#2}}
\newcommand{\f}[2]{#1(#2)}
\newcommand{\id}[1]{\text{id}_{#1}}
\renewcommand{\restriction}[2]{#1\!\upharpoonright_{#2}}

% Predicate calculus, logic
% % Bunch of synonyms
\def\then{\Rightarrow} 
\def\therefore{\Rightarrow}
\def\since{\Leftarrow}
\def\because{\Leftarrow} 

% % Halmos and Iverson
\renewcommand{\iff}{\Leftrightarrow}
\newcommand{\Iverson}[2][\big]{#1[\, #2\, #1]}

% Sets
% % Braces
\DeclarePairedDelimiter\singleton\{\}
\newcommand{\set}[3][\big]{\singleton[#1]{#2: #3}}

% % Union, intersection.
% For now, nothing gets bolded. TODO:? Minor concern. 
\DeclareMathOperator{\unionOfCollection}{\boldsymbol{\bigcup}}
\DeclareMathOperator{\intersectionOfCollection}{\boldsymbol{\bigcap}}
\def\cuts{\cap} % synonym

% % Subset, supset
\AtBeginDocument{\let\subset\subseteq
\let\supset\supseteq}
\def\contains{\supseteq} % synonym

% % Convex hull
\DeclareMathOperator\cvxhull{co} 
\newcommand{\co}[1]{\cvxhull(#1)}

% % Misc
\newcommand{\bigsetminus}[1][\big]{\mathbin{#1\backslash}}
\DeclareMathOperator\opcard{card} 

% Topology 
\newcommand{\interior}[1]{{#1}^\circ}

\AtBeginDocument{
  \DeclareRobustCommand{\closure}[1]{\overline{#1}}
}

\newcommand{\localbase}[1]{\mathscr #1}
\def\weakstar{\text{weak}^\ast\text{-}}

% Standard calculus
% % Intervals
%\newcommand{\openInterval}[2]{\left]#1, #2\right[}
%\newcommand{\openInterval}[2]{\openInterval}
%\DeclarePairedDelimiter\openinterval[]
\DeclarePairedDelimiter\openinterval{]}{[}
\newcommand{\openInterval}[3][\big]{\openinterval[#1]{#2, #3}}

%newcommand{\set}[3][\big]{\singleton[#1]{#2: #3}}

% % Usual spaces
\newcommand{\Continuous}{{\mathscr C}} 
\newcommand{\D}{{\mathscr D}} 

% % Bounds
\newcommand{\supp}[1]{\operatorname{supp}#1}

% % Usual functions and operator
\DeclareMathOperator{\cotan}{cotan}
\DeclareMathOperator\sgn{sgn}
\newcommand{\ceil}[1]{\lceil #1 \rceil}
\DeclareMathOperator\Leibnizdiff{d}
\newcommand{\diff}[1]{\Leibnizdiff{\!#1}}

% % Evaluation
\newcommand{\evalbar}[4][\Big]{#2\mathrel{#1|}_{#3}^{#4}}
\newcommand{\evalbracket}[4][\Big]{#1[\,#2\,#1]_{#3}^{#4}}

% % Limits
\newcommand{\tendsto}[2]{\xrightarrow[#1 \to #2]{}}

% % Norms
\def\hspaceNorm{\hspace{0.12 em}}

% % Alternative to \abs
\newcommand{\magnitude}[2][\big]{#1\lvert \hspaceNorm #2 \hspaceNorm #1\rvert}

\renewcommand{\norm}[2][]{
  \if\relax\detokenize{#1}\relax 
    \lVert \hspaceNorm  #2 \hspaceNorm \rVert 
  \else 
    \left\lVert \hspaceNorm #2 \hspaceNorm \right\rVert_{#1} 
  \fi
}

% % the ||| . ||| version
\newcommand{\vertiii}[1]{{\left\vert\kern-0.25ex\left\vert\kern-0.25ex\left\vert #1 
    \right\vert\kern-0.25ex\right\vert\kern-0.25ex\right\vert}}
\newcommand{\normtriple}[2]{\vertiii{\hspaceNorm #2 \hspaceNorm}_{#1}}

% Typesetting

\newcommand{\varit}[1]{\mathit{#1}}
\def\vart{\varit{t}}

\def\ie{\textit{, i.e., }}
\def\eg{\textit{, e.g., }}
\def\cf{\textit{cf.\,}} 
\def\IFF{{\bf iff }}
\def\wlg{{\bf wlg }}

% Typography
% % Greel letters.
\renewcommand{\rho}{\uprho}
\AtBeginDocument{
\renewcommand{\epsilon}{\varepsilon} 
%\renewcommand{\phi}{\varphi} % For now, \phi is OK.
}

% Citations (unchanged)
\newcommand{\citehere}[2]{\overset{#1}{#2}}
\newcommand{\citeq}[1]{\citehere{#1}{=}}
\newcommand{\citeleq}[1]{\citehere{#1}{\leq}}
\newcommand{\citegeq}[1]{\citehere{#1}{\geq}}
\newcommand{\citeleast}[1]{\citehere{#1}{<}}
\newcommand{\citegreater}[1]{\citehere{#1}{>}}
\newcommand{\citesubset}[1]{\citehere{#1}{\subset}}
\newcommand{\citesubseteq}[1]{\citehere{#1}{\subseteq}}
\newcommand{\citesupset}[1]{\citehere{#1}{\supset}}
\newcommand{\citesupseteq}[1]{\citehere{#1}{\supseteq}}
\newcommand{\citethen}[1]{\citehere{#1}{\then}}
\newcommand{\citeresult}[2]{#1 of #2}
\newcommand{\citebook}[2]{[#1] of {\cite{#2}}}
\newcommand{\citeFA}[1]{\citebook{#1}{FA}}
\newcommand{\citein}[1]{\citehere{#1}{\in}}

\newcommand{\underbarwithindex}[2]{\underline{#1}\,\!_{#2}}
\newcommand{\dy}[1]{{\function{dyadic}}(#1)}
\def\ddy{{\function{decay}}}

% --------------------------- THEOREM STYLE ----------------------------
\makeatletter
\newtheoremstyle{namedlemma} 
  {3pt} 
  {20pt} 
  {} 
  {} 
  {\bfseries} 
  {.\newline} 
  {\newline} 
  {\thmname{#1}\ \thmnumber{#2}\thmnote{\quad #3}} 
\makeatother

\theoremstyle{namedlemma}
\newtheorem{lemma}{Lemma}[chapter]

% --------------------------- DOCUMENT INFO ----------------------------
\def\ROOT{./}
\def\TITLE{Solutions to some exercises from Walter Rudin's \textit{Functional Analysis}}
\def\EMAIL{}
\def\AUTHOR{gitcordier}


\begin{document}
% Changes the "proof" in proof environment. 
% source: https://tex.stackexchange.com/questions/8089/changing-style-of-proof
\let\oldproofname=\proofname
\renewcommand{\proofname}{{\rm \small PROOF}}
%\begin{abstract}
%    \input{\ROOT/abstract.tex}
%\end{abstract}
%\title{\TITLE}
\title{\TITLE}
\author{\AUTHOR}
\date{\today}
\maketitle

% FORMAT ENUMERATION (DEFAULT. OPTIONS: ALPH, ARABIC, ROMAN,…)
%\renewcommand{\labelenumi}{$(\textit{\alph{enumi}}\,)$}
% IF LANG = @fr
%\renewcommand{\chaptername}{Chapitre}
%
\frontmatter
\tableofcontents
%\clearpage

%\printglossary
\renewcommand{\thesection}{\Roman{section}}
\renewcommand{\thesubsection}{\roman{subsection}}
% % % % % % % % % % % % % % % % % % % % % % % % % % % % % % % % % % % % % % % % % % % % % % % % % % % % % % % % % % % % % % % %
% FunctionalAnalysis 
% notations.tex
% 
% encoding: UTF-8 
% EOL: LF
%
% format: LaTeX
% indent: spaces (2)
% width: 127
% % % % % % % % % % % % % % % % % % % % % % % % % % % % % % % % % % % % % % % % % % % % % % % % % % % % % % % % % % % % % % % %
\renewcommand{\thesection}{\Roman{section}}
\renewcommand{\thesubsection}{\roman{subsection}}
%
\chapter{Notations and Assumptions}%
%\addcontentsline{toc}{chapter}{Notations and Conventions}
% % % % % % % % % % % % % % % % % % % % % % % % % % % % % % % % % % % % % % % % % % % % % % % % % % % % % % % % % % % % % % % %
% LOGIC
% % % % % % % % % % % % % % % % % % % % % % % % % % % % % % % % % % % % % % % % % % % % % % % % % % % % % % % % % % % % % % % %
\section{Logic}%
\subsection{Propositional logic operators}
Given propositional variables $\mathit{p}$, $\mathit{q}$, the boolean %
operators $\lnot$, $\lor$, $\land$, $\iff$, $\then$, %
$\Leftarrow$, are assigned boolean \textit{truth values} (``true'' and ``false'') as follows,
\begin{enumerate}
  \item[$\lnot$]{%
    $\lnot p$ and $p$ have opposite values:
  }
  \item[$\lor$]{
    The \textit{disjunction} (``or'') $p \lor q$ is true unless $p$ is false and $q$ is false.
  }
  \item[$\land$]{
    The \textit{conjunction} (``and'') $p \land q$ is true if and only if $p$ is true and $q$ is true.
  }
  \item[$\iff$]{%
    The \textit{logical equivalence} expresses \textit{tautologies}: %
    $p \iff q$ is true unless $p$ and $q$ have opposite values. It is easily checked that %
    $(p \iff q) \iff \bigl((p \then q) \land (p \IF  q) \bigr)$, see the definitions below.
  }
  \item[$\then$]{%
    The logical connection \textit{$p$ implies $q$} is supported by $\then$: $p \then q$ means \textit{if $p$ then $q$} %
    and is formally defined as $\lnot p \lor q$. Note that the ``reasoning'' $p \then q $ is always true unless %
    $p$ is true and $q$ is false. Moreover, $p \land (p \then q) \then q$ is always true. %
    This deductive rule is known as \textit{modus ponens}.
  }
  \item[$\Leftarrow$]{ $q \IF p$ means that $q$ is implied by $p$ and is defined as $ p\then q $. %
    It is commonly read aloud as ``$q$ if $p$'' or ``$q$ is a consequence of $p$''.
  }
\end{enumerate}
%
See Section 1.3 and Subsection 16.1.3 of \cite{SpecifyingSystems} for further reading.
%
\subsection{Iverson notation}%
Given a boolean expression $\phi$, %
$\Iverson{\phi}$ returns the truth value of $\phi$, encoded as follows, %
%
\begin{align} \nonumber
  \Iverson{\phi}\triangleq 
  \begin{cases}
    0 & \quad\quad \text{if } \phi \text{ is false;} \\
    1 & \quad\quad \text{if } \phi \text{ is true.}
  \end{cases}
\end{align}
%
For example, $\Iverson{1 > 0} = 1$ but $\Iverson{ \sqrt{2} \in \Q} = 0$.
\section{Special terms}
\subsection{Halmos' $\iif$ and definitions}%
\iif is a shorthand for ``if and only if". Splitting \iif into \textit{if-then} clauses shows that it is just a rewording of %
the logical equivalence $\iff$. All definitions will use the \iif format, which is consistent with the fact that %
every definition expresses a tautology.
%
\subsection{The assignment operator $\Def$}%
Given variables $\varit{a}$ and $\varit{b}$, $\triangleq$ is a specialization of $=$. We say that $a\triangleq b$ \iif %
$a$ and $b$ are assumed to be equal. Usually, $a\triangleq b$ means that $a$ is assigned the previously known value %
$b$ (some authors write $a:=b$) but this is not a limitation. Definitions can be redundant and may overlap. %
%
% % % % % % % % % % % % % % % % % % % % % % % % % % % % % % % % % % % % % % % % % % % % % % % % % % % % % % % % % % % % % % % %
% SETS
% % % % % % % % % % % % % % % % % % % % % % % % % % % % % % % % % % % % % % % % % % % % % % % % % % % % % % % % % % % % % % % %
\section{Sets}
\subsection{Subsets and supersets}%
$\subset$ and $\supset$ are the standard symbols for set ordering, as follows: %
\begin{align}
  X \subset Y & \defiif x \in X \then x\in Y \\
  Y \supset X & \defiif X \subset Y.
\end{align}
No specific symbol is reserved for strict ordering. If necessary, $X \neq Y$ will be explicitly stated. 
%
\subsection{Special mappings}
The identity $I$ is the special mapping $I_X = \set{(x,x)}{x\in X}$. Similarly, the projection %
$\pi = \set{((x, y), x)}{x \in X, y \in Y}$ always exists. Note that $I$ is the diagonal of $X^2$. %
%
\subsection{Equinumerosity}%
$X\equiv Y$ means that there exists a bijection $\phi$ that maps $X$ to $Y$, which lets us identify $X$ with $Y$. %
In a metric space context, $X\equiv Y$ means that $\phi$ is isometric. %
%
% % % % % % % % % % % % % % % % % % % % % % % % % % % % % % % % % % % % % % % % % % % % % % % % % % % % % % % % % % % % % % % %
% TVS 
% % % % % % % % % % % % % % % % % % % % % % % % % % % % % % % % % % % % % % % % % % % % % % % % % % % % % % % % % % % % % % % %
\section{Topological vector spaces}
\subsection{Scalar field}%
$\C$ extends $\R$, which implies that a property\eg linearity, that is true on $\C$ is also true on $\R$. The complex case %
is then a \textit{special case} of the real case. This restriction may be significant in some contexts. Nevertheless, the %
standard scalar field is $\C$, which means that considering $\R$ instead of $\C$ makes no difference, unless stated otherwise.%
%
% % % % % % % % % % % % % % % % % % % % % % % % % % % % % % % % % % % % % % % % % % % % % % % % % % % % % % % % % % % % % % % %
% Vector spaces
% % % % % % % % % % % % % % % % % % % % % % % % % % % % % % % % % % % % % % % % % % % % % % % % % % % % % % % % % % % % % % % %
\subsection{Vector space bases}\label{notations: vector spaces: vector space bases}
Given a vector space $X$ over $\C$ (or, more generally, over a field), a subset $B$ of $X$ is a basis of $X$ \iif %
the sum %
%
\begin{align}
  \bigl\{(z_u)_{u\in B}: z_u \in \C, \set{u}{z_u \neq 0} \text{ is finite} \bigr\} & \to X \\\nonumber
  (z_u) & \mapsto \sum_{z_u \neq 0 } z_u u
\end{align}
bijectively maps all \textit{finitely supported} $(z_u)$ \textit{onto} $X$. %
%
The axiom of choice (AC) forces %
\begin{enumerate}
  \item the existence of such $B$ %
    (the proof is similar to the second part of the Hahn-Banach theorem [3.1] of \cite{FA} with $B$ playing the role of 
    $\Lambda$); %
  \item all bases to have the same cardinal, which is called the {\it dimension} of $X$ and is denoted as $\dim(X)$. %
\end{enumerate}
%
We now turn to the finite-dimensional case. Note that the $0$-dimensional case is the degenerate case $B=\emptyset$, %
which is equivalent to $X=\singleton{0}$. Our first step is to study $\C^n$ ($n>0$) the standard $n$-dimensional vector space.
%
% % % % % % % % % % % % % % % % % % % % % % % % % % % % % % % % % % % % % % % % % % % % % % % % % % % % % % % % % % % % % % % %
% Finite dimension
% % % % % % % % % % % % % % % % % % % % % % % % % % % % % % % % % % % % % % % % % % % % % % % % % % % % % % % % % % % % % % % %
\subsection{Finite-dimensional spaces}\label{notations: vector spaces: finite-dimensional vector spaces}%
\subsubsection{The product topology of $\C^n$}%
\label{notations: vector spaces: finite-dimensional vector spaces: the product topology of Cn}
%
$\C^n$ has a standard basis $1_{\singleton{1}}, \dots, 1_{\singleton{n}} \in \{0, 1\}^{\{0, \dots, n-1\}}$ %
so that $z_k$ ($1 \leq k \leq n$) is the $k$-th component of a given %
$(z_1,\dots, z_n) = z_1 (1, 0, \dots, 0)  + \cdots  + z_n (0, \dots, 0, 1) \in \C^n$. Furthermore, %
$\C^n$ is endowed with the topology generated by all polydiscs %
%
\begin{align}
  \prod_{i=1}^{n} \underbrace{\set{z_i \in C}{\magnitude{z_i} < r_i}}_{D_{r_i}} \quad (r_i > 0 ).
\end{align}
%
Equivalently, we may equip $\C^n$ with the Euclidean norm % 
%
\begin{align}
  \norma{2}{z} \Def \sqrt{\magnitude{z_1}^2 + \cdots + \magnitude{z_n}^2} \quad \left(z = (z_1, \dots, z_n) \in \C^n\right), 
\end{align}
%
whose open balls centered at the origin are all %
%
\begin{align}
  B_r \Def \set{z\in \C^n}{\norma{2}{z} < r} \quad (r > 0).
\end{align}
%
To show the equivalence, first set $r_i = r/\sqrt{n}$. Hence %
%
\begin{align}
  \prod_{i=1}^{n} D_{r_i} \subset B_r. 
\end{align}
%
Conversely, put $r = \min\{r_1, \dots, r_n\}$ so that %
%
\begin{align}
   B_r \subset \prod_{i=1}^{n} D_{r_i} .
\end{align}
%
\subsubsection{Topology of a finite-dimensional vector space}
It is customary to identify any $n$-dimensional vector space with $\C^n$ equipped with the Euclidean norm, %
see [\ref{notations: vector spaces: finite-dimensional vector spaces: the product topology of Cn}]. %
%
To show this, pick an $n$-dimensional vector space $Y$ and let $f$ be an isomorphism of $\C^n$ onto $Y$. %
For instance, require that $u_k = f(e_k)$, like in [1.20] of \cite{FA}, as $\singleton{u_k}$ is a basis of $Y$, %
see [\ref{notations: vector spaces: vector space bases}]. It follows from [1.21] of \cite{FA} that $f$ is a homeomorphism. 
The striking consequence is that 
%
\begin{quote}
  $\set{f(U)}{U \text{ open in }\C^n}$ {is the only vector space topology for $Y$}. %
\end{quote}
%
Thus, $Y$ is necessarily locally convex and locally bounded \ie normable, see [1.39] of \cite{FA}. Note that %
$\norm{y}= \norma{2}{f^{\minus 1}(y)}$ ($y\in Y$) is an example of a norm. Additionally, $Y$ is locally compact, %
since the closed unit ball of $\C^n$ is compact. Now pick an $n$-dimensional topological vector space $W$, then repeat the %
same reasoning, first with $g: \C^n \to W$, next with $h = g\circ f^{\,\minus 1}$ playing the role of $f$. This establishes %
that the homeomorphism $h$ maps $Y$ onto $W$ and that $W$ is normable as well. It is now clear that the following assertions %
are equivalent in the finite-dimensional context: %
%
\begin{enumerate}
  \item $\dim(W) = \dim(Y),$ 
  \item $W$ and $Y$ are isomorphic to each other,%
  \item $W$ and $Y$ are homeomorphic to each other, they are normable.
\end{enumerate}
%
Furthermore, the norms on $W$ and $Y$ are \textit{equivalent}. That is, for any given norm $\norm{\, \cdot \, }$ on $Y$ and %
any given norm $\norma{W}{\,\cdot \,}$ on $W$, there exists a positive constant (termed as ``modulus of continuity'') $C= C_h$ %
such that %
\begin{align}
  \norma{W}{w} & \leq C \norm{y} \quad\quad \left((y, w) \in h \right), 
\end{align}
%
as $h$ is continuous.  The special case $W=Y$ is that all norms on $Y$ are equivalent, in the sense that %
%
\begin{align}
  \norma{{\text{copy of }Y}}{h(y)} \leq C \norm{y}.
\end{align}
%
\subsubsection{The standard norms $\norma{1}{\,\cdot\,}$, $\norma{2}{\,\cdot\,}$, $\norma{\infty}{\,\cdot\,}$} When $\C^n$ %
is equipped with standard norms $1, 2, \infty$, the sharp constant\ie the smallest $C_{i, j}$ such that %
%
\begin{align}
	\norma{j}{z} \leq C_{i, j} \norma{i}{z} %\quad \left(z \in \C^n \right)
\end{align}
%
is easily derived from definitions - see [1.19] of \cite{FA} - with the noticeable exception of $C_{2, 1} = \sqrt{n}$. %
Indeed, $\norma{1}{z} \leq \sqrt{n} \norma{2}{z}$ is a special Cauchy-Schwarz inequality, see (1) in [12.2] of \cite{FA}. %
The steps of this classical trick are left to the reader. %
%
\section{Measure theory on $\R^n$}
\subsection{Radon measures}
A positive Radon measure is a linear functional %
$\Lambda: C_c(\R^n) \to \C, \phi \mapsto \Lambda \phi$ that %
%
\renewcommand{\labelenumi}{(\roman{enumi})} 
\begin{enumerate}
  \item is positive\ie $\phi \geq 0 \then \Lambda\phi \geq 0$,
  \item is continuous in the sense that, for each compact $K \subset X$ there exists a modulus of continuity $M_K$ such that %
  \begin{align}
    \magnitude{\Lambda\phi} \leq M_K\norma{\infty}{\phi} \quad (\supp{\phi} \subset K ).
  \end{align}
\end{enumerate}
\renewcommand{\labelenumi}{(\alph{enumi})} 
% 
Theorem [2.15] of \cite{BigRudin} shows that (i) actually implies (ii), as (ii) defines Radon measures in a weaker sense, %
see \cite{AnalyseIII}. Moreover, $\Lambda$ is norm-bounded on $C_c(\R^n)$ \iif (ii) is satisfied with all %
$M_K\leq \norma{\infty}{\Lambda} <\infty$. These bounded linear functionals are at the core of the theory and [6.19] of %
\cite{BigRudin} isometrically identifies each of them with a specific regular Borel measure $\beta$. %
%
\subsection{Lebesgue integration}
Since every Radon measure $\Lambda$ has a Borel equivalent, we introduce % 
%
\begin{align}\label{definition of sum}
	\int_X \phi \diff{\Lambda} \Def \Lambda \phi, 
\end{align}
%
where the left side is a Lebesgue integral. The regularity property implies that $\Lambda$ has mass %
%
\begin{align}
	\int_X 1 \diff{\Lambda} \Def \sup \bigl\{\Lambda \phi: \phi \in C_c(\R^n), \norma{\infty}{\phi} \leq 1 \bigr\}.
\end{align}
%
The case $\diff{\Lambda} = f\diff{\beta}$, with regular Borel measure $\beta$ and $f$ Borel measurable, identifies %
\eqref{definition of sum} with the Lebesgue integral as follows:
%
\begin{align}\label{Borel integral}
	\int_X f \phi \diff{\beta} = \Lambda \phi \quad \left(\phi \in C_c(\R^n)\right).
\end{align}
%
In particular, if $f$ is positive, 
%
\begin{align}
	\int_{\R^n} f \diff{\beta} = \sup \bigl\{\Lambda \phi: \phi \in C_c(\R^n), \norma{\infty}{\phi} \leq 1 \bigr\}.
\end{align}
%
Thus, we define the following vector spaces %
%
\begin{align}
	L^1 \Def \Biggl\{f: \int_{\R^n}\magnitude{f} \diff{\beta} < \infty\Biggr\} 
  \subset \Biggl\{f: \forall K \subset X \text{ compact}: \int_{\R^n} 1_K \magnitude{f} \diff{\beta} < \infty \Biggr\} 
  \Def L^1_{\text{loc}} %
\end{align}
%
The collection $N$ of all $f$ whose magnitude $\magnitude{f}_1 = \int_{\R^n} \magnitude{f} \diff{\beta}$ is zero is a closed %
subspace of $L^1$. Moreover the quotient space $L^1/N$ is a Banach space. More specifically, %
%
\begin{align}
	\norma{1}{g + N} \Def \inf\set{\magnitude{f}_1}{ f - g  \in N } \quad (g \in L^1)
\end{align}
%
frames a complete norm for $L^1/N$. Now starting from \eqref{Borel integral} implies that $f$ and $\Lambda$ are identified %
modulo $N$\ie 
\begin{align}
	f\beta = g\beta \iff f-g \in N.
\end{align}
%
\subsection{Dirac's impulse, a physicist's detour}
Suppose we wish to record the impact of a particle on a surface. To do so, we introduce a carrier signal $\phi(t)$ that %
vanishes outside the observation interval. A given collision occurs at time $t_Q$, which means the surface absorbs a pulsed %
energy quantum $Q$. Thus, the particle contribution is encoded as the Heaviside step function $H(t) = \Iverson{t \geq 0}$, %
as time $t$ ranges over the real line. This models the fact that energy $H(t)$ only increases by an infinitely abrupt jump %
at reference time $t_Q=0$. In this detour, we heuristically conclude that %
%
\begin{align}
  \frac{\diff{H}}{\diff{t}} = \infty \times \Iverson{t=0}.
\end{align}
%
The latter informal derivative $\diff{H}/\diff{t}$ is empirically known as $\delta(t)$ the \textit{Dirac delta function}, 
or \textit{Dirac's impulse}. Interpreted as a density, $\delta$ carries unit mass because %
%
\begin{align}
  \int_{\minus \infty}^\infty \delta(t) \diff{t} = \int_{\minus \infty}^\infty \diff{H} = H(\infty) - H(\minus \infty) = 1.
\end{align}
%
This expresses that the collision releases the energy quantum. We now involve the carrier signal $\phi$, as follows: %
%
\begin{align}
  \int_{\minus \infty}^\infty \phi(t) \delta(t)\diff{t} & =  \int_{\minus \infty}^\infty  \phi \diff{H} & \\
  & =   \Delta H\phi - \int_{\minus \infty}^\infty H \diff{\phi} & [\text{integration by parts}]\\
  & = -\int_{\minus \infty}^\infty H \diff{\phi} \label{Dirac's impulse as a distribution}&[\text{$\supp{\phi}$ is compact}]\\
  & =  \phi(0) &
\end{align}
%
The key point is that the sum \eqref{Dirac's impulse as a distribution} no longer involves any $\diff{H}$. Moreover, %
we obtain $\Delta \phi = \phi(0)$, the filtered carrier signal $\phi \mapsto \phi\vert_{t=0}$. This motivates the following %
definitions, %
%
\begin{align}
  H(\phi)  & \Def \int_{\minus \infty}^\infty H \phi\diff{t}, \\
  \diff{H}(\phi) & \Def \minus \int_{\minus \infty}^\infty H \diff{\phi} = \phi(0), \\
  \delta(\phi) & \Def \phi(0).\label{Dirac's impulse definition}
\end{align}
%
This definition of $\delta$ conveys the idea of energy increasing by a fixed amount at pulsed time while being %
mathematically sound. Extension of \eqref{Dirac's impulse definition} to all $\phi \in C_c(\R^n)$ is immediate and turns %
$\delta$ into a Radon measure. Notably, its Borel-measure counterpart is the \textit{Dirac measure} %
%
\begin{align}
  \delta: E \to \Iverson{0 \in E} 
\end{align}
%
restricted to Borel sets in $\R^n$. Dirac's $\delta$ plays a pivotal role in mathematical physics, as it is the convolution %
identity. To see that, interpret the convolution $[\delta \ast \phi] (s)$ ($s \in \R$) as integration with respect to the %
Dirac measure. Hence %
%
\begin{align}
  [\delta \ast \phi](s)  = \int_{\minus \infty}^\infty \phi(s-t) \diff\delta(t) = \int_\R \phi_s \diff{\delta} 
  \citeq{\ref{definition of sum}} \delta(\phi_s) = \phi(s), 
\end{align}
%
as $\phi_s(t) = \phi(s-t)$. % 
%
% END
%



\renewcommand{\thesubsection}{\arabic{subsection}}
\renewcommand{\thesection}{\arabic{section}}
\mainmatter
%\part{Content} 
\setcounter{chapter}{0}
\chapter{Topological Vector Spaces}
%
\section{Exercise 1. Basic results}
%:1
\renewcommand{\labelenumi}{(\alph{enumi})} 
\textit{Suppose $X$ is a vector space. All sets mentioned below are understood 
  to be subsets of $X$. Prove the following statements from the axioms 
  as given as in section 1.4.
\begin{enumerate}
\item{If $x,\,y\in X$ there is a unique $z\in X$ such that $x+z=y$.}
\item{ $0\cdot x=0=\alpha\cdot 0 \quad (\alpha\in\C, x\in X)$.}
\item{ $2A\subset A+A$.}
\item{ $A$ is convex if and only if $(s+t)A=sA+tA$ %
  for all positive scalars $s$ and $t$.}
\item{ Every union (and intersection) of balanced sets is balanced.}
\item{ Every intersection of convex sets is convex.}
\item{ If $\Gamma$ is a collection of convex sets that is totally ordered by 
  set inclusion, then the union of all members of $\Gamma$ is convex.}
\item{ If $A$ and $B$ are convex, so is $A+B$.}
\item{ If $A$ and $B$ are balanced, so is $A+B$.}
\item{ Show that parts (f\,), (g) and (h) hold with subspaces in place of 
  convex sets.}
\end{enumerate}
}
%
\renewcommand{\labelenumi}{(\alph{enumi})} 
\begin{proof}
\begin{enumerate}
%: (a)
\item Such property only depends on the group structure of $X$: Each $x$ in
$X$ has an opposite $\minus x$. Let $x'$ be any opposite of $x$so that
${x-x=0}=x+x'$. %
Thus, $\minus x +x -x =  \minus x + x + x' $, %
which is equivalent to $\minus x = x'$. So is established the uniqueness of %
$\minus x $. %
%
It is now clear that $x+z=y$ \IFF $z=\minus x +y$, %
which asserts both the existence and the uniqueness of $z$.
%: (b)
\item Remark that %
%
\begin{align}
  0\cdot x & =(0+0)\cdot x=0\cdot x+0\cdot x \\
           & =(0+0)\cdot x=0 +0\cdot x 
\end{align} 
%
then conclude from (a) that $0\cdot x=0$. So, %
\begin{align} \label{inverse of x}
  0=0\cdot x=(1-1)\cdot x &=x+(\minus 1)\cdot x
  \Rightarrow \minus 1\cdot x= \minus x.
\end{align}
%
Finally, %
%
\begin{align}
  \alpha\cdot 0\overset{(\ref{inverse of x})}{=}
  \alpha\cdot (x+(\minus 1\cdot x))
  = \alpha \cdot x + \alpha \cdot (\minus 1) \cdot x 
  = (\alpha-\alpha )  \cdot x =0\cdot x = 0,
\end{align}
%
which proves (b).
%
%: (c)
\item Remark that 
%
\begin{align}
  2x =(1+1) x = x + x
\end{align}
%
for every $x$ in $X$, and so conclude that %
%
\begin{align}\label{double lies in sum}
  2A = \{2x: x\in A \} 
  = \{x + x: x \in A \} 
  \subset \{ x + y : (x,\,y) \in A^2 \} 
  = A+A
\end{align}
%
for all subsets $A$ of $X$; which proves (c). %
%: (d)
\item If $A$ is convex, then %
%
\begin{align}
  A \subset \frac{s}{s+t} A + \frac{t}{s+t} A \subset A;
\end{align}
%
which is %
%
\begin{align}
  sA + tA = (s+t)A.
\end{align}
%
Conversely, the special case $s+t=1$ is %
%
\begin{align}
  sA + (1-s)A = A.
\end{align}
%
The latter extends to $s=0$, since %
%
\begin{align}
  0A + A \overset{(b)}{=}\{0\}+A=A.
\end{align}
%
The extension to $s=1$ is analogously established %
(or simply use the fact that $+$ is commutative!).
So ends the proof. %
%: (e)
\item Let $A$ range over $B$ a collection of balanced subsetsso that %
%
\begin{align}
  \alpha \bigcap B \subset  \alpha A \subset A \subset \bigcup B 
\end{align}
%
for all scalars $\alpha$ of magnitude $\leq 1$. %
The inclusion $\alpha \bigcap B \subset A$ establishes the first part. %
Now remark that %
%
\begin{align}
  \alpha A  \subset \bigcup {B} 
\end{align}
%
implies %
%
\begin{align}
  \alpha \bigcup {B} \subset \bigcup {B};
\end{align}
which achieves the proof. %
%
%: (f)
\item Let $A$ range over $C$ a collection of convex subsetsso that %
%
\begin{align}
  (s+t) \bigcap C \subset s\bigcap C + t\bigcap C \subset  sA + tA 
  \overset{(d)}{\subset} (s+t)A
\end{align}
%
for all positives scalars $\mathit{s}$, $\mathit{t}$. %
Inclusions at both extremities force %
%
\begin{align}
  s\bigcap C  + t\bigcap C = (s+t) \bigcap C.
\end{align}
%
We now conclude from (d) that the intersection of $C$ is convex. %
So ends the proof.
%: (g)
\item Skip all trivial cases %
%
  $\Gamma = \emptyset$, %
  $\{ \emptyset\}$, %
  $\{\{x\}\}$, %
  $\{\emptyset$, %
  $\{x\}\}$ %
%
then pick $x_1, x_2$ in $\bigcup \Gamma$, 
so that each $x_i$ ($i=1, 2)$ lies in some $C_i \in \Gamma$. %
%
Since $\Gamma$ is totally ordered by set inclusion, we henceforth assume %
without loss of generality that $C_1$ is a subset of $C_2$. %
%
So, $x_1, x_2$ are now elements of the convex set $C_2$. %
Every convex combination of our $x_i$'s is then in %
$C_2 \subset \bigcup \Gamma$. Hence (g). %
%
%: (h)
\item Simply remark that 
%
\begin{align}
  s (A+B) + t (A+B) = s A+ t A +s B +t B = (s+t)(A+B)
\end{align}
%
for all positive scalars $\mathit{s}$ and $\mathit{t}$, %
then conclude from (d) that $A + B$ is convex. %
%: (i) 
\item Given any $\alpha$ from the closed unit disc, %
%
\begin{align}
\alpha(A+B)=\alpha A+ \alpha B \subset A+B. 
\end{align}
%
There is no more to prove: $A+B$ is balanced. %
%: (j)
\item Our proof will be based on the following lemma, %
%
\renewcommand{\labelenumii}{(\roman{enumii})}
\textit{
\begin{quote}
If $S$ is nonempty, then each of the following three properties
\begin{enumerate}
\item $S$ is a vector subspace of $X$;
\item $S$ is convex balanced such that $S + S = S$;
\item $S$ is convex balanced such that $\lambda S=S\quad (\lambda > 0)$
\end{enumerate}
implies the other two.
\end{quote}
}
%
To prove the lemma, let $\mathit{S}$ %
run through all nonempty subsets of $X$. %
First, assume that (i) holds: Clearly, every $S$ is convex balanced. %
Moreover, $S+S \subset S$.  Conversely, $S = S + \{0\} \subset S + S $; %
which establishes (ii). %
%
Next, assume (only) (ii): A proof by induction shows that %
%
\begin{align}\label{induction nS}
  nS = (n-1)  S + S = S + S = S \quad (n=1,2,3, \dots)
\end{align}
%
with the help of (b) and (d). %
Pick $\lambda >0$ then choose $n$ so large that $1 < n \lambda < n^2$. %
Thus, %
%
\begin{align}
  nS \overset{(\ref{induction nS})}{\subset} S 
  \subset n\,\lambda S 
  \subset  n^2 S, 
\end{align}
%
since $S$ is balanced. %
For instance, set %
%
  $n = \lceil{1/\lambda}\rceil + \lceil{\lambda}\rceil$. %
%
Dividing the latter inclusions by $n$ shows that %
%
\begin{align} 
  S \subset \lambda S \subset nS \overset{(\ref{induction nS})}{\subset} S,
\end{align}
%
which is (iii). Finally, dropping (ii) in favor of (iii) leads to %
%
\begin{align} 
  \alpha  S +\beta  S 
  \overset{(a)}{=} |\alpha | S + | \beta | S 
  \overset{(d)}{=} (|\alpha | + | \beta| )S 
  \overset{(iii)}{=} S 
  %
  \quad (|\alpha| + |\beta| > 0);
\end{align}
%
where the equality at the left holds as $S$ is balanced. %
%
Moreover (under the sole assumption that $S$ is balanced), %
this extends to $|\alpha| + |\beta| = 0$, as follows, %
%
\begin{align} 
  \alpha S + \beta S  = 0S + 0S\overset{(b)}{=} \{0\} 
  \overset{(b)}{=} 0S \subset S.
\end{align}
%
Hence (i), which achieves the lemma's proof. %
We will now offer a straightforward proof of (j). \\
\\
Let $V$ be a collection of vector spaces of $X$, %
of intersection $I$ and union $U$. 
%
First, remark that every member of $V$ is convex balanced: %
So is $I$ (combine (e) with (f)). %
%
Next, let $\mathit{Y}$ range over $V$so that %
%
\begin{align}
  I + I \subset Y + Y \subset  Y; 
\end{align}
%
which yields
%
\begin{align}
  I + I = I 
\end{align}
%
(the fact that $I  = I + \{0\} \subset I + I$ was tacitely used). %
%
It now follows from the lemma's (ii) $\Rightarrow$ (i) that %
$I$ is a vector subspace of $X$. %
%
Now temporarily assume that $S$ is totally ordered by set inclusion: %
Combining (e) with (g) establishes that $U$ is convex balanced. %
%
To show that $U$ is more specifically a vector subspace, %
we first remark that such total order implies that either %
$Z \subset Y$ or $Y \subset Z$, as $\mathit{Z}$ ranges over $V$. %
A straightforward consequence is that 
%
\begin{align}
  Y \subset Y + Z  \subset Y\cup Z.
\end{align}
%
Another one is that $Y \cup Z$ ranges over $V$ as well. %
Combined with the latter inclusions, this leads to %
%
\begin{align}
  U \subset U  + U \subset U.
\end{align}
%
It then follows from the lemma's (ii) $\Rightarrow$ (i) that %
$U$ is a vector subspace of $X$. %
%
Finally, let $\mathit{A},\mathit{B}$ run through all vector subspaces of $X$: %
Combining (h) with (i) proves that $A+B$ is convex balanced as well. %
%
Furthermore, %
%
\begin{align}
  A + B \overset{(i) \Rightarrow (ii)}{=} (A + A) + (B + B) = (A + B) + (A + B),
\end{align}
% 
where the equality at the right holds as $X$ is an abelian group. %
We now conclude from (ii) that any $A+B$ is a vector subspace of $X$. %
%
So ends the proof. %
\end{enumerate}
\end{proof}
% END
 
% 
\section{Exercise 2. Convex hull}
\textit{The convex hull of a set $A$ in a vector space $X$ is the set of all %
convex combinations of members of $A$, that is the set of all sums %
%
  $t_1 x_1 +\cdots +t_n x_n$ %
%
in which $x_i \in A,\, t_i \geq 0$, $\sum t_i = 1$; $n$ is arbitrary. 
%
Prove that the convex hull of a set $A$ is convex and that is the intersection 
of all convex sets that contain $A$.}
%
\begin{proof} The convex hull of a set $S$ will be denoted by $\co{S}$. %
Remark that $S \supseteq eq \co{S}$ %
(to see that, take $t_1 = 1$ for each $x_1$ in $S$) and that %
$\co{A} \supseteq eq \co{B}$ where $A \supseteq eq B$ (obvious). \\
\\
Our proof will directly derive from %
%
  (i) $\Rightarrow$ (iv) %
% 
in the following lemma, 
\renewcommand{\labelenumi}{(\roman{enumi})} 
\begin{quote}
\textit{Let $S$ be a subset of a vector space $X$: Its convex hull $\co{S}$ %
is convex and the following statements %
\begin{enumerate}
  \item $S$ is convex; %
  \item{
    %
      $s_1 S + \dots + s_n S = (s_1 + \cdots + s_n) S$ %
      %
      for all positive scalar variables $\mathit{s_1}, \dots, \mathit{s_n}$; %
  }
  \item{
    %
    $\mathit{t_1} S + \dots + \mathit{t_n} S = S$ %
    %
    for all positive scalar variables $\mathit{s_1}, \dots, \mathit{s_n}$ %
    such that %
    %
    $s_1 + \cdots + s_n = 1$; %
  }
  \item $\co{S} = S$
\end{enumerate}
are equivalent. %
}
\end{quote}
%
From now on, we skip the trivial case $S=\emptyset$ %
then only consider nonempty sets. %
To prove the first part, let $\mathit{a}$, $\mathit{b}$ %
range over $\co{S}$so that %
%
  $a = t_1 x_1 + \cdots + t_n x_n$ and %
  $b = t_{n+1} x_{n+1} + \cdots + t_{n+p} x_{n+p}$ %
%
for some $(\mathit{t_i}, \mathit{x_i})$. %
%
Every sum $sa + (1-s)b $ ($0\leq s \leq 1$) is then in the convex hull of %
$\{\mathit{x_1}, \dots, \mathit{x_{n+p}}\}$, since %
%
\begin{align}
  sa + (1-s)b = \sum_{i=1}^n st_i x_i + \sum_{i=n+1}^{n+p} (1-s)t_i x_i 
\end{align}
%
and
\begin{align}
  \sum_{i=1}^n st_i + \sum_{i=n+1}^{n+p} (1-s)t_i &= 
  s\sum_{i=1}^n t_i +(1-s) \sum_{i=n+1}^{n+p} t_i  = 1.
\end{align}
%
In terms of sets $S$, this reads as follows, %
%
\begin{align}
  s\co{S} + (1-s)\co{S} \subset \co{S}; 
\end{align}
%
which was our fist goal. %
We now aim at the equivalence %
%
(i) $\Rightarrow$ $\cdots$ $\Rightarrow$ (iv) $\Rightarrow$ (i): %
%
An easy proof by induction makes the implication (i) $\Rightarrow$ (ii) %
directly come from (d) of the above exercise 1, chapter 1. %
%
(iii) is a special case of (ii),
and the implication (iii) $\Rightarrow$ (iv) derives from the definition of %
the convex hull. %
%
We now close the chain with (iv) $\Rightarrow$ (i), %
by remarking that $S$ is convex whether $S = \co{S}$. %
%  
The lemma being proved, let us establish the second part. \\
\\
To do so, we start from the convexity of $\co{A}$ then set %
%
  $F = \{\co{A}\}$. %
%
We may enrich $F$ as follows,  %
\begin{align}
  B \in F \Rightarrow B \text{ is convex and contains }A.
\end{align}
Note that our initial predicate %
``[$F$ only encompasses] \textit{all convex sets that contain A}", %
is now the special case %
%
\begin{align}
  B \in F \Leftrightarrow B \text{ is convex and contains }A.
\end{align}
%
In any case, the key ingredient is that $\co{A} \in F$ implies %
%
\begin{align}
 \co{A} \supseteq  \bigcap_{B \in F} B.
\end{align}
%
Conversely, the next formula %
%
\begin{align}
  \co{A} \subset \co{B} \overset{(i) \Rightarrow (iv)}{=} B \quad (B \in F) 
\end{align}
%
is valid and implies %
\begin{align}
  \co{A} \subset \bigcap_{B \in F} B. 
\end{align}
%
So ends the proof
\end{proof}
% END
 
% 
\section{Exercise 3. Other basic results}
\renewcommand{\labelenumi}{(\alph{enumi})} 
\textit{
Let be X as topological vector space. All sets mentioned below are understood to be the subsets of X. Prove the following statements:
\begin{enumerate}
\item The convex hull of every open set is open.
\item If X is locally convex then the convex hull of every bounded set is bounded.
\item If A and B are bounded, so is A+B.
\item If A and B are compact, so is A+B.
\item If A is compact and B is closed, then A+B is closed.
\item The sum of two closed sets may fail to be closed.
\end{enumerate}
}
%: (a)
\begin{proof}
\begin{enumerate}
\item%
Pick an open set $A$ then let the variables $\mathit{V}_i$ %
($i=1, 2, \dots$) run through all open subsets of $A$so that % 
%
\begin{align}
  \co{A} \subset 
  \bigcup_{t_i} \, %
    (t_1V_1 + \cdots + t_i V_i  + \cdots) 
  \subset \co{A}
\end{align}
%
given all convex combinations %
%
  $t_1V_1 + \cdots + t_i V_i  + \cdots $. %
 %
We know from \citeFA{1.7} that those sums are open; %
which achieves the proof. %
%
%:(b)
\item Provided a bounded set $E$, %
pick $V$ a neighborhood of $0$: By (b) of Section 1.14 in \citeFA{14}, %
$V$ contains a convex neighborhood of $0$, say $W$. %
%
There so exists a positive scalar $s$ such that
%
\begin{align}
  E \subset tW \subset tV \quad (t>s); 
\end{align}
%
which yields %
%
\begin{align}
  \co{E} \subset \co{tW} = t\co{W} = t W \subset tV.
\end{align}
%
So ends the proof. %
%
%:(c)
\item At fixed $V$, neighborhood of the origin, %
we combine the continuity of $+$ with \citeFA{1.14} %
to conclude that there exists $U$ a balanced neighborhood of the origin %
such that %
%
\begin{align}
  U+U\subset V. 
\end{align}
%
Moreover, by the very definition of boundedness, %
$A \subset r U$ for some positive scalar $r$. % 
Similarly, $B \subset s U$ for some positive $s$. % 
%
Finally, 
%
\begin{align}
A+B \subset  rU + sU \subset tU + tU \subset tV \quad (t > r, s), 
\end{align}
%
since $U$ is balanced. So ends the proof. %
%
%:(d)
\item First, $A$ and $B$ are compact: So is $A\times B$. %
Next, $+$ maps continuously $A\times B$ onto $A+B$. %
In conclusion, $A+B$ is compact. %
%
%:(e)
\item From now on, we assume that neither $A$ nor $B$ is empty, %
since otherwise the result is trivial.  %
Now pick $c\in X$ outside $A+B$: %
The result will be established by showing that $c$ is not in the closure %
of $A+B$. \\
\\
To do so, we let the variable $\mathit{a}$ range over $A$: %
Every set $a+B$ is closed as well, see \citeFA{1.7}. %
%
Trivially, $a+B \neq c$: By Section \citeFA{1.10}, %
there so exists $V=V(a)$ a neighborhood of the origin such that %
%
\begin{align}\label{separation}
  (a+B + V) \cap (c+V) = \emptyset.
\end{align}
%
Moreover, there are finitely many $a+V$, say $a_1 + V_1, a_2 + V_2, \dots$, %
whose union $U$ contains the compact set $A$. Therefore, %
%
\begin{align}\label{U + B encloses A + B}
  A+B \subset U + B.
\end{align}
%
Now define %
\begin{align}
  W \triangleq V_1 \cap V_2 \cap \cdots, 
\end{align}
%
so that 
%
\begin{align}
  (a_i + B + V_i) \cap (c + W) \overset{(\ref{separation})}{=} \emptyset %
  \quad (i = 1, 2, \dots).
\end{align}
%
As a conclusion, $c$ is not in the closure of $U+B$. %
Finally, (\ref{U + B encloses A + B}) asserts that %
$c$ is not in $\overline{A+B}$ either; which achieves the proof. \\
\\
\textbf{Corollary}: If $B$ is the closure of a set $S$, then %
%
\begin{align}
  A+B \subset \overline{A+S} \subset \overline{A+B} = A + B
\end{align}
%
by \citeFA{(b) of 1.13} (since $A$ is closed; %
see Section 1.12, from the same source). %
The special case $A = \{x\}$, $B=X$ %
will occur in the proof of Exercise 15 in chapter 2. %
%
%:(f)
\item The last proof will consist in exhibiting a counterexample. %
To do so, let $f$ be any continuous mapping of the real line such that %
\renewcommand{\labelenumii}{(\roman{enumii})} 
\begin{enumerate}
  \item $f(x) + f(\minus x) \neq 0$ \quad ($x \in \R$);
  \item $f$ vanishes at infinity. 
\end{enumerate}
For instance, we may combine (ii) with $f$ even and $f>0$ by setting %
%
  $f(x) = 2^{\minus |x|}$, %
  $f(x) = e^{\minus x^2}$, %
  $f(x) = 1/(1+|x|)$, \dots, %
%
and so on. \\
\\
As a continuous function, $f$ has closed graph $G$, see \citeFA{2.14}. %
%
Moreover, (i) implies that the origin %
%
  $(0, 0) \neq \left(x-x, f(x)+ f(\minus x)\right)$ %
% 
is not in $G+G$. %
%
On the other hand, 
%
\begin{align}
  \{ \left(0, f(n) + f(\minus n)\right): n=1, 2, \dots\} \subset G + G.
\end{align}
%

%
Now the key ingredient is that % 
%
\begin{align}
  \left(0, f(n)+f(\minus n)\right) \overset{(ii)}{\tendsto{n}{\infty}} (0, 0). 
\end{align}
%
We have so constructed a sequence in $G+G$ that converges outside $G+G$. %
So ends the proof.
\end{enumerate}
\renewcommand{\labelenumi}{(\roman{enumi})} 
%
\end{proof}
%END
 
% 
\section{Exercise 4. A balanced set whose interior is not balanced}
\textit{Let be %
%
  $B = \{(z_{1}, z_{2}) \in \C^2: |z_{1}| \leq |z_{2}| \}$. %
%
Show that $B$ is balanced but that its interior is not.
}
%
\begin{proof}
It is obvious that the nonempty set $B$ contains the origin $(0,0)$. %
Additionally, its interior $B^\circ$ is nonempty as well. %
To see that, observe that the following set %
%
\begin{equation}
  \{(z_1, z_2) \in \C^2: |1- z_1| + |2- z_2] < 1/2 \} \subset B %
\end{equation}
%
is a neighborhood of $(1, 2) \in B$. %
Moreover, $B$ is balanced, since
\begin{equation}
  |\alpha z_1|  = |\alpha| |z_1| \leq  |\alpha| |z_2| = |\alpha z_2| %
  \qquad (|\alpha| \leq 1)
\end{equation}
%
for all $(z_1, z_2)$ in $B$. %
%
However, the nonempty set $B^\circ$ is not balanced, which we %
establish by showing that $(0, 0) \notin B^\circ$. %
%
To do so, assume, to reach a contradiction, %
that the origin has a neighborhood %
%
\begin{equation}
  U \Def \{(z_1, z_2) \in \C^2: |z_1| + |z_2] < r \} \subset B %
\end{equation}
%
for some positive $r$. Clearly, $U$ contains $(r/2, 0)$,  %
and that special case $(r/2, 0) \in B$ now contradicts the definition of $B$. %
So ends the proof.
\end{proof}
% END
 
% 
\section{Exercise 5. A first restatement of boundedness}
\textit{%
Consider the definition of ``bounded set'' given in Section 1.6. %
Would the content of this definition be altered if it merely %
required that to every neighborhood $V$ of 0 corresponds }some{ %
\textit{%
$t>0$ such that $E\subset tV$?
}
\begin{proof}
The answer is: No. To prove this, start from (a) of Section 1.14: %
$V$ contains $W$, a balanced neighborhood of $0$. %
%
Assume that $E$ is bounded in this weaker sense, \ie,  %
there exists a positive $t$ that satisfies %
%
\begin{equation}%
  E\subset tW.
\end{equation}
%
Thus, 
%
\begin{equation} 
  E\subset tW \subset sW \subset sV \qquad (s>t), 
\end{equation}
%
since $W$ is balanced. Thus, we recover the definition given in Section 1.6: %
The two definitions are equivalent.
\end{proof}
% END
 
% 
\section{Exercise 6. A second restatement of boundedness}
\textit{
Prove that a set $E$ in a topological vector space is bounded if and only if %
every countable subset of E is bounded.
}
\begin{proof}
It is clear that every subset of a bounded set is bounded. %
Conversely, assume that $E$ is not bounded then pick $V$ %
a neighbourhood of the origin: %
%
No counting number $n=1, 2, \dots$ satisfies %
%
  $E\subset nV$ (see Exercise 1 in Chapter 1). %
%
In other words, there exists a sequence %
%
  $\{x_1, \dots, x_n, \dots\} \subset E$ %
%
such that %
%
\begin{align}
  x_n \notin nV.
\end{align}
%
As a consequence, $x_n /n $ fails to converge to $0$ %
as $n$ tends to $\infty $. %
In contrast, $1/n$ succeeds. %
It then follows from Section 1.30 that %
%
  $\{x_1, \dots, x_n, \dots\}$ %
%
is not bounded. So ends the proof.
\end{proof}
% END
 
% 
\section{Exercise 7. Metrizability and number theory}
%\section{Exercise 7. Metrizability \& number theory}
\textit{
Let be X the vector space of all complex functions on the unit interval 
$[0, 1]$, topologized by the family of seminorms 
%
  \begin{align}
    p_{x}(f)=|f(x)| \quad (0\leq x\leq 1).\nonumber
  \end{align}
%
This topology is called the topology of pointwise convergence. 
Justify this terminology.
Show that there is a sequence $\{f_n\}$ in X such that (a) $\{f_n\}$ converges 
to $0$ as $n \to\infty$, but (b) if $\{γ_n\}$ is any sequence of scalars such 
that $γ_n\to\infty$ then $\{γ_nf_n\}$ does not converge to $0$. 
(Use the fact that the collection of all complex sequences converging to $0$ 
has the same cardinality as $[0, 1]$.)
This shows that metrizability cannot be omited in (b) of Theorem 1.28.
}
\begin{proof}
Our justification consists in proving that $\tau$-convergence and pointwise 
convergence are the same one. 
%
To do so, remark first that the family of the seminorms $p_{x}$ is separating.
By [1.37], the collection $\mathscr{B}$ of all finite intersections 
of the sets 
%
  \begin{align}
    V_{x, k} 
      \Def 
    \singleton{p_x < 2^{\minus k}} 
      \quad 
    (x \in [0, 1], k \in \N)
  \end{align}
%
is therefore a local base for a topology $\tau$ on $X$. Given 
%
  $\set{f_n}{\counting{n}}$, 
%
we put
%
\newcommand\off[1]{\function{off}(#1)}
%
\begin{align}
  \off{U} \Def \sum_{n=1}^\infty [f_n \notin U] \quad (U\in\tau),
\end{align}
%
with the convention %$\off{U}
%
  ``$\Sigma=\infty$'' %
%
whether the sum has no finite support. 
So, 
%
  \begin{align}
    %
    \label{Inequality boolean series}
    %
    \sum_{i=1}^m \off{U_{i}} 
      = 
    \sum_{n=1}^\infty \sum_{i=1}^m [f_n \notin U_{i}]
      \geq 
    \off{U_{1} \cuts \cdots \cuts U_{m}}.
  \end{align}
%
We first assume that $\singleton{f_n}$ $\tau$-converges to some $f$ in $X$, \ie
%
  \begin{align}
    \off{f+V} < \infty \quad(V \in \mathscr{B}).
  \end{align}
%
The special cases %
%
  $V_{x, 1}, V_{x, 2}, \dots, $ %
%
mean the pointwise convergence %
%
  $f_n(x) \overset{n\infty}{\longrightarrow} f(x)$. %
%
Conversely, assume that $\singleton{f_n}$ does not $\tau$-converges to any $g$ 
in $X$, \ie 
%
  \begin{align}
    %
    \label{Divergence}
    %
    \forall g \in X, \exists W \in \localbase{B}: 
      \off{g+W} = \infty. 
    %
  \end{align}
%
Given $g$, such $W$ is, by definition,  a finite intersection
%
  $
    V_{x_{1}, k_{1}} \cap \cdots \cap 
    V_{x_{m}, k_{m}} 
  $.
%
%So, (\ref{1.7 Inequality boolean series}) implies 
Thus,
%
  \begin{align}
    \sum_{i=1}^m \off{g + V_{x_{i}, k_{i}}} 
      %
        \citegeq{\ref{Inequality boolean series}} 
      %
    \off{g + W} 
    % 
      \citeq{\ref{Divergence}} 
    %
    \infty .
  \end{align}
%
One of the sum $\off{g + V_{x_{i}, k_{i}}}$ must then be $\infty$. 
In other words, there exists a point $x_i$ for which $\singleton{f_n(x_i)}$ %
does not converge to $g(x_i)$. %
$g$ being arbitrary, we so conclude that $f_n$ does not converge pointwise. %
We have just proved that 
%
  $\tau$-convergence is a rewording of pointwise convergence.
%
% SECOND PART
We now prove the second part.
%
From now on, we let %
%
  $\varit{k}$, $\varit{n}$ and $\varit{p}$ 
%
run on $\N_+$, as $\dy{x}$ denotes the usual dyadic expansion of $x$, %
so that $\dy{x}$ is an aperiodic binary sequence \iif $x$ is irrational. 
%
Define
%
\begin{align}
  %
  \label{f_n(x) definition}
  %
  f_n(x) 
    \Def 
  \begin{cases}
      \exp_2\left({\minus\sum_{k= 1}^{n} {\dy{x}_{\minus k}}}\right) 
        & 
          (x \in [0, 1]\setminus \Q )\\
      0 
        & 
          (x \in [0,1]\cap \Q), 
    \end{cases}
\end{align}
%
so that $f_n(x) \overset{n\infty}{\longrightarrow} 0$, %
%
and take % 
%
  $\gamma_n \overset{n\infty}{\longrightarrow} \infty$, \ie 
%
  at fixed $p$, $\gamma_{n}$ is greater than $2^{p}$ for almost all $n$.
%
Next, choose $n_{p}$ among those \textit{almost all} $n$ that are 
large enough to satisfy 
%
  \begin{align}
    n_{p-1} - n_{p-2} < n_{p}- n_{p-1} 
  \end{align}
%
(start with $n_{\minus 1} = n_{0} = 0$) and so obtain 
%
  \begin{align}
    2^p < \gamma_{n_{p}}:\, 
    %
      0< n_{p} - n_{p-1}\tendsto{p}{\infty} \infty.
    %
  \end{align}
%
The indicator $\chi$ of 
%
  $\{n_{1}, n_{2}, \dots\}$ in $\Z$
%
is then aperiodic, \ie 
%
  \def\xgamma{\alpha_{\gamma}}
  \begin{align}
    \xgamma 
      \Def
    \sum_{k=1}^\infty \chi_k 2^{\minus k} \in [0, 1]\setminus \Q .
      %\notin \Q
  \end{align}
%
Hence, $\chi$ is not a the infinite-support expansion of a rational number; %
which forces  %
%
  \begin{align}
    \dy{\xgamma}_{\minus k} &= \chi_{k}.
  \end{align}
%
The key ingredient is that %
%
  \begin{align}
    \chi_1 + \cdots + \chi_{n_{p}} = p\,.
  \end{align}
%
Combined with (\ref{f_n(x) definition}), it yields %
%
  \begin{align}
    f_{n_{p}}(\xgamma) = 2^{\minus p}.
  \end{align}
%
Finally,
%
  \begin{align}
    \gamma_{n_{p}} f_{n_{p}}(\xgamma) > 1.
  \end{align}
%
There so exists $\singleton{\gamma_{n_p}}$ such that %
%
  $\singleton{\gamma_{n_p} f_{\gamma_{n_p}}}$ %
%
fails to converge pointwise to $0$. %
In other words, (b) holds, which is in violent contrast with %
\citeresultFA{1.28}: $X$ is therefore not metrizable. So ends the proof.
\end{proof}
% END
 
% 
\setcounter{section}{8}
%
\section{Exercise 9. Quotient map}
\textit{Suppose
\begin{enumerate}
  \item $X$ and $Y$ are topological vector spaces,
  \item $\Lambda: X\to Y$ is linear.
  \item $N$ is a closed subspace of $X$,
  \item$\varit{\pi}: X\to X/N$ is the quotient map, and
  \item $\Lambda \varit{x}=0$ for every $\varit{x}\in N$.
\end{enumerate}
Prove that there is a unique $f:X/N\to Y$ which satisfies 
%
  $\Lambda = \mathit{f\circ \pi}$, 
%
that is, 
%
  $\Lambda \varit{x}= \mathit{f(\pi (x))}$ for all $\mathit{x\in X}$. 
%
Prove that $\varit{f}$ is linear and that $\Lambda$ is continuous %
%
if and only if %
%
  $\varit{f} $ is continuous. 
%
Also, $\Lambda$ is open if and only if $\varit{f}$ is open.
}
%
\begin{proof} Bear in mind that 
%
  $\pi$ continuously maps $X$ onto the topological (Hausdorff) space $X/N$, 
  since $N$ is closed (see \citeFA{1.41}).
%
Moreover, the equation 
% 
  $\Lambda = f \circ \pi$ 
% 
has necessarily a unique solution, which is the binary relation 
%
  \begin{equation}\label{1.9. definition of f.}
     f \Def\set{(\pi x, \Lambda x)}{x \in X} \subset X/N \times Y.
    \end{equation}
%
To ensure that $f$ is actually a mapping, simply remark that 
the linearity of $\Lambda$ implies 
%
  \begin{equation}
    %\forall ( x,  x') \in X^2: 
    %
    \Lambda x \neq \Lambda  x' \then \pi x' \neq \pi x'.
  \end{equation}
%
It straightforwardly derives from (\ref{1.9. definition of f.}) that 
$f$ inherits linearity from $\pi$ and $\Lambda$.\\
\\ 
{\bf Remark.} The special case 
%
  $N = \singleton{\Lambda = 0}$ , \ie,  $\Lambda x= 0$ \IFF $x\in N$ (\cf (e)), %
%
is the first isomorphism theorem in the topological spaces context. 
To see this, remark that this strengthening of (e) yields 
%
  \begin{equation}
    f(\pi x) = 0
      \citethen{\ref{1.9. definition of f.}}
    \Lambda x = 0
      \then 
    x \in N 
      \then 
    \pi x = N
\end{equation}
and so conclude that $f$ is also one-to-one.
%
%We have thus proved 
%
  %the first isomorphism theorem in the topological spaces context. %
%
\\\\
Now assume $f$ to be continuous. Then so is 
%
  $\Lambda = f\circ \pi $, 
% 
by \citeFA{1.41 (a)}. 
%
Conversely, 
%
if $\Lambda$ is continuous, then for each neighborhood $V$ of $0_Y$ 
there exists a neighborhood $U$ of $0_X$ such that
%
  \begin{equation}
    \Lambda(U) = f\left(\pi(U)\right) 
      \subset 
    V.
  \end{equation}
%
Since $\pi$ is open (\citeFA{1.41 (a)}), $\pi(U)$ is a neighborhood of 
%
  $N=0_{X/N}$: 
%s
This is sufficient to establish that the linear mapping $f$ is continuous.
%
If $f$ is open, so is $\Lambda = f\circ \pi$, by \citeFA{1.41 (a)}. 
%
To prove the converse, remark that 
%
  every neighborhood $W$ of $0_{X/N}$ satisfies %
%
  \begin{equation}
    W = \pi(V)
  \end{equation}
%
for some neighborhood $V$ of $0_X$. So, 
%
  \begin{equation}
    f(W) = f \left(\pi(V)\right) = \Lambda(V).
  \end{equation}
%
As a consequence, 
% 
  if $\Lambda$ is open, then $f(W)$ is a neighborhood of $0_Y$. %
%
So ends the proof.
\end{proof}


 
% 
\section{Exercise 10. An open mapping theorem}
%\section{1.10 Exercise 10. An open mapping theorem}
\textit{Suppose that X and Y are topological vector spaces,
%
  $\dim Y < \infty$,
%
$\Lambda : X \to Y$ is linear, and $\Lambda(X) = Y$.
%
  \begin{enumerate}
    \item{
      Prove that $\Lambda$ is an open mapping.}
    \item{
      Assume, in addition, that the null space of $\Lambda$ is closed, 
      and prove that $\Lambda$ is continuous.
    }
  \end{enumerate}
  %
}
%
\begin{proof}
%We discard the trivial case $\dim Y = 0$ then henceforth assume that $\dim Y$ 
%has positive dimension $n$. \\\\
%
Discard the trivial case $\Lambda = 0$ then assume that %
$\dim Y = n$ %
for some positive $n$. %
Let $e$ range over a base of $B$ of $Y$. %
Pick $W$ an arbitrary neighborhood of the origin: %
There so exists $V$ %
a balanced neighborhood of the origin such that 
%
  \begin{align}
    \label{definition of v}
    \underset{
      \text{Put } V \text{ exactly } n \text{ time(s)}
    }{
      \underbrace{V+\cdots +V}
    }\subset W, 
  \end{align}
%
since addition is continuous. %
Moreover, for each $e$, 
there exists $x_e$ in $X$ such that 
%
  $\Lambda(x_e)=e$, 
% 
simply because $\Lambda$ is onto. So,
%
  \begin{align}\label{1_10_sum}
    y = \sum_{e} y_e \cdot \Lambda x_e, 
  \end{align}
%
given any element $y=\sum_{e}y_e\cdot e$ of $Y$. %
As a finite set, $\set{x_e}{e\in B}$ is bounded: %
In particular, there exists  a positive scalar $s$ such that  
%
  \begin{align}
    \forall e\in B,  x_e \in s \cdot V.
  \end{align}
%
Combining this with (\ref{1_10_sum}) shows that 
%
  \begin{align}
    \label{y in sum of lambda V}
    y \in \sum_e y_e \cdot s \cdot \Lambda (V).
  \end{align}
%
We now come back to (\ref{definition of v}) and so conclude that %
%
  \begin{align}
    y \in \sum_e \Lambda (V) \subset \Lambda(W) 
  \end{align}
%
whether $\magnitude{y_e} < 1/s$; which proves (a).\\\\
%
%
To prove (b), assume that the null space 
%
  $\singleton{\Lambda = 0}$ %
% 
is closed and let $f, \pi$ be as in Exercise 1.9,  %
%
  $\singleton{\Lambda = 0}$ %
%
playing the role of $N$.
%%
% Isomorphism:
%  \begin{align}
%    & X \to                    X/N   \to                  Y . \\
%    & x \overset{\pi}{\mapsto} \pi x  \overset{f}{\mapsto} \Lambda x \nonumber
%  \end{align}
%
Since $\Lambda$ is onto, the first isomorphism theorem (see Exercise 1.9) 
asserts that 
%
  $f$ is an isomorphism of $X/N$ onto $Y$. 
%
Consequently, 
%
  \begin{align}
    \dim X/N= n.
  \end{align}
% 
$f$ is then an homeomorphism of 
%
  $X/N\equiv \C^{n}$ 
%
onto $Y$; see \citeresultFA{1.21}.
We have thus established that $f$ is continuous: So is $\Lambda = f\circ \pi$.
\end{proof} 
% 
\setcounter{section}{11}
%
\section{Exercise 12. Topology stays, completeness leaves}
\textit{
Suppose %
%
  $d_{1}(x,y)=|x-y|, d_{2}(x,y)=|\phi(x)-\phi(y)|$, %
%
where %
%
  $\phi(x)={x}/{(1+\lvert x \rvert )}$. %
%
Prove that $d_{1}$ and $d_{2}$ are metrics on $\R$ which induce %
the same topology, although $d_{1}$ is complete and $d_{2}$ is not.
}
\begin{proof}First, each $d_i\, (i=1,\,2)$ induces a topology $\tau_i\,$ spanned by set of open balls 
\begin{align}\label{1_12_2}
B_i(a,r)\Def \{ x\in \R:\, d_i (a,\,x) < r\} \quad (a\in \R\,,\, \, r\in \R_+ )\quad . \end{align}
Next, remark that the mapping $\phi:\,  \R\to (\minus 1\, ;\,1)\,$ is odd and that
\begin{align}\label{1_12_1}
1> \phi(x)=1-\frac{1}{x+1} \underset{x\infty}{\uparrow} 1 \quad \quad (x\> 0) \quad.
\end{align}
$\phi$ is then an $\tau_1$-homeomorphism of $\R\,$ onto $(\minus 1\,;\, 1)$. Pick $a\,$ in $\R\,$: given any positive scalar $\epsilon\,$ the $\tau_1$-continuity of $\phi\,$ supplies a positive scalar $\eta=\eta(\epsilon)\,$ so that 
\begin{align}
\forall x \in\R:\: (\, |a-x|< \eta \,\Rightarrow\, |\phi(a)-\phi(x)|< \epsilon)\quad , \end{align}
\ie
\begin{align}\label{1_12_5}
B_1(a,\eta)\subset B_2(a,\epsilon)\quad  .\end{align}
Keep $a\,$ and deduce from the $\tau_1$-continuity of $\phi^{\minus 1}:\, (\minus 1;\, 1)\to \R\,$ that there exists a positive scalar $\epsilon'\,$ such that 
\begin{align} \label{1_12_6}
\quad B_{2}(a,\epsilon')\subset B_{1}(a,\,\eta')  \quad  ,\end{align}
provided a positive scalar $\eta'\,$. The special case $\eta' \Def \eta(\epsilon)\,$ leads us to 
\begin{align}
B_{2}(a,\epsilon')\overset{(\ref{1_12_6})}{\subset} B_{1}(a,\eta ) \overset{(\ref{1_12_5})}{\subset} B_2(a,\epsilon)\quad . \end{align}
This yields
\begin{align}\tau_1=\tau_2 \quad . \end{align}
Finally, let $m\,$ and $n\,$ range $\N\,$, so that
\begin{align}
\quad d_{2}(m,\,n)=|\phi(m)-\phi(n)| {\underset{m> n \infty}{\longrightarrow}}0\quad .
\quad \end{align}
The natural numbers sequence is then a $\tau_2$-Cauchy one that $\tau_2$-diverges, since
\begin{align}
d_{2}(0,\,n) = \phi (n) \underset{n\infty}{\longrightarrow}1 \, \notin \, \phi(\R) \quad .\end{align}
Hence $d_2\,$ fails to be complete.
\end{proof}
% 
\setcounter{section}{13}
%
\section{Exercise 14. $\mathscr{D}_K$ equipped with other seminorms}
\textit{Put $K =[0, 1]$ and define $\D_K$ as in Section 1.46. %
Show that the following three families of seminorms %
(where $n = 0, 1, 2, \dots$) define the same topology on $\D_K$. 
If $D = d/dx$: 
%
\begin{enumerate}
\item{
  $\| D^n f \|_\infty = \sup\set{\left| D^n f(x)\right|}{\infty< x< \infty}$
}
\item{
  $\| D^n f \|_1 =\int_0^1 \left|D^n f(\varit{x}) \right| d\varit{x}$
}
\item{
  $\| D^n f \|_2 = \left\{
    \int_0^1 | D^n f(\varit{x}) |^2  d\varit{x} 
  \right\}^{1/2}.$
}
\end{enumerate}
%
}
%
\begin{proof}%
Let us equipp $\D_K$ with the inner product %
%
  $\bra{f}\ket{g} = \int_0^1 f \, \bar{g}$so that %
  $\bra{f}\ket{f} = \| f \|_2^2$. %
%
The following %
%
\begin{align}
  \int_0^1 1\,| D^n f| \leq \| 1 \|_2 \| D^n f \|_2 
\end{align}
%
is then a Cauchy-Schwarz inequality, see Theorem 12.2 of \cite{FA}. %
We so obtain %
%
\begin{align}\label{inequalities 1}
  \| D^n f \|_1 
    \leq 
  \| D^n f \|_2 
    \leq 
  \| D^n f \|_\infty 
    <\infty 
\end{align}
%
since $K$ has length $1$. %
%
Obviously, the support of $D^n f$ lies in $K$, hence the below equality %
%
\begin{align}\label{inequalities 2}
  |D^n f(x)|
    =
  \left|\int_{0}^x D^{n+1}f\right|
    \leq 
  \int_{0}^x |D^{n+1}f|
    \leq 
  \|D^{n+1}f \|_1.
\end{align}
%
Take the supremum over all $|D^n f(x)|$: Combining (\ref{inequalities 1}) %
with (\ref{inequalities 2}) now reads as follows, %
%
\begin{align}\label{inequalities 3}
  \|D^n f\|_1 
    \leq 
  \|D^{n} f\|_2 
    \leq 
  \|D^{n} f\|_\infty 
    \leq 
  \|D^{n+1}f\|_1 
    \leq 
  \cdots < \infty.
\end{align}
%
Finally, put %
%
\begin{align}
  V^{(i)}_n         & \triangleq \{f\in \D_K: \|f \|_i < 2^{\,\minus n}\}, \\
  \mathscr{B}^{(i)} & \triangleq \{V^{(i)}_n: n = 0, 1, 2, \dots\},
\end{align}
%
so that (\ref{inequalities 3}) is mirrored by neighborhood inclusions, %
provided $i=1, 2, \infty$:
%
\begin{align}\label{inclusions}
  V^{(1)}_n
    \supseteq  
  V^{(2)}_n 
    \supseteq  
  V^{(\infty)}_n 
    \supseteq  
  V^{(1)}_{n+1} 
    \supseteq  
  \cdots .
\end{align}
%
Their subchains $V^{(i)}_n\supseteq V^{(i)}_{n+1}$ turn $\mathscr{B}^{(i)}$ %
into a local base of a topology $\tau_i$. 
The whole chain (\ref{inclusions}) then forces %
%
\begin{align}
  \tau_1 \subset \tau_2 \subset \tau_\infty \subset \tau_1;
\end{align}
%
which achieves the proof.
\end{proof}
% END

 
% 
\setcounter{section}{15}
%
\section{Exercise 16. Uniqueness of topology for test functions}
%!TEX root = /Volumes/HD_2/Rudin/Rudin_DM.tex
\textit{
Prove that the topology of $C(\Omega)$ does not depend on the particular 
choice of $\{K_{n}\}$, as long as this sequence satisfies the conditions 
specified in section 1.44. Do the same for $C^\infty(\Omega)$ (Section 1.46).}
%
\paragraph{Lemma} Let $X$ be a topological space with a countable local base 
$\set{V_N}{\counting{N}}$. 
If 
%
  $\tilde{V}_N = V_1 \cuts \cdots \cuts V_N$, 
%
then every subsequence 
% 
  $\singleton{\tilde{V}_{\rho(N)}}$ 
%
is a also a {decreasing} (\ie 
%
  $\tilde{V}_{\rho(N)} \contains \tilde{V}_{\rho(N+1)}$)
%
local base of $X$.
%
\begin{proof}
The decreasing property is trivial. Now remark that 
%
  $V_N \contains \tilde{V}_N \contains \tilde{V}_{N+1}$:
%
The left inclusion shows that 
%
  $\singleton{\tilde{V}_N}$ 
% 
is a local base of $X$. Then so is 
%
  $\singleton{\tilde{V}_{\rho(N)}}$,
% 
since $\tilde{V}_N \contains \tilde{V}_{\rho(N)}$.
\end{proof}
%
\paragraph{Corollary}
If 
%
  $\singleton{Q_N}$ 
%
is a sequence of compacts that satisfies the conditions specified 
in section 1.44, then every subsequence 
%
  $\singleton{Q_{\rho(N)}}$ 
%
also satisfies theses conditions.
%
Furthermore, if $\tau^{Q}$ is the $C(\Omega)$'s 
(respectively $C^\infty (\Omega)$) topology of the seminorms $p^{Q}_N = p_N$, 
as defined in section 1.44 (respectively 1.46), then the seminorms 
%
  $p^{Q}_{\rho(N)}$ 
%
define the same topology $\tau^{Q}$.
%
\begin{proof}%
%
Let $X$ be $C(\Omega)$ topologized with the seminorms $p^{Q}_N$ 
(the case $X=C^\infty(\Omega)$ is proved the same way).
%
If 
  %
    $V^{Q}_N = \singleton{p^{Q}_N < 1/N}$, 
  %
then 
  %
    $\singleton{V^{Q}_N}$ 
  %
is a decreasing local base of $X$.
%
Moreover,
% 
  \begin{align}
    Q_{\rho(N)} 
      \subset 
    \interior{Q}_{\rho(N) + 1} 
      \subset 
    Q_{\rho(N) + 1} 
      \subset 
    Q_{\rho(N+ 1)},
  \end{align}
% 
and this yields
%
  \begin{align}
    Q_{\rho(N)} 
      \subset 
    \interior{Q}_{\rho(N+ 1)}.
  \end{align}
%
In other words, 
%
  $Q_{\rho(N)}$ satisfies the conditions specified in section 1.44.
%
%
  $\singleton{p^{Q}_{\rho(N)}}$
% 
then defines a topology $\tau^{Q_\rho}$ for which  
% 
  $\singleton{V^{Q}_{\rho(N)}}$ 
%
is a local base. So, 
% 
  $\tau^{Q_\rho} \subset \tau^{Q}$.
%
Conversely, the Lemma turns 
%
  $\singleton{V^{Q}_{\rho(N)}}$ 
%
into a local base of $\tau^{Q}$. Hence  
%
  $\tau^{Q}\subset \tau^{Q_\rho}$.
%
\end{proof}
\paragraph{Theorem} 
The topology of $C(\Omega)$ does not depend on the particular choice of 
$\{K_{n}\}$, as long as this sequence satisfies the conditions 
specified in section 1.44. Neither does the topology of $C^\infty (\Omega)$, 
as long as this sequence satisfies the conditions specified in section 1.46.
%
\begin{proof}%
With the Corollary's notations,
% 
  $\tau^{K} = \tau^{K_\kappa}$,
%
for every subsequence $\singleton{K_{\kappa(n)}}$.
% 
Similarly, let 
%
  $\singleton{L_n}$ 
% 
be a sequence of compact subsets of $\Omega$ that satisfies 
the condition specified in [1.44], 
so that 
%
  $\tau^{L} = \tau^{L_\lambda}$
%
for every subsequence $\singleton{L_{\lambda(n)}}$. 
%
The following definition
%
  \begin{align}
    C_{i, j} \Def K_i \setminus \interior{L_j} \quad (\counting{i, j})
  \end{align}
%
turns $\set{C_{i, j}}{\counting{j}}$ into a decreasing sequence of compacts.
%
We now suppose (to reach a contradiction) that 
% 
  no $C_{i,j}$ is empty 
% 
and so conclude that 
% 
  $\bigcap_{j=1}^\infty C_{i, j}$ 
%
contains a point that is not in any $L_j$. 
But the conditions specified in [1.44] force 
% 
  $\singleton{\interior{L}_j}$ 
%
to be an open cover.
% 
This contradiction reveals that $C_{i,j}, C_{i, j+1}, C_{i, j+2}, \dots,$  
are actually empty for some $j=j\up{i}$. We then define  
%
  $\lambda(i) = i + j\up{i}$, 
% 
so that
%
  \begin{align}
    %
    \label{1_16. K subset interior L}
    %
    K_i 
      \subset 
    \interior{L}_{\lambda(i)}.
  \end{align} 
%
Let us reiterate the above proof with $K_n$ and $L_n$ in exchanged roles 
then similarly find a subsequence $\set{\kappa(j)}{\counting{j}}$ such that 
%
  \begin{align}
  %
  \label{1_16. L subset interior K}
  %
    L_j \subset \interior{K}_{\kappa(j)}
  \end{align}
%
Combine 
%
  (\ref{1_16. K subset interior L}) with 
  (\ref{1_16. L subset interior K}) 
%
and so obtain
%
  \begin{align}
    K_1 
      \subset 
    \interior{L}_{\lambda(1)} 
      \subset 
    L_{\lambda(1)} 
      \subset 
    \interior{K}_{\kappa\circ\lambda(1)}
      \subset 
    K_{\kappa\circ\lambda(1)}
      \subset
    \interior{L}_{\lambda\circ\kappa\circ\lambda(1)}
      \subset
    \cdots
  \end{align}
%
Thus the sequence 
%
  $Q = (
    K_1, 
    L_{\lambda(1)}, 
    K_{\kappa\circ\lambda(1)}, 
    L_{\lambda\circ\kappa\circ\lambda(1)},
    \dots
  )$

satisfies the conditions specified in section 1.44. 
It now follows from the Corollary that 
%
  \begin{align}  
    \tau^{K} 
    = 
      \tau^{K_\kappa} 
    = 
      \tau^{Q} 
    = 
      \tau^{L_\lambda} 
    = \tau^{L}.
  \end{align} 
%
So ends the proof
\end{proof}
% END
 
% 
\section{Exercise 17. Derivation in some non normed space}
\textit{In the setting of Section 1.46, prove that 
  %
    $f \mapsto D^{\alpha}f$ 
  %
is a continuous mapping of 
%
  $C^{\infty}(\Omega)$ into 
  $C^{\infty}(\Omega)$ and also of 
  $\D_{K}$ into 
  $\D_{K}$, for every multi-index $\alpha$.
%
}
\begin{proof} 
In both cases, $D^\alpha$ is a linear mapping. 
It is then sufficient to establish continuousness at the origin.
%
We begin with the $C^\infty(\Omega)$ case. \\
\\
Let $U$ be an aribtray neighborhood of the origin.
There so exists $N$ such that $U$ contains
%
  \begin{align} 
    V_{N}= \set{
      \phi \in C^\infty(\Omega)
    }{
      \max
      \set{
        | D^\beta\phi(x) |
      }{
        | \beta | \leq N, x\in K_N
      }
    < 1/N
    }.
  \end{align}
%
Now pick $g$ in $V_{N+|\alpha|}$, so that
%
  \begin{align}
    \max
    \set{
      | D^\gamma g(x) |
    }{
      | \gamma | \leq N+| \alpha |, 
      x\in K_{N}
    }
    < \frac{1}{N}.
  \end{align}
%
(the fact that $K_N\subset K_{N+|\alpha|}$ was tacitely used).
%
The special case $\gamma = \beta + \alpha$ yields
%‡
\begin{align}
    \sup
    \set{
      | D^\beta D^\alpha g(x) |
    }{
      | \beta | \leq N, 
      x\in K_{N}
    }
    < \frac{1}{N}.
  \end{align}
%
We have just proved that
%
  \begin{align}
    g \in V_{N + | \alpha|}
      \then 
    D^\alpha g \in V_{N},
      \quad
      \ie
      \quad
    D^\alpha (V_{N+|\alpha|}) \subset V_N.
  \end{align}
%
The continuity of 
  $D^{\alpha}: C^\infty (\Omega) \to C^\infty (\Omega)$ 
is so established.
%%
%
We now prove the second part, \ie 
that $D^\alpha: \mathscr{D}_K \to \mathscr{D}_K$ is continuous. \\
\\
%
Let $r$ be a positive scalar. From now on, $V(r)$ will denote the 
preimage in $\D_K$ of the open disc $D(r)$  \ie 
%
\begin{align}
  V(r) \Def \set{f \in \D_K}{| D^\alpha f | < r}.
\end{align}
%
Similarly, we define $A(r)$ as the preimage of $D(r)$ in $C^\infty(\Omega)$, 
\ie 
%
\begin{align}
  A(r) \Def \set{f \in C^\infty (\Omega)}{| D^\alpha f | < r }.
\end{align}
%
The continuousness of 
%
  $D^\alpha$ in $C^\infty(\Omega)$ 
%
implies that $A(r)$ is open in $C^\infty(\Omega)$. Moreover, 
%
  $B(r) = A(r) \cap \ \D_K$. 
%
$B(r)$ is then open in $\D_K$, for all $r$. So ends the proof.
\end{proof}

%
% END 
% 

\chapter{Completeness}
\section{Exercise 3. An equicontinous sequence of measures}
% % % % % % % % % % % % % % % % % % % % % % % % % % % % % % % % % % % % % % % % % % % % % % % % % % % % % % % % % % % % % % % %
% FunctionalAnalysis 
% 2.03.tex
% 
% encoding: UTF-8 
% EOL: LF
%
% format: LaTeX
% indent: spaces (2)
% width: 127
% % % % % % % % % % % % % % % % % % % % % % % % % % % % % % % % % % % % % % % % % % % % % % % % % % % % % % % % % % % % % % % %
\textit{Put $K=\closedInterval*{\minus 1}{1}$; define $\D_{K}$ as in Section 1.46 (with $\R$ in place of $\R^n$). Suppose %
$\{f_{n}\}$ is a sequence of Lebesgue integrable functions such that %
$\Lambda(\phi) = \underset{n \to \infty}{\lim} \int_{\minus 1}^1 f_{n}(t)\phi(t) \diff{t}$ exists for every $\phi\in\D_{K}$. %
Show that $\Lambda$ is a continuous functional on $\D_{K}$. Show that there is a positive integer $p$ and a number $M<\infty$ %
such that
%
\begin{equation*}
  \magnitude[\Bigg]{\int_{\minus 1}^1 f_n (t)\phi (t) \diff{t}}\leq M \norm*[\infty]{D^p\phi}
\end{equation*}
%
for all $n$. For example, if $f_{n}(t)=n^{3}t$ on $\closedInterval*{\minus 1}{1}$ and $0$ elsewhere, show that this can be %
done with $p=1$. Construct an example where it can be done with $p=2$ but not with $p=1$.}
%
\begin{proof}
% % % % % % % % % % % % % % % % % % % % % % % % % % % % % % % % % % % % % % % % % % % % % % % % % % % % % % % % % % % % % % % %
% First part: Introduction, f_n as a bounded and signed Radon measure.
% % % % % % % % % % % % % % % % % % % % % % % % % % % % % % % % % % % % % % % % % % % % % % % % % % % % % % % % % % % % % % % %
Equipped with the supremum norm, $\Continuous_K=C(\R)\cap \set{u}{\supp{u} \subset K}$ is the copy of $C_c(K)$ in $C(\R)$. %
Each density $f_n$ is then identified modulo $N(\beta_\R)$ with the following Radon measure%
%
\begin{align}
  \mu_n: \Continuous_K &\to \C \\
  u &\mapsto \int_{\minus 1}^{1} f_n(t)u(t)\diff{t}; \nonumber
\end{align}
%
see [\ref{notations:Radon-measures}] and [\ref{notations:Lebesgue-integration}] for definitions and notations. Every %
$\mu_n$ is continuous since $\norm*{\mu_n} = \norm*[1]{f_n}$ is finite; \cf \citeFA{6.19}. Note that the dual space %
$\Continuous_K^\ast$ is a Banach space as well, by \citeFA{4.1}. Here, we consider only pointwise convergence, %
which is weaker than norm convergence. Define
%
\begin{equation}
  \Lambda_n \Def \restriction{\mu_n}{\D_K},
\end{equation}
%
and assume the pointwise convergence of the sequence $\singleton{\Lambda_n}$, that is 
%
\begin{equation}\label{2.3:assumed-pointwise-convergence}
  \Lambda_n(\phi) \tendsto{n}{\infty} \Lambda(\phi) \qquad(\phi \in \D_K).
\end{equation}
%
We show that the linear form $\Lambda$ is continuous in the topology of $\D_K$. We also construct a particular sequence %
$\{\Lambda_n\}$ whose pointwise limit $\Lambda$ is not norm-bounded. By contraposition, Theorem \citeFA{2.8} implies that %
pointwise convergence on $\D_K$ does not extend to $\Continuous_K$. This conclusion also follows from the following bounds:
%
\begin{align}
  \big\lvert \Lambda_n(\phi) \big\rvert &\leq M \norm*[\infty]{\phi'} \label{2.3:norm-p-bound1},\\
  \magnitude*{\Lambda_n(\phi')} &\leq M \norm*[\infty]{\phi''} \label{2.3:norm-p-bound2}.
\end{align}
%
Combined with the lack of boundedness at order $p=0$, \cf \eqref{2.3:norm-p-bound1}, the contraposition of Theorem %
\citeFA{2.6} implies that $\{\Lambda_n\}$ does not converge pointwise on $\Continuous_K$. In Radon measure theory, pointwise %
convergence is known as \emph{vague convergence}. The rest of the current proof has been split into the following paragraphs,
%
\begin{description}
  \item[Continuity of $\Lambda$:]{%
    The sequence $\singleton{\Lambda_n}$ has pointwise limit $\Lambda \in \D_K^\ast$.
  }
  \item[Existence of the uniform bound $\magnitude*{\Lambda_n(\phi)}\leq M\norm*{D^p \phi}$:]{%
    $\singleton{\Lambda_n}$ is equicontinuous, hence $\magnitude*{\Lambda_n(\phi)} \leq M \norm*{D^p \phi}$.
  }
  \item[Existence of $\singleton{\Lambda_n}$ with optimal uniform bound $\magnitude*{\Lambda_n(\phi)}\leq M \norm*{D \phi}$:]{
    In this example, no similar bound in $\norm*{\phi}$ exists. We show that $\Lambda$ does not extend to a Radon measure. 
  }
  \item[Case of a uniform bound in $\norm*{D^2\phi}$:]{%
    Where the smallest possible $p$ is $2$}.
\end{description}
%
We first prove that $\displaystyle{\lim_{n\to \infty}{\restriction{\Lambda_n}{\D_K}}}$ is continuous in $\D_K$'s topology.
% % % % % % % % % % % % % % % % % % % % % % % % % % % % % % % % % % % % % % % % % % % % % % % % % % % % % % % % % % % % % % % %
% Second part: The limit Lambda is continuous 
% % % % % % % % % % % % % % % % % % % % % % % % % % % % % % % % % % % % % % % % % % % % % % % % % % % % % % % % % % % % % % % %
\paragraph{Continuity of $\Lambda$.} We equip $\D_K$ with derivative norms %
$\norm*[N]{\phi}\Def \norm*[\infty]{\phi} + \norm*[\infty]{D\phi} + \cdots + \norm*[\infty]{D^N\phi}$ for all $\phi \in \D_K$ %
and all $\counting{N}$. The induced topology $\tau_K$ of $\D_K$ is the weakest topology that makes all norms %
$\norm*[N]{\cdot}$ continuous, \cf \citeFA{1.46, 6.2} and Exercise [1.16]. Equivalently, the collection of all convex %
balanced sets %
%
\begin{equation}\label{2.3:local-base-of-DK}
  V_N \Def \set[\Big]{\phi}{\phi \in \D_K, \norm*[N]{\phi}< 1/N}
\end{equation}
%
forms a local base of $\tau_K$, \cf \citeFA{6.2}. Note that $\norm*[N]{\phi} < 1$ implies $\norm*[\infty]{\phi} < 1 $: %
Every $\Lambda_n$ is then $\tau_K$-continuous by \citeFA{1.18}, since $\magnitude*{\Lambda_n}$ is bounded by %
$\norm*{\Lambda_n}$ on $V_1$. In summary: %
%
\begin{enumerate}
  \item{$\D_K$, equipped with the topology $\tau_K$, is a Fréchet space; see \citeFA{1.46}}.
    %
  \item{Every functional $\Lambda_n$ is $\tau_K$-continuous.}
    %
  \item{
      $\Lambda_n(\phi) \to \Lambda(\phi)$ pointwise on $\D_K$ (our premise).
    }
\end{enumerate}
%
The hypotheses of the Banach-Steinhaus theorem are then satisfied, which supports the following existence proof.
% % % % % % % % % % % % % % % % % % % % % % % % % % % % % % % % % % % % % % % % % % % % % % % % % % % % % % % % % % % % % % % %
% Third part: The sequence {Lambda_n} is equicontinuous, hence the uniform bound |Lambda_n .| < M |.| .
% % % % % % % % % % % % % % % % % % % % % % % % % % % % % % % % % % % % % % % % % % % % % % % % % % % % % % % % % % % % % % % %
\paragraph{Existence of the uniform bound $\magnitude*{\Lambda_n(\phi)}\leq M\norm*{D^p \phi}$.} By \citeFA{2.6, 2.8}, %
the equicontinuous sequence $\singleton{\Lambda_n}$ converges pointwise to a continuous $\Lambda$. Furthermore, the %
equicontinuity of $\singleton{\Lambda_n}$ ensures that all $\magnitude*{\Lambda_n}$ remain below $1$ on a common balanced %
neighborhood $V_p$. Hence 
%
\begin{equation}
  \frac{1}{p}\cdot\frac{\phi}{\norm*[p]{\phi} + \epsilon} \in V_p \qquad (\epsilon > 0).
\end{equation}
%
This gives $\magnitude*{\Lambda_n(\phi)}<  p\cdot \Bigl\{\norm*[p]{\phi} + \epsilon \Bigr\}$, %
which reduces to $\magnitude*{\Lambda_n(\phi)}\leq  p\norm*[p]{\phi}$. Use [\ref{annex:mean-value-with-derivatives}] with %
$I, D, \dots, D^p$ to output %
%
\begin{equation}\label{2.3:bound-from-derivative}
  \magnitude*{\Lambda_n(\phi)} \leq p(p+1) \norm*[\infty]{D^p \phi}.
\end{equation}
%
This completes the first part of the proof, with some $p$ and a positive constant $M = M(p)$. We now construct an example of 
$\singleton{\Lambda_n}$.
% % % % % % % % % % % % % % % % % % % % % % % % % % % % % % % % % % % % % % % % % % % % % % % % % % % % % % % % % % % % % % % %
% Fourth part: Counterexample
% % % % % % % % % % % % % % % % % % % % % % % % % % % % % % % % % % % % % % % % % % % % % % % % % % % % % % % % % % % % % % % %
\paragraph{Existence of $\singleton{\Lambda_n}$ with optimal uniform bound $\magnitude*{\Lambda_n(\phi)}\leq M \norm*{D\phi}$.}
Let $\psi$ be a smooth even mapping that equals $1$ on $\closedInterval*{\minus 1/2}{1/2}$, vanishes outside %
$\closedInterval*{\minus 1}{1}$, and satisfies $0\leq \psi \leq 1$ on $\R$. $\psi$ can be obtained from the general %
construction in \citeFA{1.46}. Alternatively, take the derivative of $\phi$ from Lemma %
[\ref{lemma-derivative-not-bounded-by-magnitude}], with $\tau=1/2$, $\omega=2$, and $A=1$. %
We set 
%
\begin{equation}
  f_n(t) \Def n^3t \Iverson[\Big]{\magnitude*{t} \leq 1/n} \qquad(t \in \R).
\end{equation}
%
Under the identification $C_c(K) \equiv \Continuous_K$, $\mu_n$ reads as the difference of two positive Radon measures %
$\mu_n^+$ and $\mu_n^-$, since %
%
\begin{equation}
  \mu_n(u) = \underbrace{n^3\int_{0}^{1/n} t u(t) \diff{t}}_{\mu_n^+(u)} 
    - \underbrace{n^3\int_{\minus 1/n}^0 \minus t u(t) \diff{t}}_{\mu_n^-(u)} \qquad (u \in \Continuous_K).
\end{equation} 
%
Thus, $\mu_n$ is a signed Radon measure, whose compact support $\closedInterval*{\minus 1}{1}$ shrinks to $\singleton{0}$ as %
$n \to \infty$. We see that $\norm*{\mu_n} \leq \norm*{\mu_n^+} + \norm*{\mu_n^-}$. However, the sequence $\singleton{\mu_n}$ %
is not uniformly bounded, since 
%
\begin{equation}\label{2.3:norm-of-lambda-n}
  \norm*{\mu_n} = \norm*[1]{f_n} = n = \norm*{\mu_n^+}  + \norm*{\mu_n^-}. 
\end{equation}
%
These equalities are justified by \citeFA{6.19}, but the logistic function %
$\sigma_\lambda:t\mapsto 1/\bigl(1+\exp(\minus\lambda t)\bigr)$ provides a direct proof of them. Indeed, this function is a %
standard smooth approximation of $H:t\mapsto \Iverson*{t\geq 0}$, the Heaviside step function; see [\ref{annex-Dirac}]. %
Lebesgue's dominated convergence theorem implies
%
\begin{equation}
  \mu_n\bigl(u\cdot(\,\underbrace{2 \sigma_\lambda - 1}_{\text{odd}}\,)\bigr) 
  = 2n^3 \int_0^{1/n} \underbrace{t\, \bigl(2\sigma_\lambda(t) - 1\bigr)}_{\text{even}} \diff{t} 
  \tendsto{\lambda}{\infty} n \qquad(n \geq 2). 
\end{equation}
%
We first observe that this $\{\mu_n\}$ cannot converge vaguely: this would, by Theorem \citeFA{2.6}, imply %
$\sup_n{\norm*{\mu_n}}\ < \infty$, which would contradict \eqref{2.3:norm-of-lambda-n}. However, we investigate further %
to establish weaker convergence and boundedness in $\tau_K$. The mean value theorem implies that, given $\phi \in \D_K$, 
%
\begin{equation}
  \phi(1/n)- \phi(\minus 1/n) = \frac{2}{n}\phi'(t_n)
\end{equation}
%
for some $t_n = t_n(\phi)\in \openInterval*{\minus 1/n}{1/n}$. Moreover, integration by parts yields %
%
\begin{align}
  \mu_n(\phi) = \Lambda_n(\phi) & = \evalbar[\bigg]{\frac{n^3}{2}t^2 \phi(t)}{\minus 1/n}{1/n} 
      - \frac{n^3}{2} \int_{\minus 1/n}^{1/n} t^2 \phi'(t) \diff{t}\\
  & = \frac{n}{2}\bigl(\phi(1/n) - \phi(\minus 1/n)\bigr)
      - \frac{n^3}{2} \int_{\minus 1/n}^{1/n} t^2 \phi'(t) \diff{t}\\
  & = \phi'(t_n) 
      - \frac{n^3}{2} \int_{\minus 1/n}^{1/n} t^2 \phi'(t) \diff{t}.\label{2.3:integration-by-part}
\end{align}
%
Note that when $\phi' = 1$ in a neighborhood of $0$, \eg,  for $\phi(t)= \psi(t) \cdot t$, the latter equality reduces to %
%
\begin{equation}
  \Lambda_n(\phi) = 1 - \frac{n^3}{2} \int_{\minus 1/n}^{1/n} t^2 \diff{t} = \frac{2}{3}.
\end{equation}
%
This suggests that continuity of $\phi'$ dictates $\Lambda_n(\phi) \to\frac{2}{3}\phi'(0)$. We establish this convergence %
in two steps. First, 
%
\begin{align}
  \Lambda_n(\phi) - \frac{2}{3}\phi'(0) & = \phi'(t_n) - \phi'(0) - \frac{n^3}{2} \int_{\minus 1/n}^{1/n} t^2 \phi'(t)  \diff{t}
    + \phi'(0) \underbrace{\frac{n^3}{2} \int_{\minus 1/n}^{1/n} t^{2} \diff{t}}_{1/3} \\
  & = \phi'(t_n) 
    - \phi'(0)
  - \frac{n^3}{2} \int_{\minus 1/n}^{1/n} t^2 \bigl(\phi'(t) - \phi'(0)\bigr)  \diff{t}.
\end{align}
%
Next, taking absolute values gives %
%
\begin{equation}
  \magnitude*{\Lambda_n(\phi) - \frac{2}{3}\phi'(0)} \leq \magnitude*{\phi'(t_n) - \phi'(0)} 
  + \frac{1}{3}\max_{\closedInterval*{\minus 1}{1}}{\magnitude*{\phi' - \phi'(0)}} \tendsto{n}{\infty} 0.
\end{equation}
%
As a result, %
%
\begin{equation}\label{2.3:convergence-to-dirac}
  \Lambda_n(\phi) \tendsto{n}{\infty} \minus \frac{2}{3}\delta'(\phi) \qquad (\phi\in \D_K), 
\end{equation}
%
where $\delta': \phi \mapsto \minus \phi'(0)$ is the \emph{derivative} of the \textit{Dirac measure} %
$\delta: \phi \mapsto \phi(0)$; see \citeFA{6.1, 6.9} and [\ref{annex-Dirac}]. The previous reasoning shows that 
the limit $\Lambda = \minus \frac{2}{3}\delta'$ is $\tau_K$-continuous. In addition, \eqref{2.3:integration-by-part} %
justifies
%
\begin{equation}
  \magnitude*{\Lambda_n(\phi)} \leq \magnitude*{\phi'(t_n)} 
  + \frac{1}{3} \max_{\closedInterval*{\minus 1}{1}}{\magnitude*{\phi'}}.
\end{equation}
%
A simpler bound is
%
\begin{equation}\label{2.3:upper-bound}
  \magnitude*{\Lambda_n(\phi)} \leq \frac{4}{3} \norm*[\infty]{\phi'}. 
\end{equation}
%
This is a concrete instance of \eqref{2.3:bound-from-derivative}, with $p=1$ and $M=4/3$. To establish %
\eqref{2.3:norm-p-bound1}, it suffices to prove that no reduction to order $p=0$ is possible. To do so, we first assume, %
to reach a contradiction, that there exists $M$ such that 
%
\begin{equation}\label{2.3:upper-bound-assumption}
  \magnitude*{\Lambda_n(\phi)} \leq M \norm*[\infty]{\phi} \qquad (\phi \in \D_K, \, \counting{n}).
\end{equation}
%
Next, we choose %
\begin{equation}\label{2.3:zn-definition}
  \phi_n \Def \psi \cdot \tilde{\phi}_n,
\end{equation}
%
where $\tilde{\phi}_n$ is $\phi$ from Lemma [\ref{lemma-derivative-not-bounded-by-magnitude}] with %
$\tau= 1/n = 1/ \omega = 1/A$. Hence $\norm*[\infty]{\phi_n} < 2$. In contrast, for $n \geq 4$, $\Lambda_n (\phi_n)$ is 
%
\begin{equation} 
  \frac{n}{2} \bigl(\,\underbrace{\tilde{\phi}_n(1/n)}_{1} 
  - \underbrace{\tilde{\phi}_n(\minus 1 /n)}_{\minus 1}\,\bigr) 
  - \underbrace{\frac{n^3}{2} \int_{\minus 1/n}^{1/n} t^2 \tilde{\phi}'(t) \diff{t}}_{\frac{1}{3}n} = \frac{2}{3}n.
\end{equation}
%
Our assumption \eqref{2.3:upper-bound-assumption} then reads as
% 
\begin{equation}\label{2.3:upper-bound-contradiction2}
  \frac{2}{3}n \leq 2M < \infty. 
\end{equation}
%
The behavior $\frac{2}{3}n \to\infty$ provides the desired contradiction. Combining \eqref{2.3:convergence-to-dirac} with %
\eqref{2.3:upper-bound-contradiction2} gives%
%
\begin{equation}
  \magnitude*{\Lambda(\phi_n)} \geq \magnitude*{\Lambda_n (\phi_n)}  - \magnitude*{ \Lambda_n(\phi_n) - \Lambda(\phi_n)}
  \tendsto{n}{\infty} \infty.
\end{equation}
%
This shows that $\Lambda$ has no continuity bound for $K$ because $\norm*[\infty]{\phi_n} < 2$ while %
$\magnitude*{\Lambda_n (\phi_n)}\to \infty$. Therefore, $\Lambda$ cannot be extended as a Radon measure; %
see [\ref{notations:Radon-measures}], equation \eqref{notations:definition-of-Radon-measure}. A direct way to see this is to %
pick $\phi_\omega$ from Lemma [\ref{lemma-derivative-not-bounded-by-magnitude}], so that 
%
\begin{equation}
  \Lambda(\psi\cdot \phi_\omega) = \frac{2}{3} \omega \tendsto{\omega}{\infty} \infty 
\end{equation}
%
contrasts with $\norm*[\infty]{\psi\cdot \phi_\omega}=1$. Thus, we have exhibited a sequence of Radon measures $\mu_n$ that 
%
\begin{enumerate} 
  \item{
    does not converge vaguely to any Radon measure,
  }
  \item{but converges pointwise on $\D_K$ to $\Lambda \in \D_K^\ast$, in the specific $\D_K$'s topology; %
    see \eqref{2.3:local-base-of-DK}, \eqref{2.3:convergence-to-dirac}, and \eqref{2.3:upper-bound}.
  }
\end{enumerate}
%
We now propose a second example, with $p=2$ as the smallest admissible order.
%
% % % % % % % % % % % % % % % % % % % % % % % % % % % % % % % % % % % % % % % % % % % % % % % % % % % % % % % % % % % % % % % %
% Fifth part: Second Counterexample
% % % % % % % % % % % % % % % % % % % % % % % % % % % % % % % % % % % % % % % % % % % % % % % % % % % % % % % % % % % % % % % %
\paragraph{Case of a uniform bound in $\norm*{D^2\phi}$.} We define the \emph{derivative}
%
\begin{align}
  \Lambda_n': \D_K &\to \C\\
  \phi &\mapsto \minus \Lambda_n(\phi'); \nonumber
\end{align}
%
see \citeFA{6.1}. We have proved that every $\Lambda_n$ is continuous. So is the derivative operator in $\D_K$; see Exercise %
[1.17]. Therefore, $\Lambda_n'$ is continuous. Applying \eqref{2.3:convergence-to-dirac} with $\phi'$ yields %
%
\begin{equation}
  \Lambda_n'(\phi) \tendsto{n}{\infty}  \minus \frac{2}{3} \phi''(0).
\end{equation}
%
Furthermore, Theorem \citeFA{2.8} implies that the limit $\minus \frac{2}{3} \phi''(0)$ is $\tau_K$ continuous. 
Additionally, it follows from \eqref{2.3:upper-bound} that the bound \eqref{2.3:norm-p-bound2} is %
%
\begin{equation}
  \magnitude*{\Lambda_n'(\phi)} \leq \frac{4\,}{3} \norm*[\infty]{\phi''}.
\end{equation}
%
To prove this, it suffices to show that $2$ is the smallest suitable $p$. First, we assume, to reach a contradiction, that
%
\begin{equation}
  \magnitude*{\Lambda_n(\phi')} \leq M \norm*[\infty]{\phi'} \qquad (\phi \in \D_K, \, \counting{n}).
\end{equation}
%
Next, let $\Phi_n$ be the antiderivative of $\phi_n$ that vanishes at $\minus 1$; see \eqref{2.3:zn-definition}. The oddness %
of $\Phi_n$ ($\psi$ is even) implies that $\supp{\Phi_n} \subset \closedInterval*{\minus 1}{1}$. So, under our assumption, 
%
\begin{equation}
  \magnitude*{\Lambda_n'(\Phi_n)} = \magnitude*{\Lambda_n(\Phi_n')} \leq M \norm*[\infty]{\Phi_n'}. 
\end{equation}
%
Equivalently,
% 
\begin{equation}
  \magnitude*{\Lambda_n(\phi_n)} \leq M \norm*[\infty]{\phi_n}, 
\end{equation}
%
which has already been disproved. To reach one final contradiction, assume that there exists $M = M(0)$ so that 
%
\begin{equation}
  \magnitude*{\Lambda_n'(\phi)}  \leq M \norm*[\infty]{\phi} \qquad(\phi \in \D_K, \, \counting{n}).
\end{equation}
%
Lemma [\ref{lemma-derivative-not-bounded-by-magnitude}] implies that %
%
\begin{equation}
  \magnitude*{\Lambda_n'(\phi)}  \leq M \norm*[\infty]{\phi'}.
\end{equation}
%
This contradiction concludes the proof.
\end{proof}
% END
%
\section{Exercise 6. Fourier series may diverge at $0$}
\textit{
Define the Fourier coefficient $\hat{f}(n)$ of a function %
%
  $f\in L^2(T)$ ($T$ is the unit circle) %
%
by
%
  \begin{align*}
  \hat{f}(n) 
    = 
  \frac{1}{2\pi} \int_{\minus \pi}^{\pi} %
  %
    f(e^{i\theta}) e^{\minus i n\theta} d\theta
  %
  \end{align*}
%
for all $n\in \Z$ (the integers). Put 
%
  \begin{align*}
    \Lambda_n f =\sum_{k=\minus n }^{n} \hat{f}(k).
  \end{align*}
%
Prove that %
  $\set{f \in {L}^2(T)}{\underset{n\infty}{\lim}\,\Lambda_n f \text{ exists}} $ %
%
is a dense subspace of $L^2(T)$ of the first category. %
}
%
\begin{proof}Let $(f,g)$ denote the inner product of $f$ and $g$ in $L^2(T)$, %
so that 
  \begin{align}
    \Lambda_n f = \sum_{k=\minus n}^n (f, e_k). 
  \end{align}
%
Moreover, as a finite sum of linear functional $f \mapsto (f, e_k)$, %
$\Lambda_n$ is continuous. 
Now remark that 
% 
  \begin{align}
    P \Def \text{span}\{ e_k: k\in \Z\} 
      \subset 
    F \Def \set{
        f \in {L}^2(T)
      }{
        \lim_{n\infty} \Lambda_n f \text{ exists}
    } \subset {L}^2(T), 
  \end{align}
%
then conclude that %
%
  \begin{align}
    {L}^2(T)= \overline{P} = \overline{ F},
  \end{align}
%
with the help of the Fejér's theorem\footnote{
%
  See 4.25 of \cite{BigRudin}.}. %
%
We have so established that $F$ is dense in $L^2(T)$. %
We now prove the second part. %
\newline\newline\noindent
%
Obviously, $\sum_{k\in \Z} \| e_k\|^2$ diverges. %
Thus, the theorem \citeresultFA{12.6} forces one of the series %
%
  \begin{align}
    \sum_{k\in \Z} (g, e_k) \quad (g \in L^2(T))
  \end{align}
%
to diverge. %
The subspace $F$ is then a proper subset of ${L}^2(T)$. %
It now follows from (b) of \citeresultFA{2.7} that %
%
  $F$ is not of the second category. %
%
\end{proof}










\section{Exercise 9. Boundedness without closedness}
% % % % % % % % % % % % % % % % % % % % % % % % % % % % % % % % % % % % % % % % % % % % % % % % % % % % % % % % % % % % % % % %
% FunctionalAnalysis 
% 2_09.tex
% 
% encoding: UTF-8 
% EOL: LF
%
% format: LaTeX
% indent: spaces (2)
% width: 127
% % % % % % % % % % % % % % % % % % % % % % % % % % % % % % % % % % % % % % % % % % % % % % % % % % % % % % % % % % % % % % % %
\textit{
Suppose $X,Y,Z$ are Banach spaces and 
%
\begin{equation*}
  B:X\times Y \to Z
\end{equation*}
%
is bilinear and continuous. Prove that there exists $M<\infty$ such that 
%
\begin{equation*}
  \norm*{B(x, y)}
  \leq 
  M \norm*{x}\norm*{y} \qquad (x\in X, y\in Y).
\end{equation*}
%
Is completeness needed here?}
%
\begin{proof}%
% % % % % % % 
% FIRST PART 
% % % % % % % 
Completeness is not required. To show this, we only assume that $X$, $Y$, and $Z$ are normed spaces. Since $B$ is continuous, 
there exists $r > 0$ such that 
%
\begin{equation}\label{bound for B}
  \norm*{B(x, y)} < 1 
\end{equation}
%
whenever $\norm*{x} + \norm*{y} < r$. Therefore, any $0 < s < r$ yields%
%
\begin{equation}
  \norm*{B(x, y)} = \frac{4\norm*{x}\norm*{y}}{s^2} \cdot \norm[\bigg]{B\left(
    \frac{s}{2} \cdot \frac{x}{\norm*{x}}, %
    \frac{s}{2} \cdot \frac{y}{\norm*{y}}
  \right)}
  < \frac{4 \norm*{x} \norm*{y}}{s^2} \qquad (x \neq 0, y \neq 0)
\end{equation}
%
when %
$ \norm[\Big]{\frac{s}{2} \cdot \frac{x}{\norm*{x}}}
+ \norm[\Big]{\frac{s}{2} \cdot \frac{y}{\norm*{y}}}
= s < r$. 
%
This establishes the existence of $M=4/r^2$, since
%
\begin{equation}
    \norm*{B(x, y)} \leq %
    \frac{\!\!\!4\!}{\,s^{\,2}}\norm*{x}\norm*{y} 
    \tendsto{s}{r}
    \frac{\!\!\!4\!}{\,r^{\,2}}\norm*{x}\norm*{y}. 
\end{equation}
% 
% % % % % % % %
% SECOND PART 
% % % % % % % %
Consider the example where $X = Y = Z = C_c(\R)$ equipped with the supremum norm. The existence of $M=1$ follows from
%
\begin{equation}
  \norm*[\infty]{fg} \leq \norm*[\infty]{F} \cdot \norm*[\infty]{g}. %
\end{equation}
%
However, $C_c(\R)$ is not complete; see \citebook{5.4.4}{AnalyseIII}. We prove that the bilinear product %
%
\begin{align}
  B: C_c(\R)^{2} & \to  C_c(\R) \\
  (f, g) & \mapsto f \cdot g \nonumber
\end{align}
%
is nevertheless continuous. To do so, we start with any
%
\begin{equation}
  0 < r < \frac{\epsilon}{1 + \norm*[\infty]{f} + \norm*[\infty]{g}} < \epsilon < 1.
\end{equation}
%
Next, we choose $(u, v) \in C_c(\R)^2$ with  
%
\begin{equation}
  \norm*[\infty]{f-u} + \norm*[\infty]{g-v} < r, %
\end{equation}
%
so that $\norm*[\infty]{fg - uv} < \epsilon$. Explicitly, we have
%
\begin{align}
\norm*[\infty]{fg - uv} & = \norm*[\infty]{(f-u) \cdot g + u \cdot (g-v)} &\\
  & \leq \norm*[\infty]{f-u} \cdot \norm*[\infty]{g} + \norm*[\infty]{u} \cdot \norm*[\infty]{g-v} &\\
  & < r \cdot \norm*[\infty]{g} + \left(r + \norm*[\infty]{f}\right) \cdot r & 
    \left(\text{using $\norm*[\infty]{u} \leq r + \norm*[\infty]{f}$}\right)\\
  & < r \cdot \left(r + \norm*[\infty]{f} + \norm*[\infty]{g} \right)  & \\
  & < \epsilon \frac{r + \norm*[\infty]{f} + \norm*[\infty]{g}}{ 1+ \norm*[\infty]{f} + \norm*[\infty]{g}}  \\
  & < \epsilon & \left(\text{because $r < 1$}\right).
\end{align}
%
Since $\epsilon$ is arbitrary, this establishes that $B$ is continuous.
\end{proof}
% END
% 
\chapter{Convexity}
\section{Exercise 3. }
\input{\ROOT/chapter_3/3_03.tex}
\newpage
\section{Exercise 11. Meagerness of the polar}
\textit{\noindent
Let $X$ be an infinite-dimensional Fréchet space. %
Prove that $X^\ast$, with its $\weakstar$topology, %
is of the first category in itself.
}
\newline\newline\noindent
This is actually a consequence of the below lemma,  %
which we prove first. %
The proof that $X^\ast$ is of the first category in itself comes right after, %
as a corollary.%
%
\paragraph{Lemma.}%
If $X$ is an infinite dimensional topological vector space whose dual %
%
  $X^\ast$ %
%
separates points on $X$, then the polar
%
  \begin{align}
    K_A\Def \{ \Lambda \in X^\ast:\, \magnitude{\Lambda} \leq 1 \text{ on } A\}
  \end{align} 
%
of any absorbing subset $A$ is a $\weakstar$closed set that has empty interior.
%
\begin{proof}%
Let $x$ range over $X$. The linear form %
%
  $\Lambda \mapsto \Lambda x$ %
%
is $\weakstar$continuous; see \citeresultFA{3.14}. %
Therefore, %
%
  $P_x=\{ \Lambda \in X^\ast:\, \lvert \delta_x \,\Lambda  \,\rvert \leq 1\}$ %
%
is $\weakstar$ closed: %
%In particular, every $P_a$ ($a \in A$) is $\weakstar$closed:  %
As intersection of all $\set{P_a}{a\in A}$, %
$K_A$ is also a $\weakstar$closed set. %
We now prove the second half of the statement. % %
%
\newline\newline\noindent
%
From now on, $X$ is assumed to be endowed with its weak topology: %
$X$ is then locally convex, but its dual space is still %
%
  $X^\ast$ (see \citeresultFA{3.11}). %
%
Put %
%
  \begin{align}
    W \Def \bigcap_{x\in F} \set{\Lambda \in X^\ast}{\magnitude{\Lambda x} < r_x}, 
  \end{align}
%
where $r_x$ runs on $\R_+$, %
as $F$ runs through the nonempty finite subsets of $X$. %
%
Clearly, the collection of all such $W$ is a local base of $X^\ast$. %
Pick one of those $W$ and remark that the following subspace %
%
  \begin{align}
    M \Def \text{span}(F)
  \end{align}
%
is finite dimensional. Therefore, it is closed, by \citeresultFA{1.21 (b)}. %
%
Assume, to reach a contradiction, that $A\subset M$: %
Each $x$ is then in $tM=M$ for some $t=t(x)>0$, since $A$ is absorbing. %
So, $X=M$ is finite dimensional, which is a desired contradiction.
%
We have just established that $A\setminus M$ is nonempty: %
Now pick $\varit{a}$ in $A\setminus M$ and so conclude that %f
%
  \begin{align}
    b\Def \frac{a}{t}\in A
  \end{align}
%
Remark that $b\notin M$ (otherwise, $t b = a \in tM=M$ would hold): %
By the Hahn-Banach theorem \citeresultFA{3.5}, %
there exists $\Lambda_b$ in $X^\ast$ such that 
%
  \begin{align}
    \label{3.11. Polar_4.}\Lambda_b b    > & \,\,2 \\
    \label{3.11. Polar_4bis.}\Lambda_b (M) = & \singleton{0}.
  \end{align}
%
The latter equality implies that $\Lambda_b$ vanishes on $F$; %
hence $\Lambda_b$ is an element of $W$. %
On the other hand, given an arbitrary $f\in K_A$, %
the following inequalities  %
%
  \begin{align}
    \magnitude{\Lambda_b (b) + f(b)} 
      \geq 
    2 - \magnitude{f(b)} 
      >
    1.
  \end{align}
%
show that $f + \Lambda_b$ is off $K_A$. %
%
We have thus proved that
%
  \begin{align}
  f+ W\not\subset K_A.
  \end{align}
%
Since $W$ and $f$ are both arbitrary, this achieves the proof. %
\end{proof}
%
%\newline\newline
\noindent
We now give a proof of the original statement:
%
\paragraph{Corollary.}%
Under the same assumptions: If $X$ is Fréchet, %
then $X^\ast$ is meager in itself.
%
\begin{proof}%
From now on, $X^\ast$ is only endowed with its $\weakstar$topology. %
Let $d$ be an invariant distance that is compatible with the topology of $X$, %
so that the following sets
%
  \begin{align}
    B_n \Def \set{x\in X}{d(0, x) <  2^{\minus n}}\quad (\counting{n}) 
  \end{align}
%
form a local base of $X$. %
%
If $\Lambda$ is in $X^\ast$, then %
%
  \begin{align}
    \magnitude{\Lambda} <  m \text{ on } B_n
  \end{align}
%
for some $(n, m) \in \singleton{1, 2, 3, \dots}^2$; see \citeresultFA{1.18}. %
%
Hence, $X^\ast$ is the union of all % 
%
  \begin{align}\label{3.11. Countable union.}
    m\cdot K_n \quad (\counting{m,n}), 
  \end{align}
%
where $K_n$ is the polar of $B_n$. %
Clearly, showing that $K_n$ is nowhere dense will achieve the proof. %
To do so, remark that the above lemma asserts that %
%
  \begin{align}
    \left({\overline{K}_n}\right)^\circ = \left({{K}_n}\right)^\circ =\emptyset, 
  \end{align}
%
\ie %
%
  \begin{align}\label{3.11. Nowhere dense.}
    \left(\,{\overline{m\cdot K_n}}\, \right)^\circ 
      = 
    %\left(\, m \cdot \overline{{{K}_n}} \, \right)^\circ
    %  = 
    %\left(m \cdot K_n \right)^\circ 
    %  = 
    m\cdot \left({{K}_n}\right)^\circ 
      = 
    \emptyset.
  \end{align}
%%
%
So ends the proof.% 
\end{proof}
\newpage
\chapter{Banach Spaces}
%
Throughout this set of exercises, $X$ and $Y$ denote Banach spaces, unless the contrary is explicitly stated.
\section{Exercise 1. Basic results}
%!TEX root = /Volumes/HD_2/Rudin/Rudin_DM.tex
\textit{
Let $\phi$ be the embedding of $X$ into $X^{\,\ast\ast}$ decribed in Section 4.5. Let $\tau$ be the weak topology of $X$, and let $\sigma$ be the weak$^{\,\ast}$- topology of $X^{\,\ast\ast}$- the one induced by $X^{\,\ast}$.
\begin{enumerate}
  \item Prove that $\phi$ is an homeomorphism of $(X,\, \tau)$ onto a dense subspace of $(X^{\,\ast\ast},\, \sigma)$.
  \item If $B$ is the closed unit ball of $X$, prove that $\phi(B\,)$ is $\sigma$-dense in the closed unit ball of $X^{\,\ast\ast}$. (Use the Hahn-Banach separation theorem.)
  \item Use (a), (b), and the Banach-Alaoglu theorem to prove that $X$ is reflexive if and only if $B$ is weakly compact.
  \item Deduce from (c) that every norm-closed subspace of a reflexive space is reflexive.
  \item If $X$ is reflexive and $Y$ is a closed subspace of $X$, prove that $X/Y$ is reflexive.
  \item Prove that $X$ is reflexive if and only $X^{\,\ast} $ if reflexive. \\ Suggestion: One half follows from (c); for the other half, apply (d) to the subspace $\phi(X\,)$ of $X^{\,\ast\ast}$.
\end{enumerate}
}

\begin{proof}
Let $\psi$ be the isometric embedding of $X^\ast$ into $X^{\ast\ast\ast}$. %
The dual space of $(X^{\ast\ast},\sigma)$ is then $\psi(X^\ast)$. \\
\\
%
It is sufficient to prove that
\begin{align}
  \phi^{\minus 1}: \phi(X) \to & X \\
  \phi (x) \mapsto & x 
\end{align}
%
is an homeomorphism (with respect to $\tau$ and $\sigma$). We first consider
\begin{align}
  V \Def & \set{
      x^{\ast\ast} \in X^{\ast\ast} }{ 
      \magnitude{ \bra{x^{\ast\ast} }\ket{\psi x^\ast}} < r
    } & (x^{\ast} \in X^{\ast}, r > 0); \\
  U \Def & \set{
      x \in X }{
      \magnitude{ \bra{x}\ket{x^\ast} } < r 
    } & (x^\ast \in X^\ast, r > 0 ).
\end{align}
and remark that the so defined $V\,$'s (respectively $U\,$'s) shape a local subbase $\mathscr{S}_\sigma$ (respectively $\mathscr{S}_\tau$) of $\sigma$ (respectively $\tau$). We now observe that 
\begin{align}
U=\phi^{\minus 1} \left(V \cap \phi(X\,) \right) = \phi^{\minus 1} (V)  \cap X \quad ( V\in \mathscr{S}_\sigma\,,\,\,U\in \mathscr{S}_\tau) \quad ,
\end{align}
since $\phi^{\minus 1}$ is one-to-one. This remains true whether we enrich each subbase $\mathscr{S}$ with all finite intersections of its own elements, for the same reason. It then follows from the very definition of a local base of a weak / weak$^\ast$-topology that $\phi^{\minus 1}$ and its inverse $\phi$ are continuous.\\
\\
The second part of (a) is a special case of [3.5] and is so proved. First, it is evident that 
\begin{align}
\overline{\phi(X\,)}_\sigma\subseteq X^{\,\ast\ast}\quad .
\end{align}
and we now assume- to reach a contradiction- that $(X^{\,\ast\ast},\, \sigma)$ contains a point $z^{\,\ast\ast}$ outside the $\sigma$-closure of $\phi(X\,)$. By [3.5], there so exists $y^{\,\ast}$ in $X^{\,\ast}$ such that 
\begin{align}
\label{4_1_6} \langle \phi x,\, \psi y^{\,\ast} \rangle=&\langle  y^{\,\ast} ,\,\phi x \rangle= \langle x,\, y^{\,\ast}\rangle= 0\quad (x\in X\,)\quad ; \\
\label{4_1_7} \langle z^{\,\ast\ast},\, \psi y^{\,\ast} \rangle=&\,1
\end{align}
(\ref{4_1_6}) forces $y^{\,\ast}$ to be a the zero of $X^{\,\ast}$. The functional $\psi y^{\,\ast}$ is then the zero of $X^{\,\ast\ast\ast}$: (\ref{4_1_7}) is contradicted. Statement (a) is so proved; we next deal with (b).\\
\\
%: (b)
The unit ball $B^{\,\ast\ast}$ of $X^{\,\ast\ast}$ is weak$^\ast$-closed, by (c) of [4.3]. On the other hand,
\begin{align}
\phi(B\,)\subseteq  B^{\,\ast\ast}\quad ,
\end{align}
since $\phi$ is isometric. Hence
\begin{align}
\overline{\phi(B\,)}_\sigma\subseteq \overline{ (B^{\,\ast\ast})}_\sigma=  B^{\,\ast\ast}\quad .
\end{align}
Now suppose, to reach a contradiction, that $B^{\,\ast\ast}\setminus \overline{\phi(B\,)}_\sigma$ contains a vector $z^{\,\ast\ast}$. By [3.7], there exists $y^{\,\ast}$ in $X^{\,\ast}$ such that 
\begin{align}
 \label{4_1_10} \lvert \psi y^{\,\ast}  \rvert \leq &1 \quad  \text{on}\,\,\overline{\phi(B\,)}_\sigma \quad  ; \\
   \label{4_1_11}  \langle z^{\,\ast\ast},\, \psi y^{\,\ast} \rangle > &1\quad .
\end{align}
It follows from (\ref{4_1_10}) that 
\begin{align}
\lvert \, \psi y^{\,\ast}\,   \rvert \leq 1 \text{  on }\phi(B\,)\, , \, \, \ie \lvert\, y^{\,\ast}  \, \rvert \leq 1  \text{  on } B\quad .
\end{align}
We have so proved that 
\begin{align}
y^{\,\ast} \in B^\ast \quad .
\end{align}
Since $ z^{\,\ast\ast}$ lies in $B^{\,\ast\ast}$, it is now clear that 
\begin{align}
\lvert \langle z^{\,\ast\ast},\,  \psi y^{\,\ast} \rangle\rvert \leq 1\quad; 
\end{align}
what it contradicts (\ref{4_1_11}), and thus proves (b). We now aim at (c).\\
\\
%: c
It follows from (a) that
\begin{align}\label{4_15}
B\text{ is weakly compact if and only if }\phi(B\,) \text{ is weak}^\ast\text{-compact. }
\end{align}
If $B$ is weakly compact, then $\phi(B\,)$ is weak$^\ast$-closed. So,
\begin{align}
\phi(B\,)= \overline{\phi(B\,)}_\sigma\overset{(b)}{=} B^{\,\ast\ast}\quad .
\end{align}
$\phi$ is therefore onto, \ie $X$ is reflexive.\\
Conversely, keep $\phi$ as onto: one easily checks that $\phi(B\,)=B^{\,\ast\ast}$. The image $\phi (B\,)$ is then weak$^\ast$-compact by (c) of [4.3]. The conclusion now follows from (\ref{4_15}).\\
\\
%: (d)
Next, let $X$ be a reflexive space $X$, whose closed unit ball is $B$. Let $Y$ be a norm-closed subspace of $X$: $Y$ is then weakly closed (\cf [3.12]). On the other hand, it follows from (c) that $B$ is weakly compact. We now conclude that the closed unit ball $B\cap Y$ of $Y$ is weakly compact. We again use (c) to conclude that $Y$ is reflexive. (d) is therefore established. Now proceed to (e).\\
\\
%: (e)
Let $\equiv$ stand for ``isometrically isomorphic" and apply twice [4.9] to obtain, first
\begin{align}\label{4_17}
(X/Y\,)^\ast \equiv  Y^\bot  \quad ,
\end{align}
next,
\begin{align}\label{4_18}
(X/Y\,)^{\ast\ast}  \equiv  (Y^\bot)^\ast  
 \equiv  X^{\,\ast\ast} / (Y^\bot)^\bot  
 \equiv X/Y\quad .
\end{align}
Combining (\ref{4_17}) with (\ref{4_18}) makes (e) to hold.\\
\\
It remains to prove (f). To do so, we state the following trivial lemma (L)
\begin{quotation}
\noindent  { \CMUCS Given a reflexive Banach space $Z$, the weak$^{\,\ast}$-topology of $Z^{\,\ast}$ is its weak one.}
\end{quotation}
%Its proof is nothing else but a tiny rewriting: if $Z^{\ast\ast}\equiv Z$, then $ Z^\ast_{\text{w}^\ast}=Z^\ast_{\text{w}}$.\\
Assume first that $X$ is reflexive. Since $B^{\,\ast}$ is weak$^\ast$ compact, by (c) of [4.3], (L) implies that $B^{\,\ast}$ is also weakly compact. Then (c) turns $X^{\,\ast}$ into a reflexive space. \\
\\
Conversely, let $X^{\,\ast}$ be reflexive. What we have just proved that makes $X^{\,\ast\ast}$ reflexive. On the other hand, $\phi(X\,)$ is a norm-closed subspace of $X^{\,\ast\ast}$; \cf [4.5]. Hence $\phi(X\,)$ is reflexive, by (d). It now follows from (c) that $B^{\,\ast\ast} \cap \phi(X\,)$ is weakly compact, \ie weak$^\ast$-compact (to see this, apply (L) with $Z=X^{\,\ast}$). \\
\\
By (a), $B$ is therefore weakly compact, \ie $X$ is reflexive; see (c). So ends the proof.
\end{proof}



 












 
% 
\setcounter{section}{12}
% 
\section{Exercise 13. Operator compactness in a Hilbert space}
%!TEX root = /Volumes/HD_2/Rudin/Rudin_DM.tex
{ \CMUCS 
\begin{enumerate}
\renewcommand{\labelenumi}{(\alph{enumi})}
\item Suppose $T\in \mathscr{B}(X,Y), T_n \in  \mathscr{B}(X,Y)\, $ for $n=1,\, 2, \, 3,\, \dots$, each $T_n\,$ has finite-dimensional range, and $\lim \| T-T_n\|=0\, $. Prove that $T$ is compact.
\item Assume $Y$ is a Hilbert space, and prove the converse of (a): Every compact $T\in  \mathscr{B}(X,Y)$ can be approximated in the operator norm by operators with finite-dimensional ranges. Hint: In a Hilbert space there are linear projections of norm 1 onto any closed subspace. (See theorems 5.16, 12.4.)
\end{enumerate}
}
%: a
\paragraph{PROOF.} Since each $T_n$ is compact, (a) follows from (c) of [4.18]. Besides, we take the opportunity to alternatively prove that the compact operators subspace is norm closed.\\
\\
Reset every $T_n$ as a compact operator. Let $\{x_{\,0}^{\,i}:\, i\in \N\}$ be in $U$ the open unit ball of $X$. Since $T_1$ is compact, $\{x^i_0\}$ contains a subsequence $\{x^{\,i}_{\,1}:\, i\in \N\}$ such that $\{T_1x^{i}_{\,1}\}$ converges to a point $y_{\,1}$ of $Y$. The same reasoning can be recursively applied to $T_{n}$ and $\{x^{\,i}_{\, n-1}\}\subset U$ so that $\{T_{n\,}x^{\,i}_{\, n}\}$ tends to some $y_{n}$ of $Y$, as $\{x^{\,i}_{\, n}\}$ is a subsequence of $\{x^{\,i}_{\,n-1}\}$. Then
\begin{align}\label{4_13_a_0}
T_{n\,} x^{i}_p \,\underset{i\to \infty}{\longrightarrow}\,y_n\quad (p\> n=1,\, 2,\, 3,\, \dotsb )\quad .
\end{align}
Applied with $\{x^{\,i}_{\,n}:\, (n,i\,)\in\N^2\}$, a Cantor's diagonal process therefore provides a subsequence $\{\tilde{x}_{\, j}:\, j\in \N\} $ such that
\begin{align}\label{4_13_a_1}
&T_{j\,} \tilde{x}_k \underset{k\to \infty}{\longrightarrow}y_{j}\quad ;\\
&T_{j\,} \tilde{x}_{\, j} \underset{j\to \infty}{\longrightarrow} y_{j}\quad .
\end{align}
%Since $\|T-T_k\|\underset{k\to \infty}{\longrightarrow}0$, (\ref{4_13_a_1}) implies that
%\begin{align}\label{4_13_a_2}
%T \tilde{x}_k  -y_k  \underset{k\to \infty}{\longrightarrow}0\quad.
%\end{align}
We now easily obtain
\begin{align}\label{4_13_a_3}
\| T_{j\,} \tilde{x}_{j\,}  -T_{\, k} \tilde{x}_{\, k}\| \<
\| T_{j\,} \tilde{x}_{j\,}- y_{j\,}\| +
\| y_{j\,} -T_{j\,} \tilde{x}_{k\,}\| +
\| T_{j\,} - T_{ k\,}\|   \underset{k> j\to \infty}{\longrightarrow}0\quad .
\end{align}
$\{T_{ j\,} \tilde{x}_{\, j} \}$ is then a Cauchy sequence. So is $\{T \,\tilde{x}_{\, j}\}$, since $\| T-T_{j}\,\| \to 0$. On the other hand, $Y$ is complete: (a) is then proved and we now establish the counterpart in a Hilbert space.\\
\\
%: b
Fix $\eps$ as a positive scalar. Since $T$ is compact, $Y$ contains a finite set $C$ such that 
\begin{align}\label{4_13_b_1}
T(U\,)\subset \bigcup_{c\in C} B(c,\, \eps)\quad .
\end{align}
As a Hilbert space, $Y$ contains a \textsl{maximal orthonormal set} (or \textsl{Hilbert basis}) $M$. This implies that $\text{span}(M)$ is dense in $Y$; \cf 4.18 \& 4.22 of \cite{Big_Rudin}. The finiteness of $C$ forces $M$ to enclose a finite set $S$ so that 
\begin{align}\label{4_13_b_2}
\forall c\in C, \, \exists  s(c\,) \in \text{span} (S\,):\, \|c - s(c\,)\| <\eps\quad .
\end{align}
Let $x$ be in $U$. It follows from (\ref{4_13_b_1}) that 
\begin{align}\label{4_13_b_3}
\|Tx - c_x \| < \eps
\end{align}
for some $c_x$ of $C$. We now combine (\ref{4_13_b_2}) and (\ref{4_13_b_3}) to obtain
\begin{align}
\|Tx - s(c_x) \| \<  \| Tx - c_x \| + \| c_x - s(c_x)\| < 2\eps
\end{align}
As a finite-dimensional subspace, $\text{span}(S\,)$ is closed (see footnote 4, Exercise 1.10). We so obtain
\begin{align}
Y=\text{span}(S\,)\oplus \text{span}(S\,)^\bot \quad ,
\end{align}
by [12.4]. There so exists a unique projection projection $\pi=\pi(\eps)$ of $Y$ onto itself (see [5.6] for the definition) such that
\begin{align}
\pi (Y) = \text{span}(S\,)\, ,\,  \,  (I-\pi)(Y\,) = \text{span}(S\,)^\bot\quad .
\end{align}
It is easily checked that $\pi$ has norm $1$. Moreover,
\begin{align}
\pi s = s \quad (s\in\text{span}(S\,))\quad .
\end{align}
Thus,
\begin{align}
(I-\pi) (Tx\,)= (I-\pi ) (Tx - s(c_x)) \quad (x\in U\,)\quad .
\end{align}
Then,
\begin{align}
\|( I- \pi )(T x) \| \< \| I- \pi \| \, \| Tx -s(c_x)\| < 4 \eps  \quad (x\in U)\quad 
\end{align}
(the fact that $\pi$ has norm $1$ is hidden in the right side inequality). We have just so proved that 
\begin{align}
\| T-\pi\circ T\, \| \in \underset{\eps \sim 0}{O}(\eps) \quad . 
\end{align}
That is particularly true whether $\eps=\eps_0,\, \eps_1,\, \eps_2,\, \dots ,\, \eps_n \underset{n\to \infty}{\longrightarrow} 0\,$. Let so $T_n$ be $ \pi(\eps_n) \circ T\,$ and conclude that these (compact) operators approximate $T$ in the desired fashion, \ie
\begin{align}
\| T-T_n \| \underset{n \to \infty}{\longrightarrow} 0\quad .
\end{align}
\QED
 
%
\setcounter{section}{14}
% 
\section{Exercise 15. Hilbert-Schmidt operators}
%!TEX root = /Volumes/HD_2/Rudin/Rudin_DM.tex
{ \CMUCS Suppose $\mu$ is a finite (or $\sigma$-finite) positive measure on a measure space $\Omega$, $\mu\times\mu$ is the corresponding product measure on $\Omega\times\Omega$, and $K\in L^2(\mu\times\mu)$. Define 
\begin{align*}
(Tf\,)(s\,)=\int_\Omega K(s,t\,) f\,(t\,) d\mu(t\,)\quad [\,f\in L^2(\mu)\,].
\end{align*}
\begin{enumerate}
\renewcommand{\labelenumi}{(\alph{enumi})}
\item Prove that $T\in \mathscr{B}(L^2(\mu))$ and that 
\begin{align*}
\| T\, \|^2 \< \int_\Omega\int_\Omega \lvert K (s,t\,)\rvert ^2 d\mu (s\,) d\mu(t\,).
\end{align*}
\item Suppose $a_i$, $b_i$ are members of $L^2(\mu)$, for $1\< i\< n$, put $K_1=\sum a_i(s\,)b_i(t\,)$ and define $T_1$ in terms of $K_1$ a $T$ was defined in terms of $K$. Prove that $\dim \mathscr{R}(T_1)\< n$.
\item Deduce that $T$ is a compact operator in $L^2(\mu)$. Hint: Use exercise 13.
\item Suppose $\lambda\in \C,\, \lambda\neq 0$. Prove: Either the equation 
\begin{align*}Tf-\lambda f=g\end{align*}
has a unique solution $f\in L^2(\mu)$ for every $g\in L^2(\mu)$ or there are infinitely many solutions for some $g$ and none for others. (This is known as the \textsl{Fredholm alternative}.).
\item Describe the adjoint of $T$.
 \end{enumerate}
}
%: a 
\paragraph{PROOF.} Let $X$ (respectively $P\,$) be the Banach space $L^2(\mu)$ (respectively $L^2(\mu\times \mu)\,$). A consequence of the Radon-Nikodym theorem (\cf 6.16 of \cite{Big_Rudin}\,) is that there exists a group isomorphism $\rho:\, X\to \, X^{\,\ast},\, f\mapsto f^{\,\,\ast}$ such that
\begin{align}
\langle u ,\, f^{\,\,\ast}  \rangle =\int_\Omega u\mdot f \, \d\mu  \quad (u\in X,\, f\in X)\quad .
\end{align}
Define a.e $K_s,\, K_t:\, \Omega\to \C$ by setting
\begin{align}
K_s(t\,)\Def K_t(s\,)\Def  K(s,\, t\,)\, \, \text{ a.e}\quad \left((s,t\,)\in \Omega\right)\quad .
\end{align}
$T$ is clearly linear. Moreover, 
\begin{align}
\lvert (T f\,)(s\,) \rvert = \lvert \langle K_s,\, f^{\,\,\ast} \rangle \rvert \< \| K_s\|_X\quad(\|\,f\,\,\|_X < 1) \quad
\end{align}
(the latter inequality is a Cauchy-Schwarz one). Now apply the Fubini's theorem with $\lvert K\,\rvert^2$ to obtain
\begin{align}
\|Tf\,\,\|^2_X \< \int_\Omega \| K_s\|^2_X \, \,\d \mu (s\,)  =\| K\,\|_P^2 \,< \infty \quad(\|\,f\,\,\|_X < 1)\quad  .
\end{align}
(a) is then proved. \\
\\
%: b
To show (b), remark that
\begin{align}
\int_\Omega a_i (s\,) \mdot b_i \mdot f  \,\,\d\mu \,\,\in \C\mdot  a_i(s\,)\,\,\,\text{a.e}\quad \quad (f\in X, \, s\in \Omega)  \quad.
\end{align}
It is now clear that $T$ maps any $f$ of $X$ into $\C \mdot a_1+\dotsb+ \C\mdot a_n$. We so conclude that $\dim R(T_1)\< n $. \\
\\
%: c
We now aim at (c). The current part refers to Exercise 4.13. $X$ is also a Hilbert space and so contains a Hilbert basis $M$. Define a.e
\begin{align}
a_b: \Omega \to &\,\C   \\
 s \mapsto &\,(K_s,\, b) \nonumber
\end{align}
whenever $b$ ranges $M$. Hence,
\begin{align}
K_s= \sum_{b\in M} a_b(s\,) \mdot b \, \text{ a.e} \quad (s\in \Omega) \quad.
\end{align}
Provided any positive scalar $\eps$, there so exists a finite subset $S=S(\eps)$ of $M$ such that
\begin{align}
\| K_s - \sum_{b\in S} a_b(s\,) \mdot b\,\|_X < \eps \quad  (s\in \Omega)\quad .
\end{align}
Remark that $\underset{b\in S}{\sum} a_b \mdot b$ matches the definition of $K_1$; \cf(b): from now on, 
\begin{align}
K_1\Def \sum_{b\in S} a_b \mdot b\quad .
\end{align}
It follows from (b) that
\begin{align}
\dim R(K_1) < \infty \quad.
\end{align}
Now turn back to (a), with $K-K_1$ playing the role of $K$, and so obtain
\begin{align}
\|T-T_1\| < \eps \mu(\Omega)\< \infty \quad .
\end{align}
For if $\mu$ is finite, use (a) of Exercise 4.13 to conclude that $T$ is compact. Assume henceforth that $\mu$ is not (necessarily) finite and pick $\delta$ in $\R_+$. The simple functions (with finite measure support\,) form a dense family of an $L^p$ space ($1\< p<\infty$); \cf 3.13 of \cite{Big_Rudin}. It then exists a simple function $K_\delta$ of $L^2(\mu\times \mu)$ such that 
\begin{align}
(\mu\times\mu)\left(\{K_\delta\neq 0\}\right) <\infty \,, \,\, \| K-K_\delta \|_P <\delta\quad .
\end{align}
Define an operator $T_\delta$ in terms of $K_\delta$ as $T$ was defined in terms of $K$, and proceed as in (a) with $T-T_\delta$ instead of $T$. Then
\begin{align}\label{4_15_13}
\|T-T_\delta\| < \delta\quad .
\end{align}
The key ingredient is that $K_\delta$ can be identified with an element of the finite measure space $L^2(\{K_\delta\neq 0\},\mu\times\mu)\,$. What we have attempted to approximate $T$ by $T_1$ can therefore be reiterated (with $K_\delta$ playing the role of $K$) to achieve an approximation $T_{\delta,1}$ of $T_\delta$ so that
\begin{align}\label{4_15_14}
\|T_{\delta}-T_{\delta,1}\| < \eps\quad .
\end{align}
It now follows from (\ref{4_15_13}) and (\ref{4_15_14}) that
\begin{align}
\|T-T_{\delta,1}\|\< \|T-T_{\delta}\| +\|T_{\delta}-T_{\delta,1}\| < \eps+\delta\quad .
\end{align}
Since $\eps$ and $\delta$ were arbitrary, the $\sigma$-finite case is proved. We now establish (d).\\
\\
%: d
Provided  $g$ of $X$, let $E_g$ be the following equation on $X$
\begin{align}Tf-\lambda f =g \quad ,\end{align}
whose solution set is denoted by $S_g\,$. Note that $S_0$ is $\ker (T-\lambda)$ and discard the trivial case $S_0=X\,$\footnote{\eg$ X=L^2(\{0\},\, \delta)\,$, so that $I=\lambda^{\moins 1} T $ is compact. }: each $f$ of $X $ lies in $S_{\,Tf-\lambda f\,} $, as some $Tf-\lambda f\,$'s are nonzero. Some $S_g$'s are then nonempty. Remark that 
\begin{align}\label{4_15_17}
S_g= f + S_0 \quad (f\in S_g) \quad 
\end{align}
in such case. Furthermore, the equality $\beta= \alpha $ of [4.25] yields 
\begin{align}
(T-\lambda I\,)(X)\neq X \, \ssi\, S_0 \neq\{0\}  \quad .
\end{align}
So if $T-\lambda I$ is not onto, not only some $S_g$'s are empty, but also $S_0\neq\{0\}$. Every nonempty $S_g$ (such sets always exist, see above) is then infinite, by (\ref{4_15_17}).\\
Otherwise, $T-\lambda I$ is bijective and every equation $E_g$ has then a unique solution $f$. The Fredholm alternative is so proved. \\
\\
Our last step is the description of $T^{\,\ast}$. Let $S:\, X\to X$ be such that
\begin{align}
(Sf\,)(t\,)\Def \int_\Omega K_t \mdot f \,\,\, \text{a.e}\quad \quad (\,f\in X,\, t\in \Omega)
\end{align}
Proceed as in (a), with $S$ instead of $T$: $S$ lies in $\mathscr{B}(X)$. Next, we claim that 
\begin{align}
\langle u,\, T^{\,\ast} f^{\,\,\ast}  \rangle = &\, \langle Tu,\,f^{\,\,\ast}   \rangle\\
=&  \int_{\Omega} (T  u ) \mdot  f \,\, \,\d\mu \\
\label{int_Fubini}=& \int_{\Omega^2} K\mdot f \mdot  u \,\,\, \d(\mu\times\mu) \\
=&  \int_{\Omega} (S  f\, ) \mdot  u \,\,\, \d\mu\,  \\
=&\, \langle u,\, (S f\, )^\ast \rangle \quad ,
\end{align}
whenever $u$ and $f$ run through the closed unit ball of $X$. Since $\|T\, \|$, $\| T^{\,\ast} \|$ are equal and finite, only exactness of (\ref{int_Fubini}) is possibly in doubt; the below justification dissipates it. In conclusion,
\begin{align}
 T^{\,\ast}   =  \rho S \rho^{\moins 1} \quad .
\end{align}
Informally, 
\begin{align}
T^{\,\ast} = S\quad .
\end{align}
\\
\underline{Justification of (\ref{int_Fubini})}. The current proof shall be complete once we have justified (\ref{int_Fubini}). To do so, keep $u$ and $f$ as above. Let us introduce
\begin{align}
A(s\,)\Def \int_\Omega \lvert K_s(t\,) \mdot u (t\,)\rvert  \,\, \d\mu(t\,) \, \,\,\text{a.e}  \quad (s\in \Omega)\quad ,
\end{align}
to make hold the following Cauchy-Schwarz inequality
\begin{align}
A(s\,)\< \| K_s\|_X  \quad (s\in \Omega)\quad .
\end{align}
Thus,
\begin{align}
 \int_{\Omega^2} \lvert K(s,\,t\,) \, u(t\,)\,  f\,(s\,) \rvert \,\d\mu(s\,)\d\mu(t\,) 
 = &   \int_\Omega   \lvert \,f\,(s\,) \rvert \,A(s\,) \, \d\mu(s\,) \\
 \< &  \int_\Omega   \lvert \,f\,(s\,) \rvert  \, \| K_s\|_X\,\d\mu(s\,) \\
 \label{4_15_complement_2} \< & \left[ \int_\Omega \| K_s\|_X^2\,\,\d\mu(s\,) \right]^{\frac{1}{2}} 
 =      \| K\,\|_P < \infty \quad .
 \end{align}
The inequality in (\ref{4_15_complement_2}) is a Cauchy-Schwarz one, the following equality follows from the Fubini's theorem. This achieves the proof.\QED









% 

%
\setcounter{chapter}{5}
%
\chapter{Distributions}
%!TEX root = /Volumes/HD_2/Rudin/Rudin_DM.tex
\section{Exercise 1. Test functions are almost polynomial}
%{\it  Suppose $f$ is a complex continuous function in $\R^n$, with compact support. \!\!Prove that $\psi P_j\to \!f$ uniformly on $\R^n$, for some $\psi\in \mathscr{D}$ and for some sequence $\{P_j\,\}$ of polynomials.}
\begin{proof} According to 1.16, $\Omega$ is union of a compact sets sequence $\{K_{i\,}\}$ and $\text{supp} (f\,)$ lies in some $K= K_{i\,}$so that $f\,$ is embedded in $\mathscr{D}(\Omega)\,$. We can apply [1.10] to ensure that $\Omega$ encloses a compact set $S=\overline{K +B(\epsilon)}$ for sufficiently small $\epsilon>0\,$.\\
\\
One easily checks that the Stone-Weierstraß theorem [5.7] can be applied with the subalgebra $ \{ g\in C(S\,) :\, g \text{ is polynomial}\,\}$ of $C(S\,)\,$.
There exists a sequence $\{P_j:\, j\in \N\}$ of $\R[X_1,\dotsc,\, X_n]\,$ such that 
\begin{align}\label{6_1_1}
\sup_S \lvert\, f-P_{j\,} \rvert  \underset{j\infty}{\longrightarrow} 0\qquad.
\end{align} 
 By [6.20], the open set $K +B(\epsilon)$ has a local partition of unity $\{\psi_i\}\subset \mathscr{D}(\Omega)\,$. Moreover, there exists an integer $l$ such that $\psi=\psi_1+\dotsb+\psi_l\,$ equals $1$ on $K\,$. Hence
\begin{align}\label{6_1_2}
\| \, f - \psi P_{j\,}\|_\infty=  \, \|  \,\psi f - \psi P_{j\,}\|_\infty 
=  &  \, \sup_S  \lvert \, \psi f -\! \psi P_{j\,}\rvert  \\
   =  &  \,   \sup_S  \lvert  \, f -  \, P_{j\,} \rvert    \, \overset{(\ref{6_1_1})}{\underset{j\infty}{\longrightarrow}} 0\qquad .
\end{align} 
\end{proof}

\setcounter{section}{5} 
\section{Exercise 6. Around the supports of some distributions}
%%!TEX root = /Volumes/HD_2/Rudin/Rudin_DM.tex
\renewcommand{\labelenumi}{(\alph{enumi})} 
{\CMUCS
\begin{enumerate}
\item Suppose that $c_m=\exp\{\minus (m!)!\}$, $m=0,\, 1,\, 2,\, \dots\, $. Does the series
\begin{align*}{
\sum_{m=0}^\infty c_m (D^m\phi)(0)
}\end{align*}
converges for every $\phi\in C^{\,\infty} (R)$?
\item Let $\Omega$ be open in $\R^n$, suppose $\Lambda_i\in \mathscr{D}'(\Omega)$, and suppose that all $\Lambda_i$ have their supports in some fixed compact $K\subseteq \Omega$. Prove that the sequence $\{\Lambda_i\}$ cannot converge in $\mathscr{D}'(\Omega)$ unless the orders of the $\Lambda_j$ are bounded. Hint: Use the Banach-Steinhaus theorem.
\item Can the assumption about the supports be dropped in (b)?
\end{enumerate}}
\paragraph{PROOF.} The answer is: no. Let us establish this assertion. Assume, to reach a contradiction, that the above series converges for every smooth $\phi:\, \R\to\,\C\,$. \\
\\
The sequence $\{c_m\, (D^m\phi)(0)\}$ so converges to $0$. Nevertheless, it is proved in [1.46] that $C^{\,\infty}(\Omega)$ is not locally bounded. In other words, it is always possible to excavate a $\phi\,$ for which the magnitude of the $m$-th derivative at $0$ is as large as we please\footnote{indeed [1.46] provides sufficient tools for constructive proof of this; see the $\phi_j-\check{\phi}_j$ involved in (\ref{2_3_phicheck}).}, \eg greater than $1/c_m$.
 A desired contradiction is then reached. We now prove (b), again by contradiction.\\
\\
To do so we assume $\{\Lambda_j\}$ to converge to some $\Lambda$ of $\mathscr{D}'(\Omega)$ and we let $Q$ run through the compact sets of $\Omega\,$. Next, we define
\begin{align}{\label{6_6_1}
S(T,\, Q\,)\Def \{N\in \N, \, \exists C\in \R_+:\, \lvert T\phi\, \rvert \leq C\, \| \phi \|_N \,\text{ for all }\phi \text{ of } \mathscr{D}_Q \}\quad (T\in \mathscr{D}(\Omega))\quad .
}\end{align}
Such subset of $\N$ has a minimum $\omega(T,\, Q)$. The following value
\begin{align}{\label{6_6_2}
\omega (T\,) \Def \max\{ \omega(T,\, Q\,): \, Q\subseteq \Omega\, , \,\, Q \text{ compact }\}\leq \infty
}\end{align}
is then the order of $T$. Assume, to reach a contradiction, that, after passage to a subsequence,
\begin{align}{\label{6_6_3}
\omega (\Lambda_{j\,},\, Q_j\,) =j\quad (j=1,\, 2,\, 3,\, \dots)
}\end{align}
for some compact $Q=Q_j\,$. By (a) of [6.24], $Q_j$ cuts $\text{supp}\Lambda_j\,$, say in $p_j\,$. Since $K$ encloses $\text{supp}\Lambda_j\,$, $\{p_j\}$ tends, after passage to a subsequence, to some $p$ of $K\,$.
Choose a positive scalar $r$ so that 
\begin{align}{\label{6_6_5}
\overline{B}(p,\, r)\Def \{ x\in \R^n:\, \lvert x-p\,\rvert\leq r\}\subseteq \Omega \quad .
}\end{align}
Such closed ball $\overline{B}(p,\, r)$ is a compact subset of $\Omega$. By (b) of [6.5] (which refers to [1.46])  $\mathscr{D}_{\overline{B}(p,\, r)}$ is then a Fréchet space. It now follows from [2.6] that $\{\Lambda_j\}$ is equicontinuous on $\mathscr{D}_{\overline{B}(p,\, r)}\,$. There so exists\footnote{For more details, see Exercise 2.3.} a nonnegative integer $N$ such that 
\begin{align}{\label{6_6_6}
\lvert \Lambda \phi \,\rvert \leq C\, \| \phi \|_N \quad (\phi \in \mathscr{D}_{\overline{B}(p,\, r)})
}\end{align}
for some positive constant $C$. On the other hand, $\overline{B}(p,\, r)$ contains almost all the $p_j$'s. Hence
\begin{align}{\label{6_6_7} 
\lvert \Lambda_N\,\phi \,\rvert >  C\, \| \phi \|_N 
}\end{align}
for some $\phi$ of $\mathscr{D}_{\overline{B}(p,r)}\,$. (b) is then established.\\
\\
To prove (c), we introduce a sequence $\{x_m:\, m\in \Z\}$ of $ \Omega$ that has no limit point. Let $\{ \alpha_m:\, m\in \Z\}$ be in $\N$ and so define\footnote{As $\Omega=\R\,$, the case $\alpha_m= m$ is the ``counterpart" of the series of (a) and the case $(x_m,\, \alpha_m)= (m,\, 0)$ is the \textsl{Dirac comb}.}
\begin{align}{
\Lambda:\, \mathscr{D}(\Omega) \to &  \, \C   \qquad\qquad\qquad\qquad .\\
 \phi \mapsto & \,  \sum_{m=\minus \infty}^\infty (D^{\,\alpha_m}\phi)(x_m)  \nonumber
}\end{align}
$\Lambda$ belongs to $\mathscr{D}'(\Omega)$, since $\{x_m\}$ has no limit point. Next, we easily check that
\begin{align}{
\Lambda_j:\, \mathscr{D}(\Omega) \to   & \, \C  \qquad\qquad\qquad\qquad(j\in \N)\\
 \phi \mapsto &  \,  \sum_{\lvert m\rvert \leq j} (D^{\,\alpha_m}\phi)(x_m) \nonumber
}\end{align}
is also a distribution and that $\{\Lambda_j\}$ tends to $\Lambda$ in $\mathscr{D}'(\Omega)$. Nevertheless, no $\Lambda_j$'s can have common support because $\{x_m\}$ has no limit point. Our assumption can therefore be dropped.\QED















\setcounter{section}{8} 
\section{Exercise 9. Convergence in $\mathscr{D}(\Omega)\,$ vs. convergence in $\mathscr{D}'(\Omega)\,$}
%\renewcommand{\labelenumi}{(\alph{enumi})} 
{\it  
\begin{enumerate}
\item Prove that a set $E\subset \mathcal{D}(\Omega)$ is bounded if and only if 
\begin{align*}
\sup \{\lvert\Lambda \phi\rvert:\, \phi \in E\,\}< \infty
\end{align*} 
for every $\Lambda \in \mathcal{D}(\Omega)$.
\item Suppose $\{\phi_j\}$ is a sequence in $\mathcal{D}(\Omega)$ such that $\{\Lambda \phi_j\}$ is a bounded sequence of numbers, for every $\Lambda \in \mathcal{D}'(\Omega)$. Prove that some subsequence of $\{\phi_j\}$ converges, in the topology of $\mathcal{D}(\Omega)$.
\item Suppose $\{\Lambda _j\}$ is a sequence in $\mathcal{D}'(\Omega)$ such that $\{\Lambda_{j\,} \phi\}$ is bounded, for every $\phi \in \mathcal{D}(\Omega)$. Prove that some subsequence of $\{\Lambda _j\}$ converges in $\mathcal{D}'(\Omega)$ and that the convergence is uniform on every bounded subset of $\mathcal{D}(\Omega)$. Hint: By the Banach-Steinhaus theorem, the restrictions of the $\Lambda_j$ to $\mathcal{D}_K$ are equicontinuous. Apply Ascoli's theorem.
\end{enumerate}}
\begin{proof} 
%: (a)
Since $\mathcal{D}(\Omega)$ is a locally convex space (see (b) of [6.4]), [3.18] states that $E\,$ is bounded if and only if it is weakly bounded. That is (a). \\
\\
%: (b) 
To prove (b), we first use (a) to conclude that $E= \{\phi_j:\, j\in \N\}$ is bounded: so is $\overline{E}$. %
By (c) of [6.5], there exists some $ \mathcal{D}_K$ that contains $\overline{E}$. %
Since $ \mathcal{D}_K$ has the Heine-Borel property (see [1.46]), $\overline{E}$ is $\tau_K$-compact. %
Apply [A4] with the metrizable space $\mathcal{D}_K$ (see [1.46]) to conclude that $\overline{E}$ has a $\tau_K$ limit point. %
It then follows from (b) of [6.5] that (b) holds.
\end{proof}
\setcounter{section}{16} 
\section{Exercise 17. }
%\input{\ROOT/chapter_6/6_17.tex}

 

\backmatter
%\part{Annex}
\chapter{Annex}
%\addcontentsline{toc}{chapter}{Annex}

\newcounter{annex_chapter}
\newcounter{annex_section}

\setcounter{annex_chapter}{1}
\setcounter{annex_section}{1}
\setcounter{equation}{0}

\renewcommand{\thechapter}{\Alph{annex_chapter}} 
\renewcommand{\thesection}{\thechapter.\arabic{annex_section}} 
\renewcommand{\theequation}{\thechapter.\arabic{equation}}
%
\section{Mean value and bounded derivatives}
% % % % % % % % % % % % % % % % % % % % % % % % % % % % % % % % % % % % % % % % % % % % % % % % % % % % % % % % % % % % % % % %
% FunctionalAnalysis 
% FA_annex_mean-value.tex
% 
% encoding: UTF-8 
% EOL: LF
%
% format: LaTeX
% indent: spaces (2)
% width: 127
% % % % % % % % % % % % % % % % % % % % % % % % % % % % % % % % % % % % % % % % % % % % % % % % % % % % % % % % % % % % % % % %
\begin{lemma}[A mean value inequality for higher-order derivatives]\label{mean-value-with-derivatives}
If $\phi\in\D_{[a, b]}$, then
%
\begin{equation}
  \norm[\infty]{D^k \phi} \leq \norm[\infty]{D^p \phi} \left(\frac{b-a}{2}\right)^{p-k}
\end{equation}
%
for all $k \leq p$ in $\N$.
\end{lemma}
%
\begin{proof}
Choose $a < x_0 \leq (a+b)/2$ first. By the mean value theorem, there exists $a < x_1 < x_0$ such that
%
\begin{equation}
  \phi(x_0) = \phi(x_0) - \phi(a)=  D\phi(x_1)(x_0 - a).
\end{equation}
%
Repeating the same reasoning for $D\phi, D^2 \phi, \dots, D^p \phi \in \D_{[a, b]}$ yields % 
%
\begin{align}
  \phi(x_0) & = D^0 \phi(x_0) \\ 
  & = D^1\phi(x_1)(x_0 - a) \\
  &  \nonumber \mspace{10mu} \vdots \\
  & = D^p\phi(x_{p})(x_{p-1} - a) \cdots (x_0-a),
\end{align}
%
for some $a < x_p <  \dots < x_1 < x_0$. Hence %
%
\begin{equation}
  \magnitude{\phi(x_0)} \leq \norm[\infty]{D^p \phi} \left(\frac{b-a}{2}\right)^p .
\end{equation}
%
Similarly, if $b > x_0 \geq (a+b)/2$ (interchanging the roles of $a$ and $b$), the same inequality holds. Thus, 
%
\begin{equation}
  \magnitude{\phi(x_0)} \leq \norm[\infty]{D^p \phi} \left(\frac{b-a}{2}\right)^p \quad (a < x_0 < b), 
\end{equation}
%
which establishes the result when $k=0$. Finally, applying the latter inequality to $D^k \phi $ in %
place of $\phi$ shows that % 
%
\begin{equation}
  \norm[\infty]{D^k \phi} \leq \norm[\infty]{D^p \phi} \left(\frac{b-a}{2}\right)^{p-k}
\end{equation}
%
for all $0 \leq k \leq p$. %
\end{proof}
%
\begin{lemma}[Impossibility of a nontrivial reversed mean value theorem]%
\label{lemma-derivative-not-bounded-by-magnitude}%
There is no general formula to estimate derivatives from the supremum bound.
\end{lemma}
%
\begin{proof}
For pulsation, or \emph{angular frequency} $\omega \to \infty$, we consider 
%
\begin{align}
\left(\phi_\omega, \psi_\omega\right): \R & \to [\minus 1, 1] \times [\minus 1, 1]\\
t & \mapsto \bigl(\sin{(\omega t)}, \cos{(\omega t)}\bigr) \nonumber
\end{align}
%
so that 
\begin{align}\renewcommand{\arraystretch}{1.8}  % or 2
  \begin{bmatrix}
    \magnitude[\big]{D^p \phi_\omega(0)} \\
    \magnitude[\big]{D^p \psi_\omega(0)}
  \end{bmatrix}
  \renewcommand{\arraystretch}{1.25}  % or 2
  = 
  \Iverson{p \equiv 0 \mod{2}} 
    \cdot
    \begin{bmatrix}
      0 \\ 
      \omega^p 
    \end{bmatrix}
    + 
    \Iverson{p \equiv 1 \mod{2}} \cdot 
    \begin{bmatrix}
    \omega^p \\
    0
  \end{bmatrix}
\end{align}
In contrast, 
%
\begin{equation}
  \norm[\infty]{\phi_\omega}  = \norm[\infty]{\psi_\omega} = 1. 
\end{equation}
%
This construction already rules out any general reversed inequality. However, the following smooth example $\phi(t)$ keeps %
constant derivative around $0$, which is more tractable. Let $p\in C^\infty(\R)$ be defined as $1$ on $]\minus \infty, 0]$, 
$0$ on $[1,\infty[$, and a strict decay on $\openinterval{0}{1}$. A standard choice is $p = 1-h$ when %
%
\begin{equation}
  h(t) \Def \frac{e^{\minus 1/t}}{e^{\minus 1/t} + e^{\minus 1/(1-t)}}
\end{equation}
%
for all $0 < t < 1$. Inspired by signal processing, we choose three positive parameters: $A$ (maximum amplitude), $\tau$ %
(delay), and $\omega$ (angular frequency). From now on, response $\phi(t)$ is the solution of 
%
\begin{align}
  \begin{cases}
    \phi(0) & = 0 \\
    \displaystyle{\frac{\diff{\phi}}{\diff{t}}(t)} & = Ap\bigl(\omega(\abs{t} - \tau)\bigr), 
  \end{cases}
\end{align}
%
when time $t$ ranges over the real line. Equivalently, the function $\phi=\phi(t)$ is odd and, as time turns positive, 
%
\begin{align}
  \phi(t)  = A \int_{0}^{t} p\bigl(\omega (s - \tau)\bigr) \diff{s} 
    =  A \min(t, \tau) + A \int_{\tau}^{\max(t,  \tau)} p\bigl(\omega (s - \tau)\bigr) \diff{s} 
\end{align}
%
Note that $\magnitude{\phi}$ has maximum %
%
\begin{align}
  \norm[\infty]{\phi} & = \tau A  + A \int_{\tau}^{\tau + 1/\omega} p\bigl(\omega (s - \tau)\bigr) \diff{s} \\
    & = \tau A  + \frac{A}{\omega} \int_{0}^{1} p(u)\diff{u} \\
    & < \tau A + \frac{A}{\omega}.
\end{align}
%
The special case $\tau= 1/\omega= 1/A$ is of great interest. In this case, amplitude at $t=0$ is $A \to \infty$, as %
$\tau \to 0$. In contrast, $\norm[\infty]{\phi} < 2$.  
\end{proof}
% END
% 
%
\setcounter{annex_section}{2}
\section{Dirac's impulse, a physicist's detour}\label{annex-Dirac}
% % % % % % % % % % % % % % % % % % % % % % % % % % % % % % % % % % % % % % % % % % % % % % % % % % % % % % % % % % % % % % % %
% FunctionalAnalysis 
% FA_annex_Dirac.tex
% 
% encoding: UTF-8 
% EOL: LF
%
% format: LaTeX
% indent: spaces (2)
% width: 127
% % % % % % % % % % % % % % % % % % % % % % % % % % % % % % % % % % % % % % % % % % % % % % % % % % % % % % % % % % % % % % % %
\newcommand{\heuristic}[1]{\mathcal{#1}}
Consider a physical example: a particle colliding with a surface, which absorbs a unit of energy at impact time $t=0$. 
We start with $H$ the \emph{Heaviside step function} $t \mapsto \Iverson*{t \geq 0}$, so that $H(t)$ indicates whether the %
particle has contributed its energy by time $t$. This formalism expresses that 
%
\begin{enumerate}
\item{The energy is transferred by an instantaneous jump at time $t=0$.}
\item{The energy is conserved over time.}
\end{enumerate}
%
Heuristically, we write, 
%
\begin{equation}
  \int_{\R} \heuristic{dH} = 1, \qquad \heuristic{\frac{dH}{dt}} =
  \begin{cases}
  \infty & t = 0 \\
  0 & t \neq 0
  \end{cases}
  \end{equation} 
%
These properties cannot coexist in standard calculus. Nevertheless, the informal measure $\heuristic{dH}$ describes the %
\emph{Dirac $\delta$ function}. When identified with a positive Borel measure, $\delta$ has total mass $1$ because %
%
\begin{equation}\label{mass-of-dirac}
  \int_{\R} \diff{\delta} = \int_{\R} \heuristic{dH} = \evalbracket{H}{\minus \infty}{\infty} = 1. 
\end{equation}
%
Physically, integrating $\delta$ over time recaptures all the energy. Let $W$ be the observation window, which is adjusted so %
that either $0 \in \interior{W}$ or $0 \notin \closure{W}$. Next, consider any smooth real function $\phi$ with compact %
support in $\interior{W}$ as a test signal. In this formalism, the integral $\int_{\R} \phi \diff{\delta}$ represents the %
detector's response to the collision when $\phi$ is not identically zero. If $\max{\abs{\phi}} = 1$, then Lebesgue's %
dominated convergence theorem ensures
%
\begin{equation}
  \sup_\phi \int_{\interior{W}} \abs{\phi}\diff{\delta}= \int_{\interior{W}}\diff{\delta}= \Iverson[\Big]{0 \in \interior{W}}.
\end{equation}
%
We now make the model rigorous by eliminating the heuristic $\heuristic{dH}$, as follows:
%
\begin{align}
  \int_{\R} \phi \diff{\delta} &=\int_{\R} \phi\,\heuristic{dH} & \bigl(\text{generalization of \eqref{mass-of-dirac}}\bigr)\\
  & =  \evalbracket{H\phi}{\minus \infty}{\infty} - \int_{\R} H \diff{\phi} & (\text{integration by parts})\\
  & = \minus \int_{\R} H \diff{\phi} \label{Dirac-impulse-distribution1}&(\text{$\supp{\phi}$ is compact})\\
  & =  \phi(0) &
\end{align}
%
The key point is that the right-hand side in \eqref{Dirac-impulse-distribution1} is valid in standard calculus. %
Moreover, we obtain all filtered responses as the evaluation functional $\phi \mapsto \phi(0)$. This motivates the %
following definitions: %
%
\begin{align}
  \Lambda_H(\phi) &\Def \int_{\R} H \phi & (\text{expresses $H$})\\
  \Lambda_H'(\phi) &\Def \minus \int_{\R} H \diff{\phi} = \phi(0) & 
    (\text{the \textit{weak derivative} of $H$: impulse at $0$})\\
  \delta(\phi) &\Def \phi(0) \label{definition-of-dirac} & (\text{$\delta$ now has a rigorous definition}).
\end{align}
%
The functional $\delta:\phi\mapsto\phi(0)$ is rigorously defined and so represents an instantaneous energy injection at $t=0$. 
Its extension to all $\phi \in C_c(\R)$ turns $\delta$ into a positive Radon measure of norm/total variation %
$\norm*{\delta} = 1$ and support $\singleton{0}$. In the sense of distribution theory, $\delta$ is a (tempered) distribution %
of order $0$; see \citeFA{Chapters 6 and 7}. Notably, its Borel-measure counterpart is the \emph{Dirac measure} %
%
\begin{equation}
  \delta: E \mapsto \Iverson[\Big]{0 \in E} 
\end{equation}
%
restricted to Borel sets in $\R$; hence the special case of \eqref{definition-of-sum}
%
\begin{equation}\label{filtering}
  \int_{\R} \phi \diff{\delta}= \delta(\phi) = \phi(0).
\end{equation}
%
Convolution of $\delta$ with translated signal $\phi_t:s \mapsto \phi(t-s)$ extends \eqref{filtering}, as follows: 
%
\begin{equation}
  [\delta \ast \phi](t) \Def \int_{\R} \phi_t \diff{\delta} 
  = \delta(\phi_t) = \phi(t).
\end{equation}
%
The Radon measure $\delta$ now serves as the convolution identity.
% END
%%
%
% END
% 

% --- Bibliography ---
\bibliographystyle{plain}
\bibliography{bibliography}
\addcontentsline{toc}{chapter}{Bibliography}
\end{document}
% Made on \XeTeX
% END
% 
