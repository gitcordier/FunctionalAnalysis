%\section{Exercise 7. Metrizability \& number theory}
\textit{
Let be X the vector space of all complex functions on the unit interval 
$[0, 1]$, topologized by the family of seminorms 
%
  \begin{align}
    p_{x}(f)=|f(x)| \quad (0\leq x\leq 1).\nonumber
  \end{align}
%
This topology is called the topology of pointwise convergence. 
Justify this terminology.
Show that there is a sequence $\{f_n\}$ in X such that (a) $\{f_n\}$ converges 
to $0$ as $n \to\infty$, but (b) if $\{γ_n\}$ is any sequence of scalars such 
that $γ_n\to\infty$ then $\{γ_nf_n\}$ does not converge to $0$. 
(Use the fact that the collection of all complex sequences converging to $0$ 
has the same cardinality as $[0, 1]$.)
This shows that metrizability cannot be omited in (b) of Theorem 1.28.
}
\begin{proof}
Our justification consists in proving that $\tau$-convergence and pointwise 
convergence are the same one. 
%
To do so, remark first that the family of the seminorms $p_{x}$ is separating.
By [1.37], the collection $\mathscr{B}$ of all finite intersections 
of the sets 
%
  \begin{align}
    V\up{(x, k} 
      \Def 
    \singleton{p_x < 2^{\minus k}} 
      \quad 
    (x \in [0, 1], k \in \N)
  \end{align}
%
is then a local base for a topology $\tau$ on $X$. Given 
%
  $\set{f_n}{\counting{n}}$, 
%
we set
%
\newcommand\off[1]{\function{off}(#1)}
%
\begin{align}
  \off{U} \Def \sum_{n=1}^\infty [f_n \notin U] \quad (U\in\tau),
\end{align}
%
with the convention $\off{U}=\infty$ whether the sum has no finite support. 
So, 
%
  \begin{align}
    %
    \label{Inequality boolean series}
    %
    \sum_{i=1}^m \off{U\up{i}} 
      = 
    \sum_{n=1}^\infty \sum_{i=1}^m [f_n \notin U\up{i}]
      \geq 
    \off{U\up{1} \cuts \cdots \cuts U\up{m}}
  \end{align}
%
We first assume that $\singleton{f_n}$ $\tau$-converges to some $f$ in $X$, \ie
%
  \begin{align}
    \off{f+V} < \infty \quad(V \in \mathscr{B}).
  \end{align}
%
The special cases $V=V\up{x, k}$ mean the pointwise convergence of 
$\singleton{f_n}$. 
%
Conversely, assume that $\singleton{f_n}$ does not $\tau$-converges to any $g$ 
in $X$, \ie 
%
  \begin{align}
    %
    \label{Divergence}
    %
    \forall g \in X, \exists V\up{g} \in \localbase{B}: 
      \off{g+V\up{g}} = \infty. 
    %
  \end{align}
%
Given $g$, $V\up{g}$ is then an intersection
%
  $
    V\up{x\up{1}, k\up{1}} \cap \cdots \cap 
    V\up{x\up{m}, k\up{m}} 
  $.
%
%So, (\ref{1.7 Inequality boolean series}) implies 
Thus
%
  \begin{align}
    \sum_{i=1}^m \off{g + V\up{x\up{i}, k\up{i}}} 
      %
        \citegeq{\ref{Inequality boolean series}} 
      %
    \off{g + V\up{g}} 
    % 
      \citeq{\ref{Divergence}} 
    %
    \infty .
  \end{align}
%
One of the sum $\off{g + V\up{x\up{i}, k\up{i}}}$ must then be $\infty$. 
This implies that convergence of $f_n$ to $g$ fails at point $x_i$.
$g$ being arbitrary, we so conclude that $f_n$ does not converge pointwise.
We have just proved that 
%
  $\tau$-convergence is a rewording of pointwise convergence.
%
% SECOND PART
We now aim to prove the second part.
%
From now on, 
%
  $\varit{k}$, $\varit{n}$ and $\varit{p}$ 
%
run on $\N_+$. Let $\dy{x}$ be the usual dyadic expansion of a real number $x$, 
so that $\dy{x}$ is an aperiodic binary sequence \iif $x$ is irrational. 
%
Define
%
\begin{align}
  %
  \label{f_n(x) definition}
  %
  f_n(x) 
    \Def 
  \begin{cases}
      2^{-\sum_{k= 1}^{n} {\dy{x}_{-k}}} & (x \in [0, 1]\setminus \Q )\\
      0                                            & (x \in [0,1]\cap \Q)
    \end{cases}
\end{align}
%
so that $f_n(x) \tendsto{n}{\infty} 0$
%
and take scalars 
%
  $\gamma_n$ 
%
such that $\tendsto{n}{\infty} \infty$, \ie 
%
  at fixed $p$, $\gamma_{n}$ is greater than $2^{p}$ for almost all $n$.
%
Next, choose $n\up{p}$ among those \textit{almost all} $n$ that are 
large enough to satisfy 
%
  \begin{align}
    n\up{p-1} - n\up{p-2} < n\up{p}- n\up{p-1} 
  \end{align}
%
(start with $n\up{\minus 1} = n\up{0} = 0$) and so obtain 
%
  \begin{align}
    2^p < \gamma_{n\up{p}}:\, 
    %
      0< n\up{p} - n\up{p-1}\tendsto{p}{\infty} \infty.
    %
  \end{align}
%
The indicator $\chi$ of 
%
  $\{n\up{1}, n\up{2}, \dots\}$
%
is then aperiodic, \ie 
%
  \def\xgamma{x^{(\gamma)}}
  \begin{align}
    \xgamma 
      \Def
    \sum_{k=1}^\infty \chi_k 2^{\minus k} 
      %\notin \Q
  \end{align}
is irrational. Consequently,
%
  \begin{align}
    \dy{\xgamma}_{\minus k} &= \chi_{k}.
  \end{align}
%
We now easily see that
%
  \begin{align}
    \chi_1 + \cdots + \chi_{n\up{p}} = p, 
  \end{align}
%
which, combined with (\ref{f_n(x) definition}), yields
%
  \begin{align}
    f_{n\up{p}}(\xgamma) = 2^{\minus p}.
  \end{align}
%
Finally,
%
  \begin{align}
    \gamma_{n\up{p}} f_{n\up{p}}(\xgamma) > 1.
  \end{align}
%
We have so established that the subsequence 
%
  $\singleton{\gamma_{n\up{p}}f_{n\up{p}}}$ 
% 
does not tend pointwise to $0$, hence neither does the whole sequence 
%
  $\singleton{\gamma_{n}f_n}$.
%
In other words, (b) holds, which is in violent contrast with [1.28]: 
$X$ is then not metrizable. So ends the proof.
\end{proof}
% END
