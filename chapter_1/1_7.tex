% % % % % % % % % % % % % % % % % % % % % % % % % % % % % % % % % % % % % % % % % % % % % % % % % % % % % % % % % % % % % % % %
% FunctionalAnalysis 
% 1_7.tex
% 
% encoding: UTF-8 
% EOL: LF
%
% format: LaTeX
% indent: spaces (2)
% width: 12
% % % % % % % % % % % % % % % % % % % % % % % % % % % % % % % % % % % % % % % % % % % % % % % % % % % % % % % % % % % % % % % %
\textit{%
Let be $X$ the vector space of all complex functions on the unit interval $[0, 1]$, topologized by the family of seminorms %
%
\begin{align}
  p_x(f)= \magnitude{f(x)} \quad (0\leq x\leq 1).\nonumber
\end{align}
%
This topology is called the topology of pointwise convergence. Justify this terminology. Show that there is a sequence % 
$\singleton{f_n}$ in X such that (a) $\singleton{f_n}$ converges to $0$ as $n \to \infty$, but (b) if $\{\gamma_n\}$ is any %
sequence of scalars such that $\gamma_n \to \infty$ then $\{\gamma_n f_n\}$ does not converge to $0$. (Use the fact that %
the collection of all complex sequences converging to $0$ has the same cardinality as $[0, 1]$.) This shows that %
metrizability cannot be omited in (b) of Theorem 1.28.
}
%

% % % % % % % % % % % % % % % % % % % % % % % % % % % % % % % % % % % % % % % % % % % % % % % % % % % % % % % % % % % % % % % %
% First part: Justification of the terminology
% % % % % % % % % % % % % % % % % % % % % % % % % % % % % % % % % % % % % % % % % % % % % % % % % % % % % % % % % % % % % % % %
\subsection{Justification of the terminology}
\begin{proof}
% % % % % % % % % % % % % % % % % % % % % % % % % % % % % % % % % % % % % % % % % % % % % % % % % % % % % % % % % % % % % % % %
% A. tau-convergence => pointwise convergence. 
% % % % % % % % % % % % % % % % % % % % % % % % % % % % % % % % % % % % % % % % % % % % % % % % % % % % % % % % % % % % % % % %
The family of the seminorms $p_x$ is separating: The collection $\mathscr{B}$ of all finite intersections of the sets %
%
\begin{align}
  V(x,k) \triangleq \{p_x < 2^{\minus k}\} 
  \quad (x \in [0, 1], k=1, 2, 3, \dots)
\end{align}
%
is therefore a local base for a topology $\tau$ on $X$; see Section 1.37 of \cite{FA}. So, 
%
\begin{align}
  %
  \label{Inequality boolean series}
  %
  \sum_{n=1}^\infty              \bigl[f_n \notin \cap_{i=1}^m U_i\bigr] 
  \leq 
  \sum_{n=1}^\infty \sum_{i=1}^m \bigl[f_n \notin U_i\bigr] 
  = 
  \sum_{i=1}^m \sum_{n=1}^\infty \bigl[f_n \notin U_i\bigr] \quad (f_n \in X, U_i \in \tau).
\end{align}
%
Now assume that $\{f_n\}$ $\tau$-converges to some $f$, \ie 
%
\begin{align}
  %
  \sum_{n=1}^\infty \bigl[f_n \notin f + W\bigr] < \infty \quad (W \in \mathscr{B}).
\end{align}
%
The special case $W = V(x, k)$ means that, given $k$, $\magnitude{f_n(x) - f(x)} < 2^{\minus k}$ for almost all $n$. %
In other words, $\{f_n(x)\}$ converges to $f(x)$. %
% % % % % % % % % % % % % % % % % % % % % % % % % % % % % % % % % % % % % % % % % % % % % % % % % % % % % % % % % % % % % % % %
% B. tau-divergence => pointwise divergence. 
% % % % % % % % % % % % % % % % % % % % % % % % % % % % % % % % % % % % % % % % % % % % % % % % % % % % % % % % % % % % % % % %
Conversely, assume that $\{f_n\}$ does not $\tau$-converges in $X$, \ie %
%
\begin{align}
  %
  \label{Divergence}
  %
  \forall f \in X, \exists W \in \localbase{B}: \sum_{n=1}^\infty \bigl[f_n \notin f +  W\bigr] = \infty. 
  %
\end{align}
%
$W$ is then the intersection of some $V(x_1, k_1), \dots, V(x_m, k_m)$. Hence %
%
\begin{align}
  \infty 
  \citeq{\ref{Divergence}}                  \sum_{n=1}^\infty \bigl[f_n \notin f + W\bigr] 
  \citeleq{\ref{Inequality boolean series}} \sum_{i=1}^m \sum_{n=1}^\infty \bigl[f_n \notin f + V(x_i, k_i)\bigr].
\end{align}
%
It is now clear that % 
%
\begin{align}
  \sum_{n=1}^\infty \bigl[f_n \notin f + V(x_i, k_i)\bigr] = \infty .
\end{align}
%
for some $i$. In other words, $\{f_n(x_i)\}$ fails to converge to $f(x_i)$. The overall conclusion is that %
$\tau$-convergence is the $X$'s version of pointwise convergence. %
\end{proof}
%
% % % % % % % % % % % % % % % % % % % % % % % % % % % % % % % % % % % % % % % % % % % % % % % % % % % % % % % % % % % % % % % %
% SECOND PART
% % % % % % % % % % % % % % % % % % % % % % % % % % % % % % % % % % % % % % % % % % % % % % % % % % % % % % % % % % % % % % % %
%
% % % % % % % % % % % % % % % % % % % % % % % % % % % % % % % % % % % % % % % % % % % % % % % % % % % % % % % % % % % % % % % %
% First proof: Use the given hint
% % % % % % % % % % % % % % % % % % % % % % % % % % % % % % % % % % % % % % % % % % % % % % % % % % % % % % % % % % % % % % % %
\subsection{Proof (with the given hint)}
We now prove the second part by constructing a specific sequence $\{f_n\}$ that simultaneously satisfies (a) and (b). %
Indeed, the hint says that there exists a one-to-one and \textit{onto} mapping %
%
\begin{align}
  \varphi: [0, 1] & \to \set{\theta}{\theta \in \R^{\N_+}, \lim_{\infty} \theta = 0}.\\
          x       & \mapsto (\theta_1, \dots, \theta_n, \dots) \nonumber
\end{align}
%
\begin{proof}
We set (under the same notation)%
%
\begin{align}
  f_n(x) & \Def \theta_n \tendsto{n}{\infty} 0 \quad \bigl(x = \varphi^{\minus 1}(\theta_1, \dots, \theta_n, \dots)\bigr)
\end{align}
%
so that %
%
  $x_\gamma = \varphi^{\minus 1}\Bigl( 1 / \sqrt{1 + \magnitude{\gamma_1}}, \dots,1 / \sqrt{1 + \magnitude{\gamma_n}}, \dots\Bigr)$ %
%
implies %
%
\begin{align}
  \gamma_n f_n(x_\gamma) = \gamma_n / \sqrt{1 + \magnitude{\gamma_n}} %
  \underset{\infty}{\sim} \sqrt{\gamma_n} %
  \tendsto{n}{\infty} \infty, 
\end{align}
%
provided $\gamma_n \to \infty$. This proves (b), since 
%$\bigl(\gamma_n f_n(x_\gamma)\bigr)_n $ $\singleton{\gamma_nf_n(x_\gamma)}_{n=1}^{\infty}$ 
$\set{\gamma_n f_n(x_\gamma)}{\counting{n}}$ diverges.
\end{proof}
%
% % % % % % % % % % % % % % % % % % % % % % % % % % % % % % % % % % % % % % % % % % % % % % % % % % % % % % % % % % % % % % % %
% Second proof: No hint, the hard way.
% % % % % % % % % % % % % % % % % % % % % % % % % % % % % % % % % % % % % % % % % % % % % % % % % % % % % % % % % % % % % % % %
\subsection{Proving it the hard way (no hint)}
We will use the following simple proposition about binary expansions: each irrational has an eventually aperiodic binary %
expansion. %
%
\begin{proof}
Formally, there exists a one-to-one and \textit{onto} mapping %
%
\def\bin{\text{\fw bin}}
\begin{align}
  \bin: [0, 1] \setminus \Q &\to \bigl\{\beta \in \{0, 1\}^{\N_+}: \beta \text{ is eventually aperiodic}\bigr\}\\
    v &\mapsto (\beta_1, \dots, \beta_n, \dots)\nonumber
\end{align}
%%
where $(\beta_1, \dots, \beta_n, \dots)$ satisfies %
%
\begin{align}
  \label{definition of alpha}
  v &= \beta_1 / 2 + \cdots + \beta_n / 2^n + \cdots .
\end{align}
%
First, note that $\beta_1 + \beta_2 + \beta_3 + \cdots = \infty$ then define (under the same notation)%
%
\begin{align}
  \label{definition of f_n(alpha)}
   f_n(v)  &\Def 1/2^{\beta_1 + \cdots + \beta_n} \tendsto{n}{\infty} 0.	
\end{align}
%
Now pick an arbitrary $\gamma_n \to \infty$: Given a positive integer $k$, $\gamma_n > 4^k$ for almost all $n$, %
say $\tilde{n}$. Next, choose $n=n_k$ from $\singleton{\tilde{n}}$ so large that %
%
\begin{align}
  \label{definition of n_k}
  n_{k+1} - n_k > k + 1 \to \infty .
\end{align}
%
This way, $\chi = 1_{\{ n_1, n_2, \dots \}}$ \textit{is not eventually periodic on }$\N_+$! Moreover, one easily checks that %
the specialization $\beta = \chi = \bin (v_\gamma)$ yields %
%
\begin{align} \label{p sum of bits}
  \beta_1 + \dots + \beta_{n_1} + \dots + \beta_{n_k} = k.
\end{align}
%
Finally, combining (\ref{definition of f_n(alpha)}) with (\ref{p sum of bits}) shows that %
%
\begin{align}
  \gamma_{n_k} f_{n_{k}}(v_\gamma) 
    = {\gamma_{n_k}}/{2^k} 
    > 4^k / 2 ^k 
    > 2^k 
    \tendsto{k}{\infty}\infty.
\end{align}
%
As a conclusion, every $\gamma_n \to\infty$ contains a subsequence $\{\gamma_{n_k}\}$ that keeps %
$\gamma_{n_k} f_{n_k}(v)$ away from $0$ (for some $v=v_\gamma$) when $k \to \infty$; which is (b). 
\end{proof}
%
% END
