%\section{Exercise 7. Metrizability \& number theory}
\noindent
\textit{
Let be $X$ the vector space of all complex functions on the unit interval 
$[0, 1]$, topologized by the family of seminorms 
%
  \begin{align}
    \mathit{p_{x}(f\,)=|f(x)|} \quad\quad (0\leq x\leq 1).\nonumber
  \end{align}
%
This topology is called the topology of pointwise convergence. 
Justify this terminology.
Show that there is a sequence $\mathit{\singleton{f_n}}$ in X such that (a) %
%
$\mathit{\singleton{f_n}}$ converges to $0$ as $\mathit{n \to\infty}$, 
%
but (b) if $\mathit{\singleton{\gamma_n}}$ is any sequence of scalars such 
that $\mathit{\gamma_n\to\infty}$ then $\mathit{\singleton{\gamma_n f_n}}$ %
does not converge to $0$. 
(Use the fact that the collection of all complex sequences converging to $0$ 
has the same cardinality as $[0, 1]$.)
This shows that metrizability cannot be omited in (b) of Theorem 1.28.
}
%
\begin{proof}
The family of the seminorms $p_{\varit{x}}$ is separating: 
By \citeresultFA{1.37}, the collection $\mathscr{B}$ of all finite %
intersections of the sets %
%
  \begin{align}
    %\tensor*[_x]{V}{_k}
    V(\varit{x},\varit{k})
      \Def 
    \singleton{p_\varit{x} < 2^{\,\minus \varit{k}}} 
      \quad\quad
    (\varit{\varit{x}} \in [0, 1], \varit{k}=1, 2, 3, \dots)
  \end{align}
%
is therefore a local base for a topology $\tau$ on $X$. 
So, 
%
  \begin{align}
    %
    \label{Inequality boolean series}
    %
    \sum_{n=1}^\infty \boolean{f_n \notin \cap_{i=1}^m U_i} \leq 
    \sum_{n=1}^\infty \sum_{i=1}^m \boolean{f_n \notin U_{i}} = 
    \sum_{i=1}^m \sum_{n=1}^\infty \boolean{f_n \notin U_i} %\off{U_{i}    
    \quad\quad (f_n \in X, U_i \in \tau).
  \end{align}%
%
Now assume that $\{f_n\}$ $\tau$-converges to some $f$, \ie 
%
  \begin{align}
    %
    \sum_{n=1}^\infty \boolean{f_n \notin f + W} < \infty \quad\quad 
      (W \in \mathscr{B}).
  \end{align}
The special case %
%
$W = V(x, k)$ %
%
means that, given $k$, %
%
  $\magnitude{f_n(x) - f(x)} < 2^{\,\minus k}$ %
%
for almost all $n$,  \ie 
  $\singleton{f_n(x)}$ converges to $f(x)$. 
Conversely, %
assume that $\singleton{f_n}$ does not $\tau$-converges in $X$, \ie 
%
  \begin{align}
    %
    \label{Divergence}
    %
    \forall f \in X, \exists W \in \localbase{B}: 
      \sum_{n=1}^\infty\boolean{f_n \notin f +  W} = \infty. 
    %
  \end{align}
%
$W$ is now the (nonempty) intersection of finitely many %
%
$V(\varit{\varit{x}}, \varit{k})$, %
%
say  
%
  $ V(x_1, k_1), \dots, V(x_m, k_m)$. Thus,  %
%
  \begin{align}
    \sum_{i=1}^m \sum_{n=1}^\infty \boolean{f_n \notin f + V(x_i, k_i)}
      %
        \citegeq{\ref{Inequality boolean series}} 
      %
    \sum_{n=1}^\infty \boolean{f_n \notin f + W}
    % 
      \citeq{\ref{Divergence}} 
    %
    \infty .
  \end{align}
%
We can now conclude that, for some index $\varit{i}$, 
%
  \begin{align}
    \sum_{n=1}^\infty \boolean{
      f_n \notin f + V(x_i, k_i)
    } = \infty .
  \end{align}
In other word, %
%
$\singleton{f_n(x_i)}$ %
%
fails to converge to 
$f(x_i)$.
We have so proved that 
%
  $\tau$-convergence is a rewording of pointwise convergence.
%
% SECOND PART
We now establish the second part. \\% 
\newline\noindent
To do so, we split $\varit{x}$ into two variables: %
$\varit{r}$ if $\varit{x}$ is rational, ${\alpha}$ otherwise.
%
The proof is based on the following well-known result: Each $\alpha$ %
has a {\it unique} binary expansion. %
More precisely, there exists a bijection %
%
$b:[0, 1] \setminus \Q \to %
\set{\beta \in \{0, 1\}^{\N_+}}{\beta \text{ is not eventually periodic}}$ %
%
where %
$b(\alpha) = (\beta_1, \beta_2, \dots)$ %
is the only bit stream such that %
%
  \begin{align}
    \label{definition of alpha}
    \alpha = \sum_{k=1}^\infty \beta_k \cdot 2^{\,\minus k}.
  \end{align}
%
Remark that %
%
$b(\alpha)_1 + \dots + b(\alpha)_n \tendsto{n}{\infty}\infty$, %
%
since %
%
$b(\alpha)$
%
has infinite support, then fix %
%
  \begin{equation}
    \label{definition of f_n(alpha)}
    f_n(\alpha)\Def %
    \frac{1}{b(\alpha)_1+\cdots+ b(\alpha)_n} \tendsto{n}{\infty} 0.
  \end{equation}
%
The actual values $f_n(r)$ are of no interest, %
as long as every sequence $\set{f_n(r)}{\counting{n}}$ converges to $0$. %
For example, put $f_n(r) = r/n$, or just $f_n(r) = 0$. %
We also take % 
%
  $\gamma_n \longrightarrow \infty$, \ie % 
%
given any counting number $\varit{p}$, %
$\gamma_{n}$ is greater than $\varit{p}$ for almost all $\varit{n}$.
%
Next, we choose $n_{p}$ among those \textit{almost all} $\mathit{n}$ that are 
large enough to satisfy 
%
  \begin{align}
    \label{definition of n_p}
    n_p - n_{p-1} > p
  \end{align}
%
(start with $n_0=0$).
%
So, every list %
%
  $n_{p}, n_{p'}, n_{p''}, \dots $ %
%
that satisfies %
%
  $n_{p'}- n_{p} = n_{p''} - n_{p'} = \dots$ %
%
is finite %
(otherwise, %
    $n_{p'}-n_{p} \geq n_{p+1} - n_{p}  
    > p 
    \rightarrow  \infty %
    $
would hold; see (\ref{definition of n_p})).
In other words, {\it the distribution of %
%
$\mathit{n_1, n_2, \dots}$ %
%
displays no periodic pattern}. %
As a consequence, the {\it characteristic function} %
  $\chi: k\mapsto \boolean{k \in \singleton{n_1, n_2, \dots}}$ %
is not eventually periodic. %
Combined with (\ref{definition of alpha}), this establishes that 
%
\def\xgamma{\alpha_{\gamma}}
%
  \begin{align}
    \xgamma 
      \Def
    \sum_{k=1}^\infty \chi_k 2^{\, \minus k}
      %\notin \Q
  \end{align}
%
is irrational. Conversely, still with (\ref{definition of alpha}),  %
%
  \begin{align}
    b(\xgamma)_{k} = \chi_{k}.
  \end{align}
%
Now remark that %
%
  \begin{alignat}{2}
    \chi_1 + \cdots + \chi_{n_{1}} & &&=1 \\
    \chi_1 + \cdots + \chi_{n_{1}} & + \cdots + \chi_{n_{2}} &&=2 \\
    \nonumber \vdots&  &&\\
    \chi_1 + \cdots + \chi_{n_{1}} & + \cdots + \chi_{n_{2}} + \cdots + 
    \chi_{n_{p}}&&=p.
  \end{alignat}
%
Combined with (\ref{definition of f_n(alpha)}), this yields %
%
  \begin{align}
    \gamma_{n_p} f_{n_{p}}(\xgamma) = \frac{\gamma_{n_p}}{p} > 1.
  \end{align}
%
There so exists a subsequence $\singleton{\gamma_{n_p}}$ such that %
%
  $\singleton{\gamma_{n_p} f_{\gamma_{n_p}}}$ %
%
fails to converge pointwise to $0$. %
Since $\singleton{\gamma_{n}}$ was arbitrary, this proves (b).
\end{proof}
% END
