% % % % % % % % % % % % % % % % % % % % % % % % % % % % % % % % % % % % % % % % % % % % % % % % % % % % % % % % % % % % % % % %
% FunctionalAnalysis 
% 1_7.tex
% 
% encoding: UTF-8 
% EOL: LF
%
% format: LaTeX
% indent: spaces (2)
% width: 127
% % % % % % % % % % % % % % % % % % % % % % % % % % % % % % % % % % % % % % % % % % % % % % % % % % % % % % % % % % % % % % % %
\textit{%
Let $X$ be the vector space of all complex functions on the unit interval $[0, 1]$, topologized by the family of seminorms %
%
\begin{align}
  p_x(f)= \magnitude{f(x)} \quad (0\leq x\leq 1).\nonumber
\end{align}
%
This topology is called the topology of pointwise convergence. Justify this terminology. \\
\\
Show that there is a sequence % 
$\singleton{f_n}$ in X such that (a) $\singleton{f_n}$ converges to $0$ as $n \to \infty$, but (b) if $\singleton{\gamma_n}$ %
is any sequence of scalars such that $\gamma_n \to \infty$ then $\singleton{\gamma_n f_n}$ does not converge to $0$. %
(Use the fact that the collection of all complex sequences converging to $0$ has the same cardinality as $[0, 1]$.) %
This shows that metrizability cannot be omitted in (b) of Theorem 1.28.
}
%

% % % % % % % % % % % % % % % % % % % % % % % % % % % % % % % % % % % % % % % % % % % % % % % % % % % % % % % % % % % % % % % %
% First part: Justification of the terminology
% % % % % % % % % % % % % % % % % % % % % % % % % % % % % % % % % % % % % % % % % % % % % % % % % % % % % % % % % % % % % % % %
\subsection{Justifying the terminology}
\begin{proof}
% % % % % % % % % % % % % % % % % % % % % % % % % % % % % % % % % % % % % % % % % % % % % % % % % % % % % % % % % % % % % % % %
% A. tau-convergence => pointwise convergence. 
% % % % % % % % % % % % % % % % % % % % % % % % % % % % % % % % % % % % % % % % % % % % % % % % % % % % % % % % % % % % % % % %
The family of seminorms $p_x$ is separating: The collection $\mathscr{B}$ of all finite intersections of the sets %
%
\begin{align}
  V(x,k) \triangleq \singleton{p_x < 2^{\minus k}} 
  \quad (x \in [0, 1], k=1, 2, 3, \dots)
\end{align}
%
is therefore a local base for a topology $\tau$ on $X$, see Section 1.37 of \cite{FA}. Thus, given $\singleton{g_n} \subset X$, 
the following inequalities, expressed in Iverson's notation, hold: %
%
\begin{align}
  %
  \label{Inequality boolean series}
  %
  \sum_{n=1}^\infty \Iverson{g_n \notin \cap_{i=1}^m U_i}
  \leq 
  \sum_{n=1}^\infty \sum_{i=1}^m \Iverson{g_n \notin U_i} 
  = 
  \sum_{i=1}^m \sum_{n=1}^\infty \Iverson{g_n \notin U_i}.
\end{align}
%
Now assume that $\singleton{f_n}$ $\tau$-converges to some $f$, which can be stated as follows: %
%
\begin{align}
  %
  \sum_{n=1}^\infty \Iverson{f_n - f \notin W} < \infty \quad (W \in \mathscr{B}).
\end{align}
%
The special case $W = V(x, k)$ implies that, given $k$, $\magnitude{f_n(x)- f(x)} < 2^{\minus k}$ for all but finitely many $n$. %
In other words, $\singleton{f_n(x)}$ converges to $f(x)$. %
% % % % % % % % % % % % % % % % % % % % % % % % % % % % % % % % % % % % % % % % % % % % % % % % % % % % % % % % % % % % % % % %
% B. tau-divergence => pointwise divergence. 
% % % % % % % % % % % % % % % % % % % % % % % % % % % % % % % % % % % % % % % % % % % % % % % % % % % % % % % % % % % % % % % %
Conversely, assume that $\singleton{f_n}$ $\tau$-diverges\ie for any $f \in X$, there exist finitely many %
$V(x_1, k_1), \dots, V(x_m, k_m)$ such that %
%
\begin{align}
  %
  \label{Divergence}
  \sum_{n=1}^\infty \Iverson{f_n - f \notin \cap_{i=1}^m V(x_i, k_i)} = \infty. 
\end{align}
%
In (\ref{Inequality boolean series}), the special case $g_n = f_n - f$ and $U_i = V(x_i, k_i)$ is then %
%
\begin{align}
  \sum_{n=1}^\infty \Iverson{f_n - f \notin \cap_{i=1}^m V(x_i, k_i)} \citeleq{\ref{Inequality boolean series}} 
  \sum_{i=1}^m \sum_{n=1}^\infty \Iverson{f_n - f \notin V(x_i, k_i)}
  = \infty.
\end{align}
%
It is now obvious that % 
%
\begin{align}
  \sum_{n=1}^\infty \bigl[f_n - f \notin V(x_i, k_i)\bigr] = \infty
\end{align}
%
for some $i$, which shows that $\singleton{f_n(x_i)}$ does not converge to $f(x_i)$. Thus, $\tau$-convergence coincides with %
pointwise convergence on $X$. 
\end{proof}
%
% % % % % % % % % % % % % % % % % % % % % % % % % % % % % % % % % % % % % % % % % % % % % % % % % % % % % % % % % % % % % % % %
% SECOND PART
% % % % % % % % % % % % % % % % % % % % % % % % % % % % % % % % % % % % % % % % % % % % % % % % % % % % % % % % % % % % % % % %
%
% % % % % % % % % % % % % % % % % % % % % % % % % % % % % % % % % % % % % % % % % % % % % % % % % % % % % % % % % % % % % % % %
% First proof: Use the given hint
% % % % % % % % % % % % % % % % % % % % % % % % % % % % % % % % % % % % % % % % % % % % % % % % % % % % % % % % % % % % % % % %
\subsection{Proof (with the given hint)}
We now prove the second part by constructing a specific sequence $\singleton{f_n}$ that simultaneously satisfies (a) and (b). %
Indeed, it was hinted that there exists a one-to-one and onto mapping %  
%
\begin{align}
  \varphi: \bigl\{(\theta_n)\!: \theta_n \xrightarrow{n\infty} 0\bigr\} & \to [0, 1] \\
          (\theta_1, \dots, \theta_n, \dots) & \mapsto x. \nonumber
\end{align}
%
\begin{proof}
This allows us to set %
%
\begin{align}
  f_n(x) & \Def \theta_n \tendsto{n}{\infty} 0 \quad \bigl(x = \varphi(\theta_1, \dots, \theta_n, \dots)\bigr).
\end{align}
%
This way, the special case %
%
  $x_\gamma %
  = \varphi( 1 / \sqrt{1 + \magnitude{\gamma_1}}, \dots,1 / \sqrt{1 + \magnitude{\gamma_n}}, \dots)$ %
%
outputs %
%
\begin{align}
  \gamma_n f_n(x_\gamma) = \gamma_n / \sqrt{1 + \magnitude{\gamma_n}} %
  \tendsto{n}{\infty} \infty, 
\end{align}
%
provided $\gamma_n \to \infty$. This proves (b), since $\singleton{\gamma_n f_n(x_\gamma)}$ diverges.
\end{proof}
%
% % % % % % % % % % % % % % % % % % % % % % % % % % % % % % % % % % % % % % % % % % % % % % % % % % % % % % % % % % % % % % % %
% Second proof: No hint, the hard way.
% % % % % % % % % % % % % % % % % % % % % % % % % % % % % % % % % % % % % % % % % % % % % % % % % % % % % % % % % % % % % % % %
\subsection{Proving it the hard way (no hint)}
We will use the following simple proposition about binary expansions: Every irrational has an eventually aperiodic binary %
expansion. %
%
\begin{proof}
More precisely, there exists a bijection %
%
\def\bin{\text{\fw sum}}
\begin{align}
  \bin:  \bigl\{\beta \in \{0, 1\}^{\N_+}: \beta \text{ is eventually aperiodic}\bigr\} & \to [0, 1] \setminus \Q \\
        (\beta_1, \dots, \beta_n, \dots) & \mapsto \sum_{k=1}^\infty \beta_k 2^{\minus k}. \nonumber
\end{align}
%
Now a convenient $\singleton{f_n}$ can be defined as follows: %
%
\begin{align}
  \label{definition of f_n(alpha)} f_n(x) \Def \begin{cases} 
    2^{\minus (\beta_1 + \cdots + \beta_n)} & \bigl(x \in [0, 1]\setminus \Q, %
      \text{where } (\beta_1, \dots, \beta_n, \dots) = \bin^{\minus 1}(x)\bigr) \\
    0 & (x \in \Q).
  \end{cases} 
\end{align}
%
Indeed, we note that every bit stream $\bin^{\minus 1}(x)$ has infinitely many $1$'s, which implies that %
$f_n(x) \xrightarrow{n\infty} 0$. Next, pick an arbitrary $\gamma_n \to \infty$. Thus, for any positive integer $k$, %
$\gamma_n > 4^k$ for all sufficiently large $n$, say $n > N_k$. We actually choose $n_k > N_k$ so large that %
%
\begin{align}
  \label{definition of n_k}
  n_{k+1} - n_k > k.
\end{align}
%
The point is that $1_{\{n_1, n_2, \dots \}}: \N_{+} \to \singleton{0,1}$ is eventually aperiodic\ie has an image by $\bin$. %
Moreover, the specialization $\beta = 1_{\{n_1, n_2, \dots \}}$ implies %
%
\begin{align} \label{p sum of bits}
  \beta_1 + \dots + \beta_{n_1} + \dots + \beta_{n_k} = k.
\end{align}
%
Finally, keep this special $\beta$ so that combining (\ref{definition of f_n(alpha)}) with (\ref{p sum of bits}) yields %
%
\begin{align}
  \gamma_{n_k} f_{n_{k}}(\bin(\beta))
    = {\gamma_{n_k}}/{2^k} 
    > 4^k / 2^k 
    > 2^k 
    \tendsto{k}{\infty}\infty.
\end{align}
%
In conclusion, every $\gamma_n \to\infty$ contains a subsequence $\singleton{\gamma_{n_k}}$ that makes %
$\singleton{\gamma_{n_k}f_{n_k}}$ diverge. This is (b). 
\end{proof}
%
% END
