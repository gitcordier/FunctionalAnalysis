\textit{
Let be $X$ the vector space of all complex functions on the unit interval %
$[0, 1]$, topologized by the family of seminorms %
%
\begin{align}
  p_x(f)= |f(x)\ \quad (0\leq x\leq 1).\nonumber
\end{align}
%
This topology is called the topology of pointwise convergence. %
Justify this terminology. %
%
Show that there is a sequence $\{f_n\}$ in X such that (a) %
%
  $\{f_n\}$ converges to $0$ as $n \to\infty$, %
%
but (b) if $\{\gamma_n\}$ is any sequence of scalars such that %
%
  $\gamma_n\to\infty$ %
%
then $\{\gamma_n f_n\}$ does not converge to $0$. %
(Use the fact that the collection of all complex sequences converging to $0$ %
has the same cardinality as $[0, 1]$.) %
%
This shows that metrizability cannot be omited in (b) of Theorem 1.28.
}
%
\begin{proof}
The family of the seminorms $p_x$ is separating: %
The collection $\mathscr{B}$ of all finite intersections of the sets %
%
\begin{align}
  V(x,k) \triangleq \{p_x < 2^{\minus k}\} 
  \quad (x \in [0, 1], k=1, 2, 3, \dots)
\end{align}
%
is therefore a local base for a topology $\tau$ on $X$; %
see Section 1.37 of \cite{FA}. %
So, 
%
\begin{align}
  %
  \label{Inequality boolean series}
  %
  \sum_{n=1}^\infty [f_n \notin \cap_{i=1}^m U_i] \leq 
  \sum_{n=1}^\infty \sum_{i=1}^m      [f_n \notin U_i] = 
  \sum_{i=1}^m      \sum_{n=1}^\infty [f_n \notin U_i]
  \quad (f_n \in X, U_i \in \tau).
\end{align}
%
Now assume that $\{f_n\}$ $\tau$-converges to some $f$, \ie 
%
\begin{align}
  %
  \sum_{n=1}^\infty [f_n \notin f + W] < \infty \quad (W \in \mathscr{B}).
\end{align}
The special case %
%
$W = V(x, k)$ %
%
means that, given $k$, %
%
$|f_n(x) - f(x)| < 2^{\minus k}$ %
%
for almost all $n$. %
In other words, $\{f_n(x)\}$ converges to $f(x)$. %
%
Conversely, assume that $\{f_n\}$ does not $\tau$-converges in $X$, \ie %
%
\begin{align}
  %
  \label{Divergence}
  %
  \forall f \in X, \exists W \in \localbase{B}: 
    \sum_{n=1}^\infty [f_n \notin f +  W] = \infty. 
  %
\end{align}
%
$W$ is now the (nonempty) intersection of finitely many $V(x, k)$, say %
%
  $V(x_1, k_1), \dots, V(x_m, k_m)$. %
%
Thus,  %
%
\begin{align}
  \sum_{i=1}^m \sum_{n=1}^\infty [f_n \notin f + V(x_i, k_i)]
    %
      \overset{(\ref{Inequality boolean series})}{\geq}
    %
  \sum_{n=1}^\infty [f_n \notin f + W]
  % 
    \overset{(\ref{Divergence})}{=}
  %
  \infty .
\end{align}
%
We can now conclude that, for some index $i$, 
%
\begin{align}
  \sum_{n=1}^\infty [f_n \notin f + V(x_i, k_i)] = \infty .
\end{align}
%
In other words, $\{f_n(x_i)\}$ fails to converge to $f(x_i)$. We have so %
proved that $\tau$-convergence is a rewording of pointwise convergence. %
%
% SECOND PART
We now establish the second part by constructing a specific sequence %
$\{f_n\}$ that satisfies both (a) and (b). \\
\\
The proof will be based on the following well-known result: %
Each irrational number $\alpha$ has a \textit{unique} binary expansion. %
More precisely, there exists a bijection %
%
\begin{align}
  b: [0, 1] \setminus \Q &\to \set{\beta \in \{0, 1\}^{\N_+}}{\beta \text{ is not eventually periodic}}\\
    \alpha &\mapsto (\beta_1, \beta_2, \dots)\nonumber
\end{align}
%%
where $(\beta_1, \beta_2, \dots)$ is the only bit stream such that %
%
\begin{align}
  \label{definition of alpha}
  \alpha = \sum_{k=1}^\infty \beta_k 2^{\minus k}.
\end{align}
%
First, remark that $\beta_1 + \beta_2 + \dots = \infty$ then define %
%
\begin{align}
  \label{definition of f_n(alpha)}
  %f_n(\Q)      &\Def  \singleton{0}, \\
  f_n(\alpha)  &\Def  \frac{1}{1 + \beta_1 + \dots + \beta_n}, \,\, f_n  = 0 \text{ on } \Q, 
\end{align}
%
so that $f_n$ tends to $0$ (pointwise). %
%
Independently, take an arbitrary % 
%
$\gamma_n \longrightarrow \infty$: % 
%
Given $\counting{p}$, $\gamma_n$ is greater than $2^p$ %
for all but finitely many $n$. %
%
Next, we choose $n_p$ among those \textit{almost all} $n$ that are 
large enough to additionally satisfy %
%
\begin{align}
  \label{definition of n_p}
  n - n_{p-1} > p, 
\end{align}
%
provided $n_0=0$. This way, the distribution of %
%
  $n_1, n_2, \dots$, %
%
\textit{displays no periodic pattern}: %
Setting $\beta_{\gamma, k} = [k \in \{n_1, n_2, \dots \}]$ yields %
%
\begin{align}
  \alpha_\gamma & \Def \sum_{k=1}^\infty \beta_{\gamma, k} 2^{\, \minus k} \notin \Q.
\end{align}
%
Moreover, %
%
\begin{align}
  p &= \beta_{\gamma, 1} + \dots + \beta_{\gamma, n_1} + \dots + \beta_{\gamma, n_p}.
\end{align}
%
Hence %
%
\begin{align}
  \gamma_{n_p} f_{n_{p}}(\alpha_\gamma) 
    = \frac{\gamma_{n_p}}{p+1}  
    > \frac{2^p}{p+1}
      \tendsto{p}{\infty} \infty.
\end{align}
%
There so exists a subsequence $\{\gamma_{n_p}\}$ that prevents %
%
$\{\gamma_{n_p} f_{\gamma_{n_p}}\}$ %
%
to converge pointwise to $0$. %
Since $\{\gamma_{n}\}$ was arbitrary, this proves (b).
\end{proof}
% END
