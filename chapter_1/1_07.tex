% % % % % % % % % % % % % % % % % % % % % % % % % % % % % % % % % % % % % % % % % % % % % % % % % % % % % % % % % % % % % % % %
% FunctionalAnalysis 
% 1_07.tex
% 
% encoding: UTF-8 
% EOL: LF
%
% format: LaTeX
% indent: spaces (2)
% width: 127
% % % % % % % % % % % % % % % % % % % % % % % % % % % % % % % % % % % % % % % % % % % % % % % % % % % % % % % % % % % % % % % %
\textit{%
Let $X$ be the vector space of all complex functions on the unit interval $[0, 1]$, topologized by the family of seminorms %
%
\begin{equation}
  p_x(f)= \magnitude{f(x)} \quad (0 \leq x \leq 1).\nonumber
\end{equation}
%
This topology is called the topology of pointwise convergence. Justify this terminology. \\
\\
Show that there is a sequence % 
$\singleton{f_n}$ in $X$ such that (a) $\singleton{f_n}$ converges to $0$ as $n \to \infty$, but (b) if $\singleton{\gamma_n}$ %
is any sequence of scalars such that $\gamma_n \to \infty$ then $\singleton{\gamma_n f_n}$ does not converge to $0$. %
(Use the fact that the collection of all complex sequences converging to $0$ has the same cardinality as $[0, 1]$.) %
This shows that metrizability cannot be omitted in (b) of Theorem 1.28.
}
%

% % % % % % % % % % % % % % % % % % % % % % % % % % % % % % % % % % % % % % % % % % % % % % % % % % % % % % % % % % % % % % % %
% First part: Justification of the terminology
% % % % % % % % % % % % % % % % % % % % % % % % % % % % % % % % % % % % % % % % % % % % % % % % % % % % % % % % % % % % % % % %
\begin{proof}
\textbf{Justifying the terminology.}
% % % % % % % % % % % % % % % % % % % % % % % % % % % % % % % % % % % % % % % % % % % % % % % % % % % % % % % % % % % % % % % %
% A. tau-convergence => pointwise convergence. 
% % % % % % % % % % % % % % % % % % % % % % % % % % % % % % % % % % % % % % % % % % % % % % % % % % % % % % % % % % % % % % % %
Since the family of seminorms $p_x$ is separating, the collection $\mathscr{B}$ of all finite intersections of the sets %
%
\begin{equation}
  V(x,k) \triangleq \singleton*{p_x < 2^{\minus k}} 
  \quad (x \in [0, 1], k=1, 2, 3, \dots)
\end{equation}
%
forms a local base for a topology $\tau$ on $X$, see Theorem \citeFA{1.37}. Hence %
%
\begin{equation}
  %
  \label{Inequality-boolean-series}
  %
  \sum_{n=1}^\infty \Iverson{g_n \notin \cap_{i=1}^m U_i}
  \leq 
  \sum_{n=1}^\infty \sum_{i=1}^m \Iverson{g_n \notin U_i} 
  = 
  \sum_{i=1}^m \sum_{n=1}^\infty \Iverson{g_n \notin U_i},
\end{equation}
%
see [\ref{Iverson-notation}] for Iverson bracket notation. Now assume that $\singleton{f_n}$ $\tau$-converges to some $f$. %
By definition, %
%
\begin{equation}
  %
  \sum_{n=1}^\infty \Iverson{f_n - f \notin W} < \infty \quad (W \in \mathscr{B}).
\end{equation}
%
The special case $W = V(x, k)$ implies that, for a fixed $k$, $\magnitude{f_n(x)- f(x)} < 2^{\minus k}$ for all but %
finitely many $n$. In other words, $\singleton{f_n(x)}$ converges to $f(x)$. %
% % % % % % % % % % % % % % % % % % % % % % % % % % % % % % % % % % % % % % % % % % % % % % % % % % % % % % % % % % % % % % % %
% B. tau-divergence => pointwise divergence. 
% % % % % % % % % % % % % % % % % % % % % % % % % % % % % % % % % % % % % % % % % % % % % % % % % % % % % % % % % % % % % % % %
Conversely, assume that $\singleton{f_n}$ does not $\tau$-converge. This implies that, for every $f \in X$, there exist %
finitely many $V(x_1, k_1), \dots, V(x_m, k_m)$ such that %
%
\begin{equation}
  %
  \label{Divergence}
  \sum_{n=1}^\infty \Iverson{f_n - f \notin \cap_{i=1}^m V(x_i, k_i)} = \infty. 
\end{equation}
%
Substituting $f_n - f$ for $f_n$ in \eqref{Inequality-boolean-series}, with $U_i = V(x_i, k_i)$, yields 
%
\begin{equation}
  \sum_{n=1}^\infty \Iverson{f_n - f \notin \cap_{i=1}^m V(x_i, k_i)} 
  \leq \sum_{i=1}^m \sum_{n=1}^\infty \Iverson{f_n - f \notin V(x_i, k_i)}
  = \infty.
\end{equation}
%
It is now obvious that % 
%
\begin{equation}
  \sum_{n=1}^\infty \bigl[f_n - f \notin V(x_i, k_i)\bigr] = \infty
\end{equation}
%
for some $i$, which shows that $\singleton{f_n(x_i)}$ does not converge to $f(x_i)$. Thus, $\tau$-convergence coincides with %
pointwise convergence on $X$. 
%
% % % % % % % % % % % % % % % % % % % % % % % % % % % % % % % % % % % % % % % % % % % % % % % % % % % % % % % % % % % % % % % %
% SECOND PART
% % % % % % % % % % % % % % % % % % % % % % % % % % % % % % % % % % % % % % % % % % % % % % % % % % % % % % % % % % % % % % % %
%
% % % % % % % % % % % % % % % % % % % % % % % % % % % % % % % % % % % % % % % % % % % % % % % % % % % % % % % % % % % % % % % %
% First proof: Use the given hint
% % % % % % % % % % % % % % % % % % % % % % % % % % % % % % % % % % % % % % % % % % % % % % % % % % % % % % % % % % % % % % % %
\par\noindent\textbf{Proof with the given hint.}
We now prove the second part by constructing a specific sequence $\singleton{f_n}$ that satisfies both (a) and (b). %
Indeed, the hint suggests that there exists a bijection %  
%
\begin{align}
  \phi: \set[\Big]{\theta_n}{\theta_n \xrightarrow{n\infty} 0} & \to [0, 1] \\
          (\theta_1, \dots, \theta_n, \dots) & \mapsto x. \nonumber
\end{align} 
% 
We set %
%
\begin{equation}
  f_n(x) \Def \theta_n \quad \bigl(x = \phi(\theta_1, \dots, \theta_n, \dots)\bigr)
\end{equation}
%
so that $\singleton{f_n}$ tends pointwise to $0$. Note that, with this construction, the following  %
%
\begin{equation}
  x_\gamma %
  \Def \phi\Bigl(1 / \sqrt{1 + \magnitude{\gamma_1}}, \dots, 1 / \sqrt{1 + \magnitude{\gamma_n}}, \dots\Bigr) %
\end{equation}
%
outputs %
%
\begin{equation}
  \gamma_n f_n(x_\gamma) = \gamma_n / \sqrt{1 + \magnitude{\gamma_n}} %
  \tendsto{n}{\infty} \infty
\end{equation}
%
when $\gamma_n \to \infty$. This proves (b), since $\singleton{\gamma_n f_n(x_\gamma)}$ diverges. We now give an alternative %
construction of $\singleton{f_n}$. This second proof requires no cardinality argument. 
% 
% % % % % % % % % % % % % % % % % % % % % % % % % % % % % % % % % % % % % % % % % % % % % % % % % % % % % % % % % % % % % % % %
% Second proof: No hint, the hard way.
% % % % % % % % % % % % % % % % % % % % % % % % % % % % % % % % % % % % % % % % % % % % % % % % % % % % % % % % % % % % % % % %
\par\noindent\textbf{Proof with binary expansions (no hint)}
We rely on the following assertion: Every irrational number has a binary expansion that is not eventually periodic. %
% 
More precisely, there exists a bijective sum  %
%
%\def\bin{\text{\fw sum}}
\def\bin{\sigma}
\begin{align}
  \bin:  \set[\Big]{\beta \in \{0, 1\}^{\N_+}}{\beta \text{ is not eventually periodic}} & \to [0, 1] \setminus \Q \\
        (\beta_1, \dots, \beta_n, \dots) & \mapsto \sum_{k=1}^\infty \beta_k 2^{\minus k}. \nonumber
\end{align}
%
A suitable $\singleton{f_n}$ can be defined as follows: %
%
\begin{equation}\label{definition of f_n(alpha)}
  f_n(x) \Def \begin{cases} 
    2^{\minus (\beta_1 + \cdots + \beta_n)} & \bigr(x = \bin(\beta_1, \dots, \beta_n, \dots) \notin \Q) \\
    0 & (x \in \Q).
  \end{cases} 
\end{equation}
%
Indeed, we note that every bit stream $\bin^{\minus 1}(x)$ has infinitely many $1$'s, which implies that %
$f_n(x) \xrightarrow{n\infty} 0$. Next, pick an arbitrary $\gamma_n \to \infty$. Thus, for any positive integer $k$, %
$\gamma_n > 4^k$ for all sufficiently large $n$, say $n > N_k$. We select $n_k > N_k$ so large that %
%
\begin{equation}\label{definition of n_k}
  n_{k+1} - n_k > k.
\end{equation}
%
The crucial point is that the sequence $1_{\{n_1, n_2, \dots \}}$ is not eventually periodic. Moreover, the particular choice %
%
\begin{equation}
  \beta^{\gamma} \Def 1_{\{n_1, n_2, \dots \}}
\end{equation}
%
implies %
%
\begin{equation}\label{p sum of bits}
  \beta^\gamma_1 + \dots + \beta^\gamma_{n_1} + \dots + \beta^\gamma_{n_k} = k.
\end{equation}
%
Finally, (\ref{definition of f_n(alpha)}) and (\ref{p sum of bits}) together yield %
%
\begin{equation}
  \gamma_{n_k} f_{n_{k}}(\bin(\beta^\gamma))
    = {\gamma_{n_k}}/{2^k} 
    > 2^k 
    \tendsto{k}{\infty}\infty.
\end{equation}
%
In conclusion, every sequence of scalars $\gamma_n$ such that $\gamma_n \to \infty$ contains a subsequence %
$\singleton{\gamma_{n_k}}$ that causes $\singleton{\gamma_{n_k}f_{n_k}}$ to diverge. This is (b). 
\end{proof}
%
% END
