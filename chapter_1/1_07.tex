%\section{Exercise 7. Metrizability \& number theory}
\textit{
Let be X the vector space of all complex functions on the unit interval 
$[0, 1]$, topologized by the family of seminorms 
%
  \begin{align}
    p_{x}(f)=|f(x)| \quad (0\leq x\leq 1).\nonumber
  \end{align}
%
This topology is called the topology of pointwise convergence. 
Justify this terminology.
Show that there is a sequence $\{f_n\}$ in X such that (a) $\{f_n\}$ converges 
to $0$ as $n \to\infty$, but (b) if $\{γ_n\}$ is any sequence of scalars such 
that $γ_n\to\infty$ then $\{γ_nf_n\}$ does not converge to $0$. 
(Use the fact that the collection of all complex sequences converging to $0$ 
has the same cardinality as $[0, 1]$.)
This shows that metrizability cannot be omited in (b) of Theorem 1.28.
}
\begin{proof}
Our justification consists in proving that $\tau$-convergence and pointwise 
convergence are the same one. 
%
To do so, remark first that the family of the seminorms $p_{x}$ is separating.
By [1.37], the collection $\mathscr{B}$ of all finite intersections 
of the sets 
%
  \begin{align}
    V_{x, k} 
      \Def 
    \singleton{p_x < 2^{\minus k}} 
      \quad 
    (x \in [0, 1], k \in \N)
  \end{align}
%
is therefore a local base for a topology $\tau$ on $X$. Given 
%
  $\set{f_n}{\counting{n}}$, 
%
we put
%
\newcommand\off[1]{\function{off}(#1)}
%
\begin{align}
  \off{U} \Def \sum_{n=1}^\infty [f_n \notin U] \quad (U\in\tau),
\end{align}
%
with the convention %$\off{U}
%
  ``$\Sigma=\infty$'' %
%
whether the sum has no finite support. 
So, 
%
  \begin{align}
    %
    \label{Inequality boolean series}
    %
    \sum_{i=1}^m \off{U_{i}} 
      = 
    \sum_{n=1}^\infty \sum_{i=1}^m [f_n \notin U_{i}]
      \geq 
    \off{U_{1} \cuts \cdots \cuts U_{m}}.
  \end{align}
%
We first assume that $\singleton{f_n}$ $\tau$-converges to some $f$ in $X$, \ie
%
  \begin{align}
    \off{f+V} < \infty \quad(V \in \mathscr{B}).
  \end{align}
%
The special cases %
%
  $V_{x, 1}, V_{x, 2}, \dots, $ %
%
mean the pointwise convergence %
%
  $f_n(x) \overset{n\infty}{\longrightarrow} f(x)$. %
%
Conversely, assume that $\singleton{f_n}$ does not $\tau$-converges to any $g$ 
in $X$, \ie 
%
  \begin{align}
    %
    \label{Divergence}
    %
    \forall g \in X, \exists W \in \localbase{B}: 
      \off{g+W} = \infty. 
    %
  \end{align}
%
Given $g$, such $W$ is, by definition,  a finite intersection
%
  $
    V_{x_{1}, k_{1}} \cap \cdots \cap 
    V_{x_{m}, k_{m}} 
  $.
%
%So, (\ref{1.7 Inequality boolean series}) implies 
Thus,
%
  \begin{align}
    \sum_{i=1}^m \off{g + V_{x_{i}, k_{i}}} 
      %
        \citegeq{\ref{Inequality boolean series}} 
      %
    \off{g + W} 
    % 
      \citeq{\ref{Divergence}} 
    %
    \infty .
  \end{align}
%
One of the sum $\off{g + V_{x_{i}, k_{i}}}$ must then be $\infty$. 
In other words, there exists a point $x_i$ for which $\singleton{f_n(x_i)}$ %
does not converge to $g(x_i)$. %
$g$ being arbitrary, we so conclude that $f_n$ does not converge pointwise. %
We have just proved that 
%
  $\tau$-convergence is a rewording of pointwise convergence.
%
% SECOND PART
We now prove the second part.
%
From now on, we let %
%
  $\varit{k}$, $\varit{n}$ and $\varit{p}$ 
%
run on $\N_+$, as $\dy{x}$ denotes the usual dyadic expansion of $x$, %
so that $\dy{x}$ is an aperiodic binary sequence \iif $x$ is irrational. 
%
Define
%
\begin{align}
  %
  \label{f_n(x) definition}
  %
  f_n(x) 
    \Def 
  \begin{cases}
      \exp_2\left({\minus\sum_{k= 1}^{n} {\dy{x}_{\minus k}}}\right) 
        & 
          (x \in [0, 1]\setminus \Q )\\
      0 
        & 
          (x \in [0,1]\cap \Q), 
    \end{cases}
\end{align}
%
so that $f_n(x) \overset{n\infty}{\longrightarrow} 0$, %
%
and take % 
%
  $\gamma_n \overset{n\infty}{\longrightarrow} \infty$, \ie 
%
  at fixed $p$, $\gamma_{n}$ is greater than $2^{p}$ for almost all $n$.
%
Next, choose $n_{p}$ among those \textit{almost all} $n$ that are 
large enough to satisfy 
%
  \begin{align}
    n_{p-1} - n_{p-2} < n_{p}- n_{p-1} 
  \end{align}
%
(start with $n_{\minus 1} = n_{0} = 0$) and so obtain 
%
  \begin{align}
    2^p < \gamma_{n_{p}}:\, 
    %
      0< n_{p} - n_{p-1}\tendsto{p}{\infty} \infty.
    %
  \end{align}
%
The indicator $\chi$ of 
%
  $\{n_{1}, n_{2}, \dots\}$ in $\Z$
%
is then aperiodic, \ie 
%
  \def\xgamma{\alpha_{\gamma}}
  \begin{align}
    \xgamma 
      \Def
    \sum_{k=1}^\infty \chi_k 2^{\minus k} \in [0, 1]\setminus \Q .
      %\notin \Q
  \end{align}
%
Hence, $\chi$ is not a the infinite-support expansion of a rational number; %
which forces  %
%
  \begin{align}
    \dy{\xgamma}_{\minus k} &= \chi_{k}.
  \end{align}
%
The key ingredient is that %
%
  \begin{align}
    \chi_1 + \cdots + \chi_{n_{p}} = p\,.
  \end{align}
%
Combined with (\ref{f_n(x) definition}), it yields %
%
  \begin{align}
    f_{n_{p}}(\xgamma) = 2^{\minus p}.
  \end{align}
%
Finally,
%
  \begin{align}
    \gamma_{n_{p}} f_{n_{p}}(\xgamma) > 1.
  \end{align}
%
There so exists $\singleton{\gamma_{n_p}}$ such that %
%
  $\singleton{\gamma_{n_p} f_{\gamma_{n_p}}}$ %
%
fails to converge pointwise to $0$. %
In other words, (b) holds, which is in violent contrast with %
\citeresultFA{1.28}: $X$ is therefore not metrizable. So ends the proof.
\end{proof}
% END
