% % % % % % % % % % % % % % % % % % % % % % % % % % % % % % % % % % % % % % % % % % % % % % % % % % % % % % % % % % % % % % % %
% FunctionalAnalysis 
% 1_07.tex
% 
% encoding: UTF-8 
% EOL: LF
%
% format: LaTeX
% indent: spaces (2)
% width: 127
% % % % % % % % % % % % % % % % % % % % % % % % % % % % % % % % % % % % % % % % % % % % % % % % % % % % % % % % % % % % % % % %
\textit{%
Let $X$ be the vector space of all complex functions on the unit interval $[0, 1]$, topologized by the family of seminorms %
%
\begin{equation}
  p_x(f)= \magnitude*{f(x)} \qquad (0 \leq x \leq 1).\nonumber
\end{equation}
%
This topology is called the topology of pointwise convergence. Justify this terminology.}

\textit{Show that there is a sequence $\singleton{f_n}$ in $X$ such that (a) $\singleton{f_n}$ converges to $0$ as %
$n \to \infty$, but (b) if $\singleton{\gamma_n}$ is any sequence of scalars such that $\gamma_n \to \infty$ then %
$\singleton{\gamma_n f_n}$ does not converge to $0$. (Use the fact that the collection of all complex sequences converging to %
$0$ has the same cardinality as $[0, 1]$.) This shows that metrizability cannot be omitted in (b) of Theorem 1.28.
}
%

% % % % % % % % % % % % % % % % % % % % % % % % % % % % % % % % % % % % % % % % % % % % % % % % % % % % % % % % % % % % % % % %
% First part: Justification of the terminology
% % % % % % % % % % % % % % % % % % % % % % % % % % % % % % % % % % % % % % % % % % % % % % % % % % % % % % % % % % % % % % % %
\begin{proof}
\textbf{Justifying the terminology.}
% % % % % % % % % % % % % % % % % % % % % % % % % % % % % % % % % % % % % % % % % % % % % % % % % % % % % % % % % % % % % % % %
% A. tau-convergence => pointwise convergence. 
% % % % % % % % % % % % % % % % % % % % % % % % % % % % % % % % % % % % % % % % % % % % % % % % % % % % % % % % % % % % % % % %
Since the family of seminorms $p_x$ is separating, the collection $\mathcal{B}$ of all finite intersections of $V$ %
that are of the form %
%
\begin{equation}
  V(x,k) \Def \singleton*{p_x < 2^{\minus k}} 
  \qquad (x \in [0, 1], k=1, 2, 3, \dots)
\end{equation}
%
is a local base for a topology $\tau$ on $X$, see Theorem \citeFA{1.37}. Additionally, note that the total number of %
occurrences when $f_n \in \cup_{i=1}^m S_i$ is bounded as follows:
%
\begin{equation}
  %
  \label{1.7:inequality-boolean-series}
  %
  \sum_{n=1}^\infty \Iverson[\big]{f_n \in \cup_{i=1}^m S_i}
  \leq 
  \sum_{n=1}^\infty \sum_{i=1}^m \Iverson[\big]{f_n \in S_i} 
  = 
  \sum_{i=1}^m \sum_{n=1}^\infty \Iverson[\big]{f_n \in S_i};
\end{equation}
%
see [\ref{Iverson-notation}] for Iverson bracket notation. Now assume that $\singleton{f_n}$ $\tau$-converges to some $f$. %
By definition, %
%
\begin{equation}
  %
  \sum_{n=1}^\infty \Iverson[\big]{f_n - f \in X \setminus W} < \infty \qquad (W \in \mathcal{B}).
\end{equation}
%
In the special case where $W = V(x, k)$, we have, for a fixed $k$, $\magnitude*{f_n(x)- f(x)} < 2^{\minus k}$ for all but %
finitely many $n$. In other words, $\singleton{f_n(x)}$ converges to $f(x)$. %
% % % % % % % % % % % % % % % % % % % % % % % % % % % % % % % % % % % % % % % % % % % % % % % % % % % % % % % % % % % % % % % %
% B. tau-divergence => pointwise divergence. 
% % % % % % % % % % % % % % % % % % % % % % % % % % % % % % % % % % % % % % % % % % % % % % % % % % % % % % % % % % % % % % % %
Conversely, assume that $\singleton{f_n}$ does not converge in $\tau$: For each $f$ there exists an intersection of finitely %
many $V(x_1, k_1), \dots, V(x_m, k_m) \in \mathcal{B}$ such that %
%
\begin{equation}
  %
  \label{1.7:divergence}
  \sum_{n=1}^\infty \Iverson[\big]{f_n - f \notin \cap_{i=1}^m V(x_i, k_i)}
  = \sum_{n=1}^\infty \Iverson[\big]{f_n - f \in \cup_{i=1}^m X \setminus V(x_i, k_i)}
  = \infty 
\end{equation}
%
Substituting $f_n - f$ for $f_n$ in \eqref{1.7:inequality-boolean-series}, with $S_i = X \setminus V(x_i, k_i)$, yields 
%
\begin{equation}
  \sum_{n=1}^\infty \Iverson[\big]{f_n - f \in \cup_{i=1}^m X \setminus V(x_i, k_i)} 
  \leq \sum_{i=1}^m \sum_{n=1}^\infty \Iverson[\big]{f_n - f \in X \setminus V(x_i, k_i)}
  = \infty.
\end{equation}
%
As a result, there exists $i \in \{1, \dots, m\}$ such that 
%
\begin{equation}
  \sum_{n=1}^\infty \Iverson[\big]{f_n - f \in X \setminus V(x_i, k_i)} = \infty.
\end{equation}
%
This shows that $\singleton{f_n(x_i)}$ does not converge to $f(x_i)$. Thus, $\tau$-convergence coincides with pointwise %
convergence on $X$. We prove the second part in two different ways. The first proof uses the suggested cardinality argument; %
the second one provides a construction based on binary expansions.

% % % % % % % % % % % % % % % % % % % % % % % % % % % % % % % % % % % % % % % % % % % % % % % % % % % % % % % % % % % % % % % %
% SECOND PART
% % % % % % % % % % % % % % % % % % % % % % % % % % % % % % % % % % % % % % % % % % % % % % % % % % % % % % % % % % % % % % % %
%
% % % % % % % % % % % % % % % % % % % % % % % % % % % % % % % % % % % % % % % % % % % % % % % % % % % % % % % % % % % % % % % %
% First proof: Use the given hint
% % % % % % % % % % % % % % % % % % % % % % % % % % % % % % % % % % % % % % % % % % % % % % % % % % % % % % % % % % % % % % % %
\textbf{Proof with the given hint.} We prove the second part by constructing a specific sequence $\singleton{f_n}$ that %
satisfies both (a) and (b). The hint suggests that there exists a bijection %  
%
\begin{align}
  \phi: \set[\big]{\theta}{\theta \in \C^{\N_+}, \lim_{\infty} \theta = 0} & \to [0, 1] \\
          (\theta_1, \dots, \theta_n, \dots) & \mapsto x. \nonumber
\end{align} 
% 
We set
%
\begin{equation}
  f_n(x) \Def \theta_n \qquad \bigl(x = \phi(\theta_1, \dots, \theta_n, \dots)\bigr), 
\end{equation}
%
so that $\singleton{f_n}$ tends pointwise to $0$. Note that, with this construction, the following  %
%
\begin{equation}
  x_\gamma %
  \Def \phi\bigl(1 / \sqrt{1 + \magnitude*{\gamma_1}}, \dots, 1 / \sqrt{1 + \magnitude*{\gamma_n}}, \dots\bigr) 
    \qquad (\gamma_n \tendsto{n}{\infty} \infty)
\end{equation}
%
outputs %
%
\begin{equation}
  \gamma_n f_n(x_\gamma) = \gamma_n / \sqrt{1 + \magnitude*{\gamma_n}} %
  \tendsto{n}{\infty} \infty. 
\end{equation}
%
This proves (b) because $\singleton[\big]{\gamma_n f_n(x_\gamma)}$ diverges. We now construct a second $\singleton{f_n}$ %
without assuming the cardinality of $\set{\theta}{\theta \in \C^{\N_+}, \lim_{\infty} \theta = 0}$.

% % % % % % % % % % % % % % % % % % % % % % % % % % % % % % % % % % % % % % % % % % % % % % % % % % % % % % % % % % % % % % % %
% Second proof: No hint, the hard way.
% % % % % % % % % % % % % % % % % % % % % % % % % % % % % % % % % % % % % % % % % % % % % % % % % % % % % % % % % % % % % % % %
\textbf{Proof with binary expansions (no hint).} We rely on the fact that every irrational number has a binary expansion %
that is not eventually periodic. More precisely, there exists a bijective map
%  
\begin{align} 
  \sigma:  \set[\big]{\beta}{\beta \in \{0, 1\}^{\N_+}, \beta \text{ not eventually periodic}} 
    & \to \closedInterval*{0}{1} \setminus \Q \\
        (\beta_1, \dots, \beta_n, \dots) & \mapsto \sum_{k=1}^\infty \beta_k 2^{\minus k}. \nonumber
\end{align}
%
A suitable $\singleton{f_n}$ is defined as follows on $\closedInterval*{0}{1}\setminus \Q$: %
%
\begin{equation}\label{1.7:definition-of-f-n-alpha}
  f_n(\alpha) \Def 2^{\,\minus (\beta_1 + \cdots + \beta_n)} 
    \qquad \bigl(\alpha = \sigma(\beta_1, \dots, \beta_n, \dots) \in \closedInterval*{0}{1} \setminus \Q \bigr).\\
\end{equation}
%
The definition of $f_n$ on $\closedInterval*{0}{1}\setminus\Q$, say $\restriction{f_n}{\closedInterval*{0}{1}\setminus\Q}= 0$, 
is of no importance. The key ingredient is that every bit stream $\sigma^{\minus 1}(\alpha)$ has infinitely many $1$'s %
(which implies that $f_n(\alpha) \to 0$, since $\beta_1 + \cdots + \beta_n \to \infty$). Pick an arbitrary %
$\gamma_n \to \infty$: For each $k\in \N_+$, choose $m_k$ such that $\gamma_n > 4^k$ for all $n > m_k$. Next, select %
$n_k > m_k$ large enough to additionally satisfy %
%
\begin{equation}\label{1.7:definition-of-n-k}
  n_{k+1} - n_k > k.
\end{equation}
%
The crucial point is that the sequence $\beta^\gamma = 1_{\{n_1, n_2, \dots \}}$ is not eventually periodic. Hence %
%
\begin{equation}
  \sigma(\beta^{\gamma}) \in \closedInterval*{0}{1} \setminus \Q.
\end{equation}
%
Moreover, a straightforward induction on $k$ shows that  %
%
\begin{equation}\label{1.7:partial-sum-of-beta}
  \beta^\gamma_1 + \dots + \beta^\gamma_{n_1} + \dots + \beta^\gamma_{n_k} = k.
\end{equation}
%
Finally, \eqref{1.7:definition-of-f-n-alpha} and \eqref{1.7:partial-sum-of-beta} together yield %
%
\begin{equation}
  \gamma_{n_k} f_{n_{k}}\bigl(\sigma(\beta^\gamma)\bigr)
    = \gamma_{n_k} \cdot 2^{\,\minus k}
    > 2^k 
    \tendsto{k}{\infty}\infty.
\end{equation}
%
In conclusion, every sequence of scalars $\singleton{\gamma_n}$ such that $\gamma_n \to \infty$ contains a subsequence %
$\singleton{\gamma_{n_k}}$ such that $\singleton{\gamma_{n_k}f_{n_k}}$ does not converge pointwise to $0$, which proves (b).
\end{proof}
%
% END
