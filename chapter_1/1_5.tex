\textit{%
Consider the definition of ``bounded set'' given in Section 1.6. %
Would the content of this definition be altered if it merely %
required that to every neighborhood $V$ of 0 corresponds }some{ %
\textit{%
$t>0$ such that $E\subset tV$?
}
\begin{proof}
The answer is: No. To prove this, start from (a) of Section 1.14: %
$V$ contains $W$, a balanced neighborhood of $0$. %
%
Assume that $E$ is bounded in this weaker sense, \ie,  %
there exists a positive $t$ that satisfies %
%
\begin{equation}%
  E\subset tW.
\end{equation}
%
Thus, 
%
\begin{equation} 
  E\subset tW \subset sW \subset sV \qquad (s>t), 
\end{equation}
%
since $W$ is balanced. Thus, we recover the definition given in Section 1.6: %
The two definitions are equivalent.
\end{proof}
% END
