\textit{%
Consider the definition of ``bounded set'' given in Section 1.6. %
Would the content of this definition be altered if it were required merely %
required that to every neighbourhood $V$ of 0 corresponds }some{ %
\textit{%
$t>0$ such that $E\subseteq tV$?
}
\begin{proof}
The answer is: No. To prove it, start from (a) of Section 1.14: %
$V$ contains $W$, a balanced neighbourhood of $0$. %
%
Assume that $E$ is bounded in this weaker sense, \ie %
there exists a positive $t$ that satisfies %
%
\begin{align}%
  E\subseteq tW.
\end{align}
%
Thus, 
%
\begin{align} 
  E\subseteq tW \subseteq sW \subseteq sV \quad (s>t), 
\end{align}
%
since $W$ is balanced. We so reach the definition given in Section 1.6: %
The two ones are equivalent.
\end{proof}
% END
