%:1
\renewcommand{\labelenumi}{(\alph{enumi})} 
\textit{Suppose $X$ is a vector space. All sets mentioned below are understood 
  to be subsets of $X$. Prove the following statements from the axioms 
  as given in section 1.4.
\begin{enumerate}
\item{If $x,\,y\in X$ there is a unique $z\in X$ such that $x+z=y$.}
\item{ $0\cdot x=0=\alpha\cdot 0 \qquad (\alpha\in\C, x\in X)$.}
\item{ $2A\subset A+A$.}
\item{ $A$ is convex if and only if $(s+t)A=sA+tA$ %
  for all positive scalars $s$ and $t$.}
\item{ Every union (and intersection) of balanced sets is balanced.}
\item{ Every intersection of convex sets is convex.}
\item{ If $\Gamma$ is a collection of convex sets that is totally ordered by 
  set inclusion, then the union of all members of $\Gamma$ is convex.}
\item{ If $A$ and $B$ are convex, so is $A+B$.}
\item{ If $A$ and $B$ are balanced, so is $A+B$.}
\item{ Show that parts (f\,), (g) and (h) hold with subspaces in place of 
  convex sets.}
\end{enumerate}
}
%
\begin{proof}
\begin{enumerate}
%: (a)
\item Such a property only depends on the group structure of $X$: Each $x$ in
$X$ has an additive inverse $\minus x$. Let $x'$ be any additive inverse of $x$, so that
${x-x=0}=x+x'$. %
Thus, $\minus x +x -x =  \minus x + x + x' $, %
which is equivalent to $\minus x = x'$. Therefore, the inverse $\minus x$ is unique.
%
It is now clear that $x+z=y$ \IFF $z=\minus x +y$, %
which asserts both the existence and the uniqueness of $z$.
%: (b)
\item Remark that %
%
\begin{align}\label{1-1-b-0x}
  0\cdot x & =(0+0)\cdot x=0\cdot x+0\cdot x \\
           & =(0+0)\cdot x=0 +0\cdot x 
\end{align} 
%
then conclude from (a) that $0\cdot x=0$. So, %
\begin{align} \label{inverse of x}
  0=0\cdot x=(1-1)\cdot x &=x+(\minus 1)\cdot x
  \Rightarrow \minus 1\cdot x= \minus x.
\end{align}
%
Finally, %
%
\begin{equation}
  \alpha\cdot 0\overset{(\ref{inverse of x})}{=}
  \alpha\cdot (x+(\minus 1\cdot x))
  = \alpha \cdot x + \alpha \cdot (\minus 1) \cdot x 
  = (\alpha-\alpha )  \cdot x =0\cdot x = 0,
\end{equation}
%
which proves (b).
%
%: (c)
\item Remark that 
%
\begin{equation}
  2x =(1+1) x = x + x
\end{equation}
%
for every $x$ in $X$, and so conclude that %
%
\begin{equation}\label{double lies in sum}
  2A = \{2x: x\in A \} 
  = \{x + x: x \in A \} 
  \subset \{ x + y : (x,\,y) \in A^2 \} 
  = A+A
\end{equation}
%
for all subsets $A$ of $X$; which proves (c). %
%: (d)
\item If $A$ is convex, then %
%
\begin{equation}
  A \subset \frac{s}{s+t} A + \frac{t}{s+t} A \subset A;
\end{equation}
%
which is %
%
\begin{equation}\label{Exercise-1-d}
  sA + tA = (s+t)A.
\end{equation}
%
Conversely, the special case $s+t=1$ is %
%
\begin{equation}
  sA + (1-s)A = A.
\end{equation}
%
The latter extends to $s=0$, since %
%
\begin{equation}
  0A + A \overset{(b)}{=}\{0\}+A=A.
\end{equation}
%
The extension to $s=1$ is analogously established %
(or simply use the fact that $+$ is commutative!).
So ends the proof. %
%: (e)
\item Let $A$ range over $B$ a collection of balanced subsets, so that %
%
\begin{equation}
  \alpha \bigcap B \subset  \alpha A \subset A \subset \bigcup B 
\end{equation}
%
for all scalars $\alpha$ of magnitude $\leq 1$. %
The inclusion $\alpha \bigcap B \subset A$ establishes the first part. %
Now remark that %
%
\begin{equation}
  \alpha A  \subset \bigcup {B} 
\end{equation}
%
implies %
%
\begin{equation}
  \alpha \bigcup {B} \subset \bigcup {B};
\end{equation}
which completes the proof. %
%
%: (f)
\item Let $A$ range over $C$ a collection of convex subsets, so that %
%
\begin{equation}
  (s+t) \bigcap C \subset s\bigcap C + t\bigcap C \subset  sA + tA 
  \overset{(d)}{\subset} (s+t)A
\end{equation}
%
for all positives scalars $\mathit{s}$, $\mathit{t}$. %
Inclusions at both extremities force %
%
\begin{equation}
  s\bigcap C  + t\bigcap C = (s+t) \bigcap C.
\end{equation}
%
We then conclude from (d) that the intersection of $C$ is convex. %
So ends the proof.
%: (g)
\item We dismiss all trivial cases %
%
  $\Gamma = \emptyset$, %
  $\{ \emptyset\}$, %
  $\{\{x\}\}$, %
  $\{\emptyset$, %
  $\{x\}\}$ %
%
then pick $x_1, x_2$ in $\bigcup \Gamma$, 
so that each $x_i$ ($i=1, 2)$ lies in some $C_i \in \Gamma$. %
%
Since $\Gamma$ is totally ordered by set inclusion, we henceforth assume %
without loss of generality that $C_1$ is a subset of $C_2$. %
%
So, $x_1, x_2$ are now elements of the convex set $C_2$. %
Every convex combination of our $x_i$'s is then in %
$C_2 \subset \bigcup \Gamma$. Hence (g). %
%
%: (h)
\item Simply remark that 
%
\begin{equation}
  s (A+B) + t (A+B) = s A+ t A +s B +t B = (s+t)(A+B)
\end{equation}
%
for all positive scalars $\mathit{s}$ and $\mathit{t}$, %
then conclude from (d) that $A + B$ is convex. %
%: (i) 
\item Given any $\alpha$ from the closed unit disc, %
%
\begin{equation}
\alpha(A+B)=\alpha A+ \alpha B \subset A+B. 
\end{equation}
%
This completes the proof: $A+B$ is balanced. %
%: (j)
\item The proof is based on Lemma [\ref{annex:vector-subspaces-as-convex-balanced-sets}]. Let $\Gamma$ be a collection of %
vector subspaces of $X$. Define $I = \bigcap \Gamma$ and $U = \bigcup \Gamma$. The intersection $I$ is convex and balanced by %
(e) and (f), because every $Y \in \Gamma$ is convex and balanced. Next, observe that
%
\begin{equation}
  I + I \subset Y + Y \subset  Y
\end{equation}
%
for all $Y \in \Gamma$. Thus,
%
\begin{equation}
  I + I \subset I.
\end{equation}
%
It now follows from the implication $(b) \then (a)$ of Lemma~[\ref{annex:vector-subspaces-as-convex-balanced-sets}] that $I$ %
is a vector subspace of $X$. We now prove the counterpart of (g) when $\Gamma$ is totally ordered by set inclusion. Combining %
(e) with (g) demonstrates that $U$ is convex and balanced. To show that $U$ is a vector subspace, we note that this total %
ordering of $\Gamma$ implies
%
\begin{equation}
  Y_1 + Y_2  \subset \max(Y_1, Y_2),
\end{equation}
%
where $Y_1$ and $Y_2$ run through $\Gamma$. Hence%
%
\begin{equation}
  U + U \subset U.
\end{equation}
%
It then follows from the implication $(b) \then (a)$ of Lemma [\ref{annex:vector-subspaces-as-convex-balanced-sets}] that $U$ %
is a vector subspace of $X$. To prove the counterpart of (h), let each $Y_i$ be a vector subspace of $X$. Taken together, %
(h) and (i) imply that $Y_1 + Y_2$ is convex and balanced. Moreover, 
%
\begin{equation}
  (Y_1 + Y_2) + (Y_1 + Y_2) = (Y_1 + Y_1) + (Y_2 + Y_2) \subset Y_1 + Y_2.
\end{equation}
% 
Finally, we conclude from $(b) \then (a)$ of Lemma [\ref{annex:vector-subspaces-as-convex-balanced-sets}] that $Y_1 + Y_2$ %
is a vector subspace of $X$. %
\end{enumerate}
\end{proof}
% END
