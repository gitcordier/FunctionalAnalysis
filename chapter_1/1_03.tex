% % % % % % % % % % % % % % % % % % % % % % % % % % % % % % % % % % % % % % % % % % % % % % % % % % % % % % % % % % % % % % % %
% FunctionalAnalysis 
% 1_03.tex
% 
% encoding: UTF-8 
% EOL: LF
%
% format: LaTeX
% indent: spaces (2)
% width: 127
% % % % % % % % % % % % % % % % % % % % % % % % % % % % % % % % % % % % % % % % % % % % % % % % % % % % % % % % % % % % % % % %
\textit{
Let be X as topological vector space. All sets mentioned below are understood to be the subsets of X. Prove the following %
statements:
\begin{enumerate}
\item The convex hull of every open set is open.
\item If X is locally convex then the convex hull of every bounded set is bounded.
\item If A and B are bounded, so is A+B.
\item If A and B are compact, so is A+B.
\item If A is compact and B is closed, then A+B is closed.
\item The sum of two closed sets may fail to be closed.
\end{enumerate}
}
%: (a)
\begin{proof}%
\textbf{Proof of (a).} Let $A$ be open, so that 
%
\begin{equation}
  \co(A) \subset \bigcup %
    \set[\Bigg]{\sum_{i=1}^n t_i V_i}{V_i \text{ open subset of } A, \sum_{i=1}^n t_i = 1, t_i > 0, n > 0} 
  \subset \co(A).
\end{equation}
%
We know from \citeFA{1.7} that every $\sum_i t_i V_i$ is open, which proves (a). %
%
%:(b)
\paragraph{Proof of (b).} Provided a bounded set $E$, pick $V$ a neighborhood of $0$: By \citeFA{1.14 (b)}, $V$ %
contains a convex neighborhood of $0$, say $W$. It follows that there exists a positive scalar $s$ such that
%
\begin{equation}
  E \subset tW \subset tV \qquad (t>s). 
\end{equation}
%
Hence %
%
\begin{equation}
  \co(E) \subset \co(tW) = t\co(W) = t W \subset tV, 
\end{equation}
%
which completes the proof. %
%
%:(c)
\paragraph{Proof of (c).} At fixed $V$, neighborhood of the origin, we combine the continuity of $+$ with \citeFA{1.14} %
to conclude that there exists $U$, a balanced neighborhood of the origin such that %
%
\begin{equation}
  U+U\subset V. 
\end{equation}
%
Moreover, by the very definition of boundedness, $A \subset r\,U$ for some positive $r$. Similarly, $B \subset s \, U$ for %
sufficiently large $s > 0$. Finally, 
%
\begin{equation}
  A+B \subset  r\,U + s\,U \subset t\,U + t\,U \subset t\,V \qquad \bigl(t > \max(r, s)\bigr), 
\end{equation}
%
since $U$ is balanced. So ends the proof. %
%
%:(d)
\paragraph{Proof of (d).} First, $A$ and $B$ are compact: So is $A\times B$. %
Next, $+$ maps continuously $A\times B$ onto $A+B$. %
In conclusion, $A+B$ is compact. %
%
%:(e)
\paragraph{Proof of (e).} From now on, we assume that neither $A$ nor $B$ is empty, %
since otherwise the result is trivial.  %
Now pick $c\in X$ outside $A+B$: %
The result will be established by showing that $c$ is not in the closure %
of $A+B$. \\
\\
To do so, we let the variable $\mathit{a}$ range over $A$: %
Every set $a+B$ is closed as well, see \citeFA{1.7}. %
%
Since $a+B \cap \singleton{c} = \emptyset$, there exists $V=V(a)$ a neighborhood of the origin such that %
%
\begin{equation}\label{separation}
  (a+B + V) \cap (c+V) = \emptyset;
\end{equation}
%
see \citeFA{1.10}. 
Moreover, there are finitely many $a+V$, say $a_1 + V_1, a_2 + V_2, \dots$, %
whose union $U$ contains the compact set $A$. Therefore, %
%
\begin{equation}\label{U + B encloses A + B}
  A+B \subset U + B.
\end{equation}
%
Now define %
\begin{equation}
  W \Def V_1 \cap V_2 \cap \cdots, 
\end{equation}
%
so that 
%
\begin{equation}
  (a_i + B + V_i) \cap (c + W) \overset{(\ref{separation})}{=} \emptyset %
  \qquad (i = 1, 2, \dots).
\end{equation}
%
In conclusion, $c$ is not in the closure of $U+B$. %
Finally, (\ref{U + B encloses A + B}) asserts that %
$c$ is not in $\overline{A+B}$ either; which achieves the proof. \\
\\
\textbf{Corollary}: If $B$ is the closure of a set $S$, then %
%
\begin{equation}
  A+B \subset \overline{A+S} \subset \overline{A+B} = A + B
\end{equation}
%
by \citeFA{(b) of 1.13} (since $A$ is closed; %
see Section 1.12, from the same source). %
The special case $A = \{x\}$, $B=X$ %
will occur in the proof of Exercise 15 in chapter 2. %
%
%:(f)
\paragraph{Proof of (f).} The final proof consists of exhibiting a counterexample. To do so, let $f$ be any continuous mapping of the %
real line such that %
%
\begin{enumerate}[label=(\roman*)]
  \item $f(x) + f(\minus x) \neq 0$ \qquad ($x \in \R$);
  \item $f$ vanishes at infinity. 
\end{enumerate}
For instance, we may combine (ii) with $f$ even and $f>0$ by setting %
%
  $f(x) = 2^{\minus |x|}$, %
  $f(x) = e^{\minus x^2}$, %
  $f(x) = 1/(1+|x|)$, \dots, %
%
and so on. \\
\\
As a continuous function, $f$ has closed graph $G$, see \citeFA{2.14}. %
%
Moreover, (i) implies that the origin %
%
  $(0, 0) \neq \left(x-x, f(x)+ f(\minus x)\right)$ %
% 
is not in $G+G$. %
%
On the other hand, 
%
\begin{equation}
  \{ \left(0, f(n) + f(\minus n)\right): n=1, 2, \dots\} \subset G + G.
\end{equation}
%

%
Now the key ingredient is that % 
%
\begin{equation}
  \left(0, f(n)+f(\minus n)\right) \overset{(ii)}{\tendsto{n}{\infty}} (0, 0). 
\end{equation}
%
We have so constructed a sequence in $G+G$ that converges outside $G+G$. %
So ends the proof.
%
\end{proof}
%END 
% 
