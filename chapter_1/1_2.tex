\textit{The convex hull of a set $A$ in a vector space $X$ is the set of all %
convex combinations of members of $A$, that is the set of all sums %
%
  $t_1 x_1 +\cdots +t_n x_n$ %
%
in which $x_i \in A,\, t_i \geq 0$, $\sum t_i = 1$; $n$ is arbitrary. 
%
Prove that the convex hull of a set $A$ is convex and that is the intersection 
of all convex sets that contain $A$.}
%
\begin{proof} The convex hull of a set $S$ will be denoted by $\co{S}$. %
Remark that $S \subset \co{S}$ %
(to see that, take $t_1 = 1$ for each $x_1$ in $S$) and that %
$\co{A} \subset \co{B}$ where $A \subset B$ (obvious).
\\
Our proof will directly derive from the following lemma, 
\renewcommand{\labelenumi}{(\roman{enumi})} 
\begin{quote}
\textit{Let $S$ be a subset of a vector space $X$: Its convex hull $\co{S}$ %
is convex and the following statements %
\begin{enumerate}
  \item $S$ is convex; %
  \item{
    %
      $s_1 S + \dots + s_n S = (s_1 + \cdots + s_n) S$ %
      %
      for all positive scalar variables $\mathit{s_1}, \dots, \mathit{s_n}$; %
  }
  \item{
    %
    $\mathit{t_1} S + \dots + \mathit{t_n} S = S$ %
    %
    for all positive scalar variables $\mathit{s_1}, \dots, \mathit{s_n}$ %
    such that %
    %
    $s_1 + \cdots + s_n = 1$; %
  }
  \item $\co{S} = S$
\end{enumerate}
are equivalent.}
\end{quote}
%
More specifically, %
%
our proof of the second part will only depend on (i) $\Rightarrow$ (iv). \\
\\
From now on, we skip the trivial case $S=\emptyset$ %
then only consider nonempty sets. %
To prove the first part, let $\mathit{a}$, $\mathit{b}$ %
run through the convex combination(s) of $S$, so that %
%
  $a = t_1 x_1 + \cdots + t_n x_n$ and %
  $b = t_{n+1} x_{n+1} + \cdots + t_{n+p} x_{n+p}$ %
%
for some $(\mathit{t_i}, \mathit{x_i})$. %
%
Every sum $sa + (1-s)b $ ($0\leq s \leq 1$) is then a convex combination of %
$\mathit{x_1}, \dots, \mathit{x_{n+p}}$, since %
%
\begin{align}
  sa + (1-s)b = \sum_{i=1}^n st_i x_i + \sum_{i=n+1}^{n+p} (1-s)t_i x_i 
\end{align}
%
and
\begin{align}
  \sum_{i=1}^n st_i + \sum_{i=n+1}^{n+p} (1-s)t_i &= 
  s\sum_{i=1}^n t_i +(1-s) \sum_{i=n+1}^{n+p} t_i  = 1.
\end{align}
%
In terms of sets $S$, this reads %
\begin{align}
  s\co{S} + (1-s)\co{S} \subset \co{S}; 
\end{align}
which was our fist goal. %
We now aim at the equivalence %
%
(i) $\Rightarrow$ $\cdots$ $\Rightarrow$ (iv) $\Rightarrow$ (i): %
%
An easy proof by induction makes the implication (i) $\Rightarrow$ (ii) %
directly come from (d) of the above exercise 1, chapter 1. %
%
(iii) is a special case of (ii),
and the implication (iii) $\Rightarrow$ (iv) derives from the definition of %
the convex hull. %
%
We now close the chain with (iv) $\Rightarrow$ (i), %
by remarking that $S$ is convex whether $S = \co{S}$. %
%  
The lemma being proved, let us establish the second part. %
To do so, start from $F\ni\co{A}$ then possibly enrich $F$ the following way: %
\begin{align}
  B \in F \Rightarrow B \text{ is convex and contains }A.
\end{align}
Note that our definition of $F$ is weaker than the primary assumption %
``[$F$ only encompasses] \textit{all convex sets that contain A}", %
which is the special case %
%
\begin{align}
  B \in F \Leftrightarrow B \text{ is convex and contains }A.
\end{align}
%
In any case, the key ingredient is that $\co{A} \in F$ implies %
%
\begin{align}
 \co{A} \supset \bigcap_{B \in F} B.
\end{align}
%
Conversely, the next formula %
%
\begin{align}
  \co{A} \subset \co{B} \overset{(i) \Rightarrow (iv)}{=} B \quad (B \in F) 
\end{align}
%
is valid and implies %
\begin{align}
  \co{A} \subset \bigcap_{B \in F} B. 
\end{align}
%
So ends the proof
\end{proof}
%