% % % % % % % % % % % % % % % % % % % % % % % % % % % % % % % % % % % % % % % % % % % % % % % % % % % % % % % % % % % % % % % %
% FunctionalAnalysis 
% 1_02.tex
% 
% encoding: UTF-8 
% EOL: LF
%
% format: LaTeX
% indent: spaces (2)
% width: 127
% % % % % % % % % % % % % % % % % % % % % % % % % % % % % % % % % % % % % % % % % % % % % % % % % % % % % % % % % % % % % % % %
\textit{The convex hull of a set $A$ in a vector space $X$ is the set of all convex combinations of members of $A$, that is %
the set of all sums $t_1 x_1 +\cdots +t_n x_n$ in which $x_i \in A,\, t_i \geq 0$, $\sum t_i = 1$; $n$ is arbitrary. %
Prove that the convex hull of a set $A$ is convex and that it is the intersection of all convex sets that contain $A$.}
%
\begin{proof}
% % % % % % % % % % % % % % % % % % % % % % % % % % % % % % % % % % % % % % % % % % % % % % % % % % % % % % % % % % % % % % % %
% First property: co(A) is convex
% % % % % % % % % % % % % % % % % % % % % % % % % % % % % % % % % % % % % % % % % % % % % % % % % % % % % % % % % % % % % % % %
With this notation, for all $s$ in the unit interval, the weighted sum
%
\begin{equation}
  s \sum_{i=1}^{n} t_i x_i + (1-s) \sum_{j=1}^{n'} t'_j x'_j = \sum_{i=1}^{n}  s t_i x_i + \sum_{j=1}^{n'} (1-s) t'_j x'_j 
\end{equation}
%
lies in $\co(A)$, because
% 
\begin{equation}
  \sum_{i=1}^{n} st_i + \sum_{j=1}^{n'} (1-s)t'_j = s \sum_{i=1}^{n} t_i +(1-s) \sum_{j=1}^{n'} t'_j  = 1.
\end{equation}
%
This shows that 
%
\begin{equation}
  s\co(A) + (1-s)\co(A) \subset \co(A) \qquad \bigl(s \in \closedInterval*{0}{1}\bigr). 
\end{equation}
%
In other words, $\co(A)$ is convex. To prove the intersection property, we establish the chain $(i)\to(ii)\to(iii)\to(iv)$ %
from the following properties
% % % % % % % % % % % % % % % % % % % % % % % % % % % % % % % % % % % % % % % % % % % % % % % % % % % % % % % % % % % % % % % %
% Second property: co(A) is the intersection of all convex sets that contain A.
% % % % % % % % % % % % % % % % % % % % % % % % % % % % % % % % % % % % % % % % % % % % % % % % % % % % % % % % % % % % % % % %
%
\begin{enumerate}[label=(\roman*)]
  \item{$A$ is convex.}
  \item{$s_1 A + \dots + s_n A = (s_1 + \cdots + s_n)\, A$ for all positive $s_i$, for all $n$.}
  \item{$t_1 A + \dots + t_n A = A$ for all positive $t_i$ ($t_1 + \cdots + t_n = 1)$, for all $n$.}
  \item{$\co(A) = A$.}
\end{enumerate}
%
A straightforward proof by induction shows that $(i) \to (ii)$, since \eqref{Exercise-1-d} gives the base case $n=2$. Next, %
(iii) is a special case of (ii), and $(iii) \to (iv)$ follows from the definition of the convex hull. Finally, consider
% 
\begin{equation} 
  \Gamma \Def \set{B}{A \subset B,\,B \text{ convex }}.
\end{equation} 
% 
We must prove that $\co(A) = \bigcap \Gamma$. Observe that
%
\begin{equation}\label{1.2:intersection-subsetof-co-A}
 \bigcap \Gamma \subset \co(A), 
\end{equation}
%
since $\co(A) \in \Gamma$. Conversely, applying $(i) \to (iv)$ with $B \in \Gamma$ playing the role of $A$ shows that
%
\begin{equation}\label{1.2:co-A-subsetof-intersection}
  \co(A) \subset \co(B) = B, 
\end{equation}
as $A \subset B$ implies $\co(A) \subset \co(B)$. In conclusion, \eqref{1.2:co-A-subsetof-intersection} and %
\eqref{1.2:intersection-subsetof-co-A} together yield
% 
\begin{equation}
  \co(A) \subset \bigcap \Gamma \subset \co(A), 
\end{equation} 
%
which completes the proof. In addition, observe that the convexity of $\co(A)$ conversely gives $(iv) \to (i)$.
\end{proof}
% END
% 
