% % % % % % % % % % % % % % % % % % % % % % % % % % % % % % % % % % % % % % % % % % % % % % % % % % % % % % % % % % % % % % % %
% FunctionalAnalysis 
% 1_02.tex
% 
% encoding: UTF-8 
% EOL: LF
%
% format: LaTeX
% indent: spaces (2)
% width: 127
% % % % % % % % % % % % % % % % % % % % % % % % % % % % % % % % % % % % % % % % % % % % % % % % % % % % % % % % % % % % % % % %
\textit{The convex hull of a set $A$ in a vector space $X$ is the set of all convex combinations of members of $A$, that is %
the set of all sums $t_1 x_1 +\cdots +t_n x_n$ in which $x_i \in A,\, t_i \geq 0$, $\sum t_i = 1$; $n$ is arbitrary. %
Prove that the convex hull of a set $A$ is convex and that it is the intersection of all convex sets that contain $A$.}
%
\begin{proof} Our proof relies on the following equivalent properties:%
%
\renewcommand{\labelenumi}{(\roman{enumi})} 
\begin{enumerate}
  \item{$A$ is convex.}
  \item{$s_1 A + \dots + s_n A = (s_1 + \cdots + s_n)\, A$ when $s_i > 0$.}
  \item{$t_1 A + \dots + t_n A = A$ when $t_i > 0$ and $\sum t_i = 1$.}
  \item{$\co(A) = A$.}
\end{enumerate}
\renewcommand{\labelenumi}{(\alph{enumi})} 
%
% % % % % % % % % % % % % % % % % % % % % % % % % % % % % % % % % % % % % % % % % % % % % % % % % % % % % % % % % % % % % % % %
% First property: co(A) is convex
% % % % % % % % % % % % % % % % % % % % % % % % % % % % % % % % % % % % % % % % % % % % % % % % % % % % % % % % % % % % % % % %
Let $y$ stand for $t_1 x_1 + \dots + t_n x_n$. If $s \in \closedInterval*{0}{1}$, then $sy + (1-s)y' \in \co(A)$, because
%
\begin{equation}
  sy + (1-s)y' = st_1 x_1 + \cdots  + st_n x_n  + (1-s)t'_1 x'_1 + \cdots + (1-s)t'_{n'} \,x'_{n'}
\end{equation}
%
and 
% 
\begin{equation}
  \sum_i st_i + \sum_i (1-s)t'_i = s \sum_i t_i +(1-s) \sum_i t'_i  = 1.
\end{equation}
%
This shows that 
%
\begin{equation}
  s\co(A) + (1-s)\co(A) \subset \co(A) \qquad (0 \leq s \leq 1). 
\end{equation}
%
In other words, $\co(A)$ is convex (which implies $(iv) \to (i)$). We now prove that $(i)\to(ii)\to(iii)\to(iv)$ to establish %
the intersection property as a corollary of $(i) \to (iv)$. 

% % % % % % % % % % % % % % % % % % % % % % % % % % % % % % % % % % % % % % % % % % % % % % % % % % % % % % % % % % % % % % % %
% Second property: co(A) is the intersection of all convex sets that contain A.
% % % % % % % % % % % % % % % % % % % % % % % % % % % % % % % % % % % % % % % % % % % % % % % % % % % % % % % % % % % % % % % %
A proof by induction shows that $(i) \to (ii)$, since (d) of [1.1] gives the base case $n=2$; see \eqref{Exercise-1-d}. %
Next, (iii) is a special case of (ii), and $(iii) \to (iv)$ derives from the definition of the convex hull. Finally, consider
% 
\begin{equation} 
  \Gamma \Def \set{B}{A \subset B,\,B \text{ convex }}.
\end{equation} 
% 
We must prove that $\co(A) = \bigcap \Gamma$. Observe that
%
\begin{equation}
 \bigcap \Gamma \subset \co(A), 
\end{equation}
%
since $\co(A) \in \Gamma$. Conversely, the implication $(i) \to (iv)$, with $B\in \Gamma$ substituted for $A$, yields
%
\begin{equation}
  \co(A) \subset \co(B) = B
\end{equation}
because $A \subset B \then \co(A) \subset \co(B)$. In conclusion,
% 
\begin{equation}
  \bigcap \Gamma \subset \co(A) \subset B, 
\end{equation} 
%
where $B$ runs through $\Gamma$. This completes the proof.
\end{proof}
% END
