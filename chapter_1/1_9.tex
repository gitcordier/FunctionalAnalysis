\textit{Suppose
\begin{enumerate}
  \item $X$ and $Y$ are topological vector spaces,
  \item $\Lambda: X\to Y$ is linear.
  \item $N$ is a closed subspace of $X$,
  \item$\varit{\pi}: X\to X/N$ is the quotient map, and
  \item $\Lambda \varit{x}=0$ for every $\varit{x}\in N$.
\end{enumerate}
Prove that there is a unique $f:X/N\to Y$ which satisfies 
%
  $\Lambda = \mathit{f\circ \pi}$, 
%
that is, 
%
  $\Lambda \varit{x}= \mathit{f(\pi (x))}$ for all $\mathit{x\in X}$. 
%
Prove that $\varit{f}$ is linear and that $\Lambda$ is continuous %
%
if and only if %
%
  $\varit{f} $ is continuous. 
%
Also, $\Lambda$ is open if and only if $\varit{f}$ is open.
}
%
\begin{proof} Bear in mind that 
%
  $\pi$ continuously maps $X$ onto the topological (Hausdorff) space $X/N$, 
  since $N$ is closed (see \citeFA{1.41}).
%
Moreover, the equation 
% 
  $\Lambda = f \circ \pi$ 
% 
has necessarily a unique solution, which is the binary relation 
%
  \begin{equation}\label{1.9. definition of f.}
     f \Def\set{(\pi x, \Lambda x)}{x \in X} \subset X/N \times Y.
    \end{equation}
%
To ensure that $f$ is actually a mapping, simply remark that 
the linearity of $\Lambda$ implies 
%
  \begin{equation}
    %\forall ( x,  x') \in X^2: 
    %
    \Lambda x \neq \Lambda  x' \then \pi x' \neq \pi x'.
  \end{equation}
%
It straightforwardly derives from (\ref{1.9. definition of f.}) that 
$f$ inherits linearity from $\pi$ and $\Lambda$.\\
\\ 
{\bf Remark.} The special case 
%
  $N = \singleton{\Lambda = 0}$ , \ie,  $\Lambda x= 0$ \IFF $x\in N$ (\cf (e)), %
%
is the first isomorphism theorem in the topological spaces context. 
To see this, remark that this strengthening of (e) yields 
%
  \begin{equation}
    f(\pi x) = 0
      \citethen{\ref{1.9. definition of f.}}
    \Lambda x = 0
      \then 
    x \in N 
      \then 
    \pi x = N
\end{equation}
and so conclude that $f$ is also one-to-one.
%
%We have thus proved 
%
  %the first isomorphism theorem in the topological spaces context. %
%
\\\\
Now assume $f$ to be continuous. Then so is 
%
  $\Lambda = f\circ \pi $, 
% 
by \citeFA{1.41 (a)}. 
%
Conversely, 
%
if $\Lambda$ is continuous, then for each neighborhood $V$ of $0_Y$ 
there exists a neighborhood $U$ of $0_X$ such that
%
  \begin{equation}
    \Lambda(U) = f\left(\pi(U)\right) 
      \subset 
    V.
  \end{equation}
%
Since $\pi$ is open (\citeFA{1.41 (a)}), $\pi(U)$ is a neighborhood of 
%
  $N=0_{X/N}$: 
%s
This is sufficient to establish that the linear mapping $f$ is continuous.
%
If $f$ is open, so is $\Lambda = f\circ \pi$, by \citeFA{1.41 (a)}. 
%
To prove the converse, remark that 
%
  every neighborhood $W$ of $0_{X/N}$ satisfies %
%
  \begin{equation}
    W = \pi(V)
  \end{equation}
%
for some neighborhood $V$ of $0_X$. So, 
%
  \begin{equation}
    f(W) = f \left(\pi(V)\right) = \Lambda(V).
  \end{equation}
%
As a consequence, 
% 
  if $\Lambda$ is open, then $f(W)$ is a neighborhood of $0_Y$. %
%
So ends the proof.
\end{proof}


