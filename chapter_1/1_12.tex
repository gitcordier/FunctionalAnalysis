\textit{Suppose %
%
  $d_1(x,y) = |x-y|, d_2(x,y) = |\varphi(x) - \varphi(y)|$, %
%
where %
%
  $\varphi(x)={x}/{(1+\lvert x \rvert )}$. %
%
Prove that $d_1$ and $d_2$ are metrics on $\R$ which induce the same %
topology, although $d_1$ is complete and $d_2$ is not.
}
%
\begin{proof}%
First, each $d_i\, (i=1, 2)$ induces a topology $\tau_i$ %
whose open balls are all %
%
\begin{align}\label{1_12_2}
  B_i(a,r) \triangleq \{x \in \R: d_i (a, x) < r \} 
  \quad (a \in \R, r > 0 ).
\end{align}
%
Next, remark that the monotonically increasing mapping %
$\varphi: \R \to ]\minus 1, 1[$ is odd and that %
%
\begin{align}\label{1_12_1}
  \varphi(x) \tendsto{x}{\infty} 1.
\end{align}
%
$\varphi$ is therefore a $\tau_1$-homeomorphsim of $\R$ onto $]\minus 1, 1[$. %
%
A first consequence is that, at fixed $a \in \R$, given any positive scalar %
$\epsilon$, the $\tau_1$-continuousness of $\varphi$ supplies an open ball %
$B_1(a,\eta)$ on which $|\varphi(a)-\varphi|<\epsilon$. %
In terms of balls $B_i$, this reads as follows, %
%
\begin{align}
  B_1(a,\eta) \subset B_2(a,\epsilon).
\end{align}
%
The second consequence is that the $\tau_1$-continuousness of %
$\varphi^{\, \minus 1}$ yields similar inclusions %
%
\begin{align} \label{1_12_6}
\quad B_2(a, \epsilon') \subset B_1 (a, \eta')
\end{align}
%
provided $\eta'>0$. At arbitrary $\epsilon$, the special case $\eta' = \eta$ %
is the concatenation %
%
\begin{align}
  B_2(a, \epsilon') \subset B_1(a,\eta) \subset B_2(a,\epsilon); 
\end{align}
%
which proves that $\tau_1 =\tau_2$. %
%
Finally, all inequalities $n < i < j$ over $\N$ together yield %
%
\begin{align}
  d_2(i, j)=|\varphi(i)-\varphi(j)| \tendsto{n}{\infty} 0.
\quad \end{align}
%
The sequence $n=0, 1, 2, \dots$ is therefore $\tau_2$-Cauchy. We will %
nevertheless establish that it $\tau_2$-diverges. To do so, we start by %
offering the $\tau_2$-converge to some $\lambda$: %
The triangle inequality immediately dismiss that assumption, as follows, %
%
\begin{align}
  d_2(0, \lambda) \geq 
  d_2(0, n) - d_2(\lambda, n) = 
  \varphi(n) -d_2(\lambda, n)  \tendsto{n}{\infty} 1.
\end{align}
%
We then conclude that $d_2$ fails to be complete. 
\end{proof}
% END
