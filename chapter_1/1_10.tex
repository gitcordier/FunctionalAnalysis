% % % % % % % % % % % % % % % % % % % % % % % % % % % % % % % % % % % % % % % % % % % % % % % % % % % % % % % % % % % % % % % %
% FunctionalAnalysis 
% 1_10.tex
% 
% encoding: UTF-8 
% EOL: LF
%
% format: LaTeX
% indent: spaces (2)
% width: 127
% % % % % % % % % % % % % % % % % % % % % % % % % % % % % % % % % % % % % % % % % % % % % % % % % % % % % % % % % % % % % % % %
\textit{Suppose that X and Y are topological vector spaces, $\dim Y < \infty$, $\Lambda : X \to Y$ is linear, and %
$\Lambda(X) = Y$.
%
\begin{enumerate}
  \item{
    Prove that $\Lambda$ is an open mapping.}
  \item{
    Assume, in addition, that the null space of $\Lambda$ is closed, 
    and prove that $\Lambda$ is continuous.
  }
\end{enumerate}
%
}
%
\begin{proof}
Discard the trivial case $\Lambda = 0$ and assume that $\dim Y > 0$. From now on, $Y$ has a basis $B= \singleton{e, e', \dots}$ %
and $W$ is an arbitrary neighborhood of $0$. Since addition is continuous, there exists a balanced open $V\subset Y$ such that %
%
  \begin{equation}
    \label{definition of v}
    \sum_{e} V \subset W. 
  \end{equation}
%
Moreover, the surjectivity of $\Lambda$ allows us to pick $x_e$ with $\Lambda x_e = e$. This implies that %
every $y = \sum_e y_e e $ ($y_e \in \C$) can be expressed as %
%
\begin{equation}\label{1_10_sum}
  y = \sum_{e} y_e \Lambda x_e. 
\end{equation}
%
Note that $\set{x_e}{e\in B}$ is bounded (as a finite set). Hence %
%
\begin{equation}
  \set{x_e}{e \in B} \subset sV 
\end{equation}
%
for some $s > 0$. Combining this with (\ref{1_10_sum}) yields % 
%
  \begin{equation}\label{y in sum of lambda V}
    y \in \sum_e s y_e  \Lambda (V).
  \end{equation}
%
Using (\ref{definition of v}) and the balancedness of $\Lambda(V)$, we conclude that
%
  \begin{equation}
    y \in \sum_e \Lambda (V) \subset \Lambda(W)  \quad (\norma{\infty}{y} < 1/s). 
  \end{equation}
%
This proves (a) in the case where $Y = \C^n$ and $B$ is the standard basis. The general case now follows from %
[\ref{notations: vector spaces: finite-dimensional vector spaces}].\\\\
%
To prove (b), assume that the null space $\{\Lambda = 0\}$ is closed and let $f, \pi$ be as in Exercise 1.9, %
$\{\Lambda = 0\}$ playing the role of $N$. Since $\Lambda$ is onto, the first isomorphism theorem (see Exercise 1.9) asserts %
that $f$ is an isomorphism of $X/N$ onto $Y$. We now conclude with the help of %
[\ref{notations: vector spaces: finite-dimensional vector spaces}] that $f$ is a homeomorphism of $X/N$ onto $Y$. %
We have thus established that $f$ is continuous: So is $\Lambda = f\circ \pi$.
\end{proof}
% END
