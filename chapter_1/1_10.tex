\textit{Suppose that X and Y are topological vector spaces,
%
  $\dim Y < \infty$,
%
$\Lambda : X \to Y$ is linear, and $\Lambda(X) = Y$.
%
\begin{enumerate}
  \item{
    Prove that $\Lambda$ is an open mapping.}
  \item{
    Assume, in addition, that the null space of $\Lambda$ is closed, 
    and prove that $\Lambda$ is continuous.
  }
\end{enumerate}
%
}
%
\begin{proof}
Discard the trivial case $\Lambda = 0$ and assume that $\dim Y = n$ %
for some positive $n$. %
Let $e$ range over a basis of $B$ of $Y$ then pick in $X$ $W$ an arbitrary %
neighborhood of the origin: There so exists $V$ a balanced neighborhood of %
the origin of $X$ such that 
%
  \begin{align}
    \label{definition of v}
    \sum_{e} V \subseteq W, 
  \end{align}
%
since addition is continuous. Moreover, for each $e$, there exists $x_e$ %
in $X$ such that 
%
  $\Lambda(x_e)=e$, 
% 
simply because $\Lambda$ is onto: Given $y$ in $Y$, %
of $e$-component(s) $y_e$, we now obtain
%
\begin{align}\label{1_10_sum}
  y = \sum_{e} y_e \Lambda (x_e).
\end{align}
%
As a finite set, $\set{x_e}{e\in B}$ is bounded: There so exists a positive %
scalar $s$ such that  
%
\begin{align}
  \forall e\in B,  x_e \in s V.
\end{align}
%
Combining this with (\ref{1_10_sum}) shows that 
%
  \begin{align}
    \label{y in sum of lambda V}
    y \in \sum_e y_e \,s \Lambda (V).
  \end{align}
%
We now come back to (\ref{definition of v}) and so conclude that %
%
  \begin{align}
    y \in \sum_e \Lambda (V) \subseteq \Lambda(W) 
  \end{align}
%
for if $|y_e| < 1/s$; which proves (a); %
see \hyperref[finite dimensional spaces]{[finite dimensional spaces]}.\\\\
%
%
To prove (b), assume that the null space 
%
  $\{\Lambda = 0\}$ %
% 
is closed and let $f, \pi$ be as in Exercise 1.9,  %
%
  $\{\Lambda = 0\}$ %
%
playing the role of $N$. Since $\Lambda$ is onto, the first isomorphism %
theorem (see Exercise 1.9) asserts that $f$ is an isomorphism of $X/N$ %
onto $Y$. Consequently, 
%
\begin{align}
  \dim X/N= n.
\end{align}
% 
$f$ is then an homeomorphism of $X/N$ onto $Y$; %
see \hyperref[finite dimensional spaces]{[finite dimensional spaces]}.
We have thus established that $f$ is continuous: So is $\Lambda = f\circ \pi$.
\end{proof}
% END
