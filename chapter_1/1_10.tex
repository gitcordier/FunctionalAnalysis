% % % % % % % % % % % % % % % % % % % % % % % % % % % % % % % % % % % % % % % % % % % % % % % % % % % % % % % % % % % % % % % %
% FunctionalAnalysis 
% 1_10.tex
% 
% encoding: UTF-8 
% EOL: LF
%
% format: LaTeX
% indent: spaces (2)
% width: 127
% % % % % % % % % % % % % % % % % % % % % % % % % % % % % % % % % % % % % % % % % % % % % % % % % % % % % % % % % % % % % % % %
\textit{Suppose that X and Y are topological vector spaces, $\dim Y < \infty$, $\Lambda : X \to Y$ is linear, and %
$\Lambda(X) = Y$.
%
\begin{enumerate}
  \item{
    Prove that $\Lambda$ is an open mapping.}
  \item{
    Assume, in addition, that the null space of $\Lambda$ is closed, 
    and prove that $\Lambda$ is continuous.
  }
\end{enumerate}
%
}
%
\begin{proof}
Let $B=\singleton{e, e', \dots}$ be a basis for $Y$, and let $W\subset X$ be an arbitrary neighborhood of the origin. Since %
addition is continuous in $X$, there exists a balanced open $V$ such that
%
  \begin{equation}
    \label{definition of v}
    \sum_e V \subset W. 
  \end{equation}
%
Note that $\Lambda(V)$ is balanced as well. Moreover, the surjective $\Lambda$ provides a vector $x_e$ such that %
$\Lambda x_e = e$. Therefore, given the coordinate representation $\sum_e y_e e$ of $y\in Y$, we have
%
\begin{equation}\label{1_10_sum}
  y = \sum_e y_e e = \sum_e y_e \Lambda x_e. 
\end{equation}
%
Note that $\set{x_e}{e\in B}$ is bounded (as a finite set). Hence %
%
\begin{equation}
  \set{x_e}{e \in B} \subset sV 
\end{equation}
%
for some $s > 0$. Combining this with (\ref{1_10_sum}) yields % 
%
  \begin{equation}\label{y in sum of lambda V}
    y \in \sum_e s y_e  \Lambda (V).
  \end{equation}
%
Using (\ref{definition of v}) and the balancedness of $\Lambda(V)$, we conclude that $\magnitude*{y_e} < 1/s$ implies
%
  \begin{equation}
    y \in \sum_e \Lambda (V) \subset \Lambda(W) .
  \end{equation}
%
This establishes (a) for $Y = \C^n$ equipped with $\norm*[\infty]{\cdot}$, when $B$ is the standard basis. The general case %
$\dim{Y} = n$ now follows from [\ref{notations: vector spaces: finite-dimensional vector spaces}]. The case $Y=\singleton{0}$ %
is trivial. \par\noindent
%
To prove (b), assume that the null space $N = \singleton{\Lambda = 0}$ is closed and let $f, \pi$ be as in Exercise 1.9. %
Since $\Lambda$ is onto, the first isomorphism theorem (see Exercise 1.9) asserts that $f$ is an isomorphism of $X/N$ onto %
$Y$. By [\ref{notations: vector spaces: finite-dimensional vector spaces}], $f$ is also a homeomorphism. We have thus %
established that $f$ is continuous; so is $\Lambda = f\circ \pi$.
\end{proof}
% END
% 