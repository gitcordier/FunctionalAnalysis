%\section{1.10 Exercise 10. An open mapping theorem}
\textit{Suppose that X and Y are topological vector spaces,
%
  $\dim Y < \infty$,
%
$\Lambda : X \to Y$ is linear, and $\Lambda(X) = Y$.
%
  \begin{enumerate}
    \item{
      Prove that $\Lambda$ is an open mapping.}
    \item{
      Assume, in addition, that the null space of $\Lambda$ is closed, 
      and prove that $\Lambda$ is continuous.
    }
  \end{enumerate}
  %
}
%
\begin{proof}
We discard the trivial case $\dim Y = 0$ then henceforth assume that $\dim Y$ 
has positive dimension $n$. \\\\
%
Let $e$ range over a base of $Y$: For each $e$, 
there exists $x_e$ in $X$ such that 
%
  $\Lambda(x_e)=e$, 
% 
since $\Lambda$ is onto. So,
%
  \begin{align}\label{1_10_sum}
    y = \sum_{e} y_e \Lambda x_e \quad (y\in Y).
  \end{align}
%
The sequence $\singleton{x_e}$ is finite; therefore it is bounded: %
Given $V$ a balanced neighborhood of the origin, there exists a positive %
scalar $s$ such that  
%
  \begin{align}
    x_e \in s V \text{ for all } x_e.
  \end{align}
%
Combining this with (\ref{1_10_sum}) shows that 
%
  \begin{align}
    y \in \sum_e \Lambda (V) \quad (y\in Y: |y_e| < s^{\minus 1}), 
  \end{align}
%
which proves (a).\\\\
%
%
To prove (b), assume that the null space 
%
  $\singleton{\Lambda = 0}$ %
% 
is closed and let $f, \pi$ be as in Exercise 1.9, with 
%
  % Ker: \ket?
  $\singleton{\Lambda = 0}$ %
%
playing the role of $N$.
%%
% Isomorphism:
%  \begin{align}
%    & X \to                    X/N   \to                  Y . \\
%    & x \overset{\pi}{\mapsto} \pi x  \overset{f}{\mapsto} \Lambda x \nonumber
%  \end{align}
%
Since $\Lambda$ is onto, the first isomorphism theorem (see Exercise 1.9) 
asserts that 
%
  $f$ is an isomorphism of $X/N$ onto $Y$. 
%
Consequently, 
%
  \begin{align}
    \dim X/N= n.
  \end{align}
% 
$f$ is then an homeomorphism of 
%
  $X/N\equiv \C^{n}$ 
%
onto $Y$; see \citeresultFA{1.21}.
We have thus established that $f$ is continuous: So is $\Lambda = f\circ \pi$.
\end{proof}