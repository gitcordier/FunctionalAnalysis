\textit{Put $K =[0, 1]$ and define $\D_K$ as in Section 1.46. %
Show that the following three families of seminorms %
(where $n = 0, 1, 2, \dots$) define the same topology on $\D_K$. 
If $D = d/dx$: 
%
\begin{enumerate}
\item{
  $\| D^n f \|_\infty = \sup\set{\left| D^n f(x)\right|}{\infty< x< \infty}$
}
\item{
  $\| D^n f \|_1 =\int_0^1 \left|D^n f(\varit{x}) \right| d\varit{x}$
}
\item{
  $\| D^n f \|_2 = \left\{
    \int_0^1 | D^n f(\varit{x}) |^2  d\varit{x} 
  \right\}^{1/2}.$
}
\end{enumerate}
%
}
%
\begin{proof}%
Let us equipp $\D_K$ with the inner product %
%
  $\bra{f}\ket{g} = \int_0^1 f \, \bar{g}$, so that %
  $\bra{f}\ket{f} = \| f \|_2$. %
%
The following %
%
\begin{align}
  \int_0^1 1\,| D^n f| \leq \| 1 \|_2 \| D^n f \|_2 
\end{align}
%
is then a Cauchy-Schwarz inequality; see Theorem 12.2 of \cite{FA}. %
We so obtain %
%
\begin{align}\label{inequalities 1}
  \| D^n f \|_1 
    \leq 
  \| D^n f \|_2 
    \leq 
  \| D^n f \|_\infty 
    <\infty 
\end{align}
%
since $K$ has length $1$. %
%
Obviously, the support of $D^n f$ lies in $K$, hence the below equality %
%
\begin{align}\label{inequalities 2}
  |D^n f(x)|
    =
  \left|\int_{0}^x D^{n+1}f\right|
    \leq 
  \int_{0}^x |D^{n+1}f|
    \leq 
  \|D^{n+1}f \|_1.
\end{align}
%
Take the supremum over all $|D^n f(x)|$: Combining (\ref{inequalities 1}) %
with (\ref{inequalities 2}) now reads as follows, %
%
\begin{align}\label{inequalities 3}
  \|D^n f\|_1 
    \leq 
  \|D^{n} f\|_2 
    \leq 
  \|D^{n} f\|_\infty 
    \leq 
  \|D^{n+1}f\|_1 
    \leq 
  \cdots < \infty.
\end{align}
%
Finally, put %
%
\begin{align}
  V^{(i)}_n         & \triangleq \{f\in \D_K: \|f \|_i < 2^{\,\minus n}\}, \\
  \mathscr{B}^{(i)} & \triangleq \{V^{(i)}_n: n = 0, 1, 2, \dots\},
\end{align}
%
so that (\ref{inequalities 3}) is mirrored by neighborhood inclusions, %
provided $i=1, 2, \infty$:
%
\begin{align}\label{inclusions}
  V^{(1)}_n
    \supseteq  
  V^{(2)}_n 
    \supseteq  
  V^{(\infty)}_n 
    \supseteq  
  V^{(1)}_{n+1} 
    \supseteq  
  \cdots .
\end{align}
%
Their subchains $V^{(i)}_n\supseteq V^{(i)}_{n+1}$ turn $\mathscr{B}^{(i)}$ %
into a local base of a topology $\tau_i$. 
The whole chain (\ref{inclusions}) then forces %
%
\begin{align}
  \tau_1 \subseteq \tau_2 \subseteq \tau_\infty \subseteq \tau_1;
\end{align}
%
which achieves the proof.
\end{proof}
% END

