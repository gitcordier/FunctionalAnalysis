\textit{Put $K =[0, 1]$ and define $\D_K$ as in Section 1.46. 
Show that the following three families of seminorms 
  (where $n = 0, 1, 2, \dots$) define the same topology on $\D_K$. 
If $D = d/dx$: 
%
  \begin{enumerate}
    \item{
      $\| D^n f \|_\infty = \sup\set{\left| D^n f(x)\right|}{\infty< x< \infty}$
    }
    \item{
      $\| D^n f \|_1 =\int_0^1 \left|D^n f(\varit{x}) \right| d\varit{x}$
    }
    \item{
      $\| D^n f \|_2 = \left\{
        \int_0^1 | D^n f(\varit{x}) |^2  d\varit{x} 
      \right\}^{1/2}.$
    }
  \end{enumerate}
  %
}
%
\begin{proof} 
First, remark that  
%
  \begin{align}\label{1_14_2}
    \| D^n f \|_1 
      \leq 
    \| D^n f \|_2 
      \leq 
    \| D^n f \|_\infty 
      <\infty 
  \end{align}
%
holds, since $K$ has length $1$ 
(the inequality on the left is a Cauchy-Schwarz one). 
Next, that the support of $D^n f$ lies in $K$; which yields % 
  \begin{align}\label{1_14_4}
    \magnitude{ D^n f(x) }
      =
    \magnitude{ \int_{0}^x D^{n+1}f }
      \leq 
    \int_{0}^x \magnitude{D^{n+1}f }
      \leq 
    \|D^{n+1}f \|_1 .
  \end{align}
%
So, 
%
  \begin{align}\label{1_14_5}
    \| D^n f \|_\infty 
      \leq 
    \| D^{n+1} f \|_1 .
  \end{align}
%
We now combine (\ref{1_14_2}) with (\ref{1_14_5}) and so obtain %
%
  \begin{align}\label{1_14_6}
    \| D^n f \|_1 
      \leq 
    \| D^{n} f \|_2 
      \leq 
    \| D^{n} f \|_\infty 
      \leq 
    \| D^{n+1}f\|_1 
      \leq 
    \cdots \quad (\integers{n}).
  \end{align}
%
Put  
  \begin{align}\label{1_14_1}
    V\up{i}_n &       \Def \set{f\in \D_K}{\norm{i}{f} < 2^{\minus n} }
      \quad(i=1,2,\infty)\\
    \mathscr{B}\up{i}&\Def \set{V\up{i}_n }{\integers{n}},
  \end{align}
so that (\ref{1_14_6}) is mirrored in terms of neighborhood inclusions, 
as follows,
%
  \begin{align}\label{1_14_7}
    V\up{1}_n
      \supset 
    V\up{2}_n 
      \supset 
    V\up{\infty}_n 
      \supset 
    V\up{1}_{n+1} 
      \supset 
    \cdots .
  \end{align}
%
Since 
  $V\up{i}_n\supset V\up{i}_{n+1}$, 
$\mathscr{B}\up{i}$ is a local base of a topology $\tau_i$. 
But the chain (\ref{1_14_7}) forces %
%
  \begin{align}
    \tau_1 = \tau_2 = \tau_\infty.
  \end{align}
%
To see that, choose a set $S$ that is $\tau_1$-open at $f$, \ie 
%
  $V\up{1}_n \subset S-f$  
%
for some $n$. Next, concatenate this with %
%
  $V\up{2}_n \subset V\up{1}_n$ (see (\ref{1_14_7})) %
%
and so obtain  
%
  $V\up{2}_n \subset S-f$; 
%
which implies that $S$ is $\tau_2$-open at $f$.
Similarly, we deduce, still from (\ref{1_14_7}), that 
\begin{align}
  \tau_2\text{-open} 
    \then 
  \tau_\infty\text{-open} 
    \then 
  \tau_1\text{-open}.
\end{align}
So ends the proof.
\end{proof}





