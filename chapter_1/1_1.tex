%:1
\renewcommand{\labelenumi}{(\alph{enumi})} 
\textit{Suppose $X$ is a vector space. All sets mentioned below are understood 
  to be subsets of $X$. Prove the following statements from the axioms 
  as given as in section 1.4.
\begin{enumerate}
\item{If $x,\,y\in X$ there is a unique $z\in X$ such that $x+z=y$.}
\item{ $0\cdot x=0=\alpha\cdot 0 \quad (\alpha\in\field, x\in X)$.}
\item{ $2A\subset A+A$.}
\item{ $A$ is convex if and only if $(s+t)A=sA+tA$ %
  for all positive scalars $s$ and $t$.}
\item{ Every union (and intersection) of balanced sets is balanced.}
\item{ Every intersection of convex sets is convex.}
\item{ If $\Gamma$ is a collection of convex sets that is totally ordered by 
  set inclusion, then the union of all members of $\Gamma$ is convex.}
\item{ If $A$ and $B$ are convex, so is $A+B$.}
\item{ If $A$ and $B$ are balanced, so is $A+B$.}
\item{ Show that parts (f\,), (g) and (h) hold with subspaces in place of 
  convex sets.}
\end{enumerate}
}
%
\renewcommand{\labelenumi}{\arabic{enumi}.} 
\begin{proof}
\begin{enumerate}
%: (a)
\item Such property only depends on the group structure of $X$: Each $x$ in
$X$ has an opposite $\minus x$. Let $x'$ be any opposite of $x$, so that
${x-x=0}=x+x'$. %
Thus, $\minus x +x -x =  \minus x + x + x' $, %
which is equivalent to $\minus x = x'$. So is established the uniqueness of %
$\minus x $. %
%
It is now clear that $x+z=y$ \iif $z=\minus x +y$, %
which asserts both the existence and the uniqueness of $z$.
%: (b)
\item Remark that %
%
\begin{align}
  0\cdot x & =(0+0)\cdot x=0\cdot x+0\cdot x \\
           & =(0+0)\cdot x=0 +0\cdot x 
\end{align} 
%
then conclude from (a) that $0\cdot x=0$. So, %
\begin{align} \label{inverse of x}
  0=0\cdot x=(1-1)\cdot x &=x+(\minus 1)\cdot x
  \Rightarrow \minus 1\cdot x= \minus x.
\end{align}
%
Finally, %
%
\begin{align}
  \alpha\cdot 0\overset{(\ref{inverse of x})}{=}
  \alpha\cdot (x+(\minus 1\cdot x))
  = \alpha \cdot x + \alpha \cdot (\minus 1) \cdot x 
  = (\alpha-\alpha )  \cdot x =0\cdot x = 0,
\end{align}
%
which proves (b).
%
%: (c)
\item Remark that 
%
\begin{align}
  2x =(1+1) x = x + x
\end{align}
%
for every $x$ in $X$, and so conclude that %
%
\begin{align}\label{double lies in sum}
  2A = \{2x: x\in A \} 
  = \{x + x: x \in A \} 
  \subset \{ x + y : (x,\,y) \in A^2 \} 
  = A+A
\end{align}
%
for all subsets $A$ of $X$; which proves (c). %
%: (d)
\item If $A$ is convex, then %
%
\begin{align}
  A \subset \frac{s}{s+t} A + \frac{t}{s+t} A \subset A;
\end{align}
%
which is %
%
\begin{align}
  sA + tA = (s+t)A.
\end{align}
%
Conversely, the special case $s+t=1$ is %
%
\begin{align}
  sA + (1-s)A = A.
\end{align}
%
The latter extends to $s=0$, since %
%
\begin{align}
  0A + A \overset{(b)}{=}\{0\}+A=A.
\end{align}
%
The extension to $s=1$ is analogously established %
(or simply use the fact that $+$ is commutative!).
So ends the proof. %
%: (e)
\item Let $A$ range over $B$ a collection of balanced subsets, so that %
%
\begin{align}
  \alpha \bigcap B \subset  \alpha A \subset A \subset \bigcup B 
\end{align}
%
for all scalars $\alpha$ of magnitude $\leq 1$. %
The inclusion $\alpha \bigcap B \subset A$ establishes the first part. %
Now remark that %
%
\begin{align}
  \alpha A  \subset \bigcup {B} 
\end{align}
%
implies %
%
\begin{align}
  \alpha \bigcup {B} \subset \bigcup {B};
\end{align}
which achieves the proof. %
%
%: (f)
\item Let $A$ range over $C$ a collection of convex subsets, so that %
%
\begin{align}
  (s+t) \bigcap C \subset s\bigcap C + t\bigcap C \subset  sA + tA 
  \overset{(d)}{=} (s+t)A
\end{align}
%
for all positives scalars $\mathit{s}$, $\mathit{t}$. Thus, %
%
\begin{align}
  (s+t) \bigcap C \subset s\bigcap C  + t \bigcap C  \subset (s+t) \bigcap C.
\end{align}
%
We now conclude from (d) that the intersection of $C$ is convex. %
So ends the proof.
%: (g)
\item Pick $x_1, x_2$ in $\bigcup \Gamma$, 
so that each $x_i$ ($i=1, 2)$ lies in some $C_i \in \Gamma$. %
%
Since $\Gamma$ is totally ordered by set inclusion, we henceforth assume %
without loss of generality that $C_1$ is a subset of $C_2$. %
%
So, $x_1, x_2$ are now elements of the convex set $C_2$. %
Every convex combination of our $x_1$'s is then in %
$C_2 \subset \bigcup \Gamma $, hence (g). %
%: (h)
\item Simply remark that 
%
\begin{align}
  s (A+B) + t (A+B) = s A+ t A +s B +t B = (s+t)(A+B)
\end{align}
%
for all positive scalars $\mathit{s}$ and $\mathit{t}$, %
then conclude from (d) that $A + B$ is convex. %
%: (i) 
\item Given any $\alpha$ from the closed unit disc, %
%
\begin{align}
\alpha(A+B)=\alpha A+ \alpha B \subset A+B.
\end{align}
%
There is no more to prove. %
%: (j)
\item Our proof will be based on the following lemma, %
%
\renewcommand{\labelenumii}{(\roman{enumii})}
\textit{
\begin{quote}
If $\emptyset \neq S\subset X$, then any assertion 
\begin{enumerate}
\item $S$ is a vector subspace of $X$;
\item $S$ is convex balanced such that $S + S = S$;
\item $S$ is convex balanced such that $\lambda S=S\quad (\lambda > 0)$
\end{enumerate}
implies the other ones. 
\end{quote}
}
%
To prove the lemma, let $\mathit{S}$ %
run through all nonempty subsets of $X$. %
First, assume that (i) holds: Clearly, every $S$ is convex balanced. %
Moreover, $S+S \subset S$.  Conversely, $S = S + \{0\} \subset S + S $; %
which establishes (ii). %
%
Next, assume (only) (ii): A proof by induction shows that %
%
\begin{align}\label{induction nS}
  nS = (n-1)  S + S = S + S = S \quad (n=1,2,3, \dots)
\end{align}
%
with the help of (b) and (d). %
The special case $n = \lceil{1/\lambda}\rceil + \lceil{\lambda}\rceil$ ($\lambda > 0$) yields %
%
\begin{align}
  nS \overset{(\ref{induction nS})}{\subset} S 
  \subset n\,\lambda S 
  \subset  n^2 S, 
\end{align}
%
since $S$ is balanced and that $1 < n \,\lambda < n^2$. %
Dividing the latter inclusions by $n$ shows that %
%
\begin{align} 
  S \subset \lambda S \subset nS \overset{(\ref{induction nS})}{\subset} S,
\end{align}
%
which is (iii). Finally, dropping (ii) in favor of (iii) leads to %
%
\begin{align} 
  \alpha  S +\beta  S 
  \overset{(a)}{=} |\alpha | S + | \beta | S 
  \overset{(d)}{=} (|\alpha | + | \beta| )S 
  \overset{(iii)}{=} S 
  %
  \quad (|\alpha| + |\beta| > 0);
\end{align}
%
where the equality at the left holds as $S$ is balanced. %
%
Moreover (under the sole assumption that $S$ is balanced), %
this extends to $|\alpha| + |\beta| = 0$, as follows, %
%
\begin{align} 
  \alpha S + \beta S  = 0S + 0S\overset{(b)}{=} \{0\} 
  \overset{(b)}{=} 0S \subset S.
\end{align}
%
Hence (i), which achieves the lemma's proof. %
We will now offer a straightforward proof of (j). \\
\\
Let $V$ be a collection of vector spaces of $X$, %
of intersection $I$ and union $U$. 
%
First, remark that every member of $V$ is convex balanced: %
So is $I$ (combine (e) with (f)). %
%
Next, let $\mathit{Y}$ range over $V$, so that %
%
\begin{align}
  I + I \subset Y + Y \subset  Y; 
\end{align}
%
which yields
%
\begin{align}
  I + I \subset I. 
\end{align}
%
Conversely, 
\begin{align}
  I  = I + \{0\} \subset I + I.
\end{align}
%
It now follows from the lemma's (ii) $\Rightarrow$ (i) that %
$I$ is a vector subspace of $X$. %
%
Now temporarily assume that $S$ is totally ordered by set inclusion: %
Combining (e) with (g) establishes that $U$ is convex balanced. %
%
To show that $U$ is more specifically a vector subspace, %
we first remark that such total order implies that either %
$Z \subset Y$ or $Y \subset Z$, as $\mathit{Z}$ ranges over $V$. %
A straightforward consequence is that 
%
\begin{align}
  Y \subset Y + Z  \subset Y\cup Z.
\end{align}
%
Another one is that $Y \cup Z$ ranges over $V$ as well. %
Combined with the latter inclusions, this leads to %
%
\begin{align}
  U \subset U  + U \subset U.
\end{align}
%
It then follows from the lemma's (ii) $\Rightarrow$ (i) that %
$U$ is a vector subspace of $X$. %
%
Finally, let $\mathit{A},\mathit{B}$ run through the vector subspaces of $X$: %
Combining (h) with (i) proves that $A+B$ is convex balanced as well. %
%
Furthermore, %
%
\begin{align}
  A + B \overset{(i) \Rightarrow (ii)}{=} (A + A) + (B + B) = (A + B) + (A + B),
\end{align}
% 
where the equality at the right holds as $X$ is an abelian group. %
We now conclude from (ii) that any $A+B$ is a vector subspace of $X$. %
%
So ends the proof. %
\end{enumerate}
\end{proof}