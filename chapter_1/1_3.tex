\renewcommand{\labelenumi}{(\alph{enumi})} 
\textit{
Let be X as topological vector space. All sets mentioned below are understood to be the subsets of X. Prove the following statements:
\begin{enumerate}
\item The convex hull of every open set is open.
\item If X is locally convex then the convex hull of every bounded set is bounded.
\item If A and B are bounded, so is A+B.
\item If A and B are compact, so is A+B.
\item If A is compact and B is closed, then A+B is closed.
\item The sum of two closed sets may fail to be closed.
\end{enumerate}
}
%: (a)
\begin{proof}
\renewcommand{\labelenumi}{(\alph{enumi})} 
\begin{enumerate}
\item%
Pick an open set $A$ then let the variables $\mathit{V}_i$ %
($i=1, 2, \dots$) run through all open subsets of $A$, so that % 
%
\begin{align}
  \co{A} \subseteq 
  \bigcup_{t_i} \, %
    (t_1V_1 + \cdots + t_i V_i  + \cdots) 
  \subseteq \co{A}
\end{align}
%
given all convex combinations %
%
  $t_1V_1 + \cdots + t_i V_i  + \cdots $. %
 %
We know from Section 1.7 of \cite{FA} that those sums are open; %
which achieves the proof. %
%
%:(b)
\item Provided a bounded set $E$, %
pick $V$ a neighbourhood of $0$: By (b) of Section 1.14 in \cite{FA}, %
$V$ contains a convex neighbourhood of $0$, say $W$. %
%
There so exists a positive scalar $s$ such that
%
\begin{align}
  E \subseteq tW \subseteq tV \quad (t>s); 
\end{align}
%
which yields %
%
\begin{align}
  \co{E} \subseteq \co{tW} = t\co{W} = t W \subseteq tV.
\end{align}
%
So ends the proof. %
%
%:(c)
\item At fixed $V$, neighbourhood of the origin, %
we combine the continuousness of $+$ with Section 1.14 of \cite{FA} %
to conclude that there exists $U$ a balanced neighborhood of the origin %
such that %
%
\begin{align}
  U+U\subseteq V. 
\end{align}
%
Moreover, by the very definition of boundedness, %
$A \subseteq r U$ for some positive scalar $r$. % 
Similarly, $B \subseteq s U$ for some positive $s$. % 
%
Finally, 
%
\begin{align}
A+B \subseteq  rU + sU \subseteq tU + tU \subseteq tV \quad (t > r, s), 
\end{align}
%
since $U$ is balanced. So ends the proof. %
%
%:(d)
\item First, $A$ and $B$ are compact: So is $A\times B$. %
Next, $+$ maps continuously $A\times B$ onto $A+B$. %
In conclusion, $A+B$ is compact. %
%
%:(e)
\item From now on, we assume that neither $A$ nor $B$ is empty, %
since otherwise the result is trivial.  %
Now pick $c\in X$ outside $A+B$: %
The result will be established by showing that $c$ is not in the closure %
of $A+B$. \\
\\
To do so, we let the variable $\mathit{a}$ range over $A$: %
Every set $a+B$ is closed as well; see Section 1.7 of \cite{FA}. %
%
Trivially, $a+B \neq c$: By Section 1.10 of \cite{FA}, %
there so exists $V=V(a)$ a neighborhood of the origin such that %
%
\begin{align}\label{separation}
  (a+B + V) \cap (c+V) = \emptyset.
\end{align}
%
Moreover, there are finitely many $a+V$, say $a_1 + V_1, a_2 + V_2, \dots$, %
whose union $U$ contains the compact set $A$. Therefore, %
%
\begin{align}\label{U + B encloses A + B}
  A+B \subseteq U + B.
\end{align}
%
Now define %
\begin{align}
  W \triangleq V_1 \cap V_2 \cap \cdots, 
\end{align}
%
so that 
%
\begin{align}
  (a_i + B + V_i) \cap (c + W) \overset{(\ref{separation})}{=} \emptyset %
  \quad (i = 1, 2, \dots).
\end{align}
%
As a conclusion, $c$ is not in the closure of $U+B$. %
Finally, (\ref{U + B encloses A + B}) asserts that %
$c$ is not in $\overline{A+B}$ either; which achieves the proof. \\
\\
\textbf{Corollary}: If $B$ is the closure of a set $S$, then %
%
\begin{align}
  A+B \subseteq \overline{A+S} \subseteq \overline{A+B} = A + B
\end{align}
%
by (b) of Section 1.13 of \cite{FA} (since $A$ is closed; %
see Section 1.12, from the same source). %
The special case $A = \{x\}$, $B=X$ %
will occur in the proof of Exercise 15 in chapter 2. %
%
%:(f)
\item The last proof will consist in exibhiting a counterexample. %
To do so, let $f$ be any continuous mapping of the real line such that %
\renewcommand{\labelenumii}{(\roman{enumii})} 
\begin{enumerate}
  \item $f(x) + f(\minus x) \neq 0$ \quad ($x \in \R$);
  \item $f$ vanishes at infinity. 
\end{enumerate}
For instance, we may combine (ii) with $f$ even and $f>0$ by setting %
%
  $f(x) = 2^{\minus |x|}$, %
  $f(x) = e^{\minus x^2}$, %
  $f(x) = 1/(1+|x|)$, \dots, %
%
and so on. \\
\\
As a continous function, $f$ has closed graph $G$; see [2.14] of \cite{FA}. %
%
Moreover, (i) implies that the origin %
%
  $(0, 0) \neq \left(x-x, f(x)+ f(\minus x)\right)$ %
% 
is not in $G+G$. %
%
On the other hand, 
%
\begin{align}
  \{ \left(0, f(n) + f(\minus n)\right): n=1, 2, \dots\} \subseteq G + G.
\end{align}
%

%
Now the key ingredient is that % 
%
\begin{align}
  \left(0, f(n)+f(\minus n)\right) \overset{(ii)}{\tendsto{n}{\infty}} (0, 0). 
\end{align}
%
We have so constructed a sequence in $G+G$ that converges outside $G+G$. %
So ends the proof.
\end{enumerate}
\end{proof}
%END
