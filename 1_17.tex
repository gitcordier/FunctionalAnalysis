\textit{In the setting of Section 1.46, prove that 
  %
    $f \mapsto D^{\alpha}f$ 
  %
is a continuous mapping of 
%
  $C^{\infty}\left(\Omega\right)$ into 
  $C^{\infty}\left(\Omega\right)$ and also of 
  $\D_{K}$ into 
  $\D_{K}$, for every multi-index $\alpha$.
%
}
\begin{proof} 
In both cases, $D^\alpha$ is a linear mapping. 
It is then sufficient to establish continuousness at the origin.
%
We begin with the $C^\infty\left(\Omega\right)$ case. \\
\\
Let $U$ be an aribtray neighborhood of the origin.
There so exists $N$ such that $U$ contains
%
  \begin{align} 
    V_{N}= \left\{
      \phi \in C^\infty\left(\Omega\right): 
      \max\set{
        | D^\beta\phi (x) |
      }{
        \magnitude{ \beta } \leq N, x\in K_N
      }
    < 1/N
    \right\}.
  \end{align}
%
Now pick $g$ in $V_{N+\magnitude{\alpha}}$, so that
%
  \begin{align}
    \max
    \set{
      \magnitude{ D^\gamma g\left(x\right) }
    }{
      \magnitude{ \gamma | \leq N+| \alpha }, 
      x\in K_{N}
    }
    < \frac{1}{N}.
  \end{align}
%
(the fact that $K_N\subset K_{N+\magnitude{\alpha}}$ was tacitely used).
%
The special case $\gamma = \beta + \alpha$ yields
%‡
\begin{align}
    \max
    \set{
      | D^\beta D^\alpha g(x)|
    }{
      \magnitude{\beta} \leq N, 
      x\in K_{N}
    }
    < \frac{1}{N}.
  \end{align}
%
We have just proved that
%
  \begin{align}\label{1.17. inclusion}
    g \in V_{N + \magnitude{\alpha}}
      \then 
    D^\alpha g \in V_{N},
      \quad
      \ie
      \quad
    D^\alpha \left(V_{N+\magnitude{\alpha}}\right) \subset V_N.
  \end{align}
%
The continuity of 
  $D^{\alpha}: C^\infty \left(\Omega\right) \to C^\infty \left(\Omega\right)$ 
is so established. \\\\
%
%
To prove the continuousness of the restriction 
%
  $D^\alpha \lvert_{\D_K}: \D_K \to \D_K$, % 
%
%
we first remark the collection of the  
%
  ${V_N \cap \D_K}$ % 
%
is a local base of the subspace topology of $\D_K$.
%
%
  $V_{N+\alpha} \cap \D_K$ % 
%
is then a neighborhood of $0$ in this topology. %
Furthermore, 
%
  \begin{align}
    %
    D^\alpha \lvert_{\D_K} \left(V_{N+\magnitude{\alpha}} \cap \D_K\right) 
    % 
    & = 
      D^\alpha\left(V_{N+\magnitude{\alpha}} \cap \D_K\right) \\
      %
    & \subset
      D^\alpha\left(V_{N+\magnitude{\alpha}}\right) 
        \cap 
      D^\alpha\left(\D_K\right) \\
      %
    & \subset 
      V_N 
        \cap 
      \D_K
        %
          \quad (\text{see }\ref{1.17. inclusion})
        %
    %
  \end{align}
%
So ends the proof.
\end{proof}
