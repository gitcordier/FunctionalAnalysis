\textit{In the setting of Section 1.46, prove that 
  %
    $f \mapsto D^{\alpha}f$ 
  %
is a continuous mapping of 
%
  $C^{\infty}(\Omega)$ into 
  $C^{\infty}(\Omega)$ and also of 
  $\D_{K}$ into 
  $\D_{K}$, for every multi-index $\alpha$.
%
}
\begin{proof} 
In both cases, $D^\alpha$ is a linear mapping. 
It is then sufficient to establish continuousness at the origin.
%
We begin with the $C^\infty(\Omega)$ case. \\
\\
Let $U$ be an aribtray neighborhood of the origin.
There so exists $N$ such that $U$ contains
%
  \begin{align} 
    V_{N}= \set{
      \phi \in C^\infty(\Omega)
    }{
      \max
      \set{
        | D^\beta\phi(x) |
      }{
        | \beta | \leq N, x\in K_N
      }
    < 1/N
    }.
  \end{align}
%
Now pick $g$ in $V_{N+|\alpha|}$, so that
%
  \begin{align}
    \max
    \set{
      | D^\gamma g(x) |
    }{
      | \gamma | \leq N+| \alpha |, 
      x\in K_{N}
    }
    < \frac{1}{N}.
  \end{align}
%
(the fact that $K_N\subset K_{N+|\alpha|}$ was tacitely used).
%
The special case $\gamma = \beta + \alpha$ yields
%‡
\begin{align}
    \sup
    \set{
      | D^\beta D^\alpha g(x) |
    }{
      | \beta | \leq N, 
      x\in K_{N}
    }
    < \frac{1}{N}.
  \end{align}
%
We have just proved that
%
  \begin{align}
    g \in V_{N + | \alpha|}
      \then 
    D^\alpha g \in V_{N},
      \quad
      \ie
      \quad
    D^\alpha (V_{N+|\alpha|}) \subset V_N.
  \end{align}
%
The continuity of 
  $D^{\alpha}: C^\infty (\Omega) \to C^\infty (\Omega)$ 
is so established.
%%
%
We now prove the second part, \ie 
that $D^\alpha: \mathscr{D}_K \to \mathscr{D}_K$ is continuous. \\
\\
%
Let $r$ be a positive scalar. From now on, $V(r)$ will denote the 
preimage in $\D_K$ of the open disc $D(r)$  \ie 
%
\begin{align}
  V(r) \Def \set{f \in \D_K}{| D^\alpha f | < r}.
\end{align}
%
Similarly, we define $A(r)$ as the preimage of $D(r)$ in $C^\infty(\Omega)$, 
\ie 
%
\begin{align}
  A(r) \Def \set{f \in C^\infty (\Omega)}{| D^\alpha f | < r }.
\end{align}
%
The continuousness of 
%
  $D^\alpha$ in $C^\infty(\Omega)$ 
%
implies that $A(r)$ is open in $C^\infty(\Omega)$. Moreover, 
%
  $B(r) = A(r) \cap \ \D_K$. 
%
$B(r)$ is then open in $\D_K$, for all $r$. So ends the proof.
\end{proof}
