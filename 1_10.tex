%\section{1.10 Exercise 10. An open mapping theorem}
\textit{Suppose that X and Y are topological vector spaces,
%
  $\dim Y < \infty$,
%
$\Lambda : X \to Y$ is linear, and $\Lambda(X) = Y$.
%
  \begin{enumerate}
    \item{
      Prove that $\Lambda$ is an open mapping.}
    \item{
      Assume, in addition, that the null space of $\Lambda$ is closed, 
      and prove that $\Lambda$ is continuous.
    }
  \end{enumerate}
  %
}
%
\begin{proof}
  \begin{enumerate}
    \item{
      Let $e$ range over a base of $Y$: 
      For each $e$, there exists $x_e$ in $X$ such that 
      %
        $\Lambda(x_e)=e$, 
      % 
      since $\Lambda$ is onto.So,
      %
        \begin{align}\label{1_10_sum}
          y = \sum_{e} y_e \Lambda x_e \quad (y\in Y).
        \end{align}
      %
      The sequence $\singleton{x_e}$ is finite hence bounded: 
        Given $V$ a balanced neighborhood of the origin, 
        there exists a positive scalar $s$ such that  
      %
        \begin{align}
          x_e \in s V
        \end{align}
      %
      for all $x_e$.
      Combining this with (\ref{1_10_sum}) shows that 
      %
        \begin{align}
          y \in \sum_e \Lambda (V) \quad (y\in Y: |y_e| < s^{\minus 1}).
        \end{align}
      %
  }
    \item{
      Since $N$ is closed, $\pi$ continously maps $X$ onto $X/N$, 
      another topological (Hausdorf) vector space, see [1.41]. 
      %
      Now take $f$ as in Exercise 9: 
      Since $\Lambda$ is onto, the first isomorphism theorem asserts that 
      %
        $f$ is an isomorphism of $X/N$ onto $Y$. 
      %
      Consequently, $X/N$ has dimension $n=\dim Y$. 
      $f$ is then an homeomorphism of 
      %
        $X/N\equiv \C^{n}$ 
      %
      onto $Y$; see [1.21].
      We have thus established that $f$ is continuous: So is $\Lambda = f\circ \pi$.
    }
  \end{enumerate}
\end{proof}