\textit{\noindent
Let $X$ be an infinite-dimensional Fréchet space. %
Prove that $X^\ast$, with its $\weakstar$topology, %
is of the first category in itself.
}
\newline\newline\noindent
This is actually a consequence of the below lemma,  %
which we prove first. %
The proof that $X^\ast$ is of the first category in itself comes right after, %
as a corollary.%
%
\paragraph{Lemma.}%
If $X$ is an infinite dimensional topological vector space whose dual %
%
  $X^\ast$ %
%
separates points on $X$, then the polar
%
  \begin{align}
    K_A\Def \{ \Lambda \in X^\ast:\, \magnitude{\Lambda} \leq 1 \text{ on } A\}
  \end{align} 
%
of any absorbing subset $A$ is a $\weakstar$closed set that has empty interior.
%
\begin{proof}%
Let $x$ range over $X$. The linear form %
%
  $\Lambda \mapsto \Lambda x$ %
%
is $\weakstar$continuous; see \citeresultFA{3.14}. %
Therefore, %
%
  $P_x=\{ \Lambda \in X^\ast:\, \lvert \delta_x \,\Lambda  \,\rvert \leq 1\}$ %
%
is $\weakstar$ closed: %
%In particular, every $P_a$ ($a \in A$) is $\weakstar$closed:  %
As intersection of all $\set{P_a}{a\in A}$, %
$K_A$ is also a $\weakstar$closed set. %
We now prove the second half of the statement. % %
%
\newline\newline\noindent
%
From now on, $X$ is assumed to be endowed with its weak topology: %
$X$ is then locally convex, but its dual space is still %
%
  $X^\ast$ (see \citeresultFA{3.11}). %
%
Put %
%
  \begin{align}
    W \Def \bigcap_{x\in F} \set{\Lambda \in X^\ast}{\magnitude{\Lambda x} < r_x}, 
  \end{align}
%
where $r_x$ runs on $\R_+$, %
as $F$ runs through the nonempty finite subsets of $X$. %
%
Clearly, the collection of all such $W$ is a local base of $X^\ast$. %
Pick one of those $W$ and remark that the following subspace %
%
  \begin{align}
    M \Def \text{span}(F)
  \end{align}
%
is finite dimensional. Therefore, it is closed, by \citeresultFA{1.21 (b)}. %
%
Assume, to reach a contradiction, that $A\subset M$: %
Each $x$ is then in $tM=M$ for some $t=t(x)>0$, since $A$ is absorbing. %
So, $X=M$ is finite dimensional, which is a desired contradiction.
%
We have just established that $A\setminus M$ is nonempty: %
Now pick $\varit{a}$ in $A\setminus M$ and so conclude that %f
%
  \begin{align}
    b\Def \frac{a}{t}\in A
  \end{align}
%
Remark that $b\notin M$ (otherwise, $t b = a \in tM=M$ would hold): %
By the Hahn-Banach theorem \citeresultFA{3.5}, %
there exists $\Lambda_b$ in $X^\ast$ such that 
%
  \begin{align}
    \label{3.11. Polar_4.}\Lambda_b b    > & \,\,2 \\
    \label{3.11. Polar_4bis.}\Lambda_b (M) = & \singleton{0}.
  \end{align}
%
The latter equality implies that $\Lambda_b$ vanishes on $F$; %
hence $\Lambda_b$ is an element of $W$. %
On the other hand, given an arbitrary $f\in K_A$, %
the following inequalities  %
%
  \begin{align}
    \magnitude{\Lambda_b (b) + f(b)} 
      \geq 
    2 - \magnitude{f(b)} 
      >
    1.
  \end{align}
%
show that $f + \Lambda_b$ is off $K_A$. %
%
We have thus proved that
%
  \begin{align}
  f+ W\not\subset K_A.
  \end{align}
%
Since $W$ and $f$ are both arbitrary, this achieves the proof. %
\end{proof}
%
%\newline\newline
\noindent
We now give a proof of the original statement:
%
\paragraph{Corollary.}%
Under the same assumptions: If $X$ is Fréchet, %
then $X^\ast$ is meager in itself.
%
\begin{proof}%
From now on, $X^\ast$ is only endowed with its $\weakstar$topology. %
Let $d$ be an invariant distance that is compatible with the topology of $X$, %
so that the following sets
%
  \begin{align}
    B_n \Def \set{x\in X}{d(0, x) <  2^{\minus n}}\quad (\counting{n}) 
  \end{align}
%
form a local base of $X$. %
%
If $\Lambda$ is in $X^\ast$, then %
%
  \begin{align}
    \magnitude{\Lambda} <  m \text{ on } B_n
  \end{align}
%
for some $(n, m) \in \singleton{1, 2, 3, \dots}^2$; see \citeresultFA{1.18}. %
%
Hence, $X^\ast$ is the union of all % 
%
  \begin{align}\label{3.11. Countable union.}
    m\cdot K_n \quad (\counting{m,n}), 
  \end{align}
%
where $K_n$ is the polar of $B_n$. %
Clearly, showing that $K_n$ is nowhere dense will achieve the proof. %
To do so, remark that the above lemma asserts that %
%
  \begin{align}
    \left({\overline{K}_n}\right)^\circ = \left({{K}_n}\right)^\circ =\emptyset, 
  \end{align}
%
\ie %
%
  \begin{align}\label{3.11. Nowhere dense.}
    \left(\,{\overline{m\cdot K_n}}\, \right)^\circ 
      = 
    %\left(\, m \cdot \overline{{{K}_n}} \, \right)^\circ
    %  = 
    %\left(m \cdot K_n \right)^\circ 
    %  = 
    m\cdot \left({{K}_n}\right)^\circ 
      = 
    \emptyset.
  \end{align}
%%
%
So ends the proof.% 
\end{proof}