\textit{%
Suppose X is a real vector space (without topology). %
Call a point $x_0\in A \subseteq X$ an internal point of $A$ if %
$A- x_0$ is an absorbing set. %
%
%\renewcommand{\labelenumi}{(\alph{enumi})}
%
  \begin{enumerate}
    \item{%
      Suppose $A$ and $B$ are disjoint convex sets in $X$, %
      and $A$ has an internal point. %
      Prove that there is a nonconstant linear functional $\Lambda$ such that %
      $\Lambda(A)\cap \Lambda(B)$ contains at most one point. %
      (The proof is similar to that of Theorem 3.4) %
    }%
    \item{%
      Show (with $X=\R^2$, for example) that it may not possible to have %
      $\Lambda (A)$ and $\Lambda (B)$ disjoint, under the hypotheses of (a). %
    }
  \end{enumerate} 
}
%
\begin{proof}%
Take $A$ and $B$ as in (a); the trivial case $B=\emptyset$ is discarded. %
Since $A-x_0$ is absorbing, so is its convex superset %
%
  $C= A-B - x_0 + b_0$ $(b_0 \in B)$. %
%
Note that $C$ contains the origin. %
Let $p$ be the Minkowski functional of $C$. Since $A$ and $B$ are disjoint, %
$b_0-x_0$ is not in $C$, hence $p(b_0-x_0) \geq1$. %
We now proceed as in the proof of the Hahn-Banach theorem \citeresultFA{3.4} %
to establish the existence of a linear functional %
%
  $\Lambda: X\to  \R$ such that %
%
  \begin{align}
    \Lambda \leq p %
  \end{align}
%
and %
%
  \begin{align}
    \Lambda(b_0-x_0) =1.
  \end{align}%
%
Then %
%
  \begin{align}
    \Lambda a -\Lambda b + 1 =\Lambda (a-b+ b_0-x_0) \leq p (a-b+ b_0-x_0) \leq 1\quad (a\in A, b \in B).
  \end{align}
%
Hence %
%
  \begin{align}\label{3_3_3}
    \Lambda a \leq \Lambda b .
  \end{align}
%
We now prove that $\Lambda (A) \cap \Lambda (B) $ contains at most one point. %
Suppose, to reach a contradiction, that this intersection contains %
%
  $y_1$ and $y_2$. %
%
There so exists %
%
  $(a_i, b_i)$ %%
%
in $A\times B$ ($i=1, 2$) such that %
%
  \begin{align}\label{3_3_4}
    \Lambda a_i= \Lambda b_i = y_i .
  \end{align}
%
Assume without loss of generality that $y_1< y_2$. Then, %
%\begin{align}
%y_1=\Lambda b_1 < \Lambda \left( \frac{1}{2} a_1\right ) + \Lambda \left( \frac{1}{2} a_2\right ) = \frac{1}{2} (y_1+y_2)  \quad .
%\end{align}
%
  \begin{align}\label{3_3_5}
    2\cdot y_1= \Lambda b_1+  \Lambda b_1 < \Lambda ( a_1 + a_2) = (y_1+y_2)  \quad .
  \end{align}
%
Remark that $a_3= \frac{1}{2} (a_1+ a_2)$ lies in the convex set $A$. %
This implies %
%
  \begin{align}
  \Lambda b_1 \overset{(\ref{3_3_5})}{<}  \Lambda a_3 \overset{(\ref{3_3_3})}{\leq} \Lambda b_1\quad ;
  \end{align}
%
which is a desired contradiction. (a) is so proved and we now deal with (b). %
%
\newline\newline\noindent
%
From now on, the space $X$ is $\R^2$. Fetch %
%
   \begin{align}
     S_1 & \Def \set{(x, y )\in \R^2}{x\leq 0, y \geq 0}, \\
     S_2 & \Def \set{(x, y )\in \R^2}{x>0, y > 0}, \\
     A   & \Def S_1\cup S_2, \\
     B   & \Def X\setminus A.
   \end{align}
%
Pick $(x_i, y_i)$ in $S_i$. %
Let $t$ range over the unit interval, and so obtain %
%
  \begin{align}
    t\cdot  \left(\begin{array}{c}x_1 \\y_1\end{array}\right)+
    (1-t)  \cdot \left(\begin{array}{c}x_2 \\ y_2\end{array}\right)= 
    \left(\begin{array}{c}t\cdot x_1+(1-t)\cdot x_2 \\t\cdot y_1 + (1-t)\cdot y_2\end{array}\right)
    \in \R\times \R_+\subseteq A.
  \end{align}
%
Thus, every segment that has an extremity in $S_1$ and the other one in $S_2$ %
lies in $A$. %
Moreover, each $S_i$ is convex. We can now conclude that $A$ is so. %
The convexity of $B$ is proved in the same manner. Furthermore, %
$A$ hosts a non degenerate triangle, \ie $A^{\circ}$ is nonempty\footnote{%
%
  For a immediate proof of this, remark that a triangle boundary is %
  compact/closed and apply [1.10] or 2.5 of \cite{BigRudin}.
}: %
$A$ contains an internal point. %
%
\newline\newline\noindent
%
Let $L$ be a vector line of $\R^2$. %
In other words, $L$ is the null space of a linear functional %
%
  $\Lambda: \R^2\to \R$ %
%
(to see this, take some nonzero $u$ in $L^\bot$ and set %
%
  $\Lambda x= (x,u)$ %
%
for all $x$ in $\R^2$). One easily checks that both $A$ and $B$ cut $L$. %
Hence %
%
  \begin{align}
    \Lambda (L)=\{0\}\subseteq \Lambda (A)\cap \Lambda (B)\neq\emptyset\quad.
    \end{align}
%
So ends the proof.
\end{proof}
