% % % % % % % % % % % % % % % % % % % % % % % % % % % % % % % % % % % % % % % % % % % % % % % % % % % % % % % % % % % % % % % %
% FunctionalAnalysis 
% FA_annex_vector_spaces.tex
% 
% encoding: UTF-8 
% EOL: LF
%
% format: LaTeX
% indent: spaces (2)
% width: 127
% % % % % % % % % % % % % % % % % % % % % % % % % % % % % % % % % % % % % % % % % % % % % % % % % % % % % % % % % % % % % % % %
\begin{lemma}[Vector subspaces as convex and balanced sets]\label{annex:vector-subspaces-as-convex-balanced-sets}
Given a vector space $X$, the following are equivalent for any nonempty $S \subset X$.
%
\begin{enumerate}
\item $S$ is a vector subspace of $X$,
\item $S$ is convex and balanced, and $S + S \subset S$,
\item $S$ is convex and balanced, and $\lambda S = S$ for all $\lambda > 0$.
\end{enumerate}
%
\end{lemma}
%
\begin{proof}It suffices to show that $(a) \then (b)$, $(b) \then (c)$, and $(c) \then (a)$. Assume (a), which implies %
$S+S \subset S$. Furthermore, $S$ is convex and balanced. Hence $(a)\then (b)$. Next, assume (b): By convexity of $S$, %
we have%
\footnote{See Exercise 1(d), equation \eqref{Exercise-1-d}.}: %
%
\begin{align}
  2 S &= S + S \\
  n S &= (n-1)S + S = S + S. && (\text{by induction on } n=2, 3, 4, \dots)
\end{align} 
% 
The assumption $S+S \subset S$ then yields $nS \subset S$ for $\counting{n}$. Now choose $\lambda > 0$ and observe that
\begin{equation}
  1 \leq \gamma=\max\{\lambda, 1/\lambda\} \leq \ceil*{\gamma}.
\end{equation}
%
Since $S$ is balanced, this implies
%
\begin{equation}
  S \subset \gamma S \subset \ceil*{\gamma} S \subset S.
\end{equation}
%
Thus, $\gamma S = S$. Furthermore, multiplying both sides by $1/\gamma$ gives
%
\begin{equation}
  S = (1/\gamma) S.
\end{equation}
%
This proves (c), because $\lambda\in\{\gamma, 1/\gamma\}$. Finally, assume (c). For any $(\alpha_1,\alpha_2)\in \C^2$, we have:
%
\begin{align}
  \alpha_1\cdot S + \alpha_2\cdot S &\subset \{1+|\alpha_1|\}\cdot S + \{1+|\alpha_2|\} \cdot S && \text{(by balancedness)}\\
  &\subset S + S && \text{(by the assumption $\lambda S = S$)}\\
  &= 2S && \text{(by convexity)}\\
  &= S. && \text{(by the assumption $\lambda S = S$)}
\end{align}
%
In conclusion, $S$ is a vector subspace of $X$. %
\end{proof}