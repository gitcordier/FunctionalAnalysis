% % % % % % % % % % % % % % % % % % % % % % % % % % % % % % % % % % % % % % % % % % % % % % % % % % % % % % % % % % % % % % % %
% FunctionalAnalysis 
% FA_annex_vector_spaces.tex
% 
% encoding: UTF-8 
% EOL: LF
%
% format: LaTeX
% indent: spaces (2)
% width: 127
% % % % % % % % % % % % % % % % % % % % % % % % % % % % % % % % % % % % % % % % % % % % % % % % % % % % % % % % % % % % % % % %
\begin{lemma}[Vector subspaces as convex and balanced sets]\label{annex:vector-subspaces-as-convex-balanced-sets}
Given a vector space $X$, the following are equivalent for any nonempty $S \subset X$.
%
\begin{enumerate}
\item $S$ is a vector subspace of $X$,
\item $S$ is convex and balanced, and $S + S \subset S$,
\item $S$ is convex and balanced, and $\lambda S = S$ for all $\lambda > 0$.
\end{enumerate}
%
\end{lemma}
%
\begin{proof}
It suffices to show that $(a) \to (b)$, $(b) \to (c)$, and $(c) \to (a)$. Assume (a), which implies $S+S \subset S$ %
and that $S$ is convex balanced. Hence $(a)\to (b)$. Next, assume (b) and proceed by induction on $n \geq 2$, as follows:
%
\begin{equation}
  n \, S \subset (n-1)\,S + S \subset S + S \subset S  \qquad (\text{by the assumption that } S + S \subset S).\\
\end{equation} 
%
The LHS expresses the convexity of $S$, see \eqref{Exercise-1-d}. Thus, $n \, S \subset S$ for all $n \geq 1$. Now choose %
$\lambda > 0$ and observe that
%
\begin{equation}
  1 \leq \gamma=\max\{\lambda, 1/\lambda\} \leq \ceil*{\gamma}.
\end{equation}
%
From the balancedness of $S$ and the property $nS \subset S$, we obtain
%
\begin{equation}
  S \subset \gamma \, S \subset \ceil*{\gamma} \,S \subset S.
\end{equation}
%
Hence 
%
\begin{equation}\label{annex:S=gamma-S}
  S = \gamma \, S.
\end{equation}
%
Multiplying both sides by $1/\gamma$ gives
%
\begin{equation}\label{annex:S=inverse-of-gamma-S}
  (1/\gamma)\, S = S.
\end{equation}
%
Equations \eqref{annex:S=gamma-S} and \eqref{annex:S=inverse-of-gamma-S} prove (c) because $\lambda\in\{\gamma, 1/\gamma\}$. %
Finally, assume (c): For any $(\alpha_1,\alpha_2)\in \C^2$, we have
%
\begin{align}
  \alpha_1 \, S + \alpha_2 \, S &\subset \{1+|\alpha_1|\} \cdot S + \{1+|\alpha_2|\} \cdot S && \text{(by balancedness)}\\
  &\subset S + S && \text{(by the assumption that $\lambda S = S$)}\\
  &= 2\, S && \text{(by convexity)}\\
  &= S. && \text{(by the assumption that $\lambda S = S$)}
\end{align}
%
In conclusion, $S$ is a vector subspace of $X$. %
\end{proof}