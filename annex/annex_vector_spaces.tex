% % % % % % % % % % % % % % % % % % % % % % % % % % % % % % % % % % % % % % % % % % % % % % % % % % % % % % % % % % % % % % % %
% FunctionalAnalysis 
% annex_vector_spaces.tex
% 
% encoding: UTF-8 
% EOL: LF
%
% format: LaTeX
% indent: spaces (2)
% width: 127
% % % % % % % % % % % % % % % % % % % % % % % % % % % % % % % % % % % % % % % % % % % % % % % % % % % % % % % % % % % % % % % %
\begin{lemma}[Vector subspaces as convex and balanced sets]\label{annex:vector-subspaces-as-convex-balanced-sets}
Given a vector space $X$, the following are equivalent for any nonempty $S \subset X$.
%
\begin{enumerate}
\item $S$ is a vector subspace of $X$,
\item $S$ is balanced, and $S + S \subset S$,
\item $S$ is convex and balanced, and $\lambda S \subset S$ for all $\lambda > 1$.
\end{enumerate}
%
\end{lemma}
%
\begin{proof}
It suffices to show that $(a) \to (b)$, $(b) \to (c)$, and $(c) \to (a)$.

Assume (a), which implies $S+S \subset S$ and that $S$ is balanced. This establishes (b); %
we now work under (b) alone. 

Combining the assumptions of (b), we obtain the convexity of $S$:
%
\begin{equation}
  (1-t)\,S + t\,S \subset S + S \subset S \qquad (0 \leq t \leq 1).
\end{equation}
%
Next, a straightforward induction on $n \in \{2, 3, 4, \dots\}$ shows that
%
\begin{equation}
  n \, S \subset (n-1)\,S + S \subset S + S \subset S. 
\end{equation} 
%
Indeed, the base case $n=2$ follows from $S + S \subset S$. Now choose $\lambda > 1$. Then 
%
\begin{equation}
  1 <  \lambda \leq \ceil*{\lambda}.
\end{equation}
%
The balancedness of $S$ and $n \,S \subset S$ together yield
%
\begin{equation}
  S \subset \lambda \, S \subset \ceil*{\lambda} \,S \subset S.
\end{equation}
%
In summary, 
%
\begin{equation}\label{annex:S=lambda-S}
  S = \lambda \, S, 
\end{equation}
%
which completes the proof of $(b) \to (c)$.

Finally, assume only (c): For any $(\alpha_1,\alpha_2)\in \C^2$, we have
%
\begin{align}
  \alpha_1 \, S + \alpha_2 \, S &\subset \singleton[\big]{1+|\alpha_1|} \cdot S 
    + \singleton[\big]{1+|\alpha_2|} \cdot S && \text{(by balancedness)}\\
  &\subset S + S && \text{(by the assumption that $\lambda S \subset S$ for all $\lambda > 1$)}\\
  &\subset 2\, S && \text{(by convexity; see \eqref{Exercise-1-d})}\\
  &\subset S. && \text{(by the assumption that $\lambda S \subset S$ for all $\lambda > 1$)}
\end{align}
%
Therefore, $S$ is a vector subspace of $X$. %
\end{proof}