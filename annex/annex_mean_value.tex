% % % % % % % % % % % % % % % % % % % % % % % % % % % % % % % % % % % % % % % % % % % % % % % % % % % % % % % % % % % % % % % %
% FunctionalAnalysis 
% FA_annex_mean_value.tex
% 
% encoding: UTF-8 
% EOL: LF
%
% format: LaTeX
% indent: spaces (2)
% width: 127
% % % % % % % % % % % % % % % % % % % % % % % % % % % % % % % % % % % % % % % % % % % % % % % % % % % % % % % % % % % % % % % %
\begin{lemma}[A mean value inequality for higher-order derivatives]\label{annex:mean-value-with-derivatives}
If $\phi\in\D_{[a, b]}$, then
%
\begin{equation}
  \norm*[\infty]{D^k \phi} \leq \norm*[\infty]{D^p \phi} \left(\frac{b-a}{2}\right)^{p-k}
\end{equation}
%
for all $k \leq p$ in $\N$.
\end{lemma}
%
\begin{proof}
First, consider $a < x_0 \leq (a+b)/2$. By the mean value theorem, there exists $a < x_1 < x_0$ such that
%
\begin{equation}
  \phi(x_0) - \underbrace{\phi(a)}_{0}=  D\phi(x_1)(x_0 - a).
\end{equation}
%
If $p> 1$, repeating the same reasoning, first for $D\phi$, then $D^2\phi$, and so on, yields
%
\begin{align}
  \phi(x_0) & = D^1 \phi(x_1) (x_0-a) \\
  & = D^2 \phi(x_2) (x_1-a) (x_0-a) \\
  &  \mspace{5mu} \vdots \nonumber \\ 
  & = \underbrace{D^p\phi(x_{p}) (x_{p-1} - a)}_{D^{p-1} \phi(x_{p-1})} \, (x_{p-2} -a) \cdots (x_0-a)
\end{align}
%
for some points $a < x_p <  \dots < x_1 < x_0$. Hence %
%
\begin{equation}
  \magnitude*{\phi(x_0)} \leq \norm*[\infty]{D^p \phi} \left(\frac{b-a}{2}\right)^p \qquad(\integers{p}).
\end{equation}
%
Similarly, if $(a+b)/2 < x_0 < b$ (with $b$ playing the role of $a$), the same inequality holds. Thus, 
%
\begin{equation}
  \magnitude*{\phi(x_0)} \leq \norm*[\infty]{D^p \phi} \left(\frac{b-a}{2}\right)^p \qquad (a < x_0 < b), 
\end{equation}
%
which establishes the result when $k=0$. Finally, applying the latter inequality to $D^k \phi $ in %
place of $\phi$ shows that % 
%
\begin{equation}
  \norm*[\infty]{D^k \phi} \leq \norm*[\infty]{D^p \phi} \left(\frac{b-a}{2}\right)^{p-k}
\end{equation}
%
for all $0 \leq k \leq p$. %
\end{proof}
%
\begin{lemma}[Higher derivatives cannot be bounded by lower derivatives]
\label{lemma-derivative-not-bounded-by-magnitude}%
There is no general formula to estimate higher-order derivatives from lower-order derivatives. This immediately implies that no reversed mean value theorem exists.
\end{lemma}
%
\begin{proof}
For \emph{angular frequency} $\omega > 1$, we consider 
%
\begin{align}
  \phi_\omega: \R & \to [\minus 1, 1]\\
  t & \mapsto \sin(\omega t), \nonumber
\end{align}
%
so that 
%
\begin{equation}
  \frac{\norm*[\infty]{D^p \phi_\omega}}{\norm*[\infty]{D^k \phi_\omega}} = \omega^{p-k} \qquad (0 \leq k < p)
\end{equation}
%
is unbounded as $p$ or $\omega$ tends to $\infty$. %
%
Note that no pointwise estimation holds either. Indeed, when it exists, the quotient %
$Q_\omega(t) = \magnitude*{\frac{D^p \phi_\omega}{D^k \phi_\omega}(t)}$ is $\omega^{p-k}$ if $p$ and $k$ have the same parity. %
Otherwise, $Q_\omega(t)$ is either $\omega^{p-k}\magnitude*{\tan(\omega t)}$ or $\omega^{p-k}\magnitude*{\cot{(\omega t)}}$. %
In all cases, $Q_\omega(t) \to \infty$ as $\omega \to \infty$ at fixed $t$. This example rules out any general inequality in %
the reversed direction. However, the following smooth example $\phi$ maintains a constant derivative around $0$, %
which is the simplest possible behavior. \par\noindent%
%
Let $\rho\in C^\infty(\R)$ be $1$ on $\openClosedInterval*{\minus\infty}{0}$, $0$ on $\closedOpenInterval*{1}{\infty}$, and %
strictly decaying on $\openInterval*{0}{1}$. A standard choice is $\rho = 1-h$ with
%
\begin{equation}
  h(t) \Def \frac{e^{\minus 1/t}}{e^{\minus 1/t} + e^{\minus 1/(1-t)}}
\end{equation}
%
for all $0 < t < 1$. Inspired by signal processing, we retain the angular frequency $\omega$ and introduce two additional %
positive parameters: maximum amplitude ($A$) and delay ($\tau$). We now define $\phi$ as the time-dependent solution of
%
\begin{equation}
  %E_\rho: 
  \begin{cases}
    \phi(0) & = 0 \\
    \diff{\phi}& = A \rho \bigl(\omega(\abs{t} - \tau)\bigr)\diff{t}.
  \end{cases}
\end{equation}
%
Equivalently, $\phi$ is odd and, for $s > 0$, 
%
\begin{align}
  \phi(s)  = A \int_{0}^{s} \rho \bigl(\omega (t - \tau)\bigr) \diff{t} 
    =  A \min(s, \tau) + A \int_{\tau}^{\max(s,  \tau)} \rho \bigl(\omega (t - \tau)\bigr) \diff{t}.
\end{align}
%
Hence
%
\begin{align}
  \norm*[\infty]{\phi} & = \phi(\tau + 1/\omega) \\
    %& =  \tau A  + A \int_{\tau}^{\tau + 1/\omega} \rho \bigl(\omega (t - \tau)\bigr) \diff{t} \\
    & = \tau A  + \frac{A}{\omega} \int_{0}^{1} \rho(u)\diff{u} \\
    & < \tau A + \frac{A}{\omega}.
\end{align}
%
The special case $\tau= 1/\omega= 1/A$ is of great interest. Indeed, %
$\restriction{D\phi}{\closedInterval*{\minus \tau}{\tau}} \equiv A\to \infty$ as $\tau \to 0$. In contrast, %
$\norm*[\infty]{\phi} < 2$. 
\end{proof}
% END
% 