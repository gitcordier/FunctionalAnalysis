% % % % % % % % % % % % % % % % % % % % % % % % % % % % % % % % % % % % % % % % % % % % % % % % % % % % % % % % % % % % % % % %
% FunctionalAnalysis 
% FA_annex_Dirac.tex
% 
% encoding: UTF-8 
% EOL: LF
%
% format: LaTeX
% indent: spaces (2)
% width: 127
% % % % % % % % % % % % % % % % % % % % % % % % % % % % % % % % % % % % % % % % % % % % % % % % % % % % % % % % % % % % % % % %
\newcommand{\heuristic}[1]{\mathcal{#1}}
Consider a physical example: a particle colliding with a surface, which absorbs a unit of energy at impact time $t=0$. 
We start with $H$ the \emph{Heaviside step function} $t \mapsto \Iverson*{t \geq 0}$, so that $H(t)$ indicates whether the %
particle has contributed its energy by time $t$. This formalism expresses that 
%
\begin{enumerate}
\item{The energy is transferred by an instantaneous jump at time $t=0$.}
\item{The energy is conserved over time.}
\end{enumerate}
%
Heuristically, we write, 
%
\begin{equation}
  \int_{\R} \heuristic{dH} = 1, \qquad \heuristic{\frac{dH}{dt}} =
  \begin{cases}
  \infty & t = 0 \\
  0 & t \neq 0
  \end{cases}
  \end{equation} 
%
These properties cannot coexist in standard calculus. Nevertheless, the informal measure $\heuristic{dH}$ describes the %
\emph{Dirac $\delta$ function}. When identified with a positive Borel measure, $\delta$ has total mass $1$ because %
%
\begin{equation}\label{mass-of-dirac}
  \int_{\R} \diff{\delta} = \int_{\R} \heuristic{dH} = \evalbracket{H}{\minus \infty}{\infty} = 1. 
\end{equation}
%
Physically, integrating $\delta$ over time recaptures all the energy. Let $W$ be the observation window, which is adjusted so %
that either $0 \in \interior{W}$ or $0 \notin \closure{W}$. Next, consider any smooth real function $\phi$ with compact %
support in $\interior{W}$ as a test signal. In this formalism, the integral $\int_{\R} \phi \diff{\delta}$ represents the %
detector's response to the collision when $\phi$ is not identically zero. If $\max{\abs{\phi}} = 1$, then Lebesgue's %
dominated convergence theorem ensures
%
\begin{equation}
  \sup_\phi \int_{\interior{W}} \abs{\phi}\diff{\delta}= \int_{\interior{W}}\diff{\delta}= \Iverson[\Big]{0 \in \interior{W}}.
\end{equation}
%
We now make the model rigorous by eliminating the heuristic $\heuristic{dH}$, as follows:
%
\begin{align}
  \int_{\R} \phi \diff{\delta} &=\int_{\R} \phi\,\heuristic{dH} & \bigl(\text{generalization of \eqref{mass-of-dirac}}\bigr)\\
  & =  \evalbracket{H\phi}{\minus \infty}{\infty} - \int_{\R} H \diff{\phi} & (\text{integration by parts})\\
  & = \minus \int_{\R} H \diff{\phi} \label{Dirac-impulse-distribution1}&(\text{$\supp{\phi}$ is compact})\\
  & =  \phi(0) &
\end{align}
%
The key point is that the right-hand side in \eqref{Dirac-impulse-distribution1} is valid in standard calculus. %
Moreover, we obtain all filtered responses as the evaluation functional $\phi \mapsto \phi(0)$. This motivates the %
following definitions: %
%
\begin{align}
  \Lambda_H(\phi) &\Def \int_{\R} H \phi & (\text{expresses $H$})\\
  \Lambda_H'(\phi) &\Def \minus \int_{\R} H \diff{\phi} = \phi(0) & 
    (\text{the \textit{weak derivative} of $H$: impulse at $0$})\\
  \delta(\phi) &\Def \phi(0) \label{definition-of-dirac} & (\text{$\delta$ now has a rigorous definition}).
\end{align}
%
The functional $\delta:\phi\mapsto\phi(0)$ is rigorously defined and so represents an instantaneous energy injection at $t=0$. 
Its extension to all $\phi \in C_c(\R)$ turns $\delta$ into a positive Radon measure of norm/total variation %
$\norm*{\delta} = 1$ and support $\singleton{0}$. In the sense of distribution theory, $\delta$ is a (tempered) distribution %
of order $0$; see \citeFA{Chapters 6 and 7}. Notably, its Borel-measure counterpart is the \emph{Dirac measure} %
%
\begin{equation}
  \delta: E \mapsto \Iverson[\Big]{0 \in E} 
\end{equation}
%
restricted to Borel sets in $\R$; hence the special case of \eqref{definition-of-sum}
%
\begin{equation}\label{filtering}
  \int_{\R} \phi \diff{\delta}= \delta(\phi) = \phi(0).
\end{equation}
%
Convolution of $\delta$ with translated signal $\phi_t:s \mapsto \phi(t-s)$ extends \eqref{filtering}, as follows: 
%
\begin{equation}
  [\delta \ast \phi](t) \Def \int_{\R} \phi_t \diff{\delta} 
  = \delta(\phi_t) = \phi(t).
\end{equation}
%
The Radon measure $\delta$ now serves as the convolution identity.
% END
%%