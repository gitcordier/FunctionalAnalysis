% section: Dirac
Suppose we wish to record the impact of a particle on a surface. To do so, we introduce the detector response $\phi$, which %
vanishes outside the observation segment $S$. A given collision occurs at impact time $t_Q$, which means the surface absorbs %
the energy quantum $Q$. Thus, the particle contribution is encoded as the Heaviside step function %
$H: t \mapsto \Iverson{t \geq 0}$, as time $t$ ranges over the real line. This models the fact that energy
%
\begin{enumerate}
\item{only changes by an infinitely abrupt increase at reference time $t_Q=0$,}
\item{is conserved over time.}
\end{enumerate}
%
We conclude that the derivative of $H$, if it could exist, would satisfy the empirical formula%
%
\newcommand{\heuristic}[1]{\mathtt{#1}}
\begin{align}
  \int_\R \heuristic{dH} =  1: \heuristic{\frac{dH}{dt}} = 
  \begin{cases}
    \infty & (t = 0) \\
    0 & (t \neq 0) \\
  \end{cases}
\end{align}
%
The informal measure $\heuristic{dH}$ describes the behavior of the \emph{Dirac $\delta$ function}. Interpreted as a Borel %
measure, $\delta$ has total mass $1$ because %
%
\begin{equation}\label{mass-of-dirac}
  \int_\R \delta = \int_\R \heuristic{dH} = H(\infty) - H(\minus \infty) = 1.
\end{equation}
%
This expresses that integrating $\delta$ over time recovers the whole energy $Q$. Lebesgue's dominated convergence theorem %
tunes our energy model, as
%
\begin{equation}
  \int_S \phi\,\delta \xrightarrow[\phi \to 1_S]{(b)} \int_S  \delta \citeq{(a)} [ 0 \in S].
\end{equation}
%
The formalism will be complete when the heuristic $\heuristic{dH}$ is discarded from $\delta$'s definition. So, 
%
\begin{align}
  \int_\R \phi\,\delta &=\int_\R \phi \, \heuristic{dH} & \bigl(\text{generalization of \eqref{mass-of-dirac}}\bigr)\\
  & =  \evalbracket{H\phi}{\minus \infty}{\infty} - \int_\R H \diff{\phi} & (\text{integration by parts})\\
  & = \minus \int_\R H \diff{\phi} \label{Dirac-impulse-distribution1}&(\text{$\supp{\phi}$ is compact})\\
  & =  \phi(0) &
\end{align}
%
The key point is that the right-hand side in \eqref{Dirac-impulse-distribution1} does not involve $\diff{\heuristic{H}}$. %
Moreover, we obtain the filtered response $\phi \mapsto \phi(0)$. This motivates the following definitions: %
%
\begin{align}
  \Lambda_H(\phi) &\Def \int_\R H \phi\diff{t} & (\text{expresses $H$, the Heaviside step function})\\
  \Lambda_H'(\phi) &\Def \minus \int_\R H \diff{\phi} = \phi(0) & 
    (\text{the \textit{weak derivative} of $H$: impulse at $0$})\\
  \delta(\phi) &\Def \phi(0) \label{definition-of-dirac} & (\text{$\delta$ now has a rigorous definition}).
\end{align}
%
The functional $\delta:\phi\mapsto \phi(0)$ is rigorously defined and conveys the idea of energy increasing by a fixed amount %
at a point in time. Its extension to all $\phi \in C_c(\R)$ turns $\delta$ into a Radon measure of norm/total variation %
$\norm{\delta} = 1$ and support $\singleton{0}$. In the sense of distribution theory, $\delta$ is a (tempered) distribution %
of order $0$; see \citeFA{Chapters 6 and 7}. Notably, its Borel-measure counterpart is the \emph{Dirac measure} %
%
\begin{equation}
  \delta: E \mapsto \Iverson{0 \in E} 
\end{equation}
%
restricted to Borel sets in $\R$. Hence the instance of \eqref{definition-of-sum}
%
\begin{equation}\label{filtering}
  \int_\R \phi \, \delta = \delta(\phi) = \phi(0).
\end{equation}
%
Convolution of $\delta$ with shifts $\phi_s:t \mapsto \phi(s-t)$ extends \eqref{filtering}, since 
%
\begin{equation}
  [\delta \ast \phi](s) \Def \int_{\minus \infty}^\infty \phi(s-t) \diff{\delta(t)} 
  = \int_\R \phi_s \diff{\delta} 
  = \delta(\phi_s) = \phi(s) \qquad (s\in \R).
\end{equation}
%
The Radon measure $\delta$ now serves as the convolution identity, which is of considerable interest.
% END
%%