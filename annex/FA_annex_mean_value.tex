% % % % % % % % % % % % % % % % % % % % % % % % % % % % % % % % % % % % % % % % % % % % % % % % % % % % % % % % % % % % % % % %
% FunctionalAnalysis 
% FA_annex_mean-value.tex
% 
% encoding: UTF-8 
% EOL: LF
%
% format: LaTeX
% indent: spaces (2)
% width: 127
% % % % % % % % % % % % % % % % % % % % % % % % % % % % % % % % % % % % % % % % % % % % % % % % % % % % % % % % % % % % % % % %
\begin{lemma}[A mean value inequality for higher-order derivatives]\label{mean-value-with-derivatives}
If $\phi\in\D_{[a, b]}$, then
%
\begin{equation}
  \norm[\infty]{D^k \phi} \leq \norm[\infty]{D^p \phi} \left(\frac{b-a}{2}\right)^{p-k}
\end{equation}
%
for all $k \leq p$ in $\N$.
\end{lemma}
%
\begin{proof}
  Let us first consider that $a < x_0 \leq (a+b)/2$. By the mean value theorem, there exists $a < x_1 < x_0$ such that
%
\begin{equation}
  \phi(x_0) = \phi(x_0) - \phi(a)=  D\phi(x_1)(x_0 - a).
\end{equation}
%
Repeating the same reasoning for $D\phi, D^2 \phi, \dots, D^p \phi \in \D_{[a, b]}$ gives % 
%
\begin{align}
  \phi(x_0) & = D^0 \phi(x_0) \\ 
  & = D^1\phi(x_1)(x_0 - a) \\
  &  \nonumber \mspace{10mu} \vdots \\
  & = D^p\phi(x_{p})(x_{p-1} - a) \cdots (x_0-a),
\end{align}
%
for some $a < x_p <  \dots < x_1 < x_0$. Hence %
%
\begin{equation}
  \magnitude{\phi(x_0)} \leq \norm[\infty]{D^p \phi} \left(\frac{b-a}{2}\right)^p .
\end{equation}
%
Similarly, if $b > x_0 \geq (a+b)/2$ (with $b$ playing the role of $a$), the same inequality holds. Thus, 
%
\begin{equation}
  \magnitude{\phi(x_0)} \leq \norm[\infty]{D^p \phi} \left(\frac{b-a}{2}\right)^p \quad (a < x_0 < b), 
\end{equation}
%
which establishes the result when $k=0$. Finally, applying the latter inequality to $D^k \phi $ in %
place of $\phi$ shows that % 
%
\begin{equation}
  \norm[\infty]{D^k \phi} \leq \norm[\infty]{D^p \phi} \left(\frac{b-a}{2}\right)^{p-k}
\end{equation}
%
for all $0 \leq k \leq p$. %
\end{proof}
%
\begin{lemma}[Impossibility of a nontrivial reversed mean value theorem]%
\label{lemma-derivative-not-bounded-by-magnitude}%
There is no general formula to estimate higher-order derivatives from lower-order derivatives.
\end{lemma}
%
\begin{proof}
For \emph{angular frequency} $\omega > 1$, we consider 
%
\begin{align}
  \phi_\omega: \R & \to [\minus 1, 1]\\
  t & \mapsto \sin(\omega t) \nonumber
\end{align}
%
so that 
%
\begin{equation}
  \frac{\norm[\infty]{D^p \phi_\omega}}{\norm[\infty]{D^k \phi_\omega}} = \magnitude{\omega}^{p-k} \qquad (0 \leq k < p)
\end{equation}
%
is unbounded as $p$ or $\omega$ tends to $\infty$. %
%
Note that no pointwise estimation holds either. Indeed, the quotient %
%
$Q(t) = \magnitude{\frac{D^p \phi_\omega}{D^k \phi_\omega}(t)}$ equals (when it exists) $\omega^{p-k}$ if $p$ and $k$ have %
same parity, or $\omega^{p-k}\tan(\omega t)$, (or $\omega^{p-k}\cot{(\omega t)}$) in the opposite case. In this last case, %
$Q(t)$ reaches all nonnegative values, which defeats any bound $Q(t) \leq M$.\par\noindent
%
This example rules out any general inequality in the reversed direction. However, the following smooth example $\phi$ %
keeps constant derivative $\phi'$ around $0$, which is more tractable. Let $\rho\in C^\infty(\R)$ be defined as $1$ on %
$]\minus \infty, 0]$, $0$ on $[1,\infty[$, and a strict decay on $\openinterval{0}{1}$. A standard choice is $\rho = 1-h$ with
%
\begin{equation}
  h(t) \Def \frac{e^{\minus 1/t}}{e^{\minus 1/t} + e^{\minus 1/(1-t)}}
\end{equation}
%
for all $0 < t < 1$. Inspired by signal processing, we choose three positive parameters: $A$ (maximum amplitude), $\tau$ %
(delay), and angular frequency $\omega$. We now define $\phi$ as the solution of 
%
\begin{align}
  E_\rho: 
  \begin{cases}
    \phi(0) & = 0 \\
    \displaystyle{\frac{\diff{\phi}}{\diff{t}}(t)} & = A \rho \bigl(\omega(\abs{t} - \tau)\bigr) 
  \end{cases}
\end{align}
%
when time $t$ ranges over the real line. Equivalently, $\phi$ is odd and, when $t$ turns positive, 
%
\begin{align}
  \phi(t)  = A \int_{0}^{t} \rho \bigl(\omega (s - \tau)\bigr) \diff{s} 
    =  A \min(t, \tau) + A \int_{\tau}^{\max(t,  \tau)} \rho \bigl(\omega (s - \tau)\bigr) \diff{s} 
\end{align}
%
Note that $\max{\magnitude{\phi}}$ over $\R$ is %
%
\begin{align}
  \norm[\infty]{\phi} & = \tau A  + A \int_{\tau}^{\tau + 1/\omega} \rho \bigl(\omega (s - \tau)\bigr) \diff{s} \\
    & = \tau A  + \frac{A}{\omega} \int_{0}^{1} \rho(u)\diff{u} \\
    & < \tau A + \frac{A}{\omega}.
\end{align}
%
The special case $\tau= 1/\omega= 1/A$ is of great interest. In this case, $\frac{\diff{\phi}}{\diff{t}}(0) = A\to \infty$, %
as $\tau \to 0$. In contrast, $\norm[\infty]{\phi} < 2$.  
\end{proof}
% END
% 