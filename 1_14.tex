\textit{Put $K =[0, 1]$ and define $\D_K$ as in Section 1.46. 
Show that the following three families of seminorms 
  (where $n = 0, 1, 2, \dots$) define the same topology on $\D_K$. 
If $D = d/dx$: 
%
  \begin{enumerate}
    \item{
      $\| D^n f \|_\infty = \sup\set{\left| D^n f(x)\right|}{\infty< x< \infty}$
    }
    \item{
      $\| D^n f \|_1 =\int_0^1 \left|D^n f(\varit{x}) \right| d\varit{x}$
    }
    \item{
      $\| D^n f \|_2 = \left\{
        \int_0^1 | D^n f(\varit{x}) |^2  d\varit{x} 
      \right\}^{1/2}.$
    }
  \end{enumerate}
  %
}
%
\begin{proof} 
First, remark that  
%
  \begin{align}\label{1_14_2}
    \| D^n f \|_1 
      \leq 
    \| D^n f \|_2 
      \leq 
    \| D^n f \|_\infty 
      <\infty 
  \end{align}
%
(the inequality on the left is a Cauchy-Schwarz one), 
since $K$ has length $1$. Next, start from 
%
  \begin{align}\label{1_14_3}
    D^n f(x) = \int_{\minus \infty}^x D^{n+1}f
  \end{align}
%
(which is true, since $f$ has a bounded support) to obtain
  \begin{align}\label{1_14_4}
    \left| D^n f(x)\right|
      \leq 
    \int_{\minus\infty}^x \left| D^{n+1}f \right| 
      \leq 
    \|D^{n+1}f \|_1 
  \end{align}
%
hence
%
  \begin{align}\label{1_14_5}
    \| D^n f \|_\infty 
      \leq 
    \| D^{n+1} f \|_1 .
  \end{align}
%
Combining (\ref{1_14_2}) with (\ref{1_14_5}) yields
%
  \begin{align}\label{1_14_6}
    \|D f \|_1 
      \leq 
    \cdots
      \leq
    \| D^n f \|_1 
      \leq 
    \| D^{n} f \|_2 
      \leq 
    \| D^{n} f \|_\infty 
      \leq 
    \| D^{n+1}f\|_1 
      \leq 
    \cdots .
  \end{align}
%
We now define 
  \begin{align}\label{1_14_1}
    V\up{i}_n &       \Def \set{f\in \D_K}{\| f \|_i <1/n}\quad(i=1,2,\infty)\\
    \mathscr{B}\up{i}&\Def \set{V\up{i}_n }{\counting{n}}
  \end{align}
so that (\ref{1_14_6}) is mirrored in terms of neighborhood inclusions, 
as follows,
%
  \begin{align}\label{1_14_7}
    V\up{1}_1 
      \supset
    \cdots 
      \supset 
    V\up{1}_n
      \supset 
    V\up{2}_n 
      \supset 
    V\up{\infty}_n 
      \supset 
    V\up{1}_{n+1} 
      \supset 
    \cdots .
  \end{align}
%
Since 
  $V\up{i}_n\supset V\up{i}_{n+1}$, 
$\mathscr{B}\up{i}$ is the local base of a topology $\tau_i$. 
But the chain (\ref{1_14_7}) forces the $\tau_i$'s to be equals. 
To see that, choose a set $S$ that is $\tau_1$-open at, say $a$: So, 
%
  $V\up{1}_n \subset S-a$  
%
for some $n$. Now $V\up{1}_n \supset V\up{2}_n$ (see (\ref{1_14_7})) forces  
%
  $V\up{2}_n \subset S-a$ , 
%
which implies that $S$ is $\tau_2$-open at $a$.
Similarly, we deduce, still from (\ref{1_14_7}), that 
\begin{align}
  \tau_2\text{-open} 
    \then 
  \tau_\infty\text{-open} 
    \then 
  \tau_1\text{-open}.
\end{align}
So ends the proof.
\end{proof}





