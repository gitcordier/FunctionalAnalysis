% % % % % % % % % % % % % % % % % % % % % % % % % % % % % % % % % % % % % % % % % % % % % % % % % % % % % % % % % % % % % % % %
% FunctionalAnalysis 
% notations.tex
% 
% encoding: UTF-8 
% EOL: LF
%
% format: LaTeX
% indent: spaces (2)
% width: 127
% % % % % % % % % % % % % % % % % % % % % % % % % % % % % % % % % % % % % % % % % % % % % % % % % % % % % % % % % % % % % % % %
\renewcommand{\thesection}{\Roman{section}}
\renewcommand{\thesubsection}{\roman{subsection}}
%
\chapter{Notations and Assumptions}%
%\addcontentsline{toc}{chapter}{Notations and Conventions}
% % % % % % % % % % % % % % % % % % % % % % % % % % % % % % % % % % % % % % % % % % % % % % % % % % % % % % % % % % % % % % % %
% LOGIC
% % % % % % % % % % % % % % % % % % % % % % % % % % % % % % % % % % % % % % % % % % % % % % % % % % % % % % % % % % % % % % % %
\section{Logic}%
\subsection{Propositional logic operators}
Given propositional variables $\mathit{p}$, $\mathit{q}$, the boolean %
operators $\lnot$, $\lor$, $\land$, $\iff$, $\then$, %
$\Leftarrow$, are assigned boolean \emph{truth values} (``true'' and ``false'') as follows,
\begin{enumerate}
  \item[$\lnot$]{%
    $\lnot p$ and $p$ have opposite values:
  }
  \item[$\lor$]{
    The \emph{disjunction} (``or'') $p \lor q$ is true unless $p$ is false and $q$ is false.
  }
  \item[$\land$]{
    The \emph{conjunction} (``and'') $p \land q$ is true if and only if $p$ is true and $q$ is true.
  }
  \item[$\iff$]{%
    The \emph{logical equivalence} expresses \emph{tautologies}: %
    $p \iff q$ is true unless $p$ and $q$ have opposite values. It is easily checked that %
    $(p \iff q) \iff \bigl((p \then q) \land (p \because  q) \bigr)$, see the definitions below.
  }
  \item[$\then$]{%
    The logical connection \emph{$p$ implies $q$} is supported by $\then$: $p \then q$ means \emph{if $p$ then $q$} %
    and is formally defined as $\lnot p \lor q$. Note that the ``reasoning'' $p \then q $ is always true unless %
    $p$ is true and $q$ is false. Moreover, $p \land (p \then q) \then q$ is always true. %
    This deductive rule is known as \emph{modus ponens}.
  }
  \item[$\Leftarrow$]{ $q \because p$ means that $q$ is implied by $p$ and is defined as $ p\then q $. %
    It is commonly read aloud as ``$q$ if $p$'' or ``$q$ is a consequence of $p$''.
  }
\end{enumerate}
%
See Section 1.3 and Subsection 16.1.3 of \cite{SpecifyingSystems} for further reading.
%
\subsection{Iverson notation}%
Given a boolean expression $\phi$, %
$\Iverson{\phi}$ returns the truth value of $\phi$, encoded as follows, %
%
\begin{align} \nonumber
  \Iverson{\phi}\triangleq 
  \begin{cases}
    0 & \quad\quad \text{if } \phi \text{ is false;} \\
    1 & \quad\quad \text{if } \phi \text{ is true.}
  \end{cases}
\end{align}
%
For example, $\Iverson{1 > 0} = 1$ but $\Iverson{ \sqrt{2} \in \Q} = 0$.
\section{Special terms}
\subsection{Halmos' \IFF and definitions}%
\IFF is a shorthand for ``if and only if". Splitting \IFF into \emph{if-then} clauses shows that it is just a rewording of %
the logical equivalence $\iff$. All definitions will use the \IFF format, which is consistent with the fact that %
every definition expresses a tautology.
%
\subsection{The assignment operator $\Def$}%
Given variables $\varit{a}$ and $\varit{b}$, $\triangleq$ is a specialization of $=$. We say that $a\triangleq b$ \IFF %
$a$ and $b$ are assumed to be equal. Usually, $a\triangleq b$ means that $a$ is assigned the previously known value %
$b$ (some authors write $a:=b$) but this is not a limitation. Definitions can be redundant and may overlap. %
%
% % % % % % % % % % % % % % % % % % % % % % % % % % % % % % % % % % % % % % % % % % % % % % % % % % % % % % % % % % % % % % % %
% SETS
% % % % % % % % % % % % % % % % % % % % % % % % % % % % % % % % % % % % % % % % % % % % % % % % % % % % % % % % % % % % % % % %
\section{Sets}
\subsection{Subsets and supersets}%
$\subset$ and $\supset$ are the standard symbols for set ordering, as follows: %
\begin{align}
  X \subset Y & \defIFF x \in X \then x\in Y \\
  Y \supset X & \defIFF X \subset Y.
\end{align}
No specific symbol is reserved for strict ordering. If necessary, $X \neq Y$ will be explicitly stated. 
%
\subsection{Special mappings}
The identity $I$ is the special mapping $I_X = \set{(x,x)}{x\in X}$. Similarly, the projection %
$\pi = \set{((x, y), x)}{x \in X, y \in Y}$ always exists. Note that $I$ is the diagonal of $X^2$. %
%
\subsection{Equinumerosity}%
$X\equiv Y$ means that there exists a bijection $\phi$ that maps $X$ to $Y$, which lets us identify $X$ with $Y$. %
In a metric space context, $X\equiv Y$ means that $\phi$ is isometric. %
%
% % % % % % % % % % % % % % % % % % % % % % % % % % % % % % % % % % % % % % % % % % % % % % % % % % % % % % % % % % % % % % % %
% TVS 
% % % % % % % % % % % % % % % % % % % % % % % % % % % % % % % % % % % % % % % % % % % % % % % % % % % % % % % % % % % % % % % %
\section{Topological vector spaces}
\subsection{Scalar field}%
$\C$ extends $\R$, which implies that a property\eg linearity, that is true on $\C$ is also true on $\R$. The complex case %
is then a \emph{special case} of the real case. This restriction may be significant in some contexts. Nevertheless, the %
standard scalar field is $\C$, which means that considering $\R$ instead of $\C$ makes no difference, unless stated otherwise.%
%
% % % % % % % % % % % % % % % % % % % % % % % % % % % % % % % % % % % % % % % % % % % % % % % % % % % % % % % % % % % % % % % %
% Vector spaces
% % % % % % % % % % % % % % % % % % % % % % % % % % % % % % % % % % % % % % % % % % % % % % % % % % % % % % % % % % % % % % % %
\subsection{Vector space bases}\label{notations: vector spaces: vector space bases}
Given a vector space $X$ over $\C$ (or, more generally, over a field), a subset $B$ of $X$ is a basis of $X$ \IFF %
the sum %
%
\begin{align}
  \setbig{(z_u)_{u\in B}}{z_u \in \C, \set{u}{z_u \neq 0} \text{ is finite}} & \to X \\\nonumber
  (z_u) & \mapsto \sum_{z_u \neq 0 } z_u u
\end{align}
bijectively maps all \emph{finitely supported} $(z_u)$ \emph{onto} $X$. %
%
The axiom of choice (AC) forces %
\begin{enumerate}
  \item the existence of such $B$ %
    (the proof is similar to the second part of the Hahn-Banach theorem [3.1] of \cite{FA} with $B$ playing the role of 
    $\Lambda$); %
  \item all bases to have the same cardinal, which is called the {\it dimension} of $X$ and is denoted as $\dim{X}$. %
\end{enumerate}
%
We now turn to the finite-dimensional case. Note that the $0$-dimensional case is the degenerate case $B=\emptyset$, %
which is equivalent to $X=\singleton{0}$. Our first step is to study $\C^n$ ($n>0$) the standard $n$-dimensional vector space.
%
% % % % % % % % % % % % % % % % % % % % % % % % % % % % % % % % % % % % % % % % % % % % % % % % % % % % % % % % % % % % % % % %
% Finite dimension
% % % % % % % % % % % % % % % % % % % % % % % % % % % % % % % % % % % % % % % % % % % % % % % % % % % % % % % % % % % % % % % %
\subsection{Finite-dimensional spaces}\label{notations: vector spaces: finite-dimensional vector spaces}%
\subsubsection{The product topology of $\C^n$}%
\label{notations: vector spaces: finite-dimensional vector spaces: the product topology of Cn}
%
$\C^n$ has a standard basis $1_{\singleton{1}}, \dots, 1_{\singleton{n}} \in \{0, 1\}^{\{0, \dots, n-1\}}$ %
so that $z_k$ ($1 \leq k \leq n$) is the $k$-th component of a given %
$(z_1,\dots, z_n) = z_1 (1, 0, \dots, 0)  + \cdots  + z_n (0, \dots, 0, 1) \in \C^n$. Furthermore, %
$\C^n$ is endowed with the topology generated by all polydiscs %
%
\begin{equation}
  \prod_{i=1}^{n} \underbrace{\set{z_i \in C}{\magnitude{z_i} < r_i}}_{D_{r_i}} \quad (r_i > 0 ).
\end{equation}
%
Equivalently, we may equip $\C^n$ with the Euclidean norm % 
%
\begin{equation}
  \norma{2}{z} \Def \sqrt{\magnitude{z_1}^2 + \cdots + \magnitude{z_n}^2} \quad \left(z = (z_1, \dots, z_n) \in \C^n\right), 
\end{equation}
%
whose open balls centered at the origin are all %
%
\begin{equation}
  B_r \Def \set{z\in \C^n}{\norma{2}{z} < r} \quad (r > 0).
\end{equation}
%
To show the equivalence, first set $r_i = r/\sqrt{n}$. Hence %
%
\begin{equation}
  \prod_{i=1}^{n} D_{r_i} \subset B_r. 
\end{equation}
%
Conversely, put $r = \min{r_1, \dots, r_n}$ so that %
%
\begin{equation}
   B_r \subset \prod_{i=1}^{n} D_{r_i} .
\end{equation}
%
\subsubsection{Topology of a finite-dimensional vector space}
It is customary to identify any $n$-dimensional vector space with $\C^n$ equipped with the Euclidean norm, %
see [\ref{notations: vector spaces: finite-dimensional vector spaces: the product topology of Cn}]. %
%
To show this, pick an $n$-dimensional vector space $Y$ and let $f$ be an isomorphism of $\C^n$ onto $Y$. %
For instance, require that $u_k = f(e_k)$, like in [1.20] of \cite{FA}, as $\singleton{u_k}$ is a basis of $Y$, %
see [\ref{notations: vector spaces: vector space bases}]. It follows from [1.21] of \cite{FA} that $f$ is a homeomorphism. 
The striking consequence is that 
%
\begin{quote}
  $\set{f(U)}{U \text{ open in }\C^n}$ {is the only vector space topology for $Y$}. %
\end{quote}
%
Thus, $Y$ is necessarily locally convex and locally bounded \ie normable, see [1.39] of \cite{FA}. Note that %
$\norm{y}= \norma{2}{f^{\minus 1}(y)}$ ($y\in Y$) is an example of a norm. Additionally, $Y$ is locally compact, %
since the closed unit ball of $\C^n$ is compact. Now pick an $n$-dimensional topological vector space $W$, then repeat the %
same reasoning, first with $g: \C^n \to W$, next with $h = g\circ f^{\,\minus 1}$ playing the role of $f$. This establishes %
that the homeomorphism $h$ maps $Y$ onto $W$ and that $W$ is normable as well. It is now clear that the following assertions %
are equivalent in the finite-dimensional context: %
%
\begin{enumerate}
  \item $\dim{W} = \dim{Y},$ 
  \item $W$ and $Y$ are isomorphic to each other,%
  \item $W$ and $Y$ are homeomorphic to each other, they are normable.
\end{enumerate}
%
Furthermore, the norms on $W$ and $Y$ are \emph{equivalent}. That is, for any given norm $\norma{Y}{\cdot}$ on $Y$ and %
any given norm $\norma{W}{\cdot}$ on $W$, there exists a positive constant (termed as \emph{modulus of continuity}) $C= C_h$ %
such that %
\begin{align}
  \norma{W}{W} & \leq C \norma{Y}{y} \quad\quad \left((y, w) \in h \right), 
\end{align}
%
as $h$ is continuous.  The special case $W=Y$ is that all norms on $Y$ are equivalent, in the sense that %
%
\begin{equation}
  \norma{{\text{\fw copy of Y}}}{h(y)} \leq C \norma{Y}{y}.
\end{equation}
%
\subsubsection{The standard norms $\norma{1}{\,\cdot\,}$, $\norma{2}{\,\cdot\,}$, $\norma{\infty}{\,\cdot\,}$} When $\C^n$ %
is equipped with standard norms $1, 2, \infty$, the sharp constant\ie the smallest $C_{i, j}$ such that %
%
\begin{equation}
	\norma{j}{z} \leq C_{i, j} \norma{i}{z} 
\end{equation}
%
is easily derived from definitions - see [1.19] of \cite{FA} - with the noticeable exception of $C_{2, 1} = \sqrt{n}$. %
Indeed, $\norma{1}{z} \leq \sqrt{n} \norma{2}{z}$ is a special Cauchy-Schwarz inequality, see (1) in [12.2] of \cite{FA}. %
The steps of this classical trick are left to the reader. %
%
\section{Measure theory on $\R^n$}
\subsection{Radon measures}
A (positive) Radon measure is a linear functional %
$\Lambda: C_c(\R^n) \to \C, \phi \mapsto \Lambda \phi$ that %
%
\renewcommand{\labelenumi}{(\roman{enumi})} 
\begin{enumerate}
  \item is positive\ie $\phi \geq 0 \then \Lambda\phi \geq 0$,
  \item is continuous in the sense that, for each compact $K \subset \R^n$ there exists a modulus of continuity $M_K$ such that %
  \begin{equation}
    \magnitude{\Lambda\phi} \leq M_K\norma{\infty}{\phi} \quad (\supp{\phi} \subset K ).
  \end{equation}
\end{enumerate}
\renewcommand{\labelenumi}{(\alph{enumi})} 
% 
Theorem [2.14] of \cite{BigRudin} shows that (i) actually implies (ii), as (ii) defines Radon measures in a weaker sense, %
see \cite{AnalyseIII}. Moreover, $\Lambda$ is bounded on $C_c(\R^n)$ with respect to the supremum norm \IFF (ii) is %
satisfied with all $M_K\leq \norma{\infty}{\Lambda} <\infty$. These bounded linear functionals are at the core of the theory, %
and according to [6.19] of \cite{BigRudin}, each of them is isometrically identified  with a specific regular Borel measure %
$\mu$. %
%
\subsection{Lebesgue integration}
Theorem [2.14] of \cite{BigRudin} states that every positive Radon measure $\Lambda$ (bounded or not) is identified with a %
specific positive and regular Borel measure $\beta$. This justifies the introduction of % 
%
\begin{equation}\label{definition-of-sum}
	\int_{\R^n} \phi \diff{\beta} \Def \Lambda \phi \quad \left(\phi \in C_c(\R^n)\right),
\end{equation}
%
where the left side is a Lebesgue integral. The regularity property implies that $\Lambda$ has mass %
%
\begin{equation}
	\int_{\R^n} 1 \diff{\beta} \Def \sup \setbig{\Lambda \phi}{\phi \in C_c(\R^n), \norma{\infty}{\phi} \leq 1}.
\end{equation}
%
When we substitute $f\diff{\beta}$ for $\diff{\beta}$, with $f$ Borel measurable, \eqref{definition-of-sum} takes the form %
%
\begin{equation}\label{Borel integral}
	\int_{\R^n} f \phi \diff{\beta} = \Lambda \phi.
\end{equation}
%
In particular, if $f$ is positive, 
%
\begin{equation}
	\int_{\R^n} f \diff{\beta} = \sup \setbig{\Lambda \phi}{\phi \in C_c(\R^n), \norma{\infty}{\phi} \leq 1}.
\end{equation}
%
Moreover, given $N = \set{f}{f = 0 \quad \beta-\text{a.e}}$, we see that, in \eqref{Borel integral}, density $f$ and measure %
$\Lambda$ are identified modulo $N$. Algebraically speaking: %
%
\begin{equation}
	f_1\beta = f_2\beta \iff f_1-f_2 \in N
\end{equation}
%
We now define the following \emph{magnitude} $\magnitude{\cdot}_1$,  %
%
\begin{equation}
  \magnitude{f}_1 \Def \int_{\R^n}\magnitude{f} \quad \left(f \text{ $\beta$-measurable}\right),
\end{equation}
%
and the associated vector spaces for calculus:
%
\begin{equation}
	\mathcal{L}^1 \Def \setbig{f}{\magnitude{f}_1  < \infty} \subset \mathcal{L}^1_{\text{loc}}
  \Def \setbig{f}{\magnitude{1_K f}_1 < \infty, K \text{ compact}}.
\end{equation}
%
The crucial points are: 
%
\begin{enumerate}
  \item $\mathcal{L}^1$, equipped with $\magnitude{\cdot}_1$, is a seminormed space, 
  \item $N$ is a closed subspace of the seminormed space $\mathcal{L}^1$.
\end{enumerate}
%
Together, these two facts imply that $L^1 = \mathcal{L}^1/N$, equipped with the norm %
%
\begin{equation}
	\norma{1}{f + N} \Def \magnitude{f}_1 \quad (f \in \mathcal{L}^1), 
\end{equation}
%
is a Banach space. % 
\subsection{Dirac's impulse, a physicist's detour}
Suppose we wish to record the impact of a particle on a surface. To do so, we introduce a carrier signal $\phi(t)$ that %
vanishes outside the observation interval. A given collision occurs at time $t_Q$, which means the surface absorbs a pulsed %
energy quantum $Q$ (normalized to 1 unit). Thus, the particle contribution is encoded as the Heaviside step function $H(t) = \Iverson{t \geq 0}$, %
as time $t$ ranges over the real line. This models the fact that energy $H(t)$ only increases by an infinitely abrupt jump %
at reference time $t_Q=0$. In this detour, we heuristically conclude that the derivative of $H(t)$, if it could exist, would %
explode by $+\infty$ at $t=0$. This motivates the empirical fiction%
%
\begin{equation}
  \frac{\diff{H}}{\diff{t}} \;\text{behaves like}\; \infty \times \Iverson{t=0}.
\end{equation}
%
The informal derivative $\diff{H}/\diff{t}$ is called $\delta(t)$, the \emph{Dirac delta function}, or %
\emph{Dirac's impulse}. Physically, the released energy is modeled by the impulse concentrated at $t=0$. Interpreted as %
a density, $\delta(t)$ has total integral $1$ because %
%
\begin{equation}
  \int_{\minus \infty}^\infty \delta(t) \diff{t} = \int_{\minus \infty}^\infty \diff{H} = H(\infty) - H(\minus \infty) = 1.
\end{equation}
%
This expresses that integration of $\delta(t)$ over time recovers the whole energy $Q$. We now apply this heuristic to the c%
arrier signal $\phi$: %
%
\begin{align}
  \int_{\minus \infty}^\infty \phi(t) \delta(t)\diff{t} & =  \int_{\minus \infty}^\infty  \phi \diff{H}  \\
  & =  \bigl[ H\phi \bigr]_{-\infty}^{\infty} - \int_{\minus \infty}^\infty H \diff{\phi} & [\text{integration by parts}]\\
  & = -\int_{\minus \infty}^\infty H \diff{\phi} \label{Dirac-impulse-distribution1}&[\text{$\supp{\phi}$ is compact}]\\
  & =  \phi(0) &
\end{align}
%
The key point is that the sum \eqref{Dirac-impulse-distribution1} no longer involves any $\diff{H}$. Moreover, %
we obtain the filtered carrier signal $\phi \mapsto \phi\vert_{t=0}$. This motivates the definition of the following linear %
functionals: %
%
\renewcommand{\labelenumi}{(\roman{enumi})}
\begin{enumerate}
  \item{
    \begin{equation*}
      \Lambda_H(\phi) \Def \int_{\minus \infty}^\infty H \phi\diff{t},
    \end{equation*}
  }
  \item{
    \begin{equation*}
      \Lambda_H'(\phi) \Def \minus \int_{\minus \infty}^\infty H \diff{\phi},
    \end{equation*}
  }
  \item{
    \begin{align}
      % Workaround: Negative hspace to force vertical alignment with above definitions
      \hspace{-0.45cm}\delta(\phi) & \Def \Lambda_H'(\phi)\nonumber \\
      &  = \phi(0).\nonumber
    \end{align}
  }
\end{enumerate}
\renewcommand{\labelenumi}{\alph{enumi}}
%
This definition of $\delta(\phi)$ conveys the idea of energy increasing by a fixed amount at pulsed time while being %
mathematically sound. Extension of (iii) to all $\phi \in C_c(\R^n)$ is immediate and turns %
$\delta$ into a Radon measure. Notably, its Borel-measure counterpart is the \emph{Dirac measure} %
%
\begin{equation}
  \delta: E \mapsto \Iverson{0 \in E} 
\end{equation}
%
restricted to Borel sets in $\R^n$. Dirac's $\delta$ plays a pivotal role in mathematical physics, as it is the convolution %
identity. To see that, we set $\phi_s(t) = \phi(s-t)$ ($s \in \R$), then we interpret the convolution $[\delta \ast \phi](s)$ %
as integration with respect to the Dirac measure. This reads as %
%
\begin{equation}
  [\delta \ast \phi](s)  \Def \int_{\minus \infty}^\infty \phi(s-t) \diff\delta(t) = \int_\R \phi_s \diff{\delta} 
  \citeq{\ref{definition-of-sum}} \delta(\phi_s) = \phi(s). 
\end{equation}
%
% END
%

