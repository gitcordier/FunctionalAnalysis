\renewcommand{\thesection}{\Roman{section}}
\renewcommand{\thesubsection}{\roman{subsection}}
%
\chapter{Notations and Conventions}%
%\addcontentsline{toc}{chapter}{Notations and Conventions}
\section{Logic}%
\subsection{Propositional logic}
Given propositional variables $\mathit{p}$, $\mathit{q}$, the boolean %
operators $\lnot$, $\lor$, $\land$, $\Leftrightarrow$, $\Rightarrow$, %
$\Leftarrow$, assign boolean \textit{truth values} as follows,
\begin{enumerate}
  \item[$\lnot$]{%
    $\lnot p$ has not the truth value of $p$.
  }
  \item[$\lor$]{
    The \textit{conjonction} $p \lor q$ is true, unless: %
    $p$ false, $q$ false.
  }
  \item[$\land$]{
    The \textit{disjunction} is false, unless: $p$ true, $q$ true.
  }
  \item[$\Leftrightarrow$]{%
    The \textit{logical equivalence} expresses \textit{tautologies}: %
    $p \Leftrightarrow q$ is true, unless: %
    $p$ has not the truth value of $q$. %
    It is easily checked that %
    %
      $(p \Leftrightarrow q) \Leftrightarrow %
        \left(
          (p \Rightarrow q) \land 
          (p \Leftarrow  q)
        \right)$; see the below definitions.
    
  }
  \item[$\Rightarrow$]{%
    The logical implication is denoted by $\Rightarrow$: %
    $p \Rightarrow q$ means %
    \textit{if (criterion/premise) $p$ then (conclusion) $q$}, or, %
    alternatively, \textit{$p$ implies $q$}. %%
    %
    $p \Rightarrow q $ is formally defined as $\lnot p \lor q$. %
    %
    Remark that the ``reasoning'' $p \Rightarrow q $ is always valid, unless: %
    $p$ true, $q$ false. Moreover, %
    %
      $p \land (p \Rightarrow q) \Rightarrow q$ %
    %
    is always true.
    }
    \item[$\Leftarrow$]{ $q \Leftarrow p$ is $ p\Rightarrow q $ read backward. 
    A common pronunciation is \textit{$q$ since $p$}.
    }
\end{enumerate}
%
For a subtle introduction to proposition logic, %
see Section 1.3 and Subsection 16.1.3 of \cite{SpecifyingSystems}.
%
\subsection{Iverson notation}%
Given a boolean expression $\varphi$, %
$\boolean{\varphi}$ returns the truth value of $\varphi$, encoded as follows, %
%
\begin{align} \nonumber
  \boolean{\varphi}\triangleq 
  \begin{cases}
    0 & \quad\quad \text{if } \varphi \text{ is false;} \\
    1 & \quad\quad \text{if } \varphi \text{ is true.}
  \end{cases}
\end{align}
%
For example, $\boolean{1 > 0} = 1$ but $\boolean{ \sqrt{2} \in \Q} = 0$.
\section{Special terms}
\subsection{Halmos' iff and definitions}%
\iif is a short for ``if and only if". %
Splitting \iif into \textit{if-then} clauses shows that it is just %
a rewording of the logical equivalence $\Leftrightarrow$. %
All definitions will use the \iif format; %
which is consistent with the fact that every definition expresses a tautology.
%
\subsection{Assigning values}%
Given variables $\varit{a}$ and $\varit{b}$, $\triangleq$ is a specialization %
of $=$. We say that $x\triangleq y$ \iif $x$ and $y$ are assumed to be equal. 
Usually, $x\triangleq y$ means that $x$ is assigned the previously known 
value $y$ (some authors write $x:=y$) but this is not a limitation. 
Definitions can be redundant and may overlap. The only restriction is that %
$x\triangleq y$ is inconsistent whether $x\neq y$.

\subsection{Equinumerosity}%
$a\equiv b$ means that there exists a bijection $\to$ that maps $a$ to $b$; %
which let us identify $a$ with $b$. %
In a metric space context, $a\equiv b$ means that $\to$ is isometric.
\section{Topological vector spaces}
%\subsection{Vector spaces}
\subsection{Product space}
\subsection{Scalar field}%
The usual (complete) scalar field is $\C$. %
A property, \eg linearity, that is true on $\C$ is also true on $\R$. %
The complex case is then a {\it special case} of the real one. %
Sometimes, this specialization is not purely formal. %
For example, theorem 12.7 of \cite{FA} asserts that, in a Hilbert space $H$ %
equipped with the inner product $\bra{\,\cdot\,}\ket{\,\cdot\,}$, %
every nonzero linear continuous operator $T$ ``breaks orthogonality'', %
in the sense that there always exists $x=x(T)$ in $H$ that satisfies %
%
  $\bra{Tx}\ket{x} \neq 0$. %
%
The proof of this theorem strongly depends on the complex field. %
Actually, a real counterpart does not exists. %
To see that, consider the $90^\circ$ rotations of the euclidian plane. %
%
Nevertheless, {\it unless the contrary is explicitely mentioned}, %
the exension to the real case will always be obvious. %
So, taking $\C$ as the scalar field shall mean %
%
\begin{quote}{\it %
Instead of letting the scalar field undefined, we choose $\C$ for the sake of %
expessivity. But considering $\R$ instead of %
$\C$ would actually make no difference here
}. %
\end{quote}
%
\subsection{Finite dimensional spaces}\label{finite dimensional spaces}%
It may be customary to identify any $n$-dimensional vector space $Y$ with %
the normed space $(\C^n, \| \,\|_{\C^n})$, %
as $\| \, \|_{\C^n}$ is an arbitray norm on $\C^n$. %
%
Indeed, those vector spaces $Y$ are actually normable spaces that are %
homeomorphic each others. %
%
This statement can be expressed as an \textit{equivalence relation} over all %
normed vector spaces $(Y, \|\,\|_Y)$. %
More precisely, given copies $Y_i$ ($i=1, 2$) of space(s) $Y$, there exists %
an isomorphism $h: Y_1 \to Y_2$ and a positive constant $C$ such that %
%
\begin{align}\label{norm equivalence 1}
  \| y_2 \|_{Y_2} \leq C \| y_1 \|_{Y_1} \quad 
  (y_i \in Y_i: y_2 = h(y_1)).
\end{align}
%
\begin{proof}
Pick a basis $F_Y$ of $Y$, as $\mathit{Y}$ run through all $n$-dimensional %
vector spaces (the trivial case $n=0$ shall be implicitely skipped). %
There so exists a one-to-one mapping of $F_{\C^n}$ onto $ F_Y$, %
and such mapping extends to an isomorphism $f: \C^n \to Y$. %
%
On the other hand, there exists an ambient topological vector space %
$(X, \tau_X)$ that induces, on $Y$, a topology %%
%
\begin{align}\label{inherited norm}
  \tau_Y \triangleq \{ Y\cap U: Y \subseteq X, U \in \tau_X\}.
\end{align}
%
For instance, take $X=Y$ then equipp $Y$ with the norm %
%
\begin{align}\label{normability}
  \| \, \|_{Y, 2}: Y & \to \R \\
            y & \mapsto \| f^{\,\minus 1} (y)\|_2 \nonumber, 
\end{align}
%
where $\| \, \|_2$ is the Euclidian norm of $\C^n$. %
%
Now assign each $Y$ a space $(X, \tau_X)$ then use Theorem 1.21 of \cite{FA} %
to conclude that $f$ is more specifically a linear homeomorphism of $\C^n$ %
(equipped with $\|\, \|_2$) onto $Y$. %
%
Given two copies $(\mathit{X}_i, \mathit{Y}_i, \mathit{f}_i)$ %
of the variables $\mathit{X}, \mathit{Y}, \mathit{f}$, %
we so obtain the following commutative diagram: %
%
\begin{equation}
\begin{tikzpicture}[-stealth,
  label/.style = { font=\footnotesize }]
  \matrix (m)
    [
      matrix of math nodes,
      row sep    = 4em,
      column sep = 4em
    ]
    {
         & \C^n &     \\
      Y_1&      & Y_2 \\
    };
  \path (m-2-1) edge node [above,label]  {$f_1^{\, \minus 1}$} (m-1-2);
  \path (m-1-2) edge node [above,label]  {$f_2$} (m-2-3);
  \path (m-2-1) edge node [above,label]  {$h$} (m-2-3);
\end{tikzpicture}
\end{equation}
%
It is now clear that all $Y$ are homeomorphic each other. %
%
The special case $Y_1 = Y_2$ means that $\tau_{Y_1} = \tau_{Y_2}$. 
In other words, each vector space $Y$ has a unique topology $\tau_Y$; %
the embedding $Y \subseteq X$ does not matter. % 
% 
$\tau_{Y}$ is induced by at least one norm $\|\,\|_Y$, \eg %
$\|\,\|_{Y, 2}$; see (\ref{normability}). %
%
We now establish the norm equivalence over all vector spaces $Y$. To do so, %
we easily check that the continous function $h$ maps the open unit sphere %
%
  $B_1 = \{ \| \, \|_{Y_1} < 1 \}$ onto %
%
a $\|\,\|_{Y_2}$-bounded set. %
%
This allows us to pick %
%
\begin{align}
  C_h \triangleq \sup_{B_1} \| h \|_{Y_2} > 0 \quad(y_1 \in Y_1), 
\end{align}
%
so that %
%
\begin{align}\label{norm equivalence 2}
  \|h(y_1)\|_{Y_{2}} \leq C_h \|y_1\|_{Y_{1}}; 
\end{align}
%
which is (\ref{norm equivalence 1}). %
%
Finally, we show that we are actually dealing with an equivalence relation. %
First, reflexivity and transitivity are obvious. Moreover, 
permuting the $\mathit{Y}_i$'s, with %
%
  $\mathit{%
  h^{\, \minus 1}, 
  B_2 = \{\| \, \|_{Y_2} < 1 \}, 
  C_{h^{\, \minus 1}} = \sup_{B_2} \| h^{\, \minus 1}\|_{Y_2}
  }$
%
playing the role of $\mathit{h}, \mathit{B_1}, \mathit{C_h}$ %
leads to a \textit{symmetrical} result; which achieves the proof.
\end{proof}
% END
