\renewcommand{\labelenumi}{\arabic{enumi}.}
\chapter{Notations and Conventions}
\section{Logic}
\begin{enumerate}
\item{{\bf Halmos' iff}:  \iif is a short for "if and only if".}
\item{{\bf Definitions (of values) with $\Def$}: Given a variables %
$\varit{a}$ and $\varit{b}$, %
$a\Def b$ means that $\varit{a}$ is defined as equal to $\varit{b}$.}
\item{{\bf Definitions (formul\ae)}: Definitions come from \iif. %
In other words, both parts (the "if \dots" part and the "only if \dots" part) %
are explicitely stated. }
\item{{\bf Iverson notation}:  Given a boolean expression $\Phi$, %
$\boolean{\Phi}$ returns the truth value of $\Phi$, encoded as follows, %
%
  \begin{align} \nonumber
    \boolean{\Phi}\Def 
    \begin{cases}
      0 & \quad\quad \text{if } \Phi \text{ is false;} \\
      1 & \quad\quad \text{if } \Phi \text{ is true.}
    \end{cases}
  \end{align}

For example, $\boolean{1 > 0} = 1$ but $\boolean{ \sqrt{2} \in \Q} = 0$.
}
\end{enumerate}
%\section{Vector spaces}
%\begin{enumerate}
%  \item{If $X$ is a vector space of base $B$ and $e$ is an element of B, }
%\end{enumerate}
