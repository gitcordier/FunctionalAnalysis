\renewcommand{\labelenumi}{\arabic{enumi}.}
\addcontentsline{toc}{chapter}{Notations and conventions}
\chapter*{Notations and Conventions}
\section{Logic}
\begin{enumerate}
\item{{\bf Halmos' iff.} \iif is a short for ``if and only if".}
\item{{\bf Definitions (of values) with $\Def$.} Given variables %
$\varit{a}$ and $\varit{b}$, %
$a\Def b$ means that $\varit{a}$ is defined as equal to $\varit{b}$.}
\item{{\bf $\equiv$.} $a\equiv b$ means that there exists a ``natural'' %
bijection $\to$ that maps $a$ to $b$; which let us identify $a$ with $b$. %
In a metric space context, $a\equiv b$ means that $\to$ is isometric.}
\item{{\bf Definitions (formul\ae).} Definitions use the \iif format. %
In other words, every definition has a ``only if''. %
}
\item{{\bf Iverson notation.} Given a boolean expression $\Phi$, %
$\boolean{\Phi}$ returns the truth value of $\Phi$, encoded as follows, %
%
  \begin{align} \nonumber
    \boolean{\Phi}\Def 
    \begin{cases}
      0 & \quad\quad \text{if } \Phi \text{ is false;} \\
      1 & \quad\quad \text{if } \Phi \text{ is true.}
    \end{cases}
  \end{align}

For example, $\boolean{1 > 0} = 1$ but $\boolean{ \sqrt{2} \in \Q} = 0$.
}
\end{enumerate}
%\section{Vector spaces}
%\begin{enumerate}
%  \item{If $X$ is a vector space of base $B$ and $e$ is an element of B, }
%\end{enumerate}
\section{Topological vector spaces}
\begin{enumerate}
\item{{\bf Product space}}
\item{{\bf Scalar field.} The usual (complete) scalar field is $\C$. %
A property, \eg linearity, that is true on $\C$ is also true on $\R$. %
The complex case is then a {\it special case} of the real one. %
Sometimes, this specialization is not purely formal. %
For example, theorem 12.7 of \cite{FA} asserts that, in a Hilbert space $H$ %
equipped with the inner product $\bra{\,\cdot\,}\ket{\,\cdot\,}$, %
every nonzero linear continuous operator $T$ ``breaks orthogonality'', %
in the sense that there always exists $x=x(T)$ in $H$ that satisfies %
%
$\bra{Tx}\ket{x} \neq 0$. %
%
The proof of this theorem strongly depends on the complex field. %
Actually, a real counterpart does not exists. %
To see that, consider the $90^\circ$ rotations of the euclidian plane. %
%
Nevertheless, {\it unless the contrary is explicitely mentioned}, %
the exension to the real case will always be obvious. %
So, taking $\C$ as the scalar field shall mean %
%
``{\it %
Instead of letting the scalar field undefined, we choose $\C$ for the sake of %
expessivity. But considering $\R$ instead of %
$\C$ would actually make no difference here}''. %
%
}
\item{{\bf Finite dimensional spaces}. %
Let $Y$ be a finite dimensional space. %
If $\dim Y = 0$, \ie  %
%
$Y$ is a group of order $1$,  %%
then $\singleton{\emptyset, Y}$ is the only possible topology for $Y$. %
%
For instance, in a quotient space $X/N$, %
the zero is $N$ and $\singleton{N}$ is zero-dimensional in $X/N$. %
%
\\
\\\noindent
Assume henceforth that $\dim Y > 0$ , \ie $Y$ has a base %
%
$F_n = \set{f_i}{i= 1,\dots, n}$ for some positive $n$. %
%
%For example, $Y$ may be the $\C_n[X] = \set{P(X)}{\deg P < n} $ %
%the set of all polynomials of degree less than $n$. %
%In such case, $\singleton{1, X, \dots, X^{n-1}}$ is a base for $Y$. %
%
The cartesian power $\C^n  = \prod_{j=1}^{n} \C$ is %
the vector space of all lists %
%
$(z_1, \dots, z_n)$, where $z_j\in \C$ %
%
(identify $\C^1$ with $\C$). %
The subset %
%
$E_n = \set{e_j}{j=1, \dots, n}$ %
%
is the {\it standard base} of $\C^n$, \ie $e_j = 1_{\singleton{j}}$. %
So, %
%
$\dim \C^n = n$. %
%%
% DEFINITION OF U AND V.
Let $u:\C^n \to Y$ be the only linear mapping that verifies all %
%
$u(e_j) = f_j$. %
%
Since $u$ is encoded as the identity matrix, % %
%
both $u$ and $v = u^{\, \minus 1}$ exist as isomorphisms. %
%
Additionally, $\C^n$ can be equipped with various norms, %
\eg the {\it p-norms} $\norm{p}{\,}$ %
(where %
%
$\norm{p}{(z_1, \dots, z_n)}^p = %
\magnitude{z_1}^p + \cdots + \magnitude{z_n}^p; p \geq 1$
%
) or %
%
$\norm{\infty}{}$ (where %
%
$\norm{\infty}{(z_1, \dots, z_n)} = \max \magnitude{z_j}$
). %
%
Note that $Y$ inherits any norm $\norm{}{}$ of $\C^n$, with %%
$\norm{}{u(z_1, \cdots, z_n)}= \norm{}{(z_1, \cdots, z_n)}$; %
%
which turns $u$ into a isometry of $\C^n$ onto $Y$. %
%
Let $\tau_{\norm{}{\,}}$ denote the topology of a norm $\norm{}{\,}$. %
%
We now go back to the proof of 1.21 of \cite{FA} and %
so equip $Y$ with a its own norm $\norm{2}{}$; %
%
which turns $u$ into a isometric isomorphim of $\C^n$ onto $Y$. %
%
$Y$ can now be seen as a topological vector space, in at least one fashion; %
namely, the space $(Y,\tau_{\norm{2}{\,}})$. %
%
Let $\tau = \tau_Y$ stand for any arbitrary topology of $Y$. %
Hence the following commutative diagram % 
%
\begin{equation}
\begin{tikzpicture}[-stealth,
  label/.style = { font=\footnotesize }]
  \matrix (m)
    [
      matrix of math nodes,
      row sep    = 4em,
      column sep = 4em
    ]
    {
                                & (\C^n, \tau_{\norm{2}{\,}})   &                    \\
     (Y, \tau_{\norm{2}{\,}})   &                        &  (Y, \tau)   \\
                                &  (\C^n, \tau_{\norm{2}{\,}})  &                      \\
      %$\C^n$ & B_1 & B_2    \\
    };
  \path (m-1-2) edge node [above,label]  {$u$} (m-2-1);
  \path (m-2-1) edge node [above,label]  {$v$} (m-3-2);
  \path (m-3-2) edge node [above,label]  {$u$} (m-2-3);
  \path (m-2-3) edge node [above,label]  {$v$} (m-1-2);
  %
  %\path (m-2-1) edge node [above,label]  {$i$} (m-2-3);
  %\path (m-2-3) edge node [above,label]  {$i$} (m-2-1);
  
\end{tikzpicture}
\end{equation}
It is now clear that the {\it identity mapping} %
%
$u\circ v$ % 
%
is an homeomorphism of $Y$ onto $Y$, which implies that %
%
$\tau = \tau_{\norm{2}{\,}}$. %
%
In other words, there is only one topology $\tau$ for $Y$, %
as a topological vector space. %
%
This topology is normable, since %
%
$\tau = \tau_{\norm{2}{\,}}$. %
%
Let $\norm{Y}{\,}$ stand for any norm of $Y$. The special case %
%
$Y=\C^n, F_n = E_n, u= i$ %
%
is of considerable interest. %
TOTO. 
Now take $\tilde{Y}$ of dimension $n$ then similarly define %
(obvious notations) %
%
$\tilde{u}$, $\tilde{v}$ and $\tilde{\tau}$.%
%
\begin{equation}
\begin{tikzpicture}[-stealth,
  label/.style = { font=\footnotesize }]
  \matrix (m)
    [
      matrix of math nodes,
      row sep    = 4em,
      column sep = 4em
    ]
    {
                & (\C^n, \tau_{\C^n})   &               \\
    (Y, \tau)   &                       &  (\tilde{Y}, \tilde{\tau})   \\
                &  (\C^n, \tau_{\C^n})  &              \\
    };
  \path (m-1-2) edge node [above,label]  {$u$} (m-2-1);
  \path (m-2-1) edge node [above,label]  {$v$} (m-3-2);
  \path (m-3-2) edge node [above,label]  {$\tilde{u}$} (m-2-3);
  \path (m-2-3) edge node [above,label]  {$\tilde{v}$} (m-1-2);
\end{tikzpicture}
\end{equation}
%
The homeomorphism between $Y$ and $\tilde{Y}$ leads to %
{\it the equivalence of norms at fixed dimension $n$}, as follows %
$A \norm{Y}{y} \leq \norm{\tilde{Y}}{ \tilde{u}\circ v (y)} \leq B \norm{Y}{y}$ ($y\in Y$) for some positive $A, B$. 
%
Equip $Y$ and $\tilde{Y}$ with the inherited norm $\norm{}{\,}$. %
$Y$ and $\tilde{Y}$ are homeomorphically isomorphic ($\equiv$) to %
%
$\C^n, \norm{}{\,}$. %
%
\begin{equation}
\begin{tikzpicture}[-stealth,
  label/.style = { font=\footnotesize }]
  \matrix (m)
    [
      matrix of math nodes,
      row sep    = 4em,
      column sep = 4em
    ]
    {
                                & (\C^n, \norm{}{\,})   &                    \\
     (Y, \norm{}{\,})    &                        &  (\tilde{Y}, \norm{}{\,})    \\
                                &  (\C^n, \norm{}{\,})   &                      \\
      %$\C^n$ & B_1 & B_2    \\
    };
  \path (m-1-2) edge node [above,label]  {$\equiv$} (m-2-1);
  \path (m-2-1) edge node [above,label]  {$\equiv$} (m-3-2);
  \path (m-3-2) edge node [above,label]  {$\equiv$} (m-2-3);
  \path (m-2-3) edge node [above,label]  {$\equiv$} (m-1-2);
\end{tikzpicture}
\end{equation}
From now the default norm will be $\norm{\infty}{}$.
}
\end{enumerate}