\renewcommand{\thesection}{\Roman{section}}
\renewcommand{\thesubsection}{\roman{subsection}}
%
\chapter{Notations and Conventions}%
%\addcontentsline{toc}{chapter}{Notations and Conventions}
\section{Logic}%
\subsection{Propositional logic}
Given propositional variables $\mathit{p}$, $\mathit{q}$, the boolean %
operators $\lnot$, $\lor$, $\land$, $\Leftrightarrow$, $\Rightarrow$, %
$\Leftarrow$, assign boolean \textit{truth values} as follows,
\begin{enumerate}
  \item[$\lnot$]{%
    $\lnot p$ has not the truth value of $p$.
  }
  \item[$\lor$]{
    The \textit{conjonction} $p \lor q$ is true, unless: %
    $p$ false, $q$ false.
  }
  \item[$\land$]{
    The \textit{disjunction} is false, unless: $p$ true, $q$ true.
  }
  \item[$\Leftrightarrow$]{%
    The \textit{logical equivalence} expresses \textit{tautologies}: %
    $p \Leftrightarrow q$ is true, unless: %
    $p$ has not the truth value of $q$. %
    It is easily checked that %
    %
      $(p \Leftrightarrow q) \Leftrightarrow %
        \left(
          (p \Rightarrow q) \land 
          (p \Leftarrow  q)
        \right)$; see the below definitions.
    
  }
  \item[$\Rightarrow$]{%
    The logical implication is denoted by $\Rightarrow$: %
    $p \Rightarrow q$ means %
    \textit{if (criterion/premise) $p$ then (conclusion) $q$}, or, %
    alternatively, \textit{$p$ implies $q$}. %%
    %
    $p \Rightarrow q $ is formally defined as $\lnot p \lor q$. %
    %
    Remark that the ``reasoning'' $p \Rightarrow q $ is always valid, unless: %
    $p$ true, $q$ false. Moreover, %
    %
      $p \land (p \Rightarrow q) \Rightarrow q$ %
    %
    is always true.
    }
    \item[$\Leftarrow$]{ $q \Leftarrow p$ is $ p\Rightarrow q $ read backward. 
    A common pronunciation is \textit{$q$ since $p$}.
    }
\end{enumerate}
%
For a subtle introduction to proposition logic, %
see Section 1.3 and Subsection 16.1.3 of \cite{SpecifyingSystems}.
%
\subsection{Iverson notation}%
Given a boolean expression $\varphi$, %
$\boolean{\varphi}$ returns the truth value of $\varphi$, encoded as follows, %
%
\begin{align} \nonumber
  \boolean{\varphi}\triangleq 
  \begin{cases}
    0 & \quad\quad \text{if } \varphi \text{ is false;} \\
    1 & \quad\quad \text{if } \varphi \text{ is true.}
  \end{cases}
\end{align}
%
For example, $\boolean{1 > 0} = 1$ but $\boolean{ \sqrt{2} \in \Q} = 0$.
\section{Special terms}
\subsection{Halmos' iff and definitions}%
\iif is a short for ``if and only if". %
Splitting \iif into \textit{if-then} clauses shows that it is just %
a rewording of the logical equivalence $\Leftrightarrow$. %
All definitions will use the \iif format; %
which is consistent with the fact that every definition expresses a tautology.
%
\subsection{Assigning values}%
Given variables $\varit{a}$ and $\varit{b}$, $\triangleq$ is a specialization %
of $=$. We say that $x\triangleq y$ \iif $x$ and $y$ are assumed to be equal. 
Usually, $x\triangleq y$ means that $x$ is assigned the previously known 
value $y$ (some authors write $x:=y$) but this is not a limitation. 
Definitions can be redundant and may overlap. The only restriction is that %
$x\triangleq y$ is inconsistent whether $x\neq y$.

\subsection{Equinumerosity}%
$a\equiv b$ means that there exists a bijection $\to$ that maps $a$ to $b$; %
which let us identify $a$ with $b$. %
In a metric space context, $a\equiv b$ means that $\to$ is isometric.
\section{Topological vector spaces}
\subsection{Scalar field}%
The usual (complete) scalar field is $\C$. %
A property, \eg linearity, that is true on $\C$ is also true on $\R$. %
The complex case is then a {\it special case} of the real one. %
Sometimes, this specialization is not harmless. %
For example, theorem 12.7 of \cite{FA} asserts that, in a Hilbert space $H$ %
equipped with the inner product $\bra{\,\cdot\,}\ket{\,\cdot\,}$, %
every nonzero linear continuous operator $T$ ``breaks orthogonality'', %
in the sense that there always exists $x=x(T)$ in $H$ that satisfies %
%
  $\bra{Tx}\ket{x} \neq 0$. %
%
The proof of this theorem strongly depends on the complex field. %
Actually, a real counterpart does not exists. %
To see that, consider the $90^\circ$ rotations of the euclidian plane. %
%
Nevertheless, {\it unless the contrary is explicitely mentioned}, %
the exension to the real case will always be obvious. %
So, taking $\C$ as the scalar field shall mean %
%
\begin{quote}{\it %
  Instead of letting the scalar field undefined, we choose $\C$ for the sake of %
  expessivity. But considering $\R$ instead of %
  $\C$ would actually make no difference here.
}
\end{quote}
%
\subsection{Vector space bases}\label{notations: vector spaces: vector space bases}
Given a vector space $X$ over $\C$ (or, more generally, over a field), %
a subset $B$ of $X$ is a basis of $X$ \iif %
$(z_u) \mapsto \sum_{u\in B} z_u u$ bijectively maps all {\it almost null} %
$z: B \to \K, u \mapsto z_u$ %
onto $X$. %
%
The axiom of choice (AC) forces %
\begin{enumerate}
  \item the existence of such $B$ %
    (the proof is similar to the second part of the Hahn-Banach theorem [3.1] of \cite{FA} %
    with $B$ playing the role of $\Lambda$); %
  \item all bases to have the same cardinal, %
    which is called the {\it dimension} of $X$ and is denoted as $\dim(X)$. %
\end{enumerate}
%
We now come to the {\it finite-dimensional} case, \ie $\dim(X)$ is a nonnegative integer $n$. %
Remark that $n=0$, \ie $B=\emptyset$, means that $X$ is a singleton. %
%
Our first step consists in studying $\C^n$, which is the standard $n$-dimensional vector space.
%
\subsection{Finite-dimensional spaces}\label{notations: vector spaces: finite-dimensional vector spaces}%
%
\quote{\it%
  From now on, the zero-dimensional case, which is trivial, shall be skipped. %
}
%
\subsubsection{The product topology of $\C^n$}\label{notations: vector spaces: finite-dimensional vector spaces: the product topology of Cn}
%
As the $n$-th power of $\C$, $\C^n$ is topologized by the polydiscs %%
\begin{align}
  \prod_{i=1}^{n} D(r_i) \quad (D(r_i) \Def \set{z_i \in C}{\magnitude{z_i} < r_i}), %
\end{align}
%as $r_i$ ranges over the real line. %%$]0, \infty[$. %
Equivalently, we may equipp $\C^n$ with the euclidian norm % %
\begin{align}
  \norma{2}{z} \Def \sqrt{\magnitude{z_1}^2 + \cdots + \magnitude{z_n}^2} \quad \left(z = (z_1, \dots, z_n) \in \C^n\right), 
\end{align}
%
whose open balls centered at the origin are all %
\begin{align}
  B(r) \Def \set{z\in \C^n}{\norma{2}{z} < r} \quad (r > 0).
\end{align}
To see such equivalence, first pick a positive $r$ then set $r_i = r/\sqrt{n}$, so that %
\begin{align}
\prod_{i=1}^{n} D(r_i) \subseteq B(r). 
\end{align}
%
Next, conversely choosing $r = \min\{r_1, \dots, r_n\}$ yields %
\begin{align}
   B(r) \subseteq \prod_{i=1}^{n} D(r_i) .
\end{align}
%
\subsubsection{Topology of a finite-dimensional vector space}
It is customary to identify any $n$-dimensional vector space %
with $\C^n$ endowed with the product topology; %
see [\ref{notations: vector spaces: finite-dimensional vector spaces: the product topology of Cn}]. %
%
To see this, pick a $n$-dimensional vector space $Y$, of basis $\{u_1, \dots, u_n\}$; %
see [\ref{notations: vector spaces: vector space bases}]. %
%
Setting $u_k = f(e_k)$ means a special case of [1.20] of \cite{FA}, %
where $f$ is an isomorphism of $\C^n$ onto $Y$. %
%
Actually, $Y$ is endowed with the topology $\set{f(U)}{U \text{ open}}$, %
%
and [1.21] of \cite{FA} shows that it is the only vector space topology for $Y$. %
%
As a consequence, this establishes that $Y$ is necessarily locally convex and bounded; \ie normable; %
see [1.39] of \cite{FA}. %
Moreover, provided a norm $\norma{Y}{\,\cdot\,}$ over $Y$, there exists a positive {\it modulus of continuity} $C=C_f$ such that %
%
\begin{align}\label{norm equivalence 1}
  \norma{Y}{y} \leq C \norma{2}{z} \quad \left((z, y) \in f\right), 
\end{align}
%
since $f$ is continuous. %
%
Now pick a $n$-dimensional topological vector space $W$ then repeat the same reasoning, %
first with some $g: \C^n \to W$, %
next with $h = g\circ f^{\,\minus 1}$, in the role of $f$: %
%
We then equip $W$ with a norm $\norma{W}{\,\cdot\,}$, %
so that %
%
\begin{align}
  \norma{W}{w} \leq C_h \norma{Y}{y} \quad \left((y, w) \in h \right)
\end{align}
%
for some positive $C_h$. %
%
The special case $g=f$ means that $Y$'s norms are equivalent, %
in the sense that there exists a positive $C_{\id{}}$ such that %
\begin{align}
  \norma{\text{id}(Y)}{y} \leq C_{\id{}} \norma{Y}{y}, 
\end{align}
%
\subsubsection{The standard norms $\norma{1}{\,\cdot\,}$, $\norma{2}{\,\cdot\,}$, $\norma{\infty}{\,\cdot\,}$}
In all pecial cases $Y=\C^n$ topologized by the standard norms $1$, $2$, $\infty$, 
the optimal modulus, \ie the smallest $C = C_{i, j}$ such that %
\begin{align}
	\norma{j}{z} \leq C_{i, j} \norma{i}{z} \quad \left(z \in \C^n \right)),
\end{align}
is easily derived from definitions - see [1.19] of \cite{FA} - %
with the noticeable exception of $C_{2, 1}=\sqrt{n}$, which is usually seen as a special {\it Cauchy-Schwarz inequality}; %
see (1) in [12.2] of \cite{FA}. %
The very steps of this classical hack are left to the reader. %
% END
