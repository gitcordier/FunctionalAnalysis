\renewcommand{\labelenumi}{\arabic{enumi}.}
%\addcontentsline{toc}{chapter}{Notations and conventions}
\chapter{Notations and Conventions}
\section*{Logic}
\begin{enumerate}
\item{{\bf Halmos' iff.} \iif is a short for ``if and only if".}
\item{{\bf Definitions (of values) with $\Def$.} Given variables %
$\varit{a}$ and $\varit{b}$, %
$a\Def b$ means that $\varit{a}$ is defined as equal to $\varit{b}$.}
\item{{\bf $\equiv$.} $a\equiv b$ means that there exists a ``natural'' %
bijection $\to$ that maps $a$ to $b$; which let us identify $a$ with $b$. %
In a metric space context, $a\equiv b$ means that $\to$ is isometric.}
\item{{\bf Definitions (formul\ae).} Definitions use the \iif format. %
In other words, every definition has a ``only if''. %
}
\item{{\bf Iverson notation.} Given a boolean expression $\Phi$, %
$\boolean{\Phi}$ returns the truth value of $\Phi$, encoded as follows, %
%
  \begin{align} \nonumber
    \boolean{\Phi}\Def 
    \begin{cases}
      0 & \quad\quad \text{if } \Phi \text{ is false;} \\
      1 & \quad\quad \text{if } \Phi \text{ is true.}
    \end{cases}
  \end{align}

For example, $\boolean{1 > 0} = 1$ but $\boolean{ \sqrt{2} \in \Q} = 0$.
}
\end{enumerate}
%\section{Vector spaces}
%\begin{enumerate}
%  \item{If $X$ is a vector space of base $B$ and $e$ is an element of B, }
%\end{enumerate}
\section*{Topological vector spaces}
\begin{enumerate}
\item{{\bf Product space}}
\item {\bf Scalar field.} The usual (complete) scalar field is $\C$. %
A property, \eg linearity, that is true on $\C$ is also true on $\R$. %
The complex case is then a {\it special case} of the real one. %
Sometimes, this specialization is not purely formal. %
For example, theorem 12.7 of \cite{FA} asserts that, in a Hilbert space $H$ %
equipped with the inner product $\bra{\,\cdot\,}\ket{\,\cdot\,}$, %
every nonzero linear continuous operator $T$ ``breaks orthogonality'', %
in the sense that there always exists $x=x(T)$ in $H$ that satisfies %
%
  $\bra{Tx}\ket{x} \neq 0$. %
%
The proof of this theorem strongly depends on the complex field. %
Actually, a real counterpart does not exists. %
To see that, consider the $90^\circ$ rotations of the euclidian plane. %
%
Nevertheless, {\it unless the contrary is explicitely mentioned}, %
the exension to the real case will always be obvious. %
So, taking $\C$ as the scalar field shall mean %
%
\begin{quote}{\it %
Instead of letting the scalar field undefined, we choose $\C$ for the sake of %
expessivity. But considering $\R$ instead of %
$\C$ would actually make no difference here
}. %
\end{quote}
%
\item {\bf Finite dimensional spaces}. %
It may be customary to identify, without loss of generality, %
any vector space of finite dimension $n$ with $\C^n$. % 
%
Such identification is relevant in the sense that all vector spaces that %
share common dimension $n$ are actually topological vector spaces that are %
homeomorphic each others. \\
\\
To see that, let $Y$ run through all $n$-dimensional subspaces of %
a given complex topological vector space $X$. %
From now on, $\C^n$ is equipped with the Euclian norm topology. %
%
It is easy to get an isomorphism $f$ of $\C^n$ onto $Y$. %
To do so, we skip the trivial case $n=0$ then pick a base $F_Y$ of $Y$. %
There so exists a one-to-one mapping of $F_{\C^n}$ onto $ F_Y$ that %
extends to an isomorphism $f: \C^n \to Y$. %
%
We now use the Section 1.21 of \cite{FA} to conclude that %
$f$ is more specifically an homeomorphism. %
%
Given two possibles copies $(Y_i, f_i)$, $(i=1, 2)$ of such pairs $(Y, f)$, %
we then obtain the following commutative diagram, %
%
\begin{equation}
\begin{tikzpicture}[-stealth,
  label/.style = { font=\footnotesize }]
  \matrix (m)
    [
      matrix of math nodes,
      row sep    = 4em,
      column sep = 4em
    ]
    {
         & \C^n &     \\
      Y_1&      & Y_2 \\
    };
  \path (m-2-1) edge node [above,label]  {$f_1^{-1}$} (m-1-2);
  \path (m-1-2) edge node [above,label]  {$f_2$} (m-2-3);
  \path (m-2-1) edge node [above,label]  {$\varphi$} (m-2-3);
\end{tikzpicture}
\end{equation}
%
It is now clear that all $Y$ are homeomorphic each other: %
Since $\C^n$ is locally convex balanced, so is every $Y$. %
%
In other words, each $\tau_Y$ is induced by norms $\| \, \|_{Y}$; %
see Section 1.39 of \cite{FA}. %
The special case $Y_1 = Y_2\,, \varphi: y \mapsto y$ %
expresses the equivalence of all such $\| \, \|_Y$, in the sense that %
\begin{align}
  A \| y\|_{Y_1} \leq \| y \|_{Y_2} \leq B\| y\|_{Y_1} \quad (y \in Y_1)
\end{align}
for some positive constants $A$, $B$. %
For instance, choose $B = \sup\{\| y \|_2: \| y \|_1 < 1\}$. %
\end{enumerate}
% END
