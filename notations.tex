% % % % % % % % % % % % % % % % % % % % % % % % % % % % % % % % % % % % % % % % % % % % % % % % % % % % % % % % % % % % % % % %
% FunctionalAnalysis 
% notations.tex
% 
% encoding: UTF-8 
% EOL: LF
%
% format: LaTeX
% indent: spaces (2)
% width: 127
% % % % % % % % % % % % % % % % % % % % % % % % % % % % % % % % % % % % % % % % % % % % % % % % % % % % % % % % % % % % % % % %
%
\chapter{Notations and Assumptions}
%\addcontentsline{toc}{chapter}{Notations and Conventions}
% % % % % % % % % % % % % % % % % % % % % % % % % % % % % % % % % % % % % % % % % % % % % % % % % % % % % % % % % % % % % % % %
% LOGIC
% % % % % % % % % % % % % % % % % % % % % % % % % % % % % % % % % % % % % % % % % % % % % % % % % % % % % % % % % % % % % % % %
\section{Special terms}
\subsection{The \IFF convention}%
\IFF is a shorthand for ``if and only if". Splitting \IFF into \emph{if-then} clauses shows that it is a natural language %
version of the logical equivalence $\iff$. All definitions are understood to be \IFF clauses, which is consistent with the %
fact that every definition expresses an equivalence.
%
\subsection{The assignment operator}%
Given variables $\varit{a}$ and $\varit{b}$, $a \Def b$ is a special form of $a = b$. We say that $a\Def b$ \IFF $a$ and $b$ %
are \emph{assumed} to be equal. Typically, $a\Def b$ is used to indicate that $\varit{a}$ is assigned the value $b$ %
(some authors write $a:=b$) but, in a different context, $a \Def b$ may also denote $a=:b$\ie $a:=b$.
%
\subsection{Iverson brackets}\label{Iverson-notation}
Given a boolean expression $\phi$, the boolean value $\Iverson*{\phi}$ encodes the truth value of $\phi$ as a \texttt{bit}:
%
\begin{equation} \nonumber
  \Iverson*{\phi}\Def 
  \begin{cases}
    0 & \qquad (\phi \text{ is false}) \\
    1 & \qquad (\phi \text{ is true}). \\
  \end{cases}
\end{equation}
%
For example, $\Iverson[\Big]{1 > 0} = 1$ but $\Iverson[\Big]{ \sqrt{2} \in \Q} = 0$. For interpretations of Boolean operators %
in logic, see \cite{SpecifyingSystems}.
% 
% % % % % % % % % % % % % % % % % % % % % % % % % % % % % % % % % % % % % % % % % % % % % % % % % % % % % % % % % % % % % % % %
% SETS
% % % % % % % % % % % % % % % % % % % % % % % % % % % % % % % % % % % % % % % % % % % % % % % % % % % % % % % % % % % % % % % %
\section{Sets}
\subsection{Subsets and supersets}%
$\subset$ and $\supset$ are the standard symbols for set ordering. No specific symbol is reserved for strict ordering, every %
constraint $X \neq Y$ will be explicitly stated when a strict subset is intended.
%
Given $A$ and $B$, $A\cup B$ is the union of $A$ and $B$. More generally, for any collection $C$, the union $U$ of $C$ is %
expressed as follows:
%
\begin{equation}
U \Def \unionOfCollection{C} \Def \bigcup_{S \in C} S.
\end{equation}
%
Similarly, $\cap$ denotes intersection in the same manner. %
%
\subsection{Special mappings}
The identity $I$ (or $id$) is the mapping $\set{(x,x)}{x\in X}$. Similarly, the projection %
$\pi = \set{((x, y), x)}{x \in X, y \in Y}$ always exists. Observe that $I$ is the diagonal of $X \times X$. %
%
\subsection{Equinumerosity}%
In a metric space context, $X\equiv Y$ means that $\phi$ is a surjective isometry. %
%
% % % % % % % % % % % % % % % % % % % % % % % % % % % % % % % % % % % % % % % % % % % % % % % % % % % % % % % % % % % % % % % %
% TVS 
% % % % % % % % % % % % % % % % % % % % % % % % % % % % % % % % % % % % % % % % % % % % % % % % % % % % % % % % % % % % % % % %
\section{Topological vector spaces}
\subsection{Scalar field}%
$\C$ extends $\R$, which implies that a property\eg linearity, that holds over $\C$ also holds over $\R$. The complex %
case is therefore \emph{stronger} than the real case. This restriction may be significant in some contexts. However, the %
standard scalar field is $\C$. Unless the field is explicitly given, as in \citeFA{3.1 and 12.7}, we assume that results for %
$\C$ apply equally to the real case. 
%
% % % % % % % % % % % % % % % % % % % % % % % % % % % % % % % % % % % % % % % % % % % % % % % % % % % % % % % % % % % % % % % %
% Vector spaces
% % % % % % % % % % % % % % % % % % % % % % % % % % % % % % % % % % % % % % % % % % % % % % % % % % % % % % % % % % % % % % % %
\subsection{Vector space bases}\label{vector-space-basis}
Given a vector space $X$, a subset $B$ of $X$ is a basis of $X$ \IFF %
the sum %
%
\begin{align}
  \set[\Big]{(z_u)_{u\in B}}{z_u \in \C, \set{u}{z_u \neq 0} \text{ is finite}} & \to X \\\nonumber
  (z_u) & \mapsto \sum_{u \in B} z_u u
\end{align}
is a bijection from all \emph{finitely supported} families $(z_u)$ onto $X$. %
%
The axiom of choice (AC) forces %$
%
\renewcommand{\labelenumi}{(\alph{enumi})} 
\begin{enumerate}
  \item the existence of such $B$ %
    (the proof is similar to the second part of the Hahn-Banach theorem \citeFA{3.1} with $B$ playing the role of 
    $\Lambda$); %
  \item all bases to have the same cardinality, which is called the \emph{dimension} of $X$ and is denoted as $\dim{X}$. %
\end{enumerate}
%
We now turn to the finite-dimensional case over the field $\C$. The zero-dimensional case is $B =\emptyset$ %
\ie $X=\singleton{0}$. Our first step is to study $\C^n$, the standard $n$-dimensional vector space, when $\counting{n}$. 
%
% % % % % % % % % % % % % % % % % % % % % % % % % % % % % % % % % % % % % % % % % % % % % % % % % % % % % % % % % % % % % % % %
% Finite dimension
% % % % % % % % % % % % % % % % % % % % % % % % % % % % % % % % % % % % % % % % % % % % % % % % % % % % % % % % % % % % % % % %
\subsection{Finite-dimensional spaces}\label{notations: vector spaces: finite-dimensional vector spaces}%
\subsubsection{The product topology of $\C^n$}
$\C^n$ has the standard basis $1_{\singleton{1}},\dots, 1_{\singleton{n}}: \{1, \dots, n\} \to \{0, 1\}$ so that the scalar %
$z_k$ is the $k$-th component of %
%
\begin{equation}
  (z_1,\dots, z_k, \dots, z_n) = 
  z_1 \cdot (\,\underbrace{1, 0, \dots}_{1_{\singleton{1}}}\,)
  + \cdots
  + z_k  \cdot (\,\underbrace{0, \dots, 1, 0, \dots}_{1_{\singleton{k}}}\,)
  + \cdots
  + z_n  \cdot (\,\underbrace{0, \dots, 1}_{1_{\singleton{n}}}\,), 
\end{equation}
%
as $z=(z_1, \dots, z_n)$ ranges over $\C^n$. A common notation is to let $e_k$ stand for $1_{\singleton{k}}$. Moreover, %
$\C^n$ is endowed with the topology generated by all polydiscs %
%
\begin{equation}
  \prod_{i=1}^{n} \{\,\underbrace{z_i \in \C: \magnitude*{z_i} < r_i}_{D_{r_i}}\,\} \qquad (r_i > 0 ).
\end{equation}
%
Equivalently, we may equip $\C^n$ with the Euclidean norm % 
%
\begin{equation}\label{Euclidean-norm}
  \norm*[2]{z} \Def \sqrt{\magnitude*{z_1}^2 + \cdots + \magnitude*{z_n}^2}, 
\end{equation}
%
whose open balls centered at the origin are all %
%
\begin{equation}
  B_r \Def \set[\Big]{z\in \C^n}{\norm*[2]{z} < r} \qquad (r > 0).
\end{equation}
%
To show the equivalence, first set $r_i = r/\sqrt{n}$. Hence %
%
\begin{equation}
  \prod_{i=1}^{n} D_{r_i} \subset B_r. 
\end{equation}
%
Conversely, put $r = \min{(r_1, \dots, r_n)}$ so that %
%
\begin{equation}
   B_r \subset \prod_{i=1}^{n} D_{r_i} .
\end{equation}
%
\subsubsection{Topology of a finite-dimensional vector space}
It is customary to identify any $n$-dimensional vector space with $\C^n$ equipped with the Euclidean norm; see %
[\ref{Euclidean-norm}]. %
%
To show this, choose an isomorphism $f: \C^n \to Y$. For instance, let $f(e_k)$ be $u_k$ as in \citeFA{1.20} when $\{u_k\}$ %
is a basis of the $n$-dimensional $Y$; see [\ref{vector-space-basis}]. It follows from \citeFA{1.21} that $f$ is a %
homeomorphism. A striking consequence is that %
%
  \emph{$\set{f(U)}{U \text{ open in }\C^n}$ is the unique topological vector space topology on $Y$}. %
%
Thus, $Y$ is necessarily locally convex and locally bounded \ie normable; see \citeFA{1.39}. Note that the formula %
$\norm*[Y]{y}= \norm*[2]{f^{\minus 1}(y)}$\;\,($y\in Y$) defines a norm. Additionally, $Y$ is locally compact, as the closed %
unit ball of $\C^n$ is compact. Choose an $n$-dimensional topological vector space $W$, and repeat the same reasoning with %
$g: \C^n \to W$, then $h = g\circ f^{\,\minus 1}$, in place of $f$. Thus, $h: Y \to W$ is a homeomorphism and $W$ is normable %
as well. The following assertions are then equivalent in the finite-dimensional context: %
%
\renewcommand{\labelenumi}{(\roman{enumi})} 
\begin{enumerate}
  \item $\dim{W} = \dim{Y},$ 
  \item $W$ and $Y$ are isomorphic,%
  \item $W$ and $Y$ are homeomorphic and normable.
\end{enumerate}
\renewcommand{\labelenumi}{(\alph{enumi})} 
%
Furthermore, the norms on $W$ and $Y$ are \emph{equivalent}. That is, for any given norm $\norm*[Y]{\cdot}$ on $Y$ and %
any given norm $\normtriple{W}{\cdot}$ on $W$, there exists a positive constant $C = C_h$ such that %
%
\begin{equation}
  \normtriple{W}{w} \leq C \normtriple{Y}{y} \qquad\left((y, w) \in h \right), 
\end{equation}
%
as $h$ is continuous. When $W=Y$, this means that all norms on $Y$ are equivalent, in the sense that %
%
\begin{equation}
  \normtriple{Y}{h(y)} \leq C \norm*[Y]{y}.
\end{equation}
%
\subsubsection{The standard norms $1$, $2$, and $\infty$}
When $\C^n$ is equipped with standard norms $\norm*[1]{\,\cdot\,}$, $\norm*[2]{\,\cdot\,}$, $\norm*[\infty]{\,\cdot\,}$, %
the least $C_{i, j}$ such that %
%
\begin{equation}
	\norm*[j]{z} \leq C_{i, j} \norm*[i]{z} 
\end{equation}
%
is easily derived from definitions - see \citeFA{1.19}, except for the case $C_{2, 1} = \sqrt{n}$, which requires the %
Cauchy-Schwarz inequality; see (1) in \citeFA{12.2}. The table below records these sharp $C_{i,j}$,
%
\begin{table}[h!]\centering
  \begin{tabular}{|c|c|c|c|}
  \hline
  \diagbox{i}{j} & 1 & 2 & $\infty$ \\ \hline
  1 & 1 & 1 & 1 \\ \hline
  2 & $\sqrt{n}$ & 1 & 1 \\ \hline
  $\infty$ & $n$ & $\sqrt{n}$ & 1 \\ \hline
  \end{tabular}
  \caption{Minimal $C_{i,j}$ for the standard norms $i, j = 1, 2, \infty$.}
  \label{tab:sharp-constants}
\end{table}
%
\subsection{Continuity and boundedness in normed spaces}%
A linear mapping $\Lambda$ is said to be \emph{bounded} \IFF \emph{$\Lambda(E)$ is a bounded set for every bounded set $E$}; %
see \citeFA{1.31}. Linearity implies that $\Lambda: X \to Y$ is bounded \IFF 
%
\begin{equation}
  \norm*{\Lambda} \Def \sup{\set[\big]{\norm*{\Lambda x}}{\norm*{x} < 1}} < \infty, 
\end{equation}
%
Given normed spaces $X$ and $Y$, bounded linear maps $\Lambda$ form a normed space $B(X,Y)$, where norm $\norm*{\Lambda}$ is %
given above; see \citeFA{4.1}. It is now easy to see that, in the current context, boundedness and continuity coincide. %
This is a particular case of \citeFA{2.8}. When $Y$ is the scalar field, this equivalence also comes from \citeFA{1.32}. %
Furthermore, we observe, that given a collection of bounded mappings $\Lambda\in B(X,Y)$, the property of equicontinuity 
now reads as 
%
\begin{equation}
  \sup_{\Lambda}{\norm*{\Lambda}} < \infty. 
\end{equation}
%
Thus, \emph{uniform boundedness} and uniform continuity coincide as well; \cf \citeFA{2.4}.
%
% % % % % % % % % % % % % % % % % % % % % % % % % % % % % % % % % % % % % % % % % % % % % % % % % % % % % % % % % % % % % % % %
% Measure theory
% % % % % % % % % % % % % % % % % % % % % % % % % % % % % % % % % % % % % % % % % % % % % % % % % % % % % % % % % % % % % % % %
\section{Measure theory}
\subsection{Radon measures}
Given a locally compact Hausdorff space $X$\eg $\R^n$, a positive Radon measure $\Lambda$ is a functional that is %
\emph{positive} on $C_c(X)$, in the sense that
%
\renewcommand{\labelenumi}{(\roman{enumi})} 
\begin{enumerate}
  \item $\phi \geq 0 \then \Lambda \phi \geq 0 \qquad \bigl(\phi \in C_c(X)\bigr)$.
\end{enumerate}
%
Theorem \citebook{2.14}{BigRudin} shows that positivity (i) implies the following \emph{continuity} property (ii):

\begin{enumerate}\setcounter{enumi}{1}
  \item For each compact $K \subset X$ there exists a ``continuity bound'' $M_K$ such that %
  \begin{equation*}
    \magnitude*{\Lambda \phi} \leq M_K\norm*[\infty]{\phi} \qquad \bigl(\phi \in C_c(X): \supp{\phi} \subset K \bigr).
  \end{equation*}
\end{enumerate}
\renewcommand{\labelenumi}{(\alph{enumi})} 
% 
Condition (ii) defines Radon measures in a weaker sense; see \cite{AnalyseIII}. Furthermore, $\Lambda$ is bounded on $C_c(X)$ %
with respect to the supremum norm \IFF (ii) is strengthened by %
$\norm*[\infty]{\Lambda}\leq\sup\set{M_K}{K \subset X\;\text{compact}} < \infty$. Such uniformly continuous functionals %
$\Lambda$ constitute the space of \emph{bounded} Radon measures. According to \citebook{6.19}{BigRudin}, each of them can be %
isometrically identified with a specific regular Borel measure $\mu$. Conversely, when the ambient space $X$ is compact, every 
Radon measure is $\norm*[\infty]{\cdot}$-bounded. For instance, this case is addressed in Exercise 3 of Chapter 2. In measure %
theory, by \emph{support} of $\mu$, we mean: 
%
\begin{equation}
  \supp{\mu} \Def X \bigsetminus \bigcup \; V, 
\end{equation}
%
where $V$ runs through all open sets of measure $0$. A very important Radon measure of support $\singleton{0}$, the %
\emph{Dirac delta function} is described in [\ref{annex-Dirac}].
%
\subsection{Lebesgue integration}
Theorem \citebook{2.14}{BigRudin} states that every positive Radon measure $\Lambda: C_c(X)\to \C$ is identified with a %
positive and regular Borel measure $\beta:X \to \R$. More precisely, % 
%
\begin{equation}\label{definition-of-sum}
	\int_{X} \phi \diff{\beta} \Def \Lambda \phi \qquad \bigl(\phi \in C_c(X)\bigr),
\end{equation}
%
where the integral on the left-hand side is a Lebesgue integral. The standard example is $X=\R$ with $\beta = \beta_\R$ the %
Lebesgue measure on Borel sets. The regularity property implies that $\beta$ has total mass %
%
\begin{equation}
	\int_{X} 1 \diff{\beta} \Def \sup \set[\big]{\Lambda \phi}{\phi \in C_c(X), \norm*[\infty]{\phi} \leq 1} \leq \infty.
\end{equation}
%
When we substitute $f\diff{\beta}$ for $\diff{\beta}$, with $f$ $X$-measurable, \eqref{definition-of-sum} takes the form %
%
\begin{equation}\label{Borel-integral}
	\int_{X} f \phi \diff{\beta} = \Lambda \phi.
\end{equation}
%
In particular, if $f$ is positive, 
%
\begin{equation}
	\int_{X} f \diff{\beta} = \sup \set{\Lambda \phi}{\phi \in C_c(X), \norm*[\infty]{\phi} \leq 1}.
\end{equation}
%
Moreover, given $N = N(\beta)$ denoting $\set{f}{f = 0 \;\,\beta\text{-a.e.}}$, we see that the density $f$ and the measure %
$\Lambda$ are identified modulo $N$ in \eqref{Borel-integral}. Algebraically speaking: %
%
\begin{equation}
	g\,\beta = f\,\beta \iff g-f \in N.
\end{equation}
%
A central question in calculus is whether the integral of the modulus, namely 
%
\begin{equation}
  \magnitude*{f}_1 \Def \int_{X}\magnitude*{f}\diff{\beta} \qquad \left(f \text{ $X$-measurable}\right), 
\end{equation}
%
is finite. This motivates the following definitions:%
%
\begin{equation}
	\mathcal{L}^1(X, \beta)\Def \set[\big]{f}{\magnitude*{f}_1  < \infty} 
  \subset \mathcal{L}^1_{\text{loc}}(X,\beta) 
  \Def \intersectionOfCollection{\set[\big]{\mathcal{L}^1(K,\beta)}{K \subset X \text{ compact.}}}.
\end{equation}
%
Standard notation $\mathcal{L}^1(\beta)$ is used to avoid redundancy, or simply $\mathcal{L}^1$ when the context is clear. %
Key points include: 
%
\renewcommand{\labelenumi}{(\alph{enumi})} 
\begin{enumerate}
  \item $\mathcal{L}^1$, equipped with $\magnitude*{\cdot}_1$, is a seminormed space, 
  \item $N$ is a closed subspace of the seminormed space $\mathcal{L}^1$.
\end{enumerate}
%
Together, these two facts imply that $L^1 = \mathcal{L}^1/N$, equipped with the norm %
%
\begin{equation}
	\norm*[1]{f + N} \Def \magnitude*{f}_1 \qquad (f \in \mathcal{L}^1), 
\end{equation}
%
is a Banach space. These definitions and properties extend to $L^p$ spaces ($p> 1$), as follows 
%
\begin{align}
  \mathcal{L}^p &\Def \set[\big]{f}{\magnitude*{f}^p \in \mathcal{L}^1}, \\
  L^p & \Def \mathcal{L}^p / N,  \\
  \norm*[p]{f+N} &\Def \left(\int_X \magnitude*{f}^p \right)^{1/p} \qquad (f \in \mathcal{L}^p).
\end{align}
%
Similar \emph{quasi-Banach} $L^p$ spaces exist for $0< p < 1$, with the notable difference that $\norm*[p]{f+N}$ no longer %
defines a norm (the triangle inequality is lost). In contrast, $L^\infty = \set{f+N}{\norm*[\infty]{f+N} < \infty}$, equipped %
with the \emph{quotient norm}
%
\begin{equation}
  \norm*[\infty]{f+N} \Def \inf\set[\big]{M}{\magnitude*{f} < M \;\,\beta \text{-a.e.}}
\end{equation}
%
is a Banach space.%
%
% END 
%
%